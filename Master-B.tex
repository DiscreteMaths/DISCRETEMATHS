

\chapter{Session 7}

%------------------------------------%

\section*{Session 07:Sequences and Series}
\begin{itemize}
\item[7A.1] Sequences
\item[7A.2] Induction
\item[7A.3] Series and the Sigma Notation
\end{itemize}

\subsection*{Recurrence Relations(7.1.1)}



$u_1 = 2$
$u_2 = u_1 + 3 = 2 +3 = 5$
$u_3 = u_2 + 3 = 5+ 3 = 8$
Airthmetic Progression


\subsection*{Proof by Induction(7.2.2)}
\begin{itemize}
\item[Step 1] Base case
\item[Step 2] Induction hypothesis
\item[Step 3] Induction step
\end{itemize}



\subsection*{Series and Sigma Notation(7.2.3)}





%------------------------------------------------------



\subsection*{Section 8 Exercises}
\begin{itemize}
\item $8^{\frac{1}{3}}$ Recall $a^{\frac{b}{c}} = a^{\frac{b}{c}}$
\item
\item
\end{itemize}



%------------------------------------%




%------------------------------------%
\subsection{Sequence and Series and Proof by Induction}


\[\sum^{n}_{i=1} (n^2) \]








%--------------------------------------------%
\newpage
\chapter{Session 9}
\section{Probability and Counting}
%-------------------------------------------------------------------------%
\newpage

\textbf{Three Steps}
\begin{description}
\item[Step 1]
\item[Step 2]
\item[Step 3]
\end{description}





%------------------------------------------- %


%------------------------- %
% Section 1
\subsection{Axioms of Probability}

The Axioms of Probability

\begin{itemize}
\item The probability of a certain event is 1.
\item The probability of an impossible event is 0.
\item 
\end{itemize}
%------------------------- %
%-------------------------- %
\section*{Session 09: Probability}
\begin{itemize}
\item[9A.1] Counting Methods
\item[9A.2] Counting using Sets
\item[9A.3] Probability
\item[9A.4] Independent Events
\end{itemize}
\begin{itemize}
\item[9B.1] 


%-----------------------------------------------------%







%-----------------------------------------------------%



\section*{Question 10}

\subsection*{Session 10: Matrices and Systems of Equations}
\begin{itemize}
\item[10A.1] Dimensions of a Matrix
\item[10A.2] Matrix Multiplication
\item[10A.3] Matrix Calculations
\item[10A.4] 
\end{itemize}

\begin{itemize}
\item[10B.1] Systems of Equations
\item[10B.2] Expression Systems of Equations as Matrices
\item[10B.3] Augmented Matrices
\item[10B.4] Guassian Elimination
\end{itemize}
%-----------------------------------%
\subsection*{Question 10A}

Say what information the first row of the matrix contains.
Find the number of edges of G.





\section{Matrices}

What are the dimensions of the following matrix


\[ \left(
\begin{array}{cc}
a_1 & a_2 \\ 
b_1 & b_2
\end{array} \right)\left(
\begin{array}{cc}
c_1 & d_1 \\ 
c_2 & d_2
\end{array} \right) = \left(
\begin{array}{cc}
(a_1 \times c_1) + (a_2 \times c_2) & (a_1 \times d_1) + (a_2 \times d_2) \\ 
(b_1 \times c_1) + (b_2 \times c_2) & (b_1 \times d_1) + (b_2 \times d_2)
\end{array} \right) \]

\bigskip
\large{
\[ \left(
\begin{array}{cc}
1 & 3 \\ 
0 & 2
\end{array} \right)\left(
\begin{array}{cc}
1 & 2 \\ 
4 & 1
\end{array} \right) = \left(
\begin{array}{cc}
(1 \times 1) + (3 \times 4) & (1 \times 2) + (3 \times 1) \\ 
(0 \times 4) + (2 \times 4) & (0 \times 2) + (2 \times 1)
\end{array} \right) = \left(
\begin{array}{cc}
14 & 5 \\ 
8 & 2
\end{array} \right) \]
}

\[ \left(
\left(
\begin{array}{cc}
1 & 2 \\ 
4 & 1
\end{array} \right)
\begin{array}{cc}
1 & 3 \\ 
0 & 2
\end{array} \right) = ? \]


%--------------------- %
% - 1.2 2010 
\begin{enumerate}[(i)]
\item
\end{enumerate}

%---------------------------------------------------------%
%\section*{Question 5}
%
%\begin{figure}
%\centering
%\includegraphics[width=0.7\linewidth]{GraphTheoryQuestion2012}
%
%\end{figure}

\section*{Question 6}
% Digraphs and Relations
% http://staff.scem.uws.edu.au/cgi-bin/cgiwrap/zhuhan/dmath/dm_readall.cgi?page=20
%============================================================================%
%===========================================================%
\section*{Question 10}
%%--------------------------------------------%
%\noindent \textbf{Part A : Matrix Operations - 4 Marks}\\ 
%\begin{figure}[h!]
%\centering
%\includegraphics[width=1.11\linewidth]{MatrixQuestion2012}
%
%\end{figure}
%--------------------------------------------%
\noindent \textbf{Part B : Gaussian Elimination - 5 Marks}\\ 
\begin{itemize}
\item[(i)] Say whether or not the graphs they represent are isomorphic.
\item[(ii)] Calculate $A^2$ and $A^4$ and say what information each gives about the graph
corresponding to A. [6]
\end{itemize}
%=========================================================================================== %
%-------------------------------------------------------------------------%
\newpage
\end{document}
