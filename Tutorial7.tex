



%Express your answers to 2 decimal places only.
\newpage
\chapter{Permutations}

%-------------------------------------%

How many anagrams (permutations of the letters) are there of the following words
\begin{enumerate}
\item ANSWER
\item PERMUTE
\item ANAGRAM
\item LITTLE
\end{enumerate}



%-------------------------------------%

{Permutations}

% -\vspace{-2cm}
\textbf{Part 1 : ANSWER}\\
Examples:
\begin{center}
ASNWRE,\;
SANERW,\;
REWSAN,\;...
\end{center}

Since ANSWER has 6 distinct letters, the number of permutations (anagrams) is

\[6! = 6\times 5 \times 4 \times 3 \times 2\times 1 = \boldsymbol{720} \]

%-------------------------------------%

{Permutations}

% -\vspace{-0.3cm}
\textbf{Part 2 : PERMUTE}\\
\begin{itemize}
\item The word PERMUTE has 7 letters, but only 6 different letters. 
\item There are 7! ways to arrange 7 letters.
\item However, interchanging the two Es does not result in a new permutation. There would be two identical anagrams.
\end{itemize}

\begin{center}
P\textcolor{red}{E}RMUT\textcolor{blue}{E}, \; MUT\textcolor{red}{E}P\textcolor{blue}{E}R, \; P\textcolor{red}{E}T\textcolor{blue}{E}MUR,\; ..\\
P\textcolor{blue}{E}RMUT\textcolor{red}{E}, \; MUT\textcolor{blue}{E}P\textcolor{red}{E}R, \; P\textcolor{blue}{E}T\textcolor{red}{E}MUR,\; ..
\end{center}

%-------------------------------------%

{Permutations}

% -\vspace{-0.3cm}
\textbf{Part 2 : PERMUTE}\\
\begin{itemize}
\item The number of permutations (anagrams) is half of 7! .

\[\frac{7!}{2} =  \frac{5040}{2} = \boldsymbol{2520} \]
\end{itemize}


%-------------------------------------%

{Permutations}

% -\vspace{-0.1cm}
\textbf{Part 3 : ANAGRAM}\\
\begin{itemize}
\item The word ANAGRAM has 7 letters, but there are three As.
\item From before, there are 7! ways to arrange 7 letters.
\item How many new permutations are found by re-arranging the As?
\end{itemize}

\begin{center}
\textcolor{red}{A}N\textcolor{blue}{A}GR\textcolor{purple}{A}M \; 
\textcolor{red}{A}N\textcolor{purple}{A}GR\textcolor{blue}{A}M \; 
\textcolor{blue}{A}N\textcolor{red}{A}GR\textcolor{purple}{A}M \; \\
\textcolor{purple}{A}N\textcolor{red}{A}GR\textcolor{blue}{A}M \; 
\textcolor{blue}{A}N\textcolor{purple}{A}GR\textcolor{red}{A}M \; 
\textcolor{purple}{A}N\textcolor{blue}{A}GR\textcolor{red}{A}M \; 
\end{center}


%-------------------------------------%

{Permutations}

% -\vspace{-0.1cm}
\textbf{Part 3 : ANAGRAM}\\
\begin{itemize}
\item We divide 7! by 3! to account for the identical anagrams.

\[\frac{7!}{3!} =  \frac{5040}{6} = \boldsymbol{840} \]
\end{itemize}

%-------------------------------------%

{Permutations}

% -\vspace{-2.3cm}
\textbf{Part 2 : PERMUTE}\\
\begin{itemize}
\item We re-express the answer from part 2 as follows:

\[\frac{7!}{2!} =  \frac{5040}{2} = \boldsymbol{2520} \]
\end{itemize}

%-------------------------------------%

{Permutations}

% -\vspace{-0.3cm}
\textbf{Part 4 : LITTLE}\\
\begin{itemize}
\item The word LITTLE has 6 letters, but there are two Ls and two Ts.
\item From before, there are 6! ways to arrange 6 letters.
\item Again, interchanging the two Ls and Ts does not result in a new permutation. 

\[\frac{6!}{2!\times 2!} =  \frac{720}{4} = \boldsymbol{180} \]
\end{itemize}

%------------------------------------------------%
\section*{Session 9 Probability}
\subsection*{Binomial Coefficients}
\begin{itemize}
\item factorials 
\[ n! = (n)\times (n-1)\times(n-2) \times \ldots \times 1 \]
\begin{itemize}
\item $5! = 5 \times 4 \times 3 \times 2 \times 1 = 120 $
\item $3! = 3 \times 2 \times 1$
\end{itemize}
\item Zero factorial
\[ 0! =  1 \]
\end{itemize}
%---------------------------------------------- %

% \[P(A |B = \frac{P(A \cap B)}{P(B)})\]

The complement rule in Probability

$P(C^{\prime}) = 1- P(C)$



If the probability of C is $70 \%$ then the probability of $C^{\prime}$ is $30\%$






\section{Counting}
% 2007 Q8
Given S is the set of all 5 digit binary strings, E is the set of a 5 digit
binary strings beginning with a 1 and F is the set of all 5 digit binary strings ending
with two zeroes.
\begin{itemize}
\item[(a)] Find the cardinality of S, E and F.
\item[(b)] Draw a Venn diagram to show the relationship between the sets S, E and F.
\item[(c)] Show the relevant number of elements in each region of your diagram.
\end{itemize}


%-------------------------------------------------------%
{
	{Basic Operations with Matrices  }
%-------------------------------------------------------%
{
	{Basic Operations with Matrices }
	
	\begin{itemize}
		\item Addition of Matrices
		\item Transpose of a Matrix
		\item Adding and Subtracting Matrices
		\item Scalar Multiplication
	\end{itemize}
}

%-------------------------------------------------------%
{
	
	{Matrix Multiplication}
	{
		
		\[ \left(
		\begin{array}{cc}
		a & b \\
		c & d \\
		\end{array}
		\right) \times \left(
		\begin{array}{cc}
		u & v \\
		w & x \\
		\end{array}
		\right)
		\]
		
		
		
		\[ \left(
		\begin{array}{cc}
		a & b \\
		c & d \\
		\end{array}
		\right) \times \left(
		\begin{array}{cc}
		u & v \\
		w & x \\
		\end{array}
		\right)
		\]
		
		
		
		\[ \left(
		\begin{array}{cc}
		3 & 1 \\
		2 & 4 \\
		\end{array}
		\right) \times \left(
		\begin{array}{cc}
		2 & 5 \\
		4 & 1 \\
		\end{array}
		\right)
		\]
		
		{Matrix Multiplication}
		
		{\huge
			
			\[ \left(
			\begin{array}{cc}
			\underline{\boldsymbol{a}}& \underline{\boldsymbol{b}} \\
			c & d \\
			\end{array}
			\right) \times \left(
			\begin{array}{cc}
			u & v \\
			w & x \\
			\end{array}
			\right)
			\]
			
			%-------------------------------------------------------%
			{
				{Matrix Multiplication}
				
				{\huge
				
						\[ \left(
						\begin{array}{cc}
						\boldsymbol{a} & \boldsymbol{b} \\
						c & d \\
						\end{array}
						\right) \times \left(
						\begin{array}{cc}
						\underline{\textcolor{green}{u}} & v \\
						\underline{\textcolor{green}{w}} & x \\
						\end{array}
						\right)
						\]
						
						
						{Matrix Multiplication}
						
						{\huge
							
							\[ \left(
							\begin{array}{cc}
							\boldsymbol{a} & \boldsymbol{b} \\
							c & d \\
							\end{array}
							\right) \times \left(
							\begin{array}{cc}
							\textcolor{green}{u} & v \\
							\textcolor{green}{w} & x \\
							\end{array}
							\right)
							\]
							
							\[ = \left(
							\begin{array}{cc}
							\boldsymbol{a}\textcolor{green}{u} + \boldsymbol{b}\textcolor{green}{w}& \ldots \ldots\\
							\ldots \ldots & \ldots \ldots \\
							\end{array}
							\right) 
							\]
							
	
	\textbf{Transpose of a Matrix}
	\begin{itemize}
		
		\item The transpose of a matrix is transformation of that matrix when all the rows are arranged into columns and columns arranged by rows.
		\item The transpose of a matrix $A$ is usually denoted $A^{T}$ or $A^{\prime}$.
		\item The relevant \texttt{R} function is \texttt{t()}.
	\end{itemize}

	{Sample Space (2)}
	Consider a random experiment in which a coin is tossed once, and a number between 1 and 4 is selected at random.
	Write out the sample space $S$ for this experiment.
	
	\bigskip
	
	\[ S = \{(H,1),(H,2),(H,3),(H,4),(T,1),(T,2),(T,3),(T,4)\} \]
	
	( $H$ and $T$ denoted `Heads' and `Tails' respectively. )
	
}
%-------------------------------------------------------%
{
	{Contingency Tables}
	Suppose there are 100 students in a first year college intake.  \begin{itemize} \item 44 are male and are studying computer science, \item 18 are male and studying statistics \item 16 are female and studying computer science, \item 22 are female and studying statistics. \end{itemize}
	\bigskip
	We assign the names $M$, $F$, $C$ and $S$ to the events that a student, randomly selected from this group, is male, female, studying computer science, and studying statistics respectively.
}
%-------------------------------------------------------%
{
	{Contingency Tables}
	The most effective way to handle this data is to draw up a table. We call this a \textbf{\emph{contingency table}}.
	\\A contingency table is a table in which all possible events (or outcomes) for one variable are listed as
	row headings, all possible events for a second variable are listed as column headings, and the value entered in
	each cell of the table is the frequency of each joint occurrence.
	
	
	\begin{center}
		\begin{tabular}{|c|c|c|c|}
			\hline
			% after \\: \hline or \cline{col1-col2} \cline{col3-col4} ...
			& C & S & Total \\ \hline
			M & 44 & 18 & 62 \\ \hline
			F & 16 & 22 & 38 \\ \hline
			Total & 60 & 40 & 100 \\ \hline
		\end{tabular}
	\end{center}
	
}
%-------------------------------------------------------%
{
	{Contingency Tables}
	It is now easy to deduce the probabilities of the respective events, by looking at the totals for each row and column.
	\begin{itemize}
		\item P(C) = 60/100 = 0.60
		\item P(S) = 40/100 = 0.40
		\item P(M) = 62/100 = 0.62
		\item P(F) = 38/100 = 0.38
	\end{itemize}
	\textbf{Remark:}\\
	The information we were originally given can also be expressed as:
	\begin{itemize}
		\item $P(C \cap M) = 44/100 = 0.44$
		\item $P(C \cap F) = 16/100 = 0.16$
		\item $P(S \cap M) = 18/100 = 0.18$
		\item $P(S \cap F) = 22/100 = 0.22$
	\end{itemize}
}

%-------------------------------------------------------%
{
	{Conditional Probability (1)}
	
	The definition of conditional probability:
	\[ P(A|B) = \frac{P(A \cap B)}{P(B)} \]
	
	\begin{itemize}
		\item $P(B)$ Probability of event B.
		\item [ $P(A)$ Probability of event A. ]
		\item $P(A|B)$ Probability of event A given that B has occurred.
		\item $P(A \cap B)$ Joint Probability of event A and event B.
		\item Will be given tomorrow.
	\end{itemize}
	
}


%-------------------------------------------------------%
{
	{Conditional Probabilities (2)}
	
	Compute the following:
	\begin{enumerate}
		\item $P(C|M)$ : Probability that a student is a computer science student, given that he is male.
		\item $P(S|M)$ : Probability that a student studies statistics, given that he is male.
		\item $P(F|S)$ : Probability that a student is female, given that she studies statistics.
	\end{enumerate}
	
}
%-------------------------------------------------------%
{
	{Conditional Probabilities (3)}
	
	\textbf{Part 1)} Probability that a student is a computer science student, given that he is male.
	\[ P(C|M) = \frac{P(C \cap M)}{P(M)}  = \frac{0.44}{0.62} = 0.71 \]
	\textbf{Part 2)} Probability that a student studies statistics, given that he is male.
	\[ P(S|M) = \frac{P(S \cap M)}{P(M)}  = \frac{0.18}{0.62} = 0.29 \]
	
}

%-------------------------------------------------------%
{
	{Conditional Probabilities (4)}
	
	\textbf{Part 3)} Probability that a student is female, given that she studies statistics.
	\[ P(F|S) = \frac{P(F \cap S)}{P(S)}  = \frac{0.22}{0.40} = 0.55 \]
	
	
	
	
}
%------------------------------------------------------------%
{
	{Bayes' Theorem}
	Bayes' Theorem is a result that allows new information to be used to update the conditional probability of an event.
	\bigskip
	
	\[ P(A|B) = \frac{P(B|A)\times P(A)}{P(B)} \]
	
	Use this theorem to compue $P(S|F)$, the probability that a student studies statistics, given that she is female.
	
	\[ P(S|F) = \frac{P(F|S)\times P(S)}{P(F)} = {0.55 \times 0.40 \over 0.38} = 0.578\]
}

%-------------------------------------------------------%
{
	{Independent Events}
	\begin{itemize}
		\item Suppose that a man and a woman each have a pack of 52 playing cards.
		\item Each draws a card from his/her pack. Find the probability that they each draw a Queen.
		\item We define the events:
		\begin{itemize} \normalsize \item A = probability that man draws a Queen = 4/52  = 1/13
			\item B = probability that woman draws a Queen = 1/13
		\end{itemize} \item Clearly events A and B are independent so:
		\[ P(A \cap B) = 1/13 \times 1/13 = 0.005917 \]
	\end{itemize}
	
}
%---------------------------------------------------------READY---------%
{
	{Expected Value and Variance of a Random Variable}
	
	The probability distribution of a discrete random variable is be tabulated as follows
	
	\begin{center}
		\begin{tabular}{|c||c|c|c|c|c|c|}
			\hline
			$x_i$  & 1 & 2 & 3 & 4 & 5 & 6 \\\hline
			$p(x_i)$ & 2/8 & 1/8& 1/8 & 1/8& c & 1/8\\
			\hline
		\end{tabular}
	\end{center}
	
	\begin{itemize}
		\item What is the value of $c$?
		\item What is expected value and variance of the outcomes?
	\end{itemize}
}
%---------------------------------------------------------READY---------%
{
	{Expected value(1)}
	\begin{itemize}
		\item Necessarily $C =0.25 = 2/8$. \\
		\item We must compute $E(X)$ as follows \[E(X) = \sum x_i p(x_i) \]
		\item That formula is \textbf{not} given in the formulae.
	\end{itemize}
	\bigskip
	$E(X) = (1 \times {2\over8}) + (2 \times {1 \over 8}) +  \ldots + (5 \times {2 \over 8}) + (6 \times {1 \over 8})$\\\bigskip
	$E(X) = 27/8 = 3.375$\bigskip
}
%------------------------------------------------------READY------------%
{
	{Variance(1)}
	\begin{itemize}
		\item The formula for computing the variance of a discrete random variable
		
		\[ V(X) = E(X^2) - E(X)^2 \]
		
		\item This is not given in the formulae for tomorrow's exam.
		
		\item We must compute $E(X^2)$
	\end{itemize}
	
	\begin{center}
		\begin{tabular}{|c||c|c|c|c|c|c|}
			\hline
			$x_i$  & 1 & 2 & 3 & 4 & 5 & 6 \\\hline
			$x^2_i$  & 1 & 4 & 9 & 16 & 25 & 36 \\\hline
			$p(x_i)$ & 2/8 & 1/8& 1/8 & 1/8& 2/8 & 1/8\\
			\hline
		\end{tabular}
	\end{center}
}
%-----------------------------------------------------READY-------------%
{

	
	\begin{itemize}
		\item $E(X^2) = (1 \times {2\over8}) + (4 \times {1 \over 8}) +  \ldots + (25 \times {2 \over 8}) + (36 \times {1 \over 8})$\bigskip
		\item $E(X^2) = {117 \over 8} = 14.625$ \bigskip
		\item $V(X) = E(X^2) - E(X)^2 = 14.625 - (3.375)^2 = 3.2344$
	\end{itemize}
	\bigskip
	
}

%---------------------------------------------------%

	Combinations formula
	\[ ^{n}C_k  = {n! \over k!  \times (n-k)!} \]
	
	\begin{itemize}
		\item Remark $n! = n \times (n-1)! $
		\item 0! = 1
	\end{itemize}



