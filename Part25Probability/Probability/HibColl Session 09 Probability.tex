\documentclass{article}

\usepackage{amsmath}
\usepackage{amssymb}
\usepackage{amsthm}
\usepackage{multicol}

\begin{document}

%-----------------------------------------------------%
\section*{Session 09: Probability}
\begin{itemize}
\item[9A.1] Counting Methods
\item[9A.2] Counting using Sets
\item[9A.3] Probability
\item[9A.4] Independent Events
\end{itemize}
\begin{itemize}
\item[9B.1] Permutation

\[ {n \choose r} = \frac{n!}{(n-r)! r!} \]


\[ {6 \choose 3} = \frac{6!}{(6-3)! 3!} = \frac{6!}{3! \times 3!}\]


\[ \frac{6!}{3! \times 3!} = \frac{6 \times 5 \times 4 \times 3!}{3! \times 3!} = \frac{120}{6} = 120\]
\end{itemize}

%\begin{multicol}{2}
\begin{itemize}
\item ${6 \choose 2} = 15$
\item ${5 \choose 2} = 10$  
\item ${4 \choose 0} = 1$  
\item ${4 \choose 3} = 4$  
\end{itemize}
%\end{multicol}

\begin{itemize}
\item pairwise disjoint sets
\item The addition principle
\end{itemize}
\subsection*{Theorem}
\[ |A \cup B| = |A| + |B| - |A \cap B|  \]

\subsection*{Probability}
\begin{itemize}
\item[9B.2] The sample space of an experiment ($S$)
\item[9B.3] The size of a sample space
\item[9B.4] Indepedent Evcents (9.3.1)
\end{itemize}
%------------------------------------------------%
\section*{Session 9 Probability}
\subsection*{Binomial Coefficients}
\begin{itemize}
\item factorials 
\[ n! = (n)\times (n-1)\times(n-2) \times \ldots \times 1 \]
\begin{itemize}
\item $5! = 5 \times 4 \times 3 \times 2 \times 1 = 120 $
\item $3! = 3 \times 2 \times 1$
\end{itemize}
\item Zero factorial
\[ 0! =  1 \]
\end{itemize}
%---------------------------------------------- %

% \[P(A |B = \frac{P(A \cap B)}{P(B)})\]

The complement rule in Probability

$P(C^{\prime}) = 1- P(C)$

 

If the probability of C is $70 \%$ then the probability of $C^{\prime}$ is $30\%$



\end{document}

