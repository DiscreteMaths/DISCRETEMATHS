\documentclass[12pt]{article}

%opening
\title{}
\author{}

\begin{document}

\section{Set Theory}
A set, in mathematics, is a collection of distinct entities, called its elements, considered as a whole. The early study of sets led to a family of paradoxes and apparent contradictions. It therefore became necessary to abandon "naïve" conceptions of sets, and a precise definition that avoids the paradoxes turns out to be a tricky matter. However, some unproblematic examples from naïve set theory will make the concept clearer. These examples will be used throughout this article:

\begin{itemize}
\item A = the set of the numbers 1, 2 and 3.
\item B = the set of primary light colours—red, green and blue.
\item C = the empty set (the set with no elements).
\item D = the set of all books in the British Library.
\item E = the set of all positive integers, 1, 2, 3, 4, and so on.
\end{itemize}

Note that the last of these sets is infinite.

A set is the collection of its elements considered as a single, abstract entity. Note that this is different from the elements themselves, and may have different properties. For example, the elements of D are flammable (they are books), but D itself is not flammable, since abstract objects cannot be burnt.

\subsection{Notation - Listing Method}

By convention, a set can be written by \textbf{listing} its elements, separated by commas, between {braces}. 

Using the sets defined above:
\begin{itemize}
\item A = $\{1, 2, 3\}$
\item B = $\{red, green, blue\}$
\item C = $\{\}$
\end{itemize}
This is impractical for large sets (D), and impossible for infinite ones (E). 
\subsection{Notation - Building Method}
Thus a set can also be described by naming a particular property that is shared by all its elements and only by them. A common notation uses a bar (|) to separate a variable name (e.g. "x") from a property of the variable that elements of the set must have. For example:
\begin{itemize}
\item D = $\{x |$ x is a book and x is in the British Library $\}$
\item E = $\{x |$ x is a positive integer$\}$
\end{itemize}
A simple translation of this notation is that "\textit{D is the set of all x, where x is a book and x is in the British Library}" or "\textit{E is the set of all x, where x is a positive integer}".

\section{Definition of a set}

A set is defined completely by its elements. Formally, sets X and Y are the same set if they have the same elements; that is, if every element of X is also an element of Y, and vice versa. For example, suppose we define:

\[ F = \{x | \mbox{(x is an integer)} \mbox{ and } 0 < x < 4)\}  \]
%Then F=A. The definition also makes it clear that {1,2,3} = (3,2,1} = {1,1,2,3,2}. Note also that if X and Y are both empty, then they are equal, justifying referring to "the" empty set.

The equivalence of empty sets has a \textbf{\textit{metaphysical}} consequence for some theories of the metaphysics of properties that define the property of being x as simply the set of all x, then if the two properties are uninstantiated or coextensive they are equivalent - under this theory, because there are no unicorns and there are no pixies, the property of being a unicorn and being a pixie are the same - but if there were a unicorns and pixies, we could tell them apart. (See Universals for more on this.)

\subsection{Elements and subsets}

The $\in$  sign indicates set membership. If x is an element (or "member") of a set X, we write $ \in X$; e.g. $3\in A$. (We may also say ``X contains x" and ``A contains 3")

A very important notion is that of a subset. X is a subset of Y, written $X \subseteq Y$ (sometimes simply as$X \subset Y$), if every element of X is also an element of Y. From before $C\subseteq A \subseteq E$.
\subsection{Sets containing Sets}


Sets can of course be elements of other sets; for example we can form the set $G = \{A,B,C,D,E\}$, whose five elements are the sets we considered earlier. Then, for instance, $A\in G$. (Note that this is very different from saying $A\subseteq G$)

%The tilde (~) is as usual used for negation; e.g. $\tilde(A\in B)$.

\subsection{Set operations}

Suppose X and Y are sets. Various operations allow us to build new sets from them.

\begin{description}
\item[Union]
The union of X and Y, written $X\cup Y$, contains all the elements in X and all those in Y. Thus $A \cup B = \{1, 2, 3, red, green, blue\}$. As A is a subset of E, the set $A \cup E$ is just E.

\item[Intersection]

The intersection of X and Y, written X∩Y, contains all the elements that are common to both X and Y. Thus {1,2,3,red,green,blue} ∩ {2,4,6,8,10} = {2}.

\item[Set difference]

The difference X minus Y, written X−Y or $X\|Y$, contains all those elements in X that are not also in Y. For example, E−A contains all integers greater than 3. A−B is just A; red, green and blue were not elements of A, so no difference is made by excluding them.
\end{description}
\newpage
%---------------------------------------------------------------------------%
\subsection{Complement and universal set}

The universal set (if it exists), usually denoted U, is a set of which everything conceivable is a member. In pure set theory, normally sets are the only objects considered (unlike here, where we have also considered numbers, colours and books, for example); in this case U would be the set of all sets. (Non-set objects, where they are allowed, are called 'urelemente' and are discussed below.)

In the presence of a universal set we can define X′, the complement of X, to be $U−X$. X′ contains everything in the universe apart from the elements of X. We could alternatively have defined it as

\[X′ = \{x | \tilde (x\in X)\}\]
and U as the complement of the empty set.

\subsection{Cardinality}

The cardinality $\|X\|$ of X is, roughly speaking, its size. The empty set has size 0, while B = $\{red, green, blue\}$ has size 3. 

This method of expressing cardinality works for finite sets but is not helpful for infinite ones. A more useful notion is that of two sets having the same size: if a direct one-to-one correspondence can be found between the elements of X and those of Y, they have the same size. 

Consider the sets, both infinite, of positive integers $\{1,2,3,4, \ldots\}$ and of even positive integers $\{2,4,6,8, \ldots\}$. One is a subset of the other. Nevertheless they have the same cardinality, as is shown by the correspondence mapping n in the former set to 2n in the latter. We can express this by writing $\|X\|$ = $\|Y\|$, or by saying that X and Y are "equinumerous".

\subsection{Power set}

The power set of X, $P(X)$, is the set whose elements are all the subsets of X. Thus \[P(A) = \{ \{\}, \{1\}, \{2\}, \{3\}, \{1,2\}, \{1,3\}, \{2,3\}, \{1,2,3\}\}\]. The power set of the empty set $P(\{\})$ = $\{\{\}\}$. 

Note that in both cases the cardinality of the power set is strictly greater than that of base set: No one-to-one correspondence exists between the set and its power set. 

%Cantor proved that this in fact holds for any set (Cantor's Theorem). This is obvious for a finite set, but Cantor's ingenious proof made no reference to the set being finite; the theorem holds even for infinite sets. This was a powerful generalisation of his previous discovery that different sizes of infinity exist.
\end{document}
