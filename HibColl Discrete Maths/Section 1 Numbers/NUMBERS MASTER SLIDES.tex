HAPTER 1. NUMBERS SYSTEMS ll.
1.2.1 Factors, multiples and primes
Definition 1.5 Suppose that m and n are positive integers. Ifn divides into m exactly, so that
the quotient m/n is also on integer, then we say that ri divides m. We also soy that n is a factor
(or a divisor) of m, and that m is a multiple ofn.
Thus, for example, we can say that “4 divides 20", “4 is a factor of 20”, "20 is a multiple of 4”.
These three sentences all have the sarne meaning. '
Definition 1.6 A prime number (or more simply, a prime) is an integer greater than 1 whose
only positive factors are itself and 1.
Notice carefully that the number 1 is not a prime. It is called a unit and has the property that
it divides every integer. A number greater than 1 that is not prime is called composite. The
smallest primes are thus 2,3,5,7,11,13,l7,..., whereas the numbers 4,6,8,9, 10, 12, 14, 15, 16, 
are composite. ‘
Example 1.22 The composite number 30 can be written as a product of two positive integers
greater than 1 in several different ways. For example, we could write
30=2><15,3O=3><10,or30=5><6.
Suppose we now repeat the process and express the composite integers in these products as products
of factors, until every factor in each product is a prime. We would then obtain
2><15 : 2><3><5
3x10 : 3><5><2
5><6 : 5><2><3.
Note that each way of splitting up 30 gives the some set of prime factors; the final expressions
differ only in the order which the primes are listed. I3
Example 1.22 illustrates the following general result. You should be familiar with its statement,
but the proof is beyond the scope of this course.
Theorem 1.7 The Fundamental Theorem of Arithmetic Every positive integer greater than
1 can be written as the product ofl and prime factors. Further, this expression is unique, except
for the order in which the prime factors occur.
The only reason for the appearance of 1 in this theorem is to accomodate the case when the integer
is itself a prime number, such as 29 for example. It does not make sense to say that 29 is a product
of 29, but we can say that 29 is a product 1 and 29. When asked to express a composite integer
as a product of prime factors, we omit the unit l. This theorem guarantees that however we go
about splitting up an integer into prime factors, we will always obtain the same prime factors.
Example 1.23 We write 280 as the product ofits prime factors. From Theorem 1.7, we can start
by writing 280 as the product of any two of its factors. An obvious factor of 280 is 10, and so we
can proceed as follows:
280 = 10 >< 28
= (2 >< 5) >< (4 >< 7)
= ‘2><5><(2><2)><'i.
To simplify the expression, we gather together all occurences of the same prime and write the
product as
280 = 23 >< 5 X 7,
using index notation to deal with repeated factors, and listing the distinct primes in ascending
order of size. I3 >
Since every positive integer corresponds in a unique way to its prime factors, the primes can be
regarded as the basic building blocks of the integers. The study of primes and the properties of



CHAPTER 1. NUMBERS SYSTEMS 12
the integers has a long history, going back at least 2500 years to the early Greek mathematicians
of the School of Pythagoras. An important modern application of prime numbers is to Public Key
Cryptography. The details of this application are outside the scope of this course‘.
1.2.2 Representing fractions
Definition 1.8 Numbers that can be ezpressed as a fraction (or ratio) m/n of two integers m and
1i, where n 9’: O, are called rational numbers.
We have two ways of representing a rational number, whatever the number base. One is to write it
in the form of a fraction m/n, where both m and n are integers (expressed in the relevant number
base] and n # O. Note that all integers are special cases of rational numbers, since we can write
an integer m as the fraction m/1.
The other method of representing a rational number is to write it in the form of a decimal, using
negative as well as positive and zero powers of the base as place values. We shall see in this section
that the decimal form of a rational number has a very special property.
1.2.3 Decimal fractions
Let us look first at the representation of rational numbers in base 10 in expanded notation. The
place values are illustrated by the following boxes.
~-I103Ii0‘l|101|10°|10"‘I10-2|10—3I---
I I I I I I I I
You will be familiar with the process of turning a fraction m/1| into a decimal by dividing n into
m. Since we are interested in what happens to the part of the expansion after the decimal point,
we shall assume that m and n are positive integers with m less than n.
The decimal expansion of a. fraction has an interesting feature: it either terminates after a finite
number of places, as for example in the expansion 17/20 = 0.85 or 3/8 = 0.375; or it has a recurring
block of figures, as for example the digit 6 in the expansion 5/12 = 0.41666... or the block 63 in
7/ll = 0.636-363.... Weinvestigate why this occurs.
lt can be shown that when we divide by a positive integer n, we can always choose the quotient
so that the remainder is either O or one of the positive integers less than n. There are thus only
n possible remainders when we divide by n. Now when we divide m.0O0 - - » by n, the remainder
at each division, becomes the “carrying number" for the next division. Since we can continue the
divisions indefinitely and there are only n possible carrying numbers, we either obtain (eventually)
a remainder 0, in which case the decimal expansion terminates, or we eventually obtain o remainder
that has occurred before. From then on, the same sequence of quotients and remainders will occur
again and again.
Example 1.24 Consider the decimal expansion of 4/T. On dividing by 7, the remainder is always
one of the six possibilities: 0,1, 2, 3,4, 5 or 6.
0.57142s51--
1|4.‘*o@o*0*0*0°0*o*0
As you see, in this case, all six possible remainders arise: the first is 4, then we have successively
remainders of 5, 1, 3, ‘2, 6; then the remainder 4 occurs again. From this point on, the same sequence
of six remainders and quotients repeats itself. Thus the decimal expansion of 4/7 has a repeated
block of six digits. EI
We now consider the converse problem. Suppose we are given a terminating or repeating decimal.
ls it possible to Write it as a fraction m/n, where m and n are integers? Clearly this is possible in
the case of a terminating decimal, since we can write it with a denominator that is a power of 10.
By a neat trick, we can show that any recurring decimal can also be expressed as a fraction m/n
and find the values of 1n and n. We explain the method in the context of the following examples.
llf you are interested in reading about this application, there is a clear and interesting explanation in Discrete
ll-lathe-maiics with Applications, by MD. Albertsorr and .l.P. I-Iutchings (see Appendix A).



CH A PTER 1 . NUMBERS S YSTEMS 14
The binary number (101.11); represents
1(22) + 0(21)+1(1)+1(2“)+1(2-2): 2’ + 2“ + 2-1+ 2"? = 5.15.
Alternatively, we could work as follows,
(101.11); : (101); +(l1/100); = 5 + 3/4 = 5175. El
The digits to the left of the point are called the integral part of the number and the digits to the
right or" the point are called the fractional part of the number. Thus, for example, the integral
part of (101.11); is (101); and the fractional part is (0.11);
Hexadecimal system
The place values in the full hexadecimal system are illustrated below.
 | 163 I 162 I 161 | le°=1 I 16* | 16'? | 16-3 | 
I I I I I I I I
Example 1.28 The number (E713B)1a represents the decimal number
14(161) + 1(1) + 3(1s-1)+11(1s-2) = 231+ 3/16 + 11/256 = 231.2a0=16s15. o
To convert the fractional part ofa binary number into the hexadecimal system, we split the binary
string into groups of four bits starting from the decimal point and working to the right. If the last
block to the right contains less than four bits, we add zeros as necessary on the end of the number
to complete the block. We then use the binary-hex conversion table given in section 1.1.41
Exaniple 1.29 Converting the binary number (101011101.11)g to hex, we have
10101110111: 0001 0101 1101 .1100 = (15D.C)15.
Converting the hex number (B35116 to binary, we have
B3.5 =101100110101:(10110011.0101)g.El
1 .3 Real numbers
Learning Objectives
After studying this section, you should be able to:
e give examples of the kinds of measurements for which the real number system is used;
o give examples of irrational numbers;
0 use and interpret greater than and less than signs;
0 express a rational number in floating point form and scientific notation.
Introduction
It is not easy to define exactly what we mean by a real number. We can visualize them geometically
as a set of numbers that can be put into one-to~one correspondence with the points of a line, infinite
in e.\'teuL This is the procedure we adopt when we mark a scale on a coordinate axis in order to
graph a function. The real numbers have the important property that they form what is called a
rxozrfinuumi The set of real numbers is therefore used when we want to measure a quantity such as
tirne. length or speed that can vary continuously‘



CHAPTER 1. NUMBERS SYSTEMS 15
1.3.1 Irrational numbers
The question arises as to whether all real numbers are rational, or whether there exist numbers that
cannot be expressed as a fraction of two integers. The ancient Greek mathematicians answered
this question in the 4th Century BC by showing that the square root of any integer that is not
an exact square cannot be expressed as a fraction. This shows the existence of real numbers that
are not rational. We call such numbers irrational. As well as numbers like \/5, V5, etc, numbers
such as 11' and e (base of the natural logarithm) are irrational numbers. It follows from Result 1.9-,
that every irrational number has a decimal expansion that neither repeats nor terminates. Thus
W€ can only ever give an approximation to the value of an irrational number in the decimal number
system, or indeed in any number system with an integer base.
1.3.2 Inequality symbols
Suppose 2 and y are two different real numbers. The fact that tr is not equal to y is expressed
symbollically as
2: ¢ y.
Note that the word distinct is often used in mathematics, rather than “different”, to describe two
or more objects of the same class that can be distinguished from one another. So, in this case, we
could have described :2 and y as “distinct numbers".
If at ¢ y, then one of 1: and y must be the greater of the two, say 2 is greater than y. We write
this as
1! > 1/.
An equivalent statement is “y is less than :r:”, written
y < 1:.
To avoid confusing these symbols, note that the pointed end (or small end) of either inequality
sign always points towards the smaller number, while the open (or larger) end points to the larger
number.
Example 1.30 The decimal expansion of 1r begins 3.1415 . . .. Thus we can say that 1r > 3.14 and
also that 'rr < 3.15. Together, these two inequalities tell us that the true value of 1r is between 3.14
and 3.15. We can express this by concatenating the two inequalities as follows:
3.14 < 1r < 3.15.
Notice that we have turned the inequality 1r > 3.14 round and expressed it as 3.14 < 1.", before
combining it with the other inequality, so that 1r is sandwiched between the smallest and the largest
of the three numbers and both inequality signs point in the same direction. D
Example 1.31 The decimal expansion of 17/9 is 1.888.... Thus 17/9 lies between 1.888 and
1.389. We can express this symbolically by writing
1.888 <17/9 < 1.889.121
The symbol “>” can be modified to express at least or, equivalently, greater than or equal ta. For
example,
:0 Z 5.3
is read “r is at least 5.3”, or “rs is greater than or equal to 5.3”. Similarly, the symbol §’ reads
“at most” or “less than or equal to”. “'
Example 1.32 Suppose a real number 2 satisfies the inequalities 14 § .1: § 25. The inequality
14 5 at tells us that 2 is at least 14. The inequality .1: § 25 means that 1' is at most 25. Thus
together they read: “.r is at least 14 and at most 25”. ,
The inequality sign “g” can also be concatenated with “<“, in either order. So, for example,
14 § :r < 25 means that ar is at least 14 and less than 25. D



CHAPTER l. NUMBERS SYSTEMS‘ 16
1.3.3 Floating-point notation
A digital device, such as a hand-held calculator or digital computer, is only able to store a finite
(and often fixed) number of digits to represent any numeral. On the other hand, all irrational
numbers and rational numbers with repeating blocks in their decimal expansion require an infinite
number of digits to represent them exactly in decimal form. Even a terminating decimal may have
more digits in its decimal expansion than the device is able to store. Thus most real numbers will
be represented in a digital computer only by an approximation, formed by cutting off the decimal
expansion after a certain number of digits. This produces an error in calculations known as a
truncation error.
Suppose that a calculator can store only 10 digits. Then, for example, the error in representing
5/12 as 0416666666 is 0000000000666“. ., or a little less than 0000000000667. This may not seem
very grave, but in a complex calculation errors may accumulate, giving a substantial error in the
solution.
It is hard to read and comprehend a number such as 0000000000667 written in this form. There are
more convenient ways of expressing very small or very large decimal numbers. The two main ones
are called scientific notation and floating-point notation. Most computers can store real numbers
in floating-point notation, so we shall describe this method in detail. In this system, a positive
number is expressed as the product of a power of 10 called the exponent, and a number 1: such
that 011 § :2 < 1 called the mantissa. A
Example 1.33 In floating point notation, the number 0000000000667 is expressed as 0.667>< 10‘9,
where -9 is the exponent and 0,667 is the mantissa; the number 146.75 is expressed as 0.14675 X 103,
where 3 is the exponent and 0.14675 is the mantissa. El
Scientific notation is based on a similar principle, and differs only in that the mantissa is a
number 2 such that 1 § 1 < 10.
Example 1.34 In scientific notation, the number 0000000000667 is expressed as 6.67 >< 104°,
where -10 is the exponent and 6.67 is the mantissa; the number 146.75 is expressed as 1.4675 >< 102,
where 2 is the exponent and 1.4675 is the mantissa. El
1.4 Exercises 1
1. (a) Write the decimal numbers 4037 and 40371 in expanded form as multiples of powers of
10.
(b) Suppose that a is the 4-digit decimal number a3a;ala0. Write a in expanded form as
multiples of powers of 10. Suppose that 1: is the 5-digit decimal number G3fl2L11£lg1- Can
you express :1: in terms of o?
2. (a) A number is expressed in base 5 as (234)5. What is it as a decimal number‘?
(b) Suppose you multiply (234)5 by 5. What would the answer be in base 5?
3. Express the following binary numbers as decimal numbers:
(11011)2; (11o0110),; (111111111),
Can you think of a quick way of doing the last one?
4. Perform the binary additions
(10111); +(1l1010)2é (1101); + (1011); —I- (1111);.
5‘ Perform the binary subtractions
(1001); - (111)g§ (110000); — (10111);
6. Perform the following binary multiplications 4
(1101); X (10l)g; (1101); >< (1101);.



