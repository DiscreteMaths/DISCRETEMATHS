%\documentclass[11pt, a4paper,dalthesis]{report}    % final
%\documentclass[11pt,a4paper,dalthesis]{report}
%\documentclass[11pt,a4paper,dalthesis]{book}

\documentclass[11pt,a4paper,titlepage,oneside,openany]{article}

\pagestyle{plain}
%\renewcommand{\baselinestretch}{1.7}

\usepackage{setspace}
%\singlespacing
\onehalfspacing
%\doublespacing
%\setstretch{1.1}

\usepackage{amsmath}
\usepackage{amssymb}
\usepackage{amsthm}
\usepackage{multicol}

\usepackage[margin=3cm]{geometry}
\usepackage{graphicx,psfrag}%\usepackage{hyperref}
\usepackage[small]{caption}
\usepackage{subfig}

\usepackage{algorithm}
\usepackage{algorithmic}
\newcommand{\theHalgorithm}{\arabic{algorithm}}

\usepackage{varioref} %NB: FIGURE LABELS MUST ALWAYS COME DIRECTLY AFTER CAPTION!!!
%\newcommand{\vref}{\ref}

\usepackage{index}
\makeindex
\newindex{sym}{adx}{and}{Symbol Index}
%\newcommand{\symindex}{\index[sym]}
%\newcommand{\symindex}[1]{\index[sym]{#1}\hfill}
\newcommand{\symindex}[1]{\index[sym]{#1}}

%\usepackage[breaklinks,dvips]{hyperref}%Always put after varioref, or you'll get nested section headings
%Make sure this is after index package too!
%\hypersetup{colorlinks=false,breaklinks=true}
%\hypersetup{colorlinks=false,breaklinks=true,pdfborder={0 0 0.15}}


%\usepackage{breakurl}

\graphicspath{{./images/}}

\usepackage[subfigure]{tocloft}%For table of contents
\setlength{\cftfignumwidth}{3em}

\input{longdiv}
\usepackage{wrapfig}


%\usepackage{index}
%\makeindex
%\usepackage{makeidx}

%\usepackage{lscape}
\usepackage{pdflscape}
\usepackage{multicol}

\usepackage[utf8]{inputenc}

%\usepackage{fullpage}

%Compulsory packages for the PhD in UL:
%\usepackage{UL Thesis}
\usepackage{natbib}

%\numberwithin{equation}{section}
\numberwithin{equation}{section}
\numberwithin{algorithm}{section}
\numberwithin{figure}{section}
\numberwithin{table}{section}
%\newcommand{\vec}[1]{\ensuremath{\math{#1}}}

%\linespread{1.6} %for double line spacing

\usepackage{afterpage}%fingers crossed

\newtheorem{thm}{Theorem}[section]
\newtheorem{defin}{Definition}[section]
\newtheorem{cor}[thm]{Corollary}
\newtheorem{lem}[thm]{Lemma}

%\newcommand{\dbar}{{\mkern+3mu\mathchar'26\mkern-12mu d}}
\newcommand{\dbar}{{\mkern+3mu\mathchar'26\mkern-12mud}}

\newcommand{\bbSigma}{{\mkern+8mu\mathsf{\Sigma}\mkern-9mu{\Sigma}}}
\newcommand{\thrfor}{{\Rightarrow}}

\newcommand{\mb}{\mathbb}
\newcommand{\bx}{\vec{x}}
\newcommand{\bxi}{\boldsymbol{\xi}}
\newcommand{\bdeta}{\boldsymbol{\eta}}
\newcommand{\bldeta}{\boldsymbol{\eta}}
\newcommand{\bgamma}{\boldsymbol{\gamma}}
\newcommand{\bTheta}{\boldsymbol{\Theta}}
\newcommand{\balpha}{\boldsymbol{\alpha}}
\newcommand{\bmu}{\boldsymbol{\mu}}
\newcommand{\bnu}{\boldsymbol{\nu}}
\newcommand{\bsigma}{\boldsymbol{\sigma}}
\newcommand{\bdiff}{\boldsymbol{\partial}}

\newcommand{\tomega}{\widetilde{\omega}}
\newcommand{\tbdeta}{\widetilde{\bdeta}}
\newcommand{\tbxi}{\widetilde{\bxi}}



\newcommand{\wv}{\vec{w}}

\newcommand{\ie}{i.e. }
\newcommand{\eg}{e.g. }
\newcommand{\etc}{etc}

\newcommand{\viceversa}{vice versa}
\newcommand{\FT}{\mathcal{F}}
\newcommand{\IFT}{\mathcal{F}^{-1}}
%\renewcommand{\vec}[1]{\boldsymbol{#1}}
\renewcommand{\vec}[1]{\mathbf{#1}}
\newcommand{\anged}[1]{\langle #1 \rangle}
\newcommand{\grv}[1]{\grave{#1}}
\newcommand{\asinh}{\sinh^{-1}}

\newcommand{\sgn}{\text{sgn}}
\newcommand{\morm}[1]{|\det #1 |}

\newcommand{\galpha}{\grv{\alpha}}
\newcommand{\gbeta}{\grv{\beta}}
%\newcommand{\rnlessO}{\mb{R}^n \setminus \vec{0}}
\usepackage{listings}

\interfootnotelinepenalty=10000

\newcommand{\sectionline}{%
  \nointerlineskip \vspace{\baselineskip}%
  \hspace{\fill}\rule{0.5\linewidth}{.7pt}\hspace{\fill}%
  \par\nointerlineskip \vspace{\baselineskip}
}

\renewcommand{\labelenumii}{\roman{enumii})}

\begin{document}

\begin{center}
  \textbf{Tutorial Sheet for Session 1}
\end{center}
\section*{Part A: Number Systems - Binary Numbers}
\begin{enumerate}
\item Express the following decimal numbers as binary numbers.
  \begin{multicols}{4}
    \begin{itemize}
    \item[i)] $(73)_{10}$
    \item[ii)] $(15)_{10}$
    \item[iii)] $(22)_{10}$
    \end{itemize}
  \end{multicols}
  
  All three answers are among the following options.
  \begin{multicols}{4}
    \begin{itemize}
    \item[a)] $(10110)_{2}$ %22
    \item[b)] $(1111)_{2}$ %15
    \item[c)] $(1001001)_{2}$ %73
    \item[d)] $(1000010)_{2}$ %64
    \end{itemize}
  \end{multicols}
  
\item Express the following binary numbers as decimal numbers.
  \begin{multicols}{4}
    \begin{itemize}
    \item[a)] $(101010)_{2}$
    \item[b)] $(10101)_{2}$
    \item[c)] $(111010)_{2}$
    \item[d)] $(11010)_{2}$
    \end{itemize}
  \end{multicols}
  \item Express the following binary numbers as decimal numbers.
  \begin{multicols}{4}
    \begin{itemize}
    \item[a)] $(110.10101)_{2}$
    \item[b)] $(101.0111)_{2}$
    \item[c)] $(111.01)_{2}$
    \item[d)] $(110.1101)_{2}$
    \end{itemize}
  \end{multicols}
\item Express the following decimal numbers as binary numbers.
  \begin{multicols}{4}
    \begin{itemize}
    \item[a)] $(27.4375)_{10}$  %
    \item[b)] $(5.625)_{10}$
    \item[c)] $(13.125)_{10}$
    \item[d)] $(11.1875)_{10}$
    \end{itemize}
  \end{multicols}
\end{enumerate}

\section*{Part B: Number Systems - Binary Arithmetic}
%http://www.csgnetwork.com/binaddsubcalc.html
(See section 1.1.3 of the text)
\begin{enumerate}
\item Perform the following binary additions.
  \begin{multicols}{2}
    \begin{itemize}
    \item[a)] $(110101)_{2}$ + $(1010111)_{2}$
    \item[b)] $(1010101)_{2}$ + $(101010)_{2}$
    \item[c)] $(11001010)_{2}$ + $(10110101)_{2}$
    \item[d)] $(1011001)_{2}$ + $(111010)_{2}$
    \end{itemize}
  \end{multicols}

\item Perform the following binary subtractions.
  \begin{multicols}{2}
    \begin{itemize}
    \item[a)] $(110101)_{2}$ - $(1010111)_{2}$
    \item[b)] $(1010101)_{2}$ - $(101010)_{2}$
    \item[c)] $(11001010)_{2}$ - $(10110101)_{2}$
    \item[d)] $(1011001)_{2}$ - $(111010)_{2}$
    \end{itemize}
  \end{multicols}


\item Perform the following binary multiplications.
\begin{multicols}{2}
\begin{itemize}
\item[a)] $(1001)_{2}\times( 1000)_{2}$  % 9 by 8
\item[b)] $(101)_{2}\times(1101)_{2}$ % 5 by 11
\item[c)] $(111)_{2}\times(1111)_{2}$ % 7 by 15
\item[d)] $(10000)_{2}\times(11001)_{2}$    %16 by 25
\end{itemize}
 \end{multicols}

\newpage

\item Perform the following binary multiplications.
%\begin{multicols}{2}
%\begin{itemize}
%\item[a)] $(1001000)_{2} \div ( 1000)_{2}$
%\item[b)] $(101101)_{2} \div (1001)_{2}$
%\item[c)] $(1001011000)_{2} \div (101000)_{2}$
%\item[d)] $(1100000)_{2} \div (10000)_{2}$
%\end{itemize}
%\end{multicols}

\begin{enumerate}
\item Which of the following binary numbers is the result of this binary division: $(10)_{2} \times ( 1101)_{2}$. % % (2) /  (13)
\begin{multicols}{2}
\begin{itemize}
\item[a)] $(11010)_{2}$ %26
\item[b)] $(11100)_{2}$ %28
\item[c)] $(10101)_{2}$ %21
\item[d)] $(11011)_2$ %27
\end{itemize}
\end{multicols}
\item Which of the following binary numbers is the result of this binary division: $(101010)_{2} \times( 111 )_{2}$. % (4) /  (6)
\begin{multicols}{2}
\begin{itemize}
\item[a)] $(11000)_{2}$ %24
\item[b)] $(11001)_{2}$ %25
\item[c)] $(10101)_{2}$ %21
\item[d)] $(11011)_2$ %27
\end{itemize}
\end{multicols}
\item Which of the following binary numbers is the result of this binary division: $(1001110)_{2}\times ( 1101 )_{2}$. % (9) /  (3)
\begin{multicols}{2}
\begin{itemize}
\item[a)] $(11000)_{2}$ %24
\item[b)] $(11001)_{2}$ %25
\item[c)] $(10101)_{2}$ %21
\item[d)] $(11011)_2$ %27
\end{itemize}
\end{multicols}
\end{enumerate}

%----------------------------------------------------------------%

\item Perform the following binary divisions.
%\begin{multicols}{2}
%\begin{itemize}
%\item[a)] $(1001000)_{2} \div ( 1000)_{2}$
%\item[b)] $(101101)_{2} \div (1001)_{2}$
%\item[c)] $(1001011000)_{2} \div (101000)_{2}$
%\item[d)] $(1100000)_{2} \div (10000)_{2}$
%\end{itemize}
%\end{multicols}

\begin{enumerate}
\item Which of the following binary numbers is the result of this binary division: $(111001)_{2} \div ( 10011)_{2}$. % (57) /  (19)
\begin{multicols}{2}
\begin{itemize}
\item[a)] $(10)_2$ %2
\item[b)] $(11)_{2}$ %3
\item[c)] $(100)_{2}$ %4
\item[d)] $(101)_{2}$ %5
\end{itemize}
\end{multicols}
\item Which of the following binary numbers is the result of this binary division: $(101010)_{2} \div ( 111 )_{2}$. % (42) /  (7)
\begin{multicols}{2}
\begin{itemize}
\item[a)] $(11)_2$ %3
\item[b)] $(100)_{2}$ %4
\item[c)] $(101)_{2}$ %5
\item[d)] $(110)_{2}$ %6
\end{itemize}
\end{multicols}
\item Which of the following binary numbers is the result of this binary division: $(1001110)_{2} \div ( 1101 )_{2}$. % (78) /  (13)
\begin{multicols}{2}
\begin{itemize}

\item[a)] $(100)_{2}$ %4
\item[b)] $(110)_{2}$ %6
\item[c)] $(111)_{2}$ %7
\item[d)] $(1001)_2$ %9
\end{itemize}
\end{multicols}
\end{enumerate}


\end{enumerate}


\section*{Part C: Number Bases - Hexadecimal}
\begin{enumerate}

\item Answer the following questions about the hexadecimal number systems
\begin{itemize}
\item[a)] How many characters are used in the hexadecimal system?
\item[b)] What is highest hexadecimal number that can be written with two characters? \item[c)] What is the equivalent number in decimal form?
\item[d)] What is the next highest hexadecimal number?
\end{itemize}


\item Which of the following are not valid hexadecimal numbers?
  \begin{multicols}{4}
    \begin{itemize}
    \item[a)] $73$
    \item[b)] $A5G$
    \item[c)] $11011$
    \item[d)] $EEF	$
    \end{itemize}
  \end{multicols}

\item Express the following decimal numbers as a hexadecimal number.
  \begin{multicols}{4}
    \begin{itemize}
    \item[a)] $(73)_{10}$
    \item[b)] $(15)_{10}$
    \item[c)] $(22)_{10}$
    \item[d)] $(121)_{10}$
    \end{itemize}
  \end{multicols}


\item Compute the following hexadecimal calculations.
  \begin{multicols}{4}
    \begin{itemize}
    \item[a)] $5D2+A30$
    \item[b)] $702+ABA$
    \item[c)] $101+111$
    \item[d)] $210+2A1$
    \end{itemize}
  \end{multicols}
\end{enumerate}








% \newpage
\section*{Part D: Natural, Rational and Real Numbers}
\begin{itemize}
\item $\mb{N}$ : natural numbers (or positive integers) $\{1,2,3,\ldots\}$
\item $\mb{Z}$ : integers $\{-3,-2,-1,0,1,2,3,\ldots\}$
\begin{itemize}
\item (The letter $\mb{Z}$ comes from the word \emph{Zahlen} which means ``numbers" in German.)
\end{itemize}
\item $\mb{Q}$ : rational numbers
\item $\mb{R}$ : real numbers
\item $\mb{N} \subseteq \mb{Z } \subseteq \mb{Q} \subseteq \mb{R}$
\begin{itemize}
\item (All natural numbers are integers. All integers are rational numbers. All rational numbers are real numbers.)
\end{itemize}
\end{itemize}
\newpage


\begin{enumerate}
\item Answer the following questions about the hexadecimal number systems
  \begin{multicols}{4}
    \begin{itemize}
    \item[a)] $18$
    \item[b)] $8.2347\ldots$
    \item[c)] $\pi$
    \item[d)] $1.33333\ldots$
    \item[e)] $17/4$
    \item[f)] $4.25$
    \item[g)] $\sqrt{\pi}$
    \item[h)] $\sqrt{25}$
    %\item[i)]
    %\item[j)] $(15)_{10}$
    %\item[k)] $\pi$
    %\item[l)] $11.132$
    \end{itemize}
  \end{multicols}
\end{enumerate}


\end{document}

% \item The Fibonacci sequence $f_n$ is defined recursively by the rule
%   \begin{equation*}
%     \begin{cases}
%       f_0&=0\\
%       f_1&=1\\
%       f_n&=a_{n-1}+a_{n-2}
%     \end{cases}
%   \end{equation*}

%   \begin{enumerate}
%   \item
%     Write a program to evaluate the Fibonacci sequence and hence evaluate $f_{50}$.
%   \item
%     Let the sequence $g_n$ be defined as the ratio
%     \begin{equation*}
%       g_n = \frac{f_{n+1}}{f_n}
%     \end{equation*}
%     Write a program to evaluate the first $50$ terms of the sequence $g_n$.
%   \item
%     Assumming that the sequence $g_n$ has a limit $\phi$, find this limit.
%   \end{enumerate}



% \pagebreak
%   \begin{center}
%     \textbf{Assignment 1}
%     Due Tuesday Week 4 (Sept $28^th$)
%   \end{center}

%   Write an Octave function which computes square roots using Heron's method to as high an accuraccy as possible. The function should be called ``sqr\_root'' and must take the following form
% \lstset{language=Octave}
% \begin{lstlisting}[title=sqr\_root.m,frame=single]
% function X=sqr_root(D)
%     ....
% endfunction
% \end{lstlisting}

% In particular, the function must be able to compute the square roots of your UL ID number, (i.e numbers of the form 10234567), as well as 0.01,0.1,10,100 and 1000 times this number; so test the function on these and many other numbers before submitting.

% Note that the stopping condition shown in the lectures will not suffice as a general stopping condition\footnote{If Octave hangs during calculation, use ``Ctrl-C'' to try to cancel the operation and return to the prompt.}. You must develop and program an alternative strategy for detecting when Heron's method has converged.

% Note also that the square of the number $X$ returned by your function must be as close to D as is possible for $X^2$ to be; that is, $X$ must be as close as it possibly can be to the true value of $\sqrt{D}$.

% Save your solutions source code in a file called ``sqr\_root.m''. To submit the assigment, you must email this solution from your UL student email to the address \emph{niall.ryan@ul.ie}. The email subject line must be ``MA4402 ASSIGNMENT SOLUTION'' and the file ``sqr\_root.m'' must be an attachment to this email.

% The email must be sent \textbf{from your UL student address}. \emph{Emails from other addresses will not be accepted}.





\end{document}

%%% Local Variables:
%%% mode: latex
%%% TeX-master: t
%%% End:
