\documentclass{article}
\usepackage{amsmath}
\usepackage{amssymb}
\usepackage{graphicx}
\begin{document}







%------------------------------------------%

\section{Hexadecimal Numbers}
\Large{
Hex Characters: 0,1,2,3,4,5,6,7,8,9,A,B,C,D,E,F}

%------------------------------------------%

\section{Types of Numbers}
\Large{
\begin{itemize}
\item Natural Numbers (1,2,3)
\item Integers (..-3,-2-1,0,1,2,3..)
\item Rational Numbers (e.g 4/7, 12/3)
\item Real Numbers (3.14151)
\end{itemize}
}
\section*{Binary and Hex}
\begin{itemize}
\item[1A.1] Coverting from Binomial to Decimal
\item[1A.2] Converting to Decimal
\item[1A.3] Priority of Operation
\item[1A.4] 
\end{itemize}
\newpage
%---------------------------------------%
\section*{Numbers}
\begin{itemize}
\item[1B.1] Real Numbers
\item[1B.2] Rational Numbers
\item[1B.3] Floating Point Aritmetic
\item[1B.4] 
\end{itemize}
%----------------------------------------------------%
\section*{Binary and Hex}
\begin{itemize}
\item[1A.1] Coverting from Binomial to Decimal
\item[1A.2] Converting to Decimal
\item[1A.3] Priority of Operation
\item[1A.4] 
\end{itemize}

\section*{Numbers}
\begin{itemize}
\item[1B.1] Real Numbers
\item[1B.2] Rational Numbers
\item[1B.3] Floating Point Aritmetic
\item[1B.4] 
\end{itemize}
\section*{Binary and Hex}
\begin{itemize}
\item[1A.1] Coverting from Binomial to Decimal
\item[1A.2] Converting to Decimal
\item[1A.3] Priority of Operation
\item[1A.4] 
\end{itemize}
%---------------------------------------%
\section*{Numbers}
\begin{itemize}
\item[1B.1] Real Numbers
\item[1B.2] Rational Numbers
\item[1B.3] Floating Point Aritmetic
\item[1B.4] 
\end{itemize}
\section*{Binary and Hex}
\begin{itemize}
\item[1A.1] Coverting from Binomial to Decimal
\item[1A.2] Converting to Decimal
\item[1A.3] Priority of Operation
\item[1A.4] 
\end{itemize}
%----------------------------------------%
\newpage
%---------------------------------------%
\subsection*{Adding Binary Numbers}


%--------------------------------------%

%--------------------------------------------------------%
\subsubsection*{Complements}


%-----------------------------------
\end{itemize}


HibColl Number systems - Exercises


%-------------%-------------%-------------%----------------
Question 2
A number is expressed in base 5 as $(234)_5$. What is it as decimal number?
Suppose you multiply $(234)_5$ by 5. what would be the answer in base 5.

%-------------%-------------%-------------%----------------
Question 3

Can you think of a quick way of doing the last one?

%-------------%-------------%-------------%----------------
Question 4


Perform the binary additions
$(10111)_2 -(111010)_2$

$(1101)_2 + (1011)_2 + (1111)_2$

%-------------%-------------%-------------%----------------
Question 5

Perform the binary subtractions
$(1001)_2 -(111)_2$
$(110000)_2 -(10111)_2$

%-------------%-------------%-------------%----------------
Question 6

Perform the binary multiplications
$(1101)_2 \times (101)_2$
$(1101)_2 \times (1101)_2$

%-------------%-------------%-------------%----------------
Question 10

Calculate $(BBB)_{16} + (A56)_{16}$
$(BBB)_{16} - (A56)_{16}$ working in hexadecimal


%-------------%-------------%-------------%----------------
Question 11

Write the hex number $(EC4)_{16}$ in binary.
Write the binary number $(11110110101|)_2$ in hex.
%-------------%-------------%-------------%----------------
Question 12

Express the decimal number 753 in binary , base 5 and hexadecimal.

%-------------%-------------%-------------%----------------
Question 13

Express 42900 as a product of its prime factors, using index notation for repeated factors.

%-------------%-------------%-------------%----------------
Question 14

Expressing the recuring decimals 0.126126126.... and 0.754545454545 as fractions in their lowest terms.

%-------------%-------------%-------------%----------------
Question 15
Given that $\pi$ is an irrational number, can you say whether $\frac{\pi}{2}$ is rational or irrational.
or is it impossible to tell?


%-------------%-------------%-------------%----------------
Question 17
(a) 5/7 lies between 0.714 and 0.715
(b) $\sqrt(2)$ is at least 1.41
(c) $\sqrt(3)$  9s at lrast 1.732 and at most 1.7322

%-------------%-------------%-------------%----------------
Question 18 

Write down the numbers 0.0000526 in floating point from
429000000

How is the number 1 expressed in floating point form

%-------------%-------------%-------------%----------------
Question 19
Deduce that eveyry composite integer $n$ has a prime factor such that $p \leq \sqrt{n}$
Decide whether 899 is a prime

%-------------%-------------%-------------%----------------
Question 20

What would be the maximum numbber of digits that a decimal fraction with denominator 13 
could have in a recurring block in theory?

Can you predict which other fractions with denominator 13 will have the same digits as 1/13 in their recurring block?




\end{document}

