\documentclass[a4paper,12pt]{article}
%%%%%%%%%%%%%%%%%%%%%%%%%%%%%%%%%%%%%%%%%%%%%%%%%%%%%%%%%%%%%%%%%%%%%%%%%%%%%%%%%%%%%%%%%%%%%%%%%%%%%%%%%%%%%%%%%%%%%%%%%%%%%%%%%%%%%%%%%%%%%%%%%%%%%%%%%%%%%%%%%%%%%%%%%%%%%%%%%%%%%%%%%%%%%%%%%%%%%%%%%%%%%%%%%%%%%%%%%%%%%%%%%%%%%%%%%%%%%%%%%%%%%%%%%%%%
\usepackage{eurosym}
\usepackage{vmargin}
\usepackage{amsmath}
\usepackage{graphics}
\usepackage{epsfig}
\usepackage{subfigure}
\usepackage{fancyhdr}
%\usepackage{listings}
\usepackage{framed}
\usepackage{graphicx}
\usepackage{amsmath}
\usepackage{chngpage}
%\usepackage{bigints}


\setcounter{MaxMatrixCols}{10}
%TCIDATA{OutputFilter=LATEX.DLL}
%TCIDATA{Version=5.00.0.2570}
%TCIDATA{<META NAME="SaveForMode" CONTENT="1">}
%TCIDATA{LastRevised=Wednesday, February 23, 2011 13:24:34}
%TCIDATA{<META NAME="GraphicsSave" CONTENT="32">}
%TCIDATA{Language=American English}

%\pagestyle{fancy}
%\setmarginsrb{20mm}{0mm}{20mm}{25mm}{12mm}{11mm}{0mm}{11mm}
%\lhead{MA4413} \rhead{Mr. Kevin O'Brien}
%\chead{Statistics For Computing}
%\input{tcilatex}

\begin{document}

% http://www.mathsisfun.com/combinatorics/combinations-permutations.html

\section*{Combinations and Permutations }
\subsection*{The factorial function}
The factorial function (symbol: !) just means to multiply a series of descending natural numbers. Examples:

\begin{itemize}
\item $4! = 4 \times 3 \times 2 \times 1 = 24$
\item $7! = 7 \times 6 \times 5 \times 4 \times 3 \times 2 \times 1 = 5,040$
\item $1! = 1$
\item $0! = 1 $
\end{itemize}
Importantly 
\[n! = n \times (n-1)!  = n \times (n-1) \times (n-2)! \]
For Example
\[6! = 6 \times 5!  = 6 \times 5 \times 4! \]
\section*{Combinations}
In mathematical terms, a combination is an subset of items from a larger set such that the order of the items does not matter.

\section*{Permutations}
There are two types of permutation:
\begin{enumerate}
\item Repetition is Allowed: such as the lock above. It could be "333".
\item No Repetition: for example the first three people in a running race. You can't be first and second.
\end{enumerate}

\subsection*{Summary}
\begin{itemize}
\item If the order doesn't matter, it is a Combination.
\item If the order does matter it is a Permutation.
\end{itemize}





\end{document}