\documentclass{beamer}

\usepackage{amsmath}
\usepackage{graphicx}
\usepackage{amssymb}

\begin{document}

%------------------------------------------------------ %
\begin{frame}
\frametitle{Permutations}
\large
\begin{itemize}
\item In how many permutations are there of counting a subset of k elements, when there are $n$ elements in total.

\item The number of permutations of a set of n elements is denoted n! (pronounced n factorial.)
\end{itemize}
\end{frame}
%------------------------------------------------------ %
\begin{frame}
\frametitle{Permutation Formula}

A formula for the number of possible permutations of k objects from a set of n. This is usually written $^nP_k$ .

\bigskip
\textbf{Formula:}	
\[ ^nP_k = \frac{n!}{(n-k)!} =  n.(n-1).(n-2).\ldots(n-k+1) \]

\end{frame}
%------------------------------------------------------ %
\begin{frame}
\frametitle{Permutation Formula}

\textbf{Example:}\\	
How many ways can 4 students from a group of 15 be lined up for a photograph?\\
\bigskip
%--------------- %
\textbf{Answer:	}\\
There are $^{15}P_4$ possible permutations of 4 students from a group of 15.
\[ ^{15}P_4 = \frac{15!}{11!} = 15\times 14\times 13\times 12 = 32760 \]
There are 32760 different lineups.
\end{frame}
%------------------------------------------------------ %
\end{document}
