\documentclass[12pt]{article}

\usepackage{amsmath}
\usepackage{amssymb}

%opening
\title{Mathematics for Computing}
%\author{Hibernia College}

\begin{document}

\begin{center}
\huge{Mathematics for Computing}\\
\LARGE{Trees}
\end{center}

\section{Trees}

Storing a list of records

Subtrees of a binary tree (8:3:1)


Example 8.2 Deleting a vertex from a graph

8.1.2 The number of edges in a tree.
Theorem 8.3 Let T be a tree with n vertices. Then T has n-1 edges.
%-----------------------
%8.1.3 Spanning Trees 
\subsection*{Spanning Trees}
Spanning Subgraph of G
Spanning Tree of G
8.2 Rooted Trees
%-------------------------------------------------------
%<page 37>
A balanced binary tree has $2^i$ nodes on all levels i apart from the highest level.

A tree in which one vertex has been singled out in this way

Let x and y be vertices of T. If the unique path from the root r
to x in T passes through y, then y is called an ancestor of x and x is a 
descendant of y.

If the vertices y and x are adjacent on the path from r to x, then
y is called the parent of x and x is called the child of y.

%-----------------------------
%< page 38 >
%Theorem 8.5

%8.2.1 Binary Trees
\subsection{Binary Trees}
A binary child is a rooted tree in which each internal node has exactly
two children, the left child and right child respectively.

%--------------------------------
< page 39 >
Balanced Binary Tree

Storing Data in a binary tree search(8:3:2)

%-----------------------------------
A tree is a connected graph that contains no cycles

A tree on n vertices has n-1 degrees





\end{document}