\documentclass[a4paper,12pt]{article}
%%%%%%%%%%%%%%%%%%%%%%%%%%%%%%%%%%%%%%%%%%%%%%%%%%%%%%%%%%%%%%%%%%%%%%%%%%%%%%%%%%%%%%%%%%%%%%%%%%%%%%%%%%%%%%%%%%%%%%%%%%%%%%%%%%%%%%%%%%%%%%%%%%%%%%%%%%%%%%%%%%%%%%%%%%%%%%%%%%%%%%%%%%%%%%%%%%%%%%%%%%%%%%%%%%%%%%%%%%%%%%%%%%%%%%%%%%%%%%%%%%%%%%%%%%%%
\usepackage{eurosym}
\usepackage{vmargin}
\usepackage{amsmath}
\usepackage{graphics}
\usepackage{epsfig}
\usepackage{subfigure}
\usepackage{fancyhdr}
%\usepackage{listings}
\usepackage{framed}
\usepackage{graphicx}
\usepackage{amsmath}
\usepackage{chngpage}
%\usepackage{bigints}


\setcounter{MaxMatrixCols}{10}
%TCIDATA{OutputFilter=LATEX.DLL}
%TCIDATA{Version=5.00.0.2570}
%TCIDATA{<META NAME="SaveForMode" CONTENT="1">}
%TCIDATA{LastRevised=Wednesday, February 23, 2011 13:24:34}
%TCIDATA{<META NAME="GraphicsSave" CONTENT="32">}
%TCIDATA{Language=American English}

%\pagestyle{fancy}
%\setmarginsrb{20mm}{0mm}{20mm}{25mm}{12mm}{11mm}{0mm}{11mm}
%\lhead{MA4413} \rhead{Mr. Kevin O'Brien}
%\chead{Statistics For Computing}
%\input{tcilatex}

\begin{document}

% http://www.mathsisfun.com/combinatorics/combinations-permutations.html


\section*{Graph Theory}
\subsection*{What is Graph Theory?}
\begin{itemize}
\item Graph theory is the study of points and lines. In particular, it involves the ways in which sets of points, called vertices, can be connected by lines or arcs, called edges. Graphs in this context differ from the more familiar coordinate plots that portray mathematical relations and functions.

\item Graphs are classified according to their complexity, the number of edges allowed between any two vertices, and whether or not directions (for example, up or down) are assigned to edges. Various sets of rules result in specific properties that can be stated as theorems.

\item Graph theory has proven useful in the design of integrated circuits ( IC s) for computers and other electronic devices. These components, more often called chip s, contain complex, layered microcircuits that can be represented as sets of points interconnected by lines or arcs. 

\item Using graph theory, engineers develop chips with maximum component density and minimum total interconnecting conductor length. This is important for optimizing processing speed and electrical efficiency.
\end{itemize}

\end{document}