\documentclass[12pt]{article}

\usepackage{amsmath}
\usepackage{amssymb}

%opening
\title{Mathematics for Computing}
%\author{Hibernia College}

\begin{document}

\begin{center}
\huge{Mathematics for Computing}\\
\LARGE{Graph Theory}
\end{center}

DRAFT
%-------------------------------------------------------- %


Graph theory is the study of points and lines. In particular, it involves the ways in which sets of points, called \textit{\textbf{vertices}}, can be connected by lines or arcs, called \textit{\textbf{edges}}.

Graphs are classified according to their complexity, the number of edges allowed between any two vertices, and whether or not directions (for example, up or down) are assigned to edges. 

%Various sets of rules result in specific properties that can be stated as theorems.

%------------------------------------------------------- %
\section*{Adjacency}
\begin{itemize}

\item[(a)] An \textit{\textbf{adjacency list}} representation of a graph is a collection of unordered lists, one for each vertex in the graph. Each list describes the set of neighbors of its vertex.

\item[(b)] An \textit{\textbf{adjacency matrix}} is a means of representing which vertices (or nodes) of a graph are adjacent to which other vertices. Another matrix representation for a graph is the incidence matrix.

\item[(c)]

\end{itemize}
%-------------------------------------------------------- %

Graph theory is the study of graphs, which are mathematical structures used to model pairwise relations between objects. A "graph" in this context is made up of "vertices" or "nodes" and lines called edges that connect them. A graph may be undirected, meaning that there is no distinction between the two vertices associated with each edge, or its edges may be directed from one vertex to another

%-------------------------------------------------------- %
\newpage

\section{Simple Connected Graphs}
Given the following definitions for simple, connected graphs:
\begin{itemize}
\item $K_n$ is a graph on $n$ vertices where each pair of vertices is connected by an edge;
\item $C_n$ is the graph with vertices $v_1, v_2, v_3, \dots, v_n$ and edges $\{v_1,v_2\}, \{v_2,v_3\}, \dots\{v_n, v_1\}$;
\item $W_n$ is the graph obtained from $C_n$ by adding an extra vertex,$v_{n+1}$, and edges
from this to each of the original vertices in $C_n$.
\end{itemize}
(a) Draw $K_4$, $C_4$, and $W_4$. 

%-------------------------------------------------------- %

% Section 5 Graph Theory

Let G be a simple graph with vertex set $V(G) = \{v1, v2, v3, v4, v5\}$ and adjacency lists as follows:
\begin{verbatim}
v1 : v2 v3 v4
v2 : v1 v3 v4 v5
v3 : v1 v2 v4
v4 : v1 v2 v3.
v5 : v2
\end{verbatim}
\begin{itemize}
\item[5.a] List the degree sequence of G. Draw the graph of G.
\item[5.b] Find two distinct paths of length 3, starting at v3 and ending at v4. Find a 4 cycle in G.
\item[5.c] Let G be a graph and let v be a vertex of G. Say what is meant by the degree of v.
State, without proving, a result connecting the degrees of the vertices of a graph G with the number of its edges.

\item[5.d]
Degree sequence 4,3,2,2,2
Degree sequence 4,3,3,2,2

\item[5.e]
K8 has degree sequence 7,7,7,7,7,7,7,7 so every vertex has degree 7.
\end{itemize}

\end{document}

A graph consists of a finite set of vertices $V$ and a finite set of Edges $E$.
%---------------------------------------------%
The degree of a vertex $v_1$ denoted by $deg(v)$ is the number of edges incident with $v$.
A graph in which every vertex has the same degree r is called \textbf{\emph{r-regular}}.
% page 73


%---------------------------------------------%
\newpage

A path is an alternating sequences of vertices and edges of the form
$v_1e_1v_2e_2 \ldots e_{k-1}v_k$

The length of a path is the number of edges in it.

\subsection{Isomorphism of a Graph}

%---------------------------------------------%
%5.3

\end{document}

%---------------------------------------------%

