\documentclass{article}
\usepackage{amsmath}
\usepackage{amssymb}
\usepackage{graphicx}
\begin{document}
\section*{DM07 Sequences and Series}

%-----------------------------------------------------%
\section*{Session 07: Sequences and Series}
\begin{itemize}
\item[7A.1] Sequences
\item[7A.2] Induction
\item[7A.3] Series and the Sigma Notation
\end{itemize}

\subsection*{Recurrence Relations(7.1.1)}

\begin{itemize}
\item $u_1 = 2$
\item $u_2 = u_1 + 3 = 2 +3 = 5$
\item $u_3 = u_2 + 3 = 5+ 3 = 8$
\end{itemize}
% Airthmetic Progression


\subsection*{Proof by Induction(7.2.2)}
\begin{itemize}
\item[\textbf{Step 1}] Base case
\item[\textbf{Step 2}] Induction hypothesis
\item[\textbf{Step 3}] Induction step
\end{itemize}



\subsection*{Series and Sigma Notation(7.2.3)}
Mathematical notation uses a symbol that compactly represents summation of many similar terms: the summation symbol, $\sum$, an enlarged form of the upright capital Greek letter \textit{Sigma}. This is defined as:
\[\sum_{i=m}^n a_i = a_m + a_{m+1} + a_{m+2} +\cdots+ a_{n-1} + a_n. \]
Where, i represents the index of summation; ai is an indexed variable representing each successive term in the series; m is the lower bound of summation, and n is the upper bound of summation. The "i = m" under the summation symbol means that the index i starts out equal to m. The index, i, is incremented by 1 for each successive term, stopping when i = n.
Here is an example showing the summation of exponential terms (all terms to the power of 2):
\[\sum_{i=3}^6 i^2 = 3^2+4^2+5^2+6^2 = 86.\]
Informal writing sometimes omits the definition of the index and bounds of summation when these are clear from context, as in:
\[\sum a_i^2 = \sum_{i=1}^n a_i^2.\]
\end{document}

