

\documentclass{beamer}

\usepackage[english]{babel}
% or whatever

\usepackage[utf8]{inputenc}
% or whatever

\usepackage{times}
\usepackage[T1]{fontenc}
% Or whatever. Note that the encoding and the font should match. If T1
\begin{document}

%-------------------------------------%
\begin{frame}
\frametitle{Partitioning of Summations}
\large

For some integers $m$ and $n$, with $m<n$.

\[ \sum^{i=n}_{i=1} u_{i} = \sum^{i=m}_{i=1} u_{i} + \sum^{i=n}_{i=m+1} u_{i}\]

Suppose $n=100$ and $m=50$

\[ \sum^{i=100}_{i=1} u_{i} = \sum^{i=50}_{i=1} u_{i} + \sum^{i=100}_{i=51} u_{i}\]

\end{frame}
%-------------------------------------%
\begin{frame}
\frametitle{Partitioning of Summations}
\large
\textbf{Example}
Evaluate the following expression:
\[ \sum^{i=100}_{i=51} (i+1) \]

\begin{description}
\item[Step 1] Evaluate this expression using the identities (notice the lower bound)
\[ \sum^{i=100}_{i=1} (i+1) \]
\item[Step 2] From the outcome of step 1, subtract the following
\[ \sum^{i=50}_{i=1} (i+1) \]
\end{description}

\end{frame}
%-------------------------------------%
\begin{frame}
\frametitle{Partitioning of Summations}
\large
\textbf{Step 1} Evaluate the following expression using the identities. In this step $n=100$
\[ \sum^{i=100}_{i=1} (i+1)  = \sum^{i=100}_{i=1} i  +  \sum^{i=100}_{i=1} 1  \]

\begin{itemize}
\item[(i)] First term \[\sum^{i=100}_{i=1} i  = \frac{100\times(100+1)}{2}  = 5050\]

\item[(ii)] Second term \[ \sum^{i=100}_{i=1} i  =  100\]
\end{itemize}

\[ \sum^{i=100}_{i=1} (i+1)  = 5050 + 100 = 5150 \]

\end{frame}
%-------------------------------------%
\begin{frame}
\frametitle{Partitioning of Summations}
\large
\textbf{Step 2} Evaluate the following expression using the identities. In this step $n=50$
\[ \sum^{i=50}_{i=1} (i+1)  = \sum^{i=50}_{i=1} i  +  \sum^{i=50}_{i=1} 1  \]

\begin{itemize}
\item[(i)] First term \[\sum^{i=50}_{i=1} i  = \frac{50\times(50+1)}{2}  = 1275\]

\item[(ii)] Second term \[ \sum^{i=50}_{i=1} i  =  50\]
\end{itemize}

\[ \sum^{i=50}_{i=1} (i+1)  = 1275 + 50 = 1325 \]

\end{frame}

%-------------------------------------%
\begin{frame}
\frametitle{Partitioning of Summations}
\large

\[ \sum^{i=100}_{i=51} (i+1) = \sum^{i=100}_{i=1} (i+1)  - \sum^{i=50}_{i=1}(i+1)   \]

\[ \sum^{i=100}_{i=51} (i+1)  = 5150 - 1325 =\boldsymbol{3825} \]
\end{frame}

%--------------------------------------------------------------%
\end{document}