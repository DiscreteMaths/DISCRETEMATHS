% small.tex
\documentclass{beamer}
\usepackage{amssymb}
\usepackage{graphicx}
\usepackage{subfigure}
\usepackage{newlfont}
\usepackage{amsmath,amsthm,amsfonts}
%\usepackage{beamerthemesplit}
\usepackage{pgf,pgfarrows,pgfnodes,pgfautomata,pgfheaps,pgfshade}
\usepackage{mathptmx}  % Font Family
\usepackage{helvet}   % Font Family
\usepackage{color}

\mode<presentation> {
 \usetheme{Default} % was
 \useinnertheme{rounded}
 \useoutertheme{infolines}
 \usefonttheme{serif}
 %\usecolortheme{wolverine}
% \usecolortheme{rose}
\usefonttheme{structurebold}
}

\setbeamercovered{dynamic}

\title[Discrete Mathemtics]{Mathemetics for Computing \\ {\normalsize Induction Day}}
\author[Kevin O'Brien]{Kevin O'Brien \\ {\scriptsize Kevin.obrien@ul.ie}}
\date{3rd November 2012}
\institute[Hibernia College]{International Program, \\ University \textit{of} London}

\renewcommand{\arraystretch}{1.5}

\begin{document}



\begin{figure}
\centering
\includegraphics[width=0.7\linewidth]{./Image3}
\label{fig:Image3}
\end{figure}
\begin{center}
\Large{Mathematics for Computing: Induction Day}
\end{center}



\titlepage



\Large{\alert{Important Question } - Do you like Maths?}



%--------------------------------------------------- %

\subsection{Mathematical Notation}
\Large{\begin{itemize}
\item Long form and short form notation
\item Variable names (examples : x and y)
\item Symbols (Logic and Set Theory)
\item Discrete Maths is quite graphical (Venn Diagrams)
\item Gates (AND, OR, XOR)
\item Different branches of mathematics has different conventions.
\end{itemize}}


%--------------------------------------------------- %
\frame{
\subsection{Overview of Module}
\begin{enumerate}
\item Number Systems
\item Set Theory
\item Logic
\item Functions
\item Introduction to Graph Theory
\item Digraphs and Relations
\item Sequences, Series and Induction
\item Trees
\item Probability
\item Matrices
\end{enumerate}
}



%-------------------------------------------------------%
\frame{
\subsection{Session 1: Number Systems}
\Large{


\end{document}





\subsection{Relational Operators}
\begin{itemize}
\item = Equal to
\item $\neq$  Not Equal to
\item < Less than
\item > greater than
\item $\[\]geq$ greater than or equal to
\item $\leq$ Not Equal to
\end{itemize}

%------------------------------------------%







%------------------------------------------%

\subsection{Floating Point Notation}
\begin{enumerate}
\item thing
\item thing
\end{enumerate}

%-------------------------------------%

\end{document}

%------------------------------------------%

\subsection{Frame Name}
\begin{enumerate}
\item
\item
\end{enumerate}

%------------------------------------------%

\subsection{Frame Name}
\begin{itemize}
\item
\item
\end{itemize}

%------------------------------------------%

\subsection{Frame Name}
\begin{itemize}
\item
\item
\end{itemize}

%------------------------------------------%


%------------------------------------------%

`
Floating Point Noation

Section 1
Section 1.2
Section 1.3
Section 14
%----------------------------------------------%

Hibernia College induction day

todays Class

Session 1
/Session 2

Number Systems

Decmal Number Systesm
Base 10

Binary Number Systems
Base 2
allowable characters are {0.1} only

Base 16 Hexadecimal

Use all of the decimal digits, in addition to 6 more A,B~,D,D,E,F

(where might you see this  - specifying colours RGB Numbers

For example FF in hexadecimal is is 255 in decimal 


Rational Numbers

Natural Numbers indies
Integers Z


%--------------------------------------------%

Computing a binary number

Useful

2^0 = 1
2^1 = 2
2^2 = 4
2^3 = 8
2^4 = 16
2^5 = 32
2^6 = 64
2^7 = 128

Firslly determine the highest power

Suppose the number we wish to convert is 58

Hwhat is the highest power of two that deivides


%-----------------------------------------%


\begin{itemize}
\end{itemize}



%-----------------------------------------%


supo





