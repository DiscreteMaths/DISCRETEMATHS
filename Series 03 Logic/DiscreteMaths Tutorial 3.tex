\documentclass[12pt]{article}

\usepackage{amsmath}
\usepackage{amssymb}

\begin{document}

\begin{center}
\huge{Mathematics for Computing}\\
\LARGE{Logic}
\end{center}

DRAFT
%---------------------------%

% Propositions
% Logical Operations
% Conditional Connectives
% Logical Proofs
% De Morgan's Law

%-----------------------------------%
\begin{frame}
\frametitle{Propositions}

\begin{itemize}
\item Contra-positives
\end{itemize}

\end{frame}
%-----------------------------------%
\begin{frame}
\frametitle{Logical Notation}

\begin{centering}
\begin{tabular}{|c|c|}
Logical & Negation \\ \hline
$p$	& $\neg p$ \\ \hline
$T$	& $F$      \\ \hline
$F$	& $T$      \\ \hline
\end{tabular}
\end{centering}

\end{frame}
%-----------------------------------%
\begin{frame}
\frametitle{Conditional Connectives}

\begin{itemize}
\item $p \rightarrow q$ is logically equivalent to $\neg(p \and \neg q)$.
\end{itemize}


\end{frame}
%-----------------------------------%


\section{Logic}
\begin{itemize}
\item The NOT operator $\neg$ (also known as the negation operator)
\end{itemize}
\subsection{Truth Tables}
{
\Large
\begin{center}
\begin{tabular}{|c|c||c|c||c|c|}
\hline $p$ & $q$ & $p \wedge q$ & $p\vee q$ & $\neg p$  & $\neg (p \wedge q)$ \\ 
\hline 0 & 0 & 0 & 0 & 1 & 1 \\ 
\hline 0 & 1 & 0 & 1 & 1 & 1 \\ 
\hline 1 & 0 & 0 & 1 & 0 & 1 \\ 
\hline 1 & 1 & 1 & 1 & 0 & 0 \\ 
\hline 
\end{tabular} 
\end{center}
}
%-------------------------%

\subsection*{Prepositional Logic}

\begin{itemize}
\item A Set is a collection of distinct and defined elements. 
\item Sets are represented by using French braces $\{ \}$ with commas to separate the elements in a Set. 
\end{itemize}
\end{document}

(2.2.2) Cardinality 
The number of distinct elements in a finite set is called its cardinality.


(2.2.3) Power Set

(2.3) Operations on Sets

(2.3.1) Complement of a set
(2.3.2) Binary Operations on Sets
-Union
-Intersection
-Set Difference
-Symmetric Difference 


A \otimes B

{2, 3, 4, 6, 7, 8}

Rules of Inclusion
{1,2,3,4,5,6}
{0,1,2,3,4,5,6,7,8,9}
{-3,0,3,6,9,12}
A \element S
A \in S
A \cup B
A \cap B


Example 1: 

If $U = {1, 2, 3, 4, 5, 6, 7, 8, 9, 10}$, \\
$A = {2, 4, 6, 8, 10}$, \\
$B = (1, 3, 6, 7, 8}$ and \\
$C = {3, 7}$, \\

find $A \cap B$, $A \cup C$, $B \cap A^{c}$, $B \cap C^{c}$

Solution: 

$U = {1, 2, 3, 4, 5, 6, 7, 8, 9, 10}$\\
$A = {2, 4, 6, 8, 10}$\\
$B = (1, 3, 6, 7, 8} $\\
$C = {3, 7}$\\

%A n B = {6, 8}
%A \cap C = {2, 3, 4, 6, 7, 8, 10}
%B n A^C = {1, 3, 7}
%B n C^C = {1, 6, 8}

%> A
%[1]  1  8  9 10
%> B
%[1] 4 7 9
%> C
%[1] 1 2 3 4 9
%> D
%[1] 2 5 6


%http://kpaprzycka.swps.edu.pl/xLogicSelfTaught/Logic02PropLogic.pdf


%------------------------%
\begin{itemize}
\item
\item
\item
\end{itemize}
%------------------------%
\subsection*{Exam Questions}


%------------------------%

\begin{itemize}

\item[(a)] 

\item[(b)]

\item[(c)]

\end{itemize}

\end{document}