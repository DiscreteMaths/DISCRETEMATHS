\documentclass[MASTER.tex]{subfiles}
\begin{document}

\section{Relations}
%Definition 1.4 

Let R be an equivalence relation defined on a set S and let x 2 S. 
Then the equivalence class of x is the subset of S containing all
elements of S which are related to x. 

We denote this by [x]. Thus
\[[x] = {y 2 S : yRx}\].

\subsection*{Theorem 1.3 }
Let R be an equivalence relation on a set S. Then:
\begin{itemize}
\item The set of distinct equivalence classes of R on S is a partition of S.
\item Two elements of S are related if and only if they belong to the same equivalence
class.
\end{itemize}
%======================================== %
\subsection*{Definition 1.6} We say that a relation R on a set S is anti-symmetric if for
all x, y 2 S such that xRy and yRx, we have x = y. 
Thus R is \textbf{anti-symmetric}
if and only if the relationship digraph of R has no directed cycles of length two.


%=================================== %
\newpage
\section{Digraphs and Relations}
% 2007 Q8
Given a flock of chickens, between any two chickens one of them is
dominant. A relation, R, is defined between chicken x and chicken y as xRy if x is
dominant over y. This gives what is known as a pecking order to the flock. Home
Farm has 5 chickens: Amy, Beth, Carol, Daisy and Eve, with the following relations:

\begin{itemize}
\item Amy is dominant over Beth and Carol
\item Beth is dominant over Eve and Carol
\item Carol is dominant over Eve and Daisy
\item Daisy is dominant over Eve, Amy and Beth
\item Eve is dominant over Amy.
\end{itemize}

\newpage


Types of Relations

\begin{itemize}
\item Antisymmetric
\item Symmetric
\item Reflexive 
\item Transitive
\end{itemize}
%---------------------------------



\end{document}