\documentclass{article}
\usepackage{amsmath}
\usepackage{amssymb}
\usepackage{graphicx}
\begin{document}
%-----------------------------------------------------%
\section*{Session 06: Digraphs and Relations}
\begin{itemize}
\item[6A.1] In-degree and out-degree
\item[6A.2]
\item[6A.3]
\end{itemize}
\subsection*{Relations}
\begin{itemize}
\item[6B.1] Equivalence RElations (6.2.2)
\item[6B.2]
\item[6B.3] Relations and Cartesian Products (6.3)
\end{itemize}


\begin{itemize}
\item[Reflexive]: 
\item[Symmetric]: 
\item[Transitive]: 
\item[Anti-symmetric]: 
\item[Equivalence Relation]: 
\item[Partial Order]:
\item[Order]:
\end{itemize}
\newpage
%---------------------------------------%
\subsection*{Directed Graphs}
A directed graph or digraph is a graph, or set of nodes connected by edges, where the edges have a direction associated with them. In formal terms a digraph is a pair $G=(V,A)$(sometimes $G=(V,E)$) of a set $V$, whose elements are called vertices or nodes, and a set $A$ of ordered pairs of vertices, called arcs, directed edges, or arrows (and sometimes simply 'edges' with the corresponding set named E instead of A).


%--------------------------------------%
\subsection*{Recurrence Relation}
In mathematics, a recurrence relation is an equation that recursively defines a sequence, once one or more initial terms are given: each further term of the sequence is defined as a function of the preceding terms.
The term difference equation sometimes (and for the purposes of this article) refers to a specific type of recurrence relation. However, "difference equation" is frequently used to refer to any recurrence relation.


%--------------------------------------------------------%

\subsubsection*{2008 Zone B question 7}
Given the set $S =\{g,e,r,b,i,l\}$.
\begin{itemize}
\item Describe how each subset of $S$ can be represented by using a 6 digit binary string
\item Write down the string corresponding to the subset $\{g,r,l\}$ and the subset corresponding to the string 010101.
\item What is the total number of subsets of S?
\end{itemize}
%--------------------------------------------------------%

% http://www.textbooksonline.tn.nic.in/books/12/std12-bm-em-1.pdf

\large
\begin{itemize}
\item A relation $R$ from a set A to a set B is a subset of the
\textbf{cartesian product} A x B. 
\item Thus R is a set of \textbf{ordered pairs} where
the first element comes from A and the second element comes
from B i.e. $(a, b)$
\end{itemize}

\large
\begin{itemize}

\item  If $(a, b) \in R$ we say that $a$ is related to $b$ and write $aRb$.
\item If $(a, b) \notin R$, we say that $a$ is not related to $b$ and write $aRb$. CHECK
\item If
$R$ is a relation from a set $A$ to itself then we say that ``$R$ is a relation
on $A$".

\item Let $A = \{2, 3, 4, 6\}$ and $B = \{4, 6, 9\}$
\item Let R be the relation from A to B defined by \textit{\textbf{xRy}} if $x$
divides $y$ exactly.


\item Let $A = \{2, 3, 4, 6\}$ and $B = \{4, 6, 9\}$
\item Let R be the relation from A to B defined by \textit{\textbf{xRy}} if $x$
divides $y$ exactly.
\item  Then
\[R = {(2, 4), (2, 6), (3, 6), (3, 9), (4, 4), (6, 6)}\]
\end{itemize}



\end{document}

