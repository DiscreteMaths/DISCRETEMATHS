\documentclass{beamer}

\usepackage{amsmath}

\begin{document}
%--------------------------------------------------------- %
\begin{frame}
\huge
\[ \mbox{Discrete Mathematics: Graphs Theory}\]
\[ \mbox{Types of Graphs}\]
\Large
\[ \mbox{www.Stats-Lab.com}\]
\[ \mbox{Twitter: StatsLabDublin}\]
\end{frame}

\begin{frame}
\Large
\textbf{Types of Graph}
\begin{itemize}
\item Bipartite Graphs
\item Path Graphs
\item Cycle Graphs
\item Null Graphs
\end{itemize}
\end{frame}
%-------------------------------------------------------- %
\begin{frame}
\Large
\vspace{-2cm}
\textbf{Bipartite Graphs}
\begin{itemize}
\item A bipartite graph is a graph whose vertex-set can be split into two sets in such a way that each edge of the graph joins a vertex in first set to a vertex in second set.
\end{itemize}

\end{frame}

%-------------------------------------------------------- %
\begin{frame}
\Large
\vspace{-2cm}
\textbf{Path Graphs}
\begin{itemize}
\item A path graph is a graph consisting of a single path. 
\item The path graph with n vertices is denoted by $P_n$.
\end{itemize}

\end{frame}
%-------------------------------------------------------- %
\begin{frame}
\Large
\vspace{-2cm}
\textbf{Cycle Graphs}
\begin{itemize}
\item A cycle graph is a graph consisting of a single cycle. 
\item The cycle graph with n vertices is denoted by $C_n$.
\end{itemize}

\end{frame}

%-------------------------------------------------------- %
\begin{frame}
\Large
\vspace{-2cm}
\textbf{Null Graphs}
\begin{itemize}
\item A null graphs is a graph containing no edges. \item The null graph with n vertices is denoted by $N_n$.
\end{itemize}

\end{frame}


\begin{frame}
END
\end{frame}
%-------------------------------------------------------- %
\end{document}