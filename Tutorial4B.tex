\documentclass[12pt]{article}
http://people.bath.ac.uk/ma2035/teaching/graph-theory/exam-sols.pdf
https://www.tutorialspoint.com/graph_theory/graph_theory_examples.htm
\usepackage{amsmath}
\usepackage{amssymb}

%opening

\begin{document}

\begin{center}
\huge{Mathematics for Computing}\\
\LARGE{Functions}
\end{center}
\tableofcontents

Both oneone and onto
one one only
neither one ontoone or 
Arrow Diagram
onto functions (4:2:2)
Invertible Functions
one-to-one correspondence
f(x) =3x-5
f-1(x) = x+5/3
Domain and Co-domain
Polynomial Functions (4:1:5)
 Linear quadratic cubic and quintic
 1- x 3 is  best described as which type of function
Absolute Value Function
Laws of Logarithms
Exponentials
O-notation (4:4:1)
Power Functions (4:4:2)
log2(x)  + log2(y) 
options : log2(xy) log2(x+y) log2(xy) 
log2(x/y)
exp(ab)

\begin{enumerate}


\item 

What conditions must be satisfied for a function to have an inverse.
\begin{enumerate}
\item One-one and onto
\item One-to-one only
\item onto only
\item Neither onto nor One-to-One
\end{enumerate}
\item 

If f is a function for which the rule is f(x) = 7/8  - x, where x is real, the rule for the inverse function f-1 is:
A	f -1(x) = 8/7 + x
B	f -1(x) = -8/x + 7 
C	f -1(x) = 2x + 73/4 
D	f -1(x) = 7/8 - x  % (This one)
E	f -1(x) = 8/7 + x

\item 
Which of the following functions is not one-to-one?
\begin{enumerate}
\item 	$f(x) = 9 - x^2, x \geq 0$
\item 	$f(x) = 1/x^2  - 9$ %  (This one)
\item	f(x) = 1 -9x
\item	$f(x) = \sqrt{x}$
\item	$f(x) = 3/x $
\end{enumerate}
\item 
The range of the function with rule f(x) = |x - 4| + 3 is:
A	(4, \infty)
B	R
C	[3, \infty) This one
D	(4, \infty)
E	(-1, \infty)

% http://www.analyzemath.com/college_algebra/problems_5.html
\item Let X = {1, 2, 3, 4} and Y={A,B,C,D}. Determine whether each relation is a function, with X -> Y.
(a) f = {(2, C), (1, D), (2, A), (3,B), (4, D)}

No:  Two different ordered pairs (2, C) and (2, A) in f have the same number 2 as their first coordinate

\item Let $X = {1, 2, 3, 4}$. Determine whether the following relation on X is a function.

(b) g = {(3, A), (4, B), (1, C)}

No The element 2 does not appear as the first coordinate in any ordered pair in g.
\end{enumerate}
\end{document}