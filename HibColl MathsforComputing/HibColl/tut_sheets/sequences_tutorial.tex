%\documentclass[11pt, a4paper,dalthesis]{report}    % final 
%\documentclass[11pt,a4paper,dalthesis]{report}
%\documentclass[11pt,a4paper,dalthesis]{book}

\documentclass[11pt,a4paper,titlepage,oneside,openany]{article}

\pagestyle{plain}
%\renewcommand{\baselinestretch}{1.7}

\usepackage{setspace}
%\singlespacing
\onehalfspacing
%\doublespacing
%\setstretch{1.1}

\usepackage{amsmath}
\usepackage{amssymb}
\usepackage{amsthm}
\usepackage{multicol}

\usepackage[margin=3cm]{geometry}
\usepackage{graphicx,psfrag}%\usepackage{hyperref}
\usepackage[small]{caption}
\usepackage{subfig}

\usepackage{algorithm}
\usepackage{algorithmic}
\newcommand{\theHalgorithm}{\arabic{algorithm}}

\usepackage{varioref} %NB: FIGURE LABELS MUST ALWAYS COME DIRECTLY AFTER CAPTION!!!
%\newcommand{\vref}{\ref}

\usepackage{index}
\makeindex
\newindex{sym}{adx}{and}{Symbol Index}
%\newcommand{\symindex}{\index[sym]}
%\newcommand{\symindex}[1]{\index[sym]{#1}\hfill}
\newcommand{\symindex}[1]{\index[sym]{#1}}

%\usepackage[breaklinks,dvips]{hyperref}%Always put after varioref, or you'll get nested section headings
%Make sure this is after index package too!
%\hypersetup{colorlinks=false,breaklinks=true}
%\hypersetup{colorlinks=false,breaklinks=true,pdfborder={0 0 0.15}}


%\usepackage{breakurl}

\graphicspath{{./images/}}

\usepackage[subfigure]{tocloft}%For table of contents
\setlength{\cftfignumwidth}{3em}

\input{longdiv}
\usepackage{wrapfig}


%\usepackage{index}
%\makeindex
%\usepackage{makeidx}

%\usepackage{lscape}
\usepackage{pdflscape}
\usepackage{multicol}

\usepackage[utf8]{inputenc}

%\usepackage{fullpage}

%Compulsory packages for the PhD in UL:
%\usepackage{UL Thesis}
\usepackage{natbib}

%\numberwithin{equation}{section}
\numberwithin{equation}{section}
\numberwithin{algorithm}{section}
\numberwithin{figure}{section}
\numberwithin{table}{section}
%\newcommand{\vec}[1]{\ensuremath{\math{#1}}}

%\linespread{1.6} %for double line spacing

\usepackage{afterpage}%fingers crossed

\newtheorem{thm}{Theorem}[section]
\newtheorem{defin}{Definition}[section]
\newtheorem{cor}[thm]{Corollary}
\newtheorem{lem}[thm]{Lemma}

%\newcommand{\dbar}{{\mkern+3mu\mathchar'26\mkern-12mu d}}
\newcommand{\dbar}{{\mkern+3mu\mathchar'26\mkern-12mud}}

\newcommand{\bbSigma}{{\mkern+8mu\mathsf{\Sigma}\mkern-9mu{\Sigma}}}
\newcommand{\thrfor}{{\Rightarrow}}

\newcommand{\mb}{\mathbb}
\newcommand{\bx}{\vec{x}}
\newcommand{\bxi}{\boldsymbol{\xi}}
\newcommand{\bdeta}{\boldsymbol{\eta}}
\newcommand{\bldeta}{\boldsymbol{\eta}}
\newcommand{\bgamma}{\boldsymbol{\gamma}}
\newcommand{\bTheta}{\boldsymbol{\Theta}}
\newcommand{\balpha}{\boldsymbol{\alpha}}
\newcommand{\bmu}{\boldsymbol{\mu}}
\newcommand{\bnu}{\boldsymbol{\nu}}
\newcommand{\bsigma}{\boldsymbol{\sigma}}
\newcommand{\bdiff}{\boldsymbol{\partial}}

\newcommand{\tomega}{\widetilde{\omega}}
\newcommand{\tbdeta}{\widetilde{\bdeta}}
\newcommand{\tbxi}{\widetilde{\bxi}}



\newcommand{\wv}{\vec{w}}

\newcommand{\ie}{i.e. }
\newcommand{\eg}{e.g. }
\newcommand{\etc}{etc}

\newcommand{\viceversa}{vice versa}
\newcommand{\FT}{\mathcal{F}}
\newcommand{\IFT}{\mathcal{F}^{-1}}
%\renewcommand{\vec}[1]{\boldsymbol{#1}}
\renewcommand{\vec}[1]{\mathbf{#1}}
\newcommand{\anged}[1]{\langle #1 \rangle}
\newcommand{\grv}[1]{\grave{#1}}
\newcommand{\asinh}{\sinh^{-1}}

\newcommand{\sgn}{\text{sgn}}
\newcommand{\morm}[1]{|\det #1 |}

\newcommand{\galpha}{\grv{\alpha}}
\newcommand{\gbeta}{\grv{\beta}}
%\newcommand{\rnlessO}{\mb{R}^n \setminus \vec{0}}
\usepackage{listings}

\interfootnotelinepenalty=10000

\newcommand{\sectionline}{%
  \nointerlineskip \vspace{\baselineskip}%
  \hspace{\fill}\rule{0.5\linewidth}{.7pt}\hspace{\fill}%
  \par\nointerlineskip \vspace{\baselineskip}
}

\renewcommand{\labelenumii}{\roman{enumii})}

\begin{document}

\begin{center}
  \textbf{Tutorial Sheet 1}
\end{center}

\begin{enumerate}
\item Evaluate the terms $a_0,a_1,a_2,a_3,a_{10},a_{100}$ in the following sequences, where possible.
  \begin{multicols}{2}
    \begin{enumerate}
    \item $a_n=3 n^2+2n +1$
    \item $a_n=\frac{n^2+7}{2n+3}$
    \item $a_n=(-1)^n n$
    \item $a_n=3^n + (-3)^n + \frac{6n^4}{3n^4},\  n\ne 0$
    \item $a_n=2^n .3^n$
    \item $a_n=6^n$
    \item $a_n=\frac{1}{n+1} + \frac{1}{n+2}$
    \item $a_n=\frac{2n+3}{n^2+3n+2}$
    \item $a_n=3n^7 + 2n^5 + n^5 +2$
    \item $a_n=n^n,\ n\ne0$
    \item $a_n=n!$
    \item $a_n=\frac{2^n+n^2+3n!}{3n!+n^2+2^n}$
    \end{enumerate}        
  \end{multicols}
  In addition, prove that the sequences vii) and viii) are equivilent. Write programs to evaluate the first $100$ terms in these sequences.
\item Evaluate the terms $a_0,a_1,a_2,a_3$ in the following sequences and write programs to evaluate the first $100$ terms in each sequence.

  \begin{multicols}{2}
    \begin{enumerate}
    \item
      \begin{equation*}
        a_n=\begin{cases}
          2+\frac{1}{2^n} &,\qquad n\text{ even}\\
          -2^n &,\qquad n\text{ odd}
        \end{cases}
      \end{equation*}
    \item
      \begin{equation*}
        b_n=\begin{cases}
          3^n-14n-7 &,\qquad n\text{ even}\\
          3^n-2n &,\qquad n\text{ odd}
        \end{cases}
      \end{equation*}
      \begin{equation*}
        a_n=\begin{cases}
          n^2+1 &,\qquad b_n>0\\
          3^n-2n &,\qquad b_n \le 0
        \end{cases}
      \end{equation*}
    \item
      \begin{equation*}
        a_n=\begin{cases}
          (n+10)^6 &,\qquad n < 2\\
          n^3 &,\qquad \text{ otherwise}
        \end{cases}
      \end{equation*}
    \item
      \begin{equation*}
        a_n=\begin{cases}
          (-1)^{n-1} \frac{\pi^{2n-1}}{(2n-1)!} &,\qquad n\text{ even}\\
          (-1)^{n+1} \frac{\pi^{2n+1}}{(2n+1)!}&,\qquad n\text{ odd}
        \end{cases}
      \end{equation*}
    \end{enumerate}        
  \end{multicols}

\item Evaluate the term $a_7$ in the following recursive sequences and write programs to evaluate the first $100$ terms in each sequence. In addition, for sequence iii), evaluate $a_{12000003}$.


  \begin{multicols}{2}
    \begin{enumerate}
    \item
      \begin{equation*}
        \begin{cases}
          a_0&=1\\
          a_n&=n^2 a_{n-1}
        \end{cases}
      \end{equation*}
    \item
      \begin{equation*}
        \begin{cases}
          a_0&=\frac{1}{2}\\
          a_n&=a_{n-1}\left(2-a_{n-1}\sqrt{2}\right)
        \end{cases}
      \end{equation*}
    \item
      \begin{equation*}
        \begin{cases}
          a_0&=-\frac{1}{4}\\
          a_1&=2\\
          a_n&=\frac{a_{n-1}}{a_{n-2}}
        \end{cases}
      \end{equation*}
      
    \end{enumerate}
  \end{multicols}

\pagebreak

\item The factorial function is defined \emph{recursively} as
  \begin{equation*}
    n! = a_n \text{, where  }
    \begin{cases}
      a_0&=1\\
      a_n&=n a_{n-1}
    \end{cases}
  \end{equation*}
The factorial function cas equivilently be defined \emph{iteratively} as the product of the first $n$ positive integers\begin{equation*}
  n! = 1\times2\times3\times4\times5\times\ldots \times (n-1)\times(n-1)\times n
\end{equation*}
Write a program which computes $n!$ in this way without using recursion.

\item Evaluate the terms $a_1,a_2,a_3,a_4,a_{1000}$ in the following sequences
   \begin{multicols}{2}
    \begin{enumerate}
    \item $a_n=\frac{(n+1)!}{n!}$
    \item $a_n=\frac{(n+2)!}{n!}$
    \item $a_n=\frac{(n+1)!}{(n+2)!}$
    \item $a_n=(-1)^n \frac{(n+3)!}{(n+2)!}$
    \end{enumerate}        
  \end{multicols}

\item For the following strictly positive sequences, evaluate the ratio $\frac{a_{n+1}}{a_n}$ and hence say whether the sequence is increasing, decreasing or neither. In addition, say whether the sequence is bounded or unbounded.
  \begin{multicols}{2}
    \begin{enumerate}
    \item $a_n=n$
    \item $a_n=\frac{n+4}{n+5}$
    \item $a_n=\frac{n+5}{n+4}$
    \item $a_n=n!$
    \item $a_n=\frac{n!}{n^n},\ n\ne 0$
    \item $a_n=\frac{2^n+1}{2^n-1}$
    \end{enumerate}        
  \end{multicols}

\item Evaluate the limits of the following sequences as $n \to \infty$.
    \begin{multicols}{2}
    \begin{enumerate}
    \item $a_n=\frac{1}{n}+\frac{1}{n^2}+\frac{1}{n!}$
    \item $a_n=\frac{2n+2}{3n+7}$
    \item $a_n=\frac{4n-3}{2n^2+2n+1}$
    \item $a_n=\frac{3n+2n^2+1}{5n^2+6}$
    \item $a_n=\frac{7n^2}{20000n+n^2}$
    \item $a_n=\frac{10^{10^{10}}n+n^2}{n^2}$
    \item $a_n=\frac{3n}{n!}$
    \item $a_n=\frac{10n!}{n^n}$
    \item $a_n=\frac{5n^4+6n^2+4}{2^n+1}$
    \item $a_n=\left(\frac{1}{4} \right)^n$
    \end{enumerate}        
  \end{multicols}

\item Assumming that the following recursive sequence has a limit $\lim_{n\to \infty}a_n = L$, find this limit in terms of $D$.
  \begin{equation*}
    \begin{cases}
      a_0&=1\\
      a_{n+1}&=\frac{2}{3}a_n + \frac{D}{3 a_n^2}
    \end{cases}
  \end{equation*}
What is the limit of the sequence defined by
  \begin{equation*}
    \begin{cases}
      a_0&=1\\
      a_{n+1}&=\frac{k-1}{k}a_n + \frac{D}{k a_n^{k-1}}
    \end{cases}
  \end{equation*}

\pagebreak

\item The Fibonacci sequence $f_n$ is defined recursively by the rule
  \begin{equation*}
    \begin{cases}
      f_0&=0\\
      f_1&=1\\
      f_n&=a_{n-1}+a_{n-2}
    \end{cases}
  \end{equation*}

  \begin{enumerate}
  \item
    Write a program to evaluate the Fibonacci sequence and hence evaluate $f_{50}$.
  \item
    Let the sequence $g_n$ be defined as the ratio
    \begin{equation*}
      g_n = \frac{f_{n+1}}{f_n}
    \end{equation*}
    Write a program to evaluate the first $50$ terms of the sequence $g_n$.
  \item
    Assumming that the sequence $g_n$ has a limit $\phi$, find this limit. 
  \end{enumerate}

% \pagebreak
%   \begin{center}
%     \textbf{Assignment 1}
%     Due Tuesday Week 4 (Sept $28^th$)   
%   \end{center}
  
%   Write an Octave function which computes square roots using Heron's method to as high an accuraccy as possible. The function should be called ``sqr\_root'' and must take the following form
% \lstset{language=Octave}
% \begin{lstlisting}[title=sqr\_root.m,frame=single]
% function X=sqr_root(D)
%     ....
% endfunction
% \end{lstlisting}

% In particular, the function must be able to compute the square roots of your UL ID number, (i.e numbers of the form 10234567), as well as 0.01,0.1,10,100 and 1000 times this number; so test the function on these and many other numbers before submitting.

% Note that the stopping condition shown in the lectures will not suffice as a general stopping condition\footnote{If Octave hangs during calculation, use ``Ctrl-C'' to try to cancel the operation and return to the prompt.}. You must develop and program an alternative strategy for detecting when Heron's method has converged.

% Note also that the square of the number $X$ returned by your function must be as close to D as is possible for $X^2$ to be; that is, $X$ must be as close as it possibly can be to the true value of $\sqrt{D}$.

% Save your solutions source code in a file called ``sqr\_root.m''. To submit the assigment, you must email this solution from your UL student email to the address \emph{niall.ryan@ul.ie}. The email subject line must be ``MA4402 ASSIGNMENT SOLUTION'' and the file ``sqr\_root.m'' must be an attachment to this email.

% The email must be sent \textbf{from your UL student address}. \emph{Emails from other addresses will not be accepted}. 

\end{enumerate}
\end{document}

%%% Local Variables: 
%%% mode: latex
%%% TeX-master: t
%%% End: 
