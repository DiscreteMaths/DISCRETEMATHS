\documentclass{beamer}

\usepackage{amsmath}

\begin{document}
%--------------------------------------------------%
\begin{frame}
\begin{center}
{ \Huge
Cartesian Product }
\\
\bigskip
{ \Large
%youtube.com/StatsLabDublin \\ \vspace{0.2cm} Twitter: @statslabdublin
}
\end{center}
\end{frame}

\begin{frame}
\frametitle{Cartesian Product}
{\LARGE
\begin{itemize}
\item Let $X$ and $Y$ be sets.
\item The \textbf{cartesian product} $X \times Y$ is the set whose elements are \textbf{all} of the ordered pairs of elements $(x,y)$ where $x \in X$ and $y \in Y$.
\end{itemize}
}

%--------------------------------------------------------%

\end{frame}


\begin{frame}
\vspace{-1.9cm}
\frametitle{Cartesian Product}
{\LARGE
\begin{itemize}
\item Let $X = \{a,b,c\}$
\item Let $Y = \{0,1\}$ 
\item The cartesian product $X \times Y$ is therefore:
\end{itemize}
}
\end{frame}

\begin{frame}
\vspace{-1.9cm}
\frametitle{Cartesian Product}
{\LARGE
\begin{itemize}
\item Importantly $X \times Y \neq Y \times X$
\item Recall: Let $X = \{a,b,c\}$ and let $Y = \{0,1\}$ 
\item The cartesian product $Y \times X$ is therefore:
\end{itemize}
}


%--------------------------------------------------------%

\end{frame}



\begin{frame}
\frametitle{The Cartesian Product}
Exercises
\end{frame}

 Discrete Maths
 
A binary relation on a set A is the collection of ordered pairs of elements of A. In other words, it is the subset of the cartesian product A2= AA

Cartesian Product
 
This is a direct product pf 2 sets
 
XY = {(x,y)| xXandyY }
 
4 suits of cards and 13 Ranks, therefore 52 element cartesian prodcut.
 
N.B         AB BA
              A=A =
 
Cartesian product is not associative
\end{document}
