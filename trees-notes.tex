
%---------------------------------------%
\chapter{Session 8}
% % Section 8 Trees
%-----------------------------------------------------%
\section*{Question 8}
\subsection*{Part A : Spanning Trees}
\begin{enumerate}
\item How many edges are in the spanning tree $T$ ?
\item What is the sum of the degree sequence of $T$?
\item Write down all the possible degree sequences for the spanning tree $T$.
\end{enumerate}
%------------------------------------%

\subsection*{Part B : Binary Search Trees}
Suppose a database, comprised of 30,000 internal nodes, is structured as a Binary Search Tree.

\begin{enumerate}
\item What is the key (number) of the Root node?
\item What are the keys of the nodes at level 1?
\item For the nodes at level1, how many subtrees are there?
\item State which nodes are in the substrees of the level 1 nodes?
\item How many nodes are the between the root (level 0) and level 7. ]
(Hint: use a summation theorem mentioned in session 7
\item What is the maximum number of searchs in this database?
\end{enumerate}




\section{Question 8B}
Suppose a database, comprised of 30,000 internal nodes, is structured as a Binary Search Tree.

\begin{enumerate}
\item What is the Key of the Root node?
\item What are the keys of the nodes at level 1?
\item For the nodes at level1, how many subtrees are there?
\item State which nodes are in the substrees of the level 1 nodes?
\item How many nodes are the between the root (level 0) and level 7. ]
(Hint: use a summation theorem mentioned in session 7
\item What is the maximum number of searchs in this database?
\end{enumerate}
%--------------------------------------------------------------------

\section*{Binary Search Trees}
What is a Binary Search Tree
\[ \lfloor  \frac{\mbox{log}_2}{T} \rfloor \]

%------------------------- %
%-------------------------- %
\section*{Session 08:Trees}
\begin{itemize}
\item[8A.1] Trees
\item[8A.2] Spanning Trees
\item[8A.3] Rooted trees
\item[8A.4] Binary Search Trees
\end{itemize}
A tree is a directed graph that contains no cycles.
%-----------------------------------------------------%
\section{Question 8}

1) Draw this tree
2) Construct all the isomorphic tress with 6 vertices which can be obtainbed by attaching a new vertex of degreee one to a vertex of T.
3) Explain briefly why the tree obtained in (ii) is not isomorphic to each other.
4) Construct a tree with 6 vertices which is not isomorphic to any tree you constructed in (ii)

Part b
Determine the number of nodes on level 5 and level 10
Find an expression in terms of $\Sigma$ and h for the number of internal nodes in such a tree.
What is the smallest possible height of such a tree if there are at least 900 internal nodes.

% %------------------------------------------------------

\section{Question 8A}
\begin{enumerate}
\item How many edges are in the spanning tree $T$ ?
\item What is the sum of the degree sequence of $T$?
\item Write down all the possible degree sequences for the spanning tree $T$.
\end{enumerate}

\section{Tree Definition}
What properties must a graph have in order for it to be a
tree?
(ii) Say, with reason, whether or not it is possible to construct a tree with
degree sequence 4, 3, 3, 1, 1.
\newpage

