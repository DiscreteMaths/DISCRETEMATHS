
\documentclass{beamer}

\usepackage{amsmath}
\usepackage{graphicx}
\usepackage{amssymb}
\begin{document}


\begin{frame}
	\frametitle{Example 2: Evaluating a Function}
	Evaluate the following function for x =1,2 and 5 respectively.
	{
		\Large
	\[ f(x) = \frac{e^x + {e^{-x}}}{2} \]
}
	Remark:  Example 1 was done in previous class.
\end{frame}
%================================================================================== %
\begin{frame}
	
	\frametitle{Example 3: Evaluating a Function}
	Evaluate the function for each of the following values : 0.5,1,1.25,2.
	{
		\Large
	\[f(x) =  \sqrt{1+e^{x}}  \]
}
	Four decimal places will suffice.
\end{frame}
\begin{frame}
	\begin{center}
		\begin{tabular}{|c|c|c|c|}
			\hline \rule[-2ex]{0pt}{5.5ex} x & $e^x$ & $1+e^x$ & $\sqrt{1+e^x}$ \\ \hline
			\hline \rule[-2ex]{0pt}{5.5ex} 0.5 & \phantom{space}  &  &  \\ 
			\hline \rule[-2ex]{0pt}{5.5ex} 1 &  &  &  \\ 
			\hline \rule[-2ex]{0pt}{5.5ex} 1.25 & \phantom{space}  &  &  \\ 
			\hline \rule[-2ex]{0pt}{5.5ex} 2 & \phantom{space}  &  & \phantom{space}  \\ 
			\hline 
		\end{tabular} 
	\end{center}
\end{frame}
%================================================================================ %
\begin{frame}
	\frametitle{Recall : Sets of Numbers}
	\begin{itemize} 
		\item $\mathbb{N}$ Set of all natural numbers
		\item $\mathbb{Z}$ Set of all integers
		\item $\mathbb{Q}$ Set of all rational numbers
		\item $\mathbb{R}$ Set of all real numbers
	\end{itemize}
	
	\bigskip There are, of course, other numbers sets, but we will not be encountering them on the course.
\end{frame}
%========================================================== %

\begin{frame}
	\frametitle{Recall : Sets of Numbers}
	\begin{itemize}
		\item $\mathbb{Z}^{+}$ Set of all positive integers
		\item $\mathbb{Z}^{-}$ Set of all negative integers
		\item $\mathbb{R}^{+}$ Set of all positive real numbers
		\item $\mathbb{R}^{-}$ Set of all negative real numbers
	\end{itemize}
\end{frame}

\begin{frame}
	\frametitle{ Topic 2 : Laws of Logarithms}
	\Large
	\begin{itemize}
		\item Law 1 : Multiplcation of Logarithms
		\[ Log(a) \times Log(b) = Log(a+b) \]
		\item Law 2 : Division of Logarithms
		\[ \frac{Log(a)}{Log(b)} = Log(a-b) \]
		\item Law 3 : Powers of Logarithms
		\[ Log(a^b) = b \times Log(a) \]
	\end{itemize}
		
\end{frame}
%======================================================================================================== %

\begin{frame}
	\frametitle{ Topic 2 : Laws of Logarithms}
	\Large
	\begin{itemize}
		\item Law 1 : Multiplcation of Logarithms
		\[ Log(a) \times Log(b) = Log(a+b) \]
		\item Law 2 : Division of Logarithms
		\[ \frac{Log(a)}{Log(b)} = Log(a-b) \]
		\item Law 3 : Powers of Logarithms
		\[ Log(a^b) = b \times Log(a) \]
	\end{itemize}
	
	
	
\end{frame}
%========================================================================================================== %
\end{document}

