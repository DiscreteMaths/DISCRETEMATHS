\documentclass{beamer}

\usepackage{graphics}

\begin{document}
	
	%=========================================================================================== %
	\subsection*{Topic 1 : Special Functions}
	\begin{frame}
		\frametitle{Special Functions}
		\begin{itemize}
			% \item The factorial Operator
			\item Absolute Value Function
			\item The Sign Function
			\item Floor and Ceiling Functions
			\item Hyperbolic Functions
		\end{itemize}	
	\end{frame}
	
	
	%---------------------------------------%
	\begin{frame}
		\frametitle{Absolute Value Function}
		\Large
		\textbf{Absolute Value Function}
		\begin{itemize}
			\item The absolute value (or modulus) $|x|$ of a real number x is the non-negative value of x without regard to its sign. 
			
			%	\item For example, $|x| = x$ for a positive x, $|x| = -x$ for a negative x (in which case −x is positive), and $|0| = 0$. 
			%	\item For example, the absolute value of 4 is 4, and the absolute value of $-4$ is also 4. 
			%\item The absolute value of a number may be thought of as its distance from zero.
		\end{itemize}
		\[|x| = \begin{cases} x, & \mbox{if }  x \ge 0  \\ -x,  & \mbox{if } x < 0. \end{cases} \]
	\end{frame}
	%---------------------------------------%
	\begin{frame}
		\frametitle{Topic 1 : Absolute Value Function}
		\Large
		\vspace{-1cm}
		\begin{itemize}
			
			\item For a positive x, $|x| = x$ 
			\item For a negative x (in which case −x is positive) $|x| = -x$ 
			\item The absolute value of 0 is 0:  $|0| = 0$. 
			\item For example, the absolute value of 4 is 4, and the absolute value of $-4$ is also 4. 
			\item IMPORTANT:  The input to this function is any real number. The output of this function will always be a positive real numbers.
			%\item The absolute value of a number may be thought of as its distance from zero.
		\end{itemize}
		
		
	\end{frame}
	\begin{frame}
		\frametitle{Topic 1 : Sign Function}
		\Large
		\textbf{Sign Function}
		\begin{itemize}
			\item The sign function $sng(x)$ of a real number x is a signed value of absolute value of 1, dependent on the sign of $x$. 
			
			%	\item For example, $|x| = x$ for a positive x, $|x| = -x$ for a negative x (in which case −x is positive), and $|0| = 0$. 
			%	\item For example, the absolute value of 4 is 4, and the absolute value of $-4$ is also 4. 
			%\item The absolute value of a number may be thought of as its distance from zero.
			\item IMPORTANT:  The input to this function is any real number. The output of this function will always be either 1 or -1.
		\end{itemize}
		\[ sgn(x) = \begin{cases} 1, & \mbox{if }  x \ge 0  \\ -1,  & \mbox{if } x < 0. \end{cases} \]
	\end{frame}
	%---------------------------------------%
	\begin{frame}
		\frametitle{Topic 1 : Floor and Ceiling Functions}
		\Large
		\begin{itemize}
			\item The floor and ceiling functions map a real number to the largest previous or the smallest following integer, respectively. \item More precisely, \[floor(x) = \lfloor x\rfloor \] is the largest integer not greater than x and \[ceiling(x) =  \lceil x \rceil \] is the smallest integer not less than x.
		\end{itemize}
		
		
	\end{frame}
	
	%---------------------------------------%
	\begin{frame}
		\frametitle{Topic 1 : Floor and Ceiling Functions}
		\textbf{Examples}
		{
			\LARGE
			\begin{eqnarray}
			\lfloor 3.14 \rfloor =& 3 \\ \bigskip
			\lceil -4.5 \rceil =& -5 \\ \bigskip
			| -4 | =&  4
			\end{eqnarray}
		}
		{\Large	
			\textbf{Remark:}  Input to the floor and ceiling function can be any really number, but outputs are always integers.
		}
	\end{frame}
	
	
\begin{frame}
	\frametitle{Floor and Ceiling Functions}
The floor and ceiling functions map a real number to the largest previous or the smallest following integer, respectively. More precisely, floor(x) = $\lfloor x\rfloor$ is the largest integer \textbf{not greater} than x and ceiling(x) =  $\lceil x \rceil$ is the smallest integer \textbf{not less} than x.

\end{frame}
%=================================================================================== %
\begin{frame}
\frametitle{Floor and Ceiling Functions}

{
\large
\begin{itemize}
\item $\lceil x\rceil$ : Ceiling function
\item $\lfloor x\rfloor$  : Floor Function
\item $\{x\}$ : Fractional Part of a number
\[\{x\} = x- \lfloor x\rfloor \]
\end{itemize}
}
\end{frame}

\end{document}
%---------------------------------------------- %
\begin{frame}
\frametitle{Examples}
{  \Large 

\begin{tabular}{|c|c|c|c|} \hline
Value x & Floor $\lfloor x\rfloor$ & Ceiling  $\lceil x\rceil$ & Fractional $ \{ x \} $\\ \hline

-1.4  &	& &	 \\
2.3&	&		& \\
7/9&		&		& \\
-16/3&		&		& \\
0 & 	&		& \\
1 & 	&		& \\
\hline
\end{tabular} 

}
\end{frame}
%---------------------------------------------- %
\begin{frame}
\frametitle{Examples}
{  \Large 

\begin{tabular}{|c|c|c|c|} \hline
Value x & Floor $\lfloor x\rfloor$ & Ceiling  $\lceil x\rceil$ & Fractional $ \{ x \} $\\ \hline
-1.4  &	-2	&-1&	 0.4\\
2.3&	&		& \\
7/9&		&		& \\
-16/3&		&		& \\
0 & 	&		& \\
1 & 	&		& \\
\hline
\end{tabular} 

}
\end{frame}
\end{document}
%---------------------------------------------- %
\begin{frame}
\frametitle{Examples}
{  \Large 

\begin{tabular}{|c|c|c|c|} \hline
Value x & Floor $\lfloor x\rfloor$ & Ceiling  $\lceil x\rceil$ & Fractional $ \{ x \} $\\ \hline
-1.4  &	-2	&-1&	 0.4\\
2.3&	2&	3	& 0.3\\
7/9&		&		& \\
-16/3&		&		& \\
0 & 	&		& \\
1 & 	&		& \\
\hline
\end{tabular} 

}
\end{frame}
%---------------------------------------------- %
\begin{frame}
\frametitle{Examples}
{  \Large 

\begin{tabular}{|c|c|c|c|} \hline
Value x & Floor $\lfloor x\rfloor$ & Ceiling  $\lceil x\rceil$ & Fractional $ \{ x \} $\\ \hline
-1.4  &	-2	&-1&	 0.4\\
2.3&	2&	3	& 0.3\\
7/9&	0&	1	& 0.7777\\
-16/3&		&		& \\
0 & 	&		& \\
1 & 	&		& \\
\hline
\end{tabular} 

}
\end{frame}
%---------------------------------------------- %
\begin{frame}
\frametitle{Examples}
{  \Large 

\begin{tabular}{|c|c|c|c|} \hline
Value x & Floor $\lfloor x\rfloor$ & Ceiling  $\lceil x\rceil$ & Fractional $ \{ x \} $\\ \hline
-1.4  &	-2	&-1&	 0.4\\
2.3&	2&	3	& 0.3\\
7/9&	0&	1	& 0.7777\\
-16/3&	-6&	-5	& 0.3333\\
0 & 	&		& \\
1 & 	&		& \\
\hline
\end{tabular} 

}
\end{frame}
%---------------------------------------------- %
\begin{frame}
\frametitle{Examples}
{  \Large 

\begin{tabular}{|c|c|c|c|} \hline
Value x & Floor $\lfloor x\rfloor$ & Ceiling  $\lceil x\rceil$ & Fractional $ \{ x \} $\\ \hline
-1.4  &	-2	&-1&	 0.4\\
2.3&	2&	3	& 0.3\\
7/9&	0&	1	& 0.7777\\
-16/3&	-6&	-5	& 0.3333\\
0 & 0&0 &0\\
1 & 	&		& \\
\hline
\end{tabular} 

}
\end{frame}
%---------------------------------------------- %
\begin{frame}
\frametitle{Examples}
{  \Large 

\begin{tabular}{|c|c|c|c|} \hline
Value x & Floor $\lfloor x\rfloor$ & Ceiling  $\lceil x\rceil$ & Fractional $ \{ x \} $\\ \hline
-1.4  &	-2	&-1&	 0.4\\
2.3&	2&	3	& 0.3\\
7/9&	0&	1	& 0.7777\\
-16/3&	-6&	-5	& 0.3333\\
0 & 0&0 &0\\
1 & 1 &1 &0 \\
\hline
\end{tabular} 

}
\end{frame}
%========================================================================= %
\end{document}
