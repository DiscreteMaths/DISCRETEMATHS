
%------------------%

% PT 2

Logarithms were introduced by John Napier in the early 17th century as a means to simplify calculations. 

They were rapidly adopted by navigators, scientists, engineers, and others to perform computations more easily, using slide rules and logarithm tables. 

% PT 3

Sound: the decibel is a logarithmic unit quantifying sound pressure and voltage ratios. 


Chemistry: pH and pOH are logarithmic measures for the acidity of an aqueous solution. Logarithms are commonplace in scientific formulae.

%--------------------------------------------------------------%
% PT 8
Scenario: During your calculations for some Mathematics problem, you end up with the following equation:

〖10〗^𝑥=550

This can be solved by converting it into log form:

log_10⁡〖550=𝑥〗


% PT 11


Your calculator can only solve logs that have a base of 10 or e.

If you encounter a log which has a base which is not 10 or e, then you need to apply the change of base formula:

%------------------------------------------------------------------------

The letter “e” represents the exponential function

𝑒≈"2.718281828..."

The exponential function is widely used in physics, chemistry and mathematics.

It is commonly used to predict growth e.g. population growth, economy growth, etc.

%---------------------------------------------------------------------
% PT 17
𝑒^𝑥=12
Can be rewritten:

log_𝑒⁡〖12=𝑥〗

2.485=𝑥

𝑥=2.485
