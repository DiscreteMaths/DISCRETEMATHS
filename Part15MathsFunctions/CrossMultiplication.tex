	
%====================================================%
\begin{frame}
	\frametitle{Tech Maths 2}
	\textbf{Cross Multiplication}
	\Large 
	\begin{itemize}
		\item Can simplify an expression by multiplying both the numerator and denominator by same term.
		\item This does not change the value of the expression.
		\item Remark
		{
			\LARGE
			\[ \frac{A}{B} + \frac{X}{Y} = \frac{AY}{BY} + \frac{BX}{BY} = \frac{AY+BX}{BY}\]
		}
	\end{itemize}
\end{frame}
%======================================================================= %


\begin{frame}
	\frametitle{Cross Mutliplication}
	\Large
	\[  \frac{p}{x+a} + \frac{q}{x+b}   = \frac{p(x+b) + q(x+a)}{(x+a)(x+b)} = \frac{(p+q)x+ (pb+aq)}{(x+a)(x+b)} \]
	
	\begin{itemize}
		\item $\{p,q,a,b\} \in R$
	\end{itemize}
	
\end{frame}
%======================================================================= 
\begin{frame}
	\frametitle{Cross Mutliplication}
	\Large
	\[  \frac{4}{x+2} + \frac{2}{x-1}   = \frac{4(x-1) + 2(x+2)}{(x+2)(x-1)} \]
	\[ = \frac{(4+2)x + (4(-1)+(2\times 2)}{(x+2)(x-1)} \]
	\[ = \frac{2x}{x^2+x-2}\]
	
\end{frame}
%====================================================%
\begin{frame}
	\frametitle{Tech Maths 2}
	\textbf{Cross-Multiplication: Example 1}
	% \textbf{Lecture 1B}
	\begin{itemize}
		\item Solve the following Equation for A and B
		\item $A,B \in \mathbb{R}$
	\end{itemize}
	
	\[ \frac{2x + 5}{x^2 - 4x - 12} = \frac{A}{x-6} + \frac{B}{x+2}\]
\end{frame}
%====================================================%
\begin{frame}
	\frametitle{Tech Maths 2}
	%\textbf{Lecture 1B}
	\textbf{Cross-Multiplication: Example 2}
	\begin{itemize}
		\item Solve the following Equation for A and B
		\item $A,B \in \mathbb{R}$
	\end{itemize}
	
	\[ \frac{5}{x^2 - 4x - 12} = \frac{A}{x-6} + \frac{B}{x+2}\]
\end{frame}


%======================================================================= %

\end{document}
