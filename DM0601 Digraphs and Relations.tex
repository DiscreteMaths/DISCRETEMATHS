\documentclass{beamer}

\usepackage{amsmath}
\usepackage{amssymb}
\usepackage{graphics}
\usepackage{graphicx}

\begin{document}
%-----------------------------------------------%
\begin{frame}
\frametitle{Digraphs and Relations}
\end{frame}
%-----------------------------------------------%
\begin{frame}
\frametitle{Digraphs and Relations}
\textbf{What is a digraph?}
\begin{itemize}
\item Short for a ``\textit{Directed Graph}".
\item Comprised of vertices and edge (common to all area of graph theory, but edges are directed - often indicated with arrows)
\item Useful for visualizing relations.
\item Very useful for a variety of IT applications (e.g. Multistate process (behaviour of AI player) and project management).
\item 
\end{itemize}
\end{frame}


%-----------------------------------------------%
\begin{frame}
\frametitle{Digraphs and Relations}
\begin{figure}
\centering
\includegraphics[width=0.8\linewidth]{./ex1}
%\caption{Example of Digraph}
\label{fig:ex1}
\end{figure}



\end{frame}
%-----------------------------------------------%
\begin{frame}
\frametitle{Digraphs and Relations}
\textbf{What is a relation?}
\begin{itemize}
\item 
\item 
\item 
\item Useful for mathematically expressing behaviour and interaction of AI players in computer games.
\end{itemize}

\end{frame}
%-----------------------------------------------%
\begin{frame}
\frametitle{Digraphs and Relations}
\textbf{Important Terminology for Relations}
\begin{itemize}
\item Reflexive,
\item Symmetrix,
\item Transitive,
\item Equivalence Relation.
\end{itemize}

\end{frame}
%-----------------------------------------------%
\end{document}
