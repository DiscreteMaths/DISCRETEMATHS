\documentclass{beamer}

\usepackage{amsmath}
\usepackage{framed}
\usepackage{amssymb}
\usepackage{graphics}

\begin{document}
\begin{frame}
\Huge
\[ \mbox{Mathematics for Computing} \]
\huge
\[ \mbox{Irrational Numbers} \]

\Large
\[ \mbox{kobriendublin.wordpress.com} \]
\[ \mbox{Twitter: @StatsLabDublin} \]

\end{frame}
%----------------------------- %
\begin{frame}
\frametitle{Irrational Numbers}
\Large
\vspace{-1cm}
Given x is the irrational positive number $\sqrt{2}$ :

\begin{itemize}
\item[(i)] Express $x^8$ in binary notation.
\item[(ii)] Is $x^8$ a rational number?
\item[(iii)] Write
$\left( \frac{1}{x} \right)^3$
 in the form $2^y$ where $y \in \mathbb{Q}$.

\end{itemize}
\end{frame} 

%----------------------------- %
\begin{frame}
\frametitle{Irrational Numbers}
\Large
\vspace{-1cm}
\textbf{Part (i)} Express $x^8$ in binary notation\\
\bigskip
\begin{itemize}
\item $(\sqrt{2})^8 = 2^4  = 16$
\item The binary equivalent of 16 is 1000.
\end{itemize}
\end{frame}
%----------------------------- %
\begin{frame}
\frametitle{Irrational Numbers}
\Large
\vspace{-1cm}
\textbf{Part (ii)} Is $x^8$ a rational number?\\

\[x^8 =16 \]

Yes 16 is an integer and therefore also a rational number.
\[x^8 \in \mathbb{Q}\]

\end{frame}



\begin{frame}
\frametitle{Irrational Numbers}
\Large
\vspace{-0.5cm}
\textbf{Part (iii)}  Write
$\left( \frac{1}{x} \right)^3$
 in the form $2^y$ where $y \in \mathbb{Q}$.

\[ \left( \frac{1}{x} \right)^3 = \left( \frac{1}{\sqrt{2}} \right)^3\]
\[\left(  \frac{1}{\sqrt{2}} \right)^3 = \left(2^{-1/2} \right)^3\]
\[ \left(2^{-1/2} \right)^3 = 2^{-3/2}\]
\[y = \frac{-3}{2} \]
\end{frame}
\end{document}
%----------------------------- %
\begin{frame}

%




\end{frame}
%----------------------------- %
\end{document}