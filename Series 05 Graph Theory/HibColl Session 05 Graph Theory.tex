\documentclass{article}
\usepackage{amsmath}
\usepackage{amssymb}
\usepackage{graphicx}
\begin{document}
\section{Set Theory}
\begin{enumerate}
\item The Universal Set $\mathcal{U}$
\item Union
\item Intersection
\item Set Difference
\item Relative Difference
\end{enumerate}

\newpage




%-----------------------------------------------------%
\section*{Session 05:Graphs}
\begin{itemize}
\item[5A.1] What is a Graph?
\item[5A.2] Paths Cycles and Connectivity
\item[5A.3] Isomorphisms of a graph
\item[5A.4] Adjacency Matrices and Adjacency Lists
\end{itemize}


\subsection*{Isomorphism}
\begin{itemize}
\item They have a different number of connected components
\item They have a different number of vertices
\item They have different degrees sequences
\item They have a different number of paths of any given length
\item They have a different number of cycles of any length.
\end{itemize}

\subsection*{Adjacency Lists}
\begin{itemize}
\item[u]: $\{v\}$
\item[v]: $\{w,x\}$
\item[w]: $\{v,x\}$
\item[z]: $\{v,w\}$
\end{itemize}




\begin{itemize}
\item Spanning Subgraphs of G.

\item a vertex is said to be an \textbf{emph{ isolated vertex}} if it has a degree of zero.
\item a vertex is said to be an \textbf{emph{ end-vertex}} if it has a degree of one.
\item a vertex is said to be an \textbf{emph{ even vertex}} if it has a degree of an even number.
\item a vertex is said to be an \textbf{emph{ odd vertex}} if it has a degree of an odd number.


\item A graph is said to be \textbf{emph{k-regular}} if the degree of each vertex is $k$. 
\item Every Graph has an even number of odd vertices.
\item A cubic graph is a graph where every vertex has degree three.
\end{itemize}


%---------------------------- %

\section*{Session 05 Graph Theory}
\begin{itemize}
\item Eulerian Path
\item Isomorphism
\item Adjacency matrices
\end{itemize}
Adjacency Matrices
\[ \left( \begin{matrix}
o & 1 & 0 & 1 & 1 \\ 
1 & 0 & 1 & 0 & 1 \\ 
1 & 1 & 0 & 1 & 1 \\ 
0 & 1 & 1 & 1 & 1 \\ 
1 & 1 & 0 & 1 & 0
\end{matrix} \right) \]

\section*{Functions}
\begin{itemize}
\item Domain of a Function
\item Range of a function
\item Inverse of a function
\end{itemize}
\begin{itemize}
\item one-one (surjective)
\item onto (bijective)
\end{itemize}
%------------------------------------------------%
\section*{Probability}
\subsection*{Binomial Coefficients}
\begin{itemize}
\item factorials 
\[ n! = (n)\times (n-1)\times(n-2) \times \ldots \times 1 \]
\begin{itemize}
\item $5! = 5 \times 4 \times 3 \times 2 \times 1 = 120 $
\item $3! = 3 \times 2 \times 1$
\end{itemize}
\item Zero factorial
\[ 0! =  1 \]
\end{itemize}
%---------------------------------------------- %

% \[P(A |B = \frac{P(A \cap B)}{P(B)})\]

The complement rule in Probability

$P(C^{\prime}) = 1- P(C)$

 

If the probability of C is $70 \%$ then the probability of $C^{\prime}$ is $30\%$
\section{Matrices}

What are the dimensions of the following matrix


\[ \left(
\begin{array}{cc}
a_1 & a_2 \\ 
b_1 & b_2
\end{array} \right)\left(
\begin{array}{cc}
c_1 & d_1 \\ 
c_2 & d_2
\end{array} \right) = \left(
\begin{array}{cc}
(a_1 \times c_1) + (a_2 \times c_2) & (a_1 \times d_1) + (a_2 \times d_2) \\ 
(b_1 \times c_1) + (b_2 \times c_2) & (b_1 \times d_1) + (b_2 \times d_2)
\end{array} \right) \]

\bigskip
\large{
\[ \left(
\begin{array}{cc}
1 & 3 \\ 
0 & 2
\end{array} \right)\left(
\begin{array}{cc}
1 & 2 \\ 
4 & 1
\end{array} \right) = \left(
\begin{array}{cc}
(1 \times 1) + (3 \times 4) & (1 \times 2) + (3 \times 1) \\ 
(0 \times 4) + (2 \times 4) & (0 \times 2) + (2 \times 1)
\end{array} \right) = \left(
\begin{array}{cc}
14 & 5 \\ 
8 & 2
\end{array} \right) \]
}

\[ \left(
\left(
\begin{array}{cc}
1 & 2 \\ 
4 & 1
\end{array} \right)
\begin{array}{cc}
1 & 3 \\ 
0 & 2
\end{array} \right) = ? \]


\end{document}

