\documentclass{beamer}

\usepackage{graphics}
\usepackage{amsmath}
\usepackage{amssymb}

\begin{document}

%Slide 1 Title Slide
\begin{frame}
\bigskip
{
\Huge
\[ \mbox{Powers and Logarithms}\]
}
{
\Large
\[ \mbox{kobriendublin.wordpress.com}\]
\[ \mbox{Youtube: StatsLabDublin}\]
}
\end{frame}


%-------------------------------------------------------%


\begin{frame}
\frametitle{Exercises}
\Large
Showing your workings, use the rules of indices and logarithms to give the following two expression in their simplest form.
\bigskip
\begin{itemize}
\item \textbf{Exercise 1}
\[ 4 \cdot 2^x - 2^{x+1} \]
\item \textbf{Exercise 2}
\[  \frac{\mbox{ln}(2) + \mbox{ln}(2^2) + \mbox{ln}(2^3)  + \mbox{ln}(2^4) + \mbox{ln}(2^5)  }  {\mbox{ln}(4)}  \]
\end{itemize}
\end{frame}

%-------------------------------------------------------%
\begin{frame}
\frametitle{Exercise 1}
\Large
\[ 4 \cdot 2^x - 2^{x+1} \]

\textbf{Remarks:}\\
\textit{(looking at the second term)}
\begin{itemize}
\item[1] Using the following rule
\[ a^b \cdot a^c = a^{(b+c)}  \] 
\item[2] Using this rule in reverse we can say
\[ 2^{x+1} = 2^x \cdot 2^1  = 2\cdot (2^x) \] 
\end{itemize}
\[ 4 \cdot 2^x - 2^{x+1} \mbox{   } = \mbox{   } (4 \cdot 2^x) -  (2\cdot 2^{x}) \]
\end{frame}

%-------------------------------------------------------%
\begin{frame}
\frametitle{Exercise 1}
\Large
\vspace{-0.9cm}

\textbf{Remarks:}
\begin{itemize}
\item[3] This expression is in the form 
\[ (a  \cdot b ) - ( c  \cdot b) \]
which can be re-expressed as follows 
\[ (a - c\sqrt{b} )  \cdot b \]
\end{itemize}
\[ (4 \cdot 2^x) -  (2\cdot 2^{x}) = (4-2)  \cdot 2^{x} \]
\[   = 2 \cdot 2^x = 2^{x+1}\]
\end{frame}
%-------------------------------------------------------%

\begin{frame}
\frametitle{Exercise 2}
\Large

\[  \frac{\mbox{ln}(2) + \mbox{ln}(2^2) + \mbox{ln}(2^3)  + \mbox{ln}(2^4) + \mbox{ln}(2^5)  }  {\mbox{ln}(4)}  \]
Useful Rule of Logarithms
\[  \mbox{ln}(a^b)  = b\cdot \mbox{ln}(a)  \]
\[  \frac{\mbox{ln}(2) + 2 \cdot \mbox{ln}(2) + 3 \cdot\mbox{ln}(2)  + 4 \cdot \mbox{ln}(2) + 5 \cdot \mbox{ln}(2)  }  {\mbox{ln}(4)}  \]
\end{frame}
%--------------------------------------%
\begin{frame}
\frametitle{Exercise 2}
\Large
Adding up all the terms in the numerator

\[  \frac{1\cdot\mbox{ln}(2) + 2 \cdot \mbox{ln}(2) + 3 \cdot\mbox{ln}(2)  + 4 \cdot \mbox{ln}(2) + 5 \cdot \mbox{ln}(2)  }  {\mbox{ln}(4)} \]  \[= \frac{15 \cdot \mbox{ln}(2) }{\mbox{ln}(4)} \]


\end{frame}

%--------------------------------------%
\begin{frame}
\frametitle{Exercise 2}
\Large
Our expression has now simplified to 
\[\frac{15 \cdot \mbox{ln}(2) }{\mbox{ln}(4)} \]

We can simplify the denominator too

\[ \mbox{ln}(4) =  \mbox{ln}(2^2) = 2 \cdot \mbox{ln}(2) \]


\end{frame}

%--------------------------------------%
\begin{frame}
\frametitle{Exercise 2}
\Large
Our expression has now simplified to 
\[\frac{15 \cdot \mbox{ln}(2) }{\mbox{ln}(4)} = \frac{15 \cdot \mbox{ln}(2) }{2 \cdot \mbox{ln}(2)} \]

We can divide above and below by $\mbox{ln}(2)$ to get our final answer


\[ \frac{15 \cdot \mbox{ln}(2) }{2 \cdot \mbox{ln}(2)} = \frac{15}{2} = 7.5 \]

\end{frame}

\begin{frame}
The End
\end{frame}

\end{document}