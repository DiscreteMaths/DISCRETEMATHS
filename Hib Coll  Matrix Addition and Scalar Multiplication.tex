\documentclass[a4]{beamer}
\usepackage{amssymb}
\usepackage{graphicx}
\usepackage{subfigure}
\usepackage{newlfont}
\usepackage{amsmath,amsthm,amsfonts}
%\usepackage{beamerthemesplit}
\usepackage{pgf,pgfarrows,pgfnodes,pgfautomata,pgfheaps,pgfshade}
\usepackage{mathptmx}  % Font Family
\usepackage{helvet}   % Font Family
\usepackage{color}

\mode<presentation> {
 \usetheme{Default} % was Frankfurt
 \useinnertheme{rounded}
 \useoutertheme{infolines}
 \usefonttheme{serif}
 %\usecolortheme{wolverine}
% \usecolortheme{rose}
\usefonttheme{structurebold}
}

\setbeamercovered{dynamic}

\title[R Tutorials]{MA4413 Statistics for Computing \\ {\normalsize MA4413 Lecture 7A : Review}}
\author[Kevin O'Brien]{Kevin O'Brien \\ {\scriptsize kevin.obrien@ul.ie}}
\date{January 2013}
\institute[Twitter: @kobriendublin]{Dept. of Mathematics \& Statistics, \\ University \textit{of} Limerick}


\renewcommand{\arraystretch}{1.5}
%------------------------------------------------------------------------%


% - Sample Space (ready)
% - Conditional Probability
% - Independent Events (ready)
% - Expected Value (ready)
% - Variance V(X) (ready)
% - Combinations (ready)
% - Binomial distribution (ready)
% - Poisson distribution (ready)
% - MB 3 Ready
% - Standardization Formula (Normal Distribution) (ready)
% - Complement and Symmetry (ready)

\begin{document}


%--------------------------------------------------------------------------------------%
\frame{
\frametitle{Binomial Distribution (1)}
\begin{itemize}
\item Identify the event that can considered the `success'.
\item (Remark : The success is usually the less likely of two complementary events.)
\item Determine the probability of a success in a single trial $p$.
\item Determine the number of independent trials $n$.
\end{itemize}

}
%--------------------------------------------------------------------------------------%
\frame{
\frametitle{Binomial Distribution (2)}
The probability of exactly k successes in a binomial experiment B(n, p) is given by
\[ P(X=k) = P(k \mbox{ successes }) = \;^nC_k  \times p^{k} \times (1-p)^{n-k}\]
Remark: This formula will be given tomorrow.
}

%--------------------------------------------------------------------------------------%
\frame{
\frametitle{Binomial Distribution (3)}

\begin{itemize}

\item Suppose we have a biased coin which yields a head only $48\%$ of the time.
\item Is this a binomial experiment?  why?
\item What is the probability of 4 heads in 7 throws?
\end{itemize}


}
%--------------------------------------------------------------------------------------%
\frame{
\frametitle{Binomial Distribution (4)}

\begin{itemize}
\item X: Number of heads thrown
\item $n$ : number of independent trials (i.e. 7)
\item $k$ : Number of successes (numeric value)
\begin{itemize}
\item Here $k$ is 4
\item Number of failures is $n-k  =3$
\end{itemize}
\item $p$ : probability of a success. (i.e. 0.48)
\item $1-p$ : probability of a failure (i.e. 0.52)
\end{itemize}

}
%--------------------------------------------------------------------------------------%
\frame{
\frametitle{Binomial Distribution (5)}

\[ P(X=4) = P(4 \mbox{ successes }) = \;^7C_4  \times (0.48)^{4} \times (0.52)^{3}\]

\bigskip

\[ P(X=4) = 35 \times 0.05308 \times  0.14061 =  \alert{0.2612} \]

Remark : must show workings.
}

%---------------------------------------------------------------------------%
\frame{
\frametitle{Poisson Distribution(1)}
The probability that there will be $k$ occurrences in a \textbf{unit time period} is denoted $P(X=k)$, and is computed as follows.
\Large
\[ P(X = k)=\frac{m^k e^{-m}}{k!} \]
\normalsize
This formula will be given tomorrow.
}
%---------------------------------------------------------------------------%
\frame{
\frametitle{Poisson Distribution(2)}
Given that there is on average 4 occurrences per day, what is the probability of one occurrences in a given day? \\ i.e. Compute $P(X=1)$ given that $m=4$
\Large
\[ P(X = 1)=\frac{4^1 e^{-4}}{1!} \]
\normalsize

The equation reduces to
\[ P(X = 1)=4 \times e^{-4} = \alert{0.07326} \]
}
%---------------------------------------------------------------------------%
\frame{
\frametitle{Poisson Distribution(3)}
What is the probability of one occurrences in a six hour period ? \\ i.e. Compute $P(X=1)$ given that $m=1$
\Large
\[ P(X = 1)=\frac{1^1 e^{-1}}{1!} \]
\normalsize
\begin{itemize}

\item $1!$ = 1
\end{itemize}
The equation reduces to
\[ P(X = 1) = e^{-1} = \alert{0.3678}\]
}
%------------------------------------------------------------------------%
\frame{
\begin{table}[ht]
\frametitle{Find $ P(Z \geq 1.27)$}
\vspace{-1.5cm}
%\caption{Standard Normal Distribution } % title of Table
\centering % used for centering table
\begin{tabular}{|c|| c c c c c c|} % centered columns (4 columns)
\hline %inserts double horizontal lines
& \ldots & \ldots & 0.06 &0.07&0.08&0.09 \\
%heading
\hline \hline% inserts single horizontal line
\ldots & \ldots & \ldots &\ldots& \ldots &\ldots&\dots \\ % inserting body of the table
1.0 & \ldots & \ldots &0.1446& 0.1423 &0.1401&0.1379 \\ % inserting body of the table
1.1 & \ldots & \ldots&0.1230& 0.1210 &0.1190&0.1170 \\ % inserting body of the table
1.2 & \ldots & \ldots&0.1038 & \alert{0.1020} & 0.1003&0.0985\\
1.3 & \ldots & \ldots &0.0869& 0.0853 &0.0838&0.0823 \\ % inserting body of the table
\ldots & \ldots &\ldots&\ldots & \ldots &\ldots&\ldots\\
\hline %inserts single line
\end{tabular}
\end{table}

Remark : Murdoch Barnes Table 3 will be given in tomorrow's exam.
}
%------------------------------------------------------------------------%
\frame{
\frametitle{The Standardization Formula}
\begin{itemize}
\item Suppose that mean $\mu = 105 $ and that standard deviation $\sigma = 8$.
\item What is the Z-score for $x_o = 117$?
\[
z_{117} = {x_o - \mu \over \sigma} = {117 - 105 \over 8} = {12 \over 8} = 1.5
\]
\item Therefore $z_{117} = 1.5$
\item Remark: $P(X \geq 117) = P(Z \geq 1.5)$.
\end{itemize}
}
%-----------------------------------------------------%
\begin{frame}
\frametitle{Complement and Symmetry Rules}
\begin{itemize}
\item \textbf{Complement Rule}: \[ P(Z \leq k) = 1-P(Z \geq k) \] for some value $k$
\item Alternatively $ P(Z \geq k) = 1-P(Z \leq k) $
\item \textbf{Symmetry Rule}: \[ P(Z \leq -k) = P(Z \geq k) \] for some value $k$
\item Alternatively $ P(Z \geq -k) = P(Z \leq k) $
\end{itemize}
\end{frame}

%-----------------------------------------------------%
\begin{frame}
\frametitle{Complement and Symmetry Rules}
\textbf{Complement Rule}

\begin{itemize}
\item $P(Z \leq 1.27) = 1-P(Z \geq 1.27) $
\item $ P(Z \geq 1.27) = 1-P(Z \leq 1.27) $
\end{itemize}

\bigskip
\textbf{Symmetry Rule}:
\begin{itemize}
\item $ P(Z \leq -1.27) = P(Z \geq 1.27) $
\item $ P(Z \geq -1.27) = P(Z \leq 1.27) $
\end{itemize}

Complement rule and Symmetry rule can be used in conjunction.
\end{frame}




%-----------------------------------------------------%
\begin{frame}
\frametitle{Complement and Symmetry Rules}

For a normally distributed random variable with mean $\mu = 1000$ and standard deviation $\sigma = 100$, compute $P(X \geq 873)$.

\begin{itemize} \item First, find the Z-value using the standardization formula.
\[
z_{873} = {x_o - \mu \over \sigma} = {873 - 1000 \over 100} = {-127 \over 100} = -1.27
\]
\item We can say $P(X \geq 873) = P(Z \geq -1.27)$.
\item Use complement rule and symmetry rule to evaluate  $P(Z \geq -1.27)$.
\item $ P(Z \geq -1.27) = P(Z \leq 1.27) = 1 - P(Z \geq 1.27) $  = 1 - 0.1020 = \alert{0.8980}.
\end{itemize}
\end{frame}
%------------------------------------------------------------------------%


%-----------------------------------------------------%
\begin{frame}
\frametitle{Interval Probability}
\begin{itemize}
\item We are often interested in the probability of being inside an interval, with lower bound $L$ and upper bound $U$.
\item It is often easier to compute the probability of the complement event, being outside the interval.
\[ P( \mbox{Inside} ) = 1 - P( \mbox{Outside} )  \]

\item Being outside the interval is the conjunction of being too low and too high.
\[ P( \mbox{Outside} ) = P( \mbox{Too Low} ) +  P( \mbox{Too High} ) \]

\item Therefore we can say
\[ P( \mbox{Inside} ) = 1- [P( \mbox{Too Low} ) +  P( \mbox{Too High} )] \]
\item $P( \mbox{Too Low} )$ = $P( X \leq L)$
\item $P( \mbox{Too High} )$ = $P( X \geq U)$
\end{itemize}
\end{frame}
\end{document}




