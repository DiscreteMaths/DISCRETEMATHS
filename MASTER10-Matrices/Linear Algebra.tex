\documentclass[12pt, a4paper]{article}
\usepackage{natbib}
\usepackage{vmargin}
\usepackage{graphicx}
\usepackage{epsfig}
\usepackage{subfigure}
%\usepackage{amscd}
\usepackage{amssymb}
\usepackage{amsbsy}
\usepackage{amsthm, amsmath}
%\usepackage[dvips]{graphicx}

\renewcommand{\baselinestretch}{1.8}

% left top textwidth textheight headheight % headsep footheight footskip
\setmargins{3.0cm}{2.5cm}{15.5 cm}{23.5cm}{0.5cm}{0cm}{1cm}{1cm}

\pagenumbering{arabic}


\begin{document}
\author{Kevin O'Brien}

\subsection{Matrices}
\[
 A =
 \begin{bmatrix}
 a_{11} & a_{12} & \cdots & a_{1n} \\
 a_{21} & a_{22} & \cdots & a_{2n} \\
 \vdots & \vdots & \ddots & \vdots \\
 a_{m1} & a_{m2} & \cdots & a_{mn}
 \end{bmatrix}\]
 
\subsection*{Augmented Matrices}
Given the matrices $A$ and $B$, where

\[
A =
  \begin{bmatrix}
    1 & 3 & 2 \\
    2 & 0 & 1 \\
    5 & 2 & 2
  \end{bmatrix}
, \quad
B =
  \begin{bmatrix}
    4 \\
    3 \\
    1
  \end{bmatrix},
\]

the augmented matrix ($A|B$) is written as

\[
(A|B)=
  \left[\begin{array}{ccc|c}
    1 & 3 & 2 & 4 \\
    2 & 0 & 1 & 3 \\
    5 & 2 & 2 & 1
  \end{array}\right].
\]

This is useful when solving systems of linear equations.


\begin{eqnarray} \nonumber
x + 2y + 3z = 0 \\ \nonumber
3x + 4y + 7z = 2 \\ \nonumber
6x + 5y + 9z = 11 \nonumber
\end{eqnarray}

the coefficients and constant terms give the matrices
\[
A =
\begin{bmatrix}
1 & 2 & 3 \\
3 & 4 & 7 \\
6 & 5 & 9
\end{bmatrix}
, \quad
B =
\begin{bmatrix}
0 \\
2 \\
11
\end{bmatrix},
\]

and hence give the augmented matrix
\[
(A|B) =
  \left[\begin{array}{ccc|c}
1 & 2 & 3 & 0 \\
3 & 4 & 7 & 2 \\
6 & 5 & 9 & 11
  \end{array}\right]
\]

\[\left[\begin{array}{ccc|c}
1 & 0 & 0 & 4 \\
0 & 1 & 0 & 1 \\
0 & 0 & 1 & -2 \\
  \end{array}\right]\]

\subsection*{Reduced Row Echelon Form}

Specifically, a matrix is in row echelon form if
\begin{itemize}
\item All nonzero rows (rows with at least one nonzero element) are above any rows of all zeroes (all zero rows, if any, belong at the bottom of the matrix).
\item The leading coefficient (the first nonzero number from the left, also called the pivot) of a nonzero row is always strictly to the right of the leading coefficient of the row above it.
\item All entries in a column below a leading entry are zeroes (implied by the first two criteria).[1]
\end{itemize}
\[\left[ \begin{array}{ccccc}
1 & a_0 & a_1 & a_2 & a_3 \\
0 & 0 & 1 & a_4 & a_5 \\
0 & 0 & 0 & 1 & a_6
\end{array} \right]\]
\newpage
\subsection*{Determinant of a $3 \times 3$ matrix}
\[ |A| = a_{11} \left|\begin{array}{cc}
a_{22} & a_{23} \\
a_{32} & a_{33} \\
\end{array}  right| - a_{12} \left| \begin{array}{cc}
a_{21} & a_{23} \\
a_{31} & a_{33} \\
\end{array} right| + a_{13} \left|\begin{array}{cc}
a_{21} & a_{22} \\
a_{31} & a_{32} \\
\end{array}  right| \]
%--------------------------------------------------%
\newpage
\subsection*{Fundamental Theorem of Linear Algebra}
\begin{itemize}
\item The rank of a matrix is $n$.
\item The determinant of the matrix is not zero.
\item The matrix is invertible.
\item The rows are linearly independent
\item There exists a solution to a set of Linear equation based on the matrix
\end{itemize}
%--------------------------------------------------%
\subsection*{Diagonalization}
\[ A = PDP^{-1} \]

By extension
\[ D = P^{-1}AP\]
%--------------------------------------------------%
\newpage
\subsection*{Systems of Linear Equations}

\[ \left(\begin{array}{ccc}
a_{11} & a_{12} &a_{13} \\
a_{21} & a_{22} &a_{23} \\
a_{31} & a_{32} &a_{33} \\ \right)
\left(\begin{array}{c}
x_{1} \\
x_{2} \\
x_{3} \\ \right) = \left(\begin{array}{c}
b_{1} \\
b_{2} \\
b_{3} \\ \right)

\]
\begin*{eqnarray}
a_{11}x_{1} + a_{12}x_{2} + a_{13}x_{3} = b_1 \\
a_{21}x_{1} + a_{22}x_{2} + a_{23}x_{3} = b_2 \\
a_{31}x_{1} + a_{32}x_{2} + a_{33}x_{3} = b_3 \\
\end{eqnarray}


\end{document}
