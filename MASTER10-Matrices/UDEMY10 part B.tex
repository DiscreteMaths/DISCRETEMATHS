\documentclass{beamer}

\usepackage{amsmath}

\begin{document}
\begin{frame}
\tableofcontents
\end{frame}
\section{Counting methods and probability}



% ClS102 Mathematics tor computing volume 2
%-----------------------------------------------------------------------------------------------------------------------------%
% Page 81
% Section 5Udemy 10.2 Matrix algebra
\begin{frame}
\frametitle{Matrix algebra}
Learning objectives for this section
\begin{itemize}
\ item  Writing a system of linear equations as a matrix equation
\ item The rules for when two matrices can be added and subtracted
\ item The rules for when two matrices can be multiplied
\ item Adding, subtracting and multiplying matrices of suitable sizes
\ item The identity matrices In and their properties
\ item Interpreting powers of the adiacency matrix of a graph/digraph
\end{itemize}
\end{frame}
%-----------------------------------------------------------------------------------------------------------------------------%
% Page 80
\begin{frame}
Definition 5.7 Let A and B be two $m \times n$ matrices. Then A = B ifand
only ifeach entry 0fA is equal to the corresponding entry of B.
Moreformally ifA = (am) and B = (bw), then A = B ifa,-J = bid
forall1§1l§mand1 §j §n.
5.2.1 Addition and subtraction of matrices
Definition 5.8 Let A and B be two m >< n matrices. Then A + B is the
m. >< n matrix obtained by adding each entry of A to the corresponding
entry of B.
Moreformally ifA = (am) and B = (hm), then A + B is the m >< n.
matrix given by A + B = (c, where cw = aw -i- b,,_, for all
lgigmandlgjgn.
Note that in order to add together two matrices they must have the
same size.
Examp|e5.8LetA=(; Z 7é)andB=(g 7} §).Then
n-I
IN
A+B:( —1)+(0—14):(1 13)
3 4 0 23 53‘
\end{frame}
%-----------------------------------------------------------------------------------------------------------------------------%
% Page 80
\begin{frame}
\frametitle{Subtracting Matrices}
Subtraction of matrices is defined in a similar manner and can also
be performed only when the two matrices are of the same size.
Examp|e5.9LetA=(; 2 ’é)andB=(g *1 :>.Then
A_B_ _1_ o—14>_ 1 35)
‘340 23-1-3‘
>-I
-IQ
80

\end{frame}
%-----------------------------------------------------------------------------------------------------------------------------%
% Page 81
\begin{frame}

5.2.2
5.2.3
Multiplication of a matrix by a constant
Definition 5.9 Let A be an m >< 7L matrix and r 6 R. Then TA is the
m >< n matrix obtained by multiplying each entry in A by r.
Equivalently, ifA = (aw-) then TA = (cw) where cm = ram for all
lgigmandlgjgn.
Example 5.10 LetA = ( é Z T3) ) and 7‘ = e2. Then
—2 e4 2
*2A‘(—e —s 0)‘

%-----------------------------------------------------------------------------------------------------------------------------%
% Page 81
\begin{frame}
\frametitle{Multiplication of two matrices}

We next define a rule for multiplying two matrices together. To
motivate this definition, we consider the system of equations
1L'1+2.’L'2 = 1
3.’E1’.'I7g = 2.
We can write this system of equations as a matrix equation involving
two 2 >< 1 matrices:
LIJ1 + 21'; _
3x1 — 1'; _
We would like to rewrite this matrix equation in the form Ax : b,
where
/_\
;/
/_\
N
;/
I\I>—\
§/
/$
I\)>—\
;/
_ 1 2 _ .I?1 _
A _ 3 *1 , x - at and b _
To do this we need to define matrix multiplication in such a way that
_ _ _ 1 2 901 _ 1r1+2ac2
)~'~*>~(3 »1)(a)~(3W)e
Definition 5.10 Let A = (am) be an m >< n matrix and B = (bu) be an
n >< p matrix. Then the matrix product AB = (rm) is the m >< p
matrix defined by
/_\
l\J>—\
11
T:.j I (li,1b1,; + f1i,2l12,] + » ~ + f1t.”l7~,,j I é (1i,k:bk.j
k-=1
f0rall1§i§mand1§j§p.
Note that in order to construct the product AB we need the number
of columns of A to be equal to the number of rows of B.
\end{frame}
%-----------------------------------------------------------------------------------------------------------------------------%
% Page 82
\begin{frame}
Example 5.11 A matrix multiplication
LetA= 1 andB= 5 1 .
2 2 3
The product AB is defined as A has 2 columns and B has 2 rows.
The resulting matrix will be a 3 >< 2-matrix as A has 3 rows and B
O>I=>~
IQ
Matrix algebra
81



5.2.4
\end{frame}
%-----------------------------------------------------------------------------------------------------------------------------%
% Page 82
\begin{frame}
(1><5)+(2><2) (l><1)+(2><3)
AB: (4><5)+(1><2) (4><1)+(1><3) =
has 2 columns:
((O><5)+(2><2) (0><1)+(2><3)
L;
/‘X
I\)
)J>I\7©
C'\\1Q
L)
The product BA is not defined in this example as B has 2 columns
while A has 3 rows.
\end{frame}
%-----------------------------------------------------------------------------------------------------------------------------%
% Page 82
\begin{frame}
To obtain the entry in the ith row and jth column of AB we
multiply together the corresponding terms in the ith row of A and
the jth column of B, and then add together the resulting products.
Powers of square matrices
A matrix is said to be square if it has the same number of rows as it
has columns. Square matrices are special in that they are the only
matrices which can be multiplied by themselves.
Example 5.12 A square matrix multiplied by itself
Consider the 2 >< 2-matrix
2 0
i_(_13)
Then A can be multiplied by itself:
r“=(~?3)(~?3)=(i 3)
Note that the size of AA is the same as the size of A. In fact, if A
and B are two square matrices of the same size, the size of AB is
the same as the size of A (and the size of B). We can thus define
powers of square matrices:
Definition 5.11 Let n and lc be two positive integers and suppose that A
is a square n >< n-THGITDC. We define
A1 := A,
and for k Z 2 we define
Ak 1: AA*"1.
Example 5.13 The powers of a square matrix
Consider the 2 >< 2-matrix
A=(~?3)
from Example 5.12 above. We have
Z _ _ 4 0
A _ AA _ ( _5 9 ) .
We can thus compute
~=A#=(_i2)(_§3)=(_£ 5)»
s2



Further,
4 _ 3 _ 2 0 8 O _ 16 0
A_AA_(—13)<—19 27 _ —65 81 '
In this way we can compute all positive powers of the matrix A.
Walks in graphs and powers of adjacency matrices
Definition 5.12 A walk in a graph G is an alternating sequence of
vertices and edges of the form
’v1@1'v2@2’vs ~ 4 - l3k—1UIc»
where e; is an edge joining vi to um. The vertices and edges in the
walk are not necessarily distinct.
Note that when the graph is simple, we may list the walk without the
edges,
’U1'l)gU3 , . . Uk,
as there is a maximum ofone edge between any pair of vertices in a
simple graph.
\end{frame}
%-----------------------------------------------------------------------------------------------------------------------------%
% Page 83
\begin{frame}
If you compare this definition to the definition of a path in Section
5.2.1 of Volume 1, you will see that a path in a graph is a walk in
which all vertices (and all edges) are distinct.
Definition 5.13 A directed walk in a digraph D is an alternating
sequence of vertices and arcs of the form
1'1@1'v2@2@'3 - - - €k—1'Uk»
where e, is an are from vi to vm. The vertices and arcs in the directed
walk are not necessarily distinct.
\end{frame}
%-----------------------------------------------------------------------------------------------------------------------------%
% Page 83
\begin{frame}
A directed path in a digraph is a directed walk in which all vertices
(and all arcs) are distinct.
Definition 5.14 The length of a walk in a graph is the number of edges
in it. The length of a directed walk in a digraph is the number of
arcs in it.
\end{frame}
%-----------------------------------------------------------------------------------------------------------------------------%
% Page 83
\begin{frame}
Suppose you want to find the number of walks of length 1 between
two vertices vi and vj in a simple (di) graph G with vertices
1:1,i1;, . . . ,u,,. Such a walk is of the form 1116177] where e is an edge
(arc) from v, to vj. The number of such walks is thus the number of
choices of e. If A = (aid) denotes the adjacency matrix of G with
respect to the natural ordering of vertices, then the entry aid is the
number of edges (arcs) from v, to vj, that is aw is the number of
walks of length 1 from 2;, to 11 J-.
Example 5.14 The number of walks of length 2 in a graph
Consider the following graph G.
I/1 112
[F
U; 1,4
Matrix algebra
83

%-----------------------------------------------------------------------------------------------------------------------------%
% Page 84
\begin{frame}
The graph G has adjacency matrix
O0»->-\O
>-\O>~Q>-I
@>—\®>—\>—\
DO»-¢OC>
OOO>—\O
A:
with respect to the natural ordering of vertices: v1, v2, 113, 04, U5.
The entry M4, 3) = 1., so there is precisely one walk of length 1 from
114 to U3 as can be easily seen from the picture of the graph. The
entry A(2, 4) = 0, so there is no walk of length 1 from 1/2 to 714 which
can also be seen from the picture.
The adjacency matrix of a graph is a square matrix, so we can
compute powers of it. Let us compute A2 for our graph G:
A2: = .
O0»-1»-1O
>-lo»-low
@)—l@>A>—\
<D<D>—\O©
CDGQPHQ
 \1
{TX
>—\>—\>—\>fll\J
O»-\>-\o;>-\
>—\CDUJ>fl>d
©>—*®>—-\>—\
>-\o>-low

\end{frame}
%-----------------------------------------------------------------------------------------------------------------------------%
% Page 84
\begin{frame}
Looking at the picture of G, there is precisely 1 walk of length 2
between v1 and 1/2, namely the path 1:1 1131/2; we note that entry
A2(1, 2) = 1.
There is no walk of length 2 between v4 and v5; we note that entry
A2(4. 5) = 0.
There are three walks of length 2 between v2 and 1:2, namely 'U2U1'U2,
'Ug'U3'U2 and '1.‘2’U5'U2; we note that entry A2(2, 2) = 3.
We leave it for you to check that A2(i, j) is the number of walks of
length 2 between vi and vj for all other values of 11 and j also.
To understand why this is so, note that any walk of length 2 from 11,
to vj is made up of a walk of length 1 from vi to some intermediate
vertex um followed by a walk of length 1 from um to '0]-. By the
Multiplication Principle the number of such walks via um is the
number of walks of length 1 from 111 to um, namely A(v1,m), times the
number of walks of length 1 from um to va, namely Mm. j). Using
the Addition principle, we get all possible walks of length 2 from iv,
to /0] by just adding the result for every possible intermediate vertex
um. Hence the number of walks of length 2 from v, to vj is
A(i,1)A(1,j) + Mi, 2)A(2,j) + . .. + A(i_5)A(5.j),
and comparing this sum with Definition 5.10, this is exactly the
entry in the ith row and jth column of A2, that is A2(i,j).
It is possible to generalise the result of Example 5.14 to walks of any
length by using induction. You can find this proof in Section 11.3 in
Epp. You will not be asked to prove the result in the examination,
but you must know how to compute the number of walks between
any pair of vertices in a simple graph:
Theorem 5.1 Let G be a simple graph with vertices 01,112, . . . gun, and let
A be the adjacency matrix of G with respect to the natural ordering of
the vertices, then for all positive integers k, the number of walks of
length k in Gfrom 11,- to 1:] is A"‘(1',j).
84



If a graph is simple, it has no loops, hence a walk of length 2
between two distinct vertices will be a path of length 2. Thus we get
the following result:
Corollary 5.2 Let G be a simple graph with vertices 1/'1, 2:2, . . . , um and
let A be the adjacency matrix of G with respect to the natural ordering
of the vertices, then the number of paths of length 2 in G from vi to vi
where i 9‘ j is A2(i,_j).
The result for digraphs is similar:
Theorem 5.3 Let D be a digraph with vertices 1/1, 1:2, . . . gun, and let A
be the adjacency matrix of G with respect to the natural ordering of
the vertices, then for all positive integers k, the number of directed
walks oflength k in Dfrom 1;; to vj is A"'(1l,j).
Example 5.15 The number of walks in a digraph
Consider the following digraph D.
v1 112
l/3
173 ’l 4
The digraph D has adjacency matrix
A = K
with respect to the natural ordering of vertices: 111, 212, 113,114, 1/'5.
@©®>—\@
@@>—\@@
@@@(D>4
C><Dr-\<DC>
CD@@>—\Q
Let us compute A2 for our digraph D:
Hence there is just one directed walk of length 2 from 1:1 to 1:4 for
example, while there is no directed walk of length 2 from 113 to 112.
@@@>—\@
@@>—\@@
@@®®>A
@©>i®@
@@@>—\@
 >
Z?
@@>—\@@
©@@@>—\
@@®>—\@
©@@@>—\
@@>—\®@
Let us also compute A3 for our graph D:
A3 = AA2 =
©@@@>—\ ©@@>—\©
@@®>—\@ @®>—\@@
©@>—\@@ ©©@©>—\
@@®>—\@ @®>—\@@
©@@@>—\ @@@>—\@
L)
.
@®>—\@@
@@@@>—l
@©®>—\@
@@@@>—\
@@>—\@©
Li}
Matrix algebra
85



5.2.5
ClS102 Mathematics ior computing volume 2
Hence there is just one directed walk of length 3 from v; to v4 for
example, while there is no directed walk of length 3 from 1:2 to 1:3.
Rules of arithmetic for matrices
Most of the rules of arithmetic you are used to from your experience
with numbers hold also for matrices:
Rule 5.4 If A,B and C are matrices of the same size then
(A+B)+C=A+(B+C).
Rule 5.5 If A,B and C are matrices such that the products AB and BC
both exist (that is, the number of columns ofA is the same as the
number of rows ofB and the number of columns ofB is the same as
the number of rows of C), then
(AB)C = A(BC).
Rule 5.6 Suppose that A and B are matrices of the same size,
(a) IfC is a matrix such that CA and CB exist (that is, the number of
columns ofC is the same as the number of rows ofA and B), then
C(A+ B) = CA+CB.
(b) If D is a matrix such that AD and BD exist (that is, the number
of rows ofD is the same as the number cf columns ofA and B),
then
(A + B)D = AD + BD.
Rule 5.7 If A and B are matrices of the same size then
A + B = B + A.
Proofs of these rules are straightforward from the definitions of the
various arithmetic operations on matrices. You will not be required
to prove these rules in the examination, but knowing them can ease
computations with matrices, and you are therefore advised to learn
and use them.
It is important to realise that Rule 5.7 does not have a multiplicative
equivalent. Unlike multiplication of numbers, matrix multiplication
need not always be commutative, i.e. the two products AB and BA
may yield different matrices. You have already seen in Example 5.11
that the product BA is not necessarily defined even though AB can
be computed. Worse still, even when AB and BA are both defined,
they are not necessarily equal.
Example 5.16 Matrix multiplication is not commutative
If A and B are n >< n matrices, then we can calculate both the
product AB and the product BA. However, the two products AB
and BA more often than not field different matrices. To see an
example of this let
A=(é %)andB=(2 
86



5.2.6
Then
10
AB_(10)
and O
O
BA _ ( 1 1 ),
so clearly AB 9! BA in this case.
Identity matrices
Definition 5.15 The n >< n identity matrix L, is the n >< n matrix with
1’s along its diagonal and 0’s elsewhere. Thus L, = (rim) where
5, _ 1ifi=j
“'1' 0ifrt¢j.
Lemma 5.8 Let A be an m >< 11 matrix. Then
S0, for example,
©®>—\
@>—\@
>—\®@
Li
AL, = A = L,,A,
Proof. We show that ALL = A. To do this we have to show that AL,
has the same size as A and that the corresponding entries in AL,
and A are equal. Since A is m >< n and LL is n >< 71, we have ALL is
m >< n, Hence AL‘ has the same size as A. Let
A = ('1'1,.1) T 1r= <@.7>, Mn = rm - Then
1,
I12] = Z@@,k5k.r = flmlm = flu
r=1
since 5;w : 0 if lc 7/j and 5]-7] : 1. Thus AL, and A have the same
entries. Hence AI" = A.|
Example 5.17 Properties of identity matrices We illustrate Lemma 5.8 by
the 2 >< 3 matrix
3 2 O
A _ ( 1 O 1 )
Recall the two identity matrices
I3 = ( and I2 = 0 1
Then
12,4 = (
_ 1-3+0-1 1-2+0-0 1~o+0-1
’ 0<3+1-1 0-2+1-0 0»0+1<1
= ( =A.
CJ<D>—‘
O>—\ O>—\O
>—'@ >—‘®@
\_/ \ ,
/“\
>-rue
CJIQ
>-0
\_/
/3
>4
Q
\_/
>~UJ
©I\J
>~@
\_/
Matrix algebra
87



C|S102 Mathematics ior computing volume 2
Also
.413
/—\ /—\
>_\(.o >-\ >-it»
O ON
>_\ >—\<3
\_/
ooti
®>~®
>-10¢:
L2
3-1+2-0+0-0 3<0+2»1+O<0 3»O+2-0+
-1+0-O+1~0 1-0+0-1+1-O 1-O+0~0+1
< 
5.3 Exercises on Chapter 5
5.3.1 True/False questions
In each of the following questions, decide whether the given
statements are true or false.
1. The equation 1-1 + 212 — mt-3 = O is linear.
Let
A =
2.
3.
4.
5.
6.
7.
8.
9.
10. AZ=(
 >5 B=< >
@@>d
>—\@@
(1-)©l\l
(D >~
<3 DJ
O
>4
and C =
1.3
A has 2 leading entries.
A is in row echelon form.
B is in row echelon form.
BC : CB.
A + BC can be computed.
AC can be computed.
CA can be computed.
@©>—\
>d@®
\O®>J>
\_/
5.3.2 Longer exercises
Question 1
Use Gaussian elimination to find all solutions to the following
systems of equations.
(a)
88
211 + 1'2 = 2
at] — ac; = 1
$1 + Z12 = 1
O1
132
(245)



