\documentclass[a4paper,12pt]{article}
%%%%%%%%%%%%%%%%%%%%%%%%%%%%%%%%%%%%%%%%%%%%%%%%%%%%%%%%%%%%%%%%%%%%%%%%%%%%%%%%%%%%%%%%%%%%%%%%%%%%%%%%%%%%%%%%%%%%%%%%%%%%%%%%%%%%%%%%%%%%%%%%%%%%%%%%%%%%%%%%%%%%%%%%%%%%%%%%%%%%%%%%%%%%%%%%%%%%%%%%%%%%%%%%%%%%%%%%%%%%%%%%%%%%%%%%%%%%%%%%%%%%%%%%%%%%
\usepackage{eurosym}
\usepackage{vmargin}
\usepackage{amsmath}
\usepackage{graphics}
\usepackage{epsfig}
\usepackage{subfigure}
\usepackage{fancyhdr}
%\usepackage{listings}
\usepackage{framed}
\usepackage{graphicx}
\usepackage{amsmath}
\usepackage{chngpage}
%\usepackage{bigints}


\setcounter{MaxMatrixCols}{10}
%TCIDATA{OutputFilter=LATEX.DLL}
%TCIDATA{Version=5.00.0.2570}
%TCIDATA{<META NAME="SaveForMode" CONTENT="1">}
%TCIDATA{LastRevised=Wednesday, February 23, 2011 13:24:34}
%TCIDATA{<META NAME="GraphicsSave" CONTENT="32">}
%TCIDATA{Language=American English}

%\pagestyle{fancy}
%\setmarginsrb{20mm}{0mm}{20mm}{25mm}{12mm}{11mm}{0mm}{11mm}
%\lhead{MA4413} \rhead{Mr. Kevin O'Brien}
%\chead{Statistics For Computing}
%\input{tcilatex}

\begin{document}

% http://www.mathsisfun.com/combinatorics/combinations-permutations.html


\section*{Laws of Logarithms}
\subsection*{Basic Rules}
\begin{itemize}
\item[(1)] Multiplication inside the log can be turned into addition outside the log, and vice versa.
\[\mbox{log}_b(mn) = \mbox{log}_b(m) + \mbox{log}_b(n)\]

\item[(2)]  Division inside the log can be turned into subtraction outside the log, and vice versa.
\[\mbox{log}_b(m/n) = \mbox{log}_b(m) – \mbox{log}_b(n)\]

\item[(3)] An exponent on everything inside a log can be moved out front as a multiplier, and vice versa.
\[mbox{log}_b(m^n) = n \times \mbox{log}_b(m)\]
\end{itemize}

\end{document}
3) 