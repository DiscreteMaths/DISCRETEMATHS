\documentclass{article}
\usepackage{amsmath}
\usepackage{amssymb}
\usepackage{graphicx}
\begin{document}
\section{Set Theory}
\begin{enumerate}
\item The Universal Set $\mathcal{U}$
\item Union
\item Intersection
\item Set Difference
\item Relative Difference
\end{enumerate}

\newpage




\subsection*{Encoding and Decoding Functions (4.2)}


\subsection*{Onto Functions (4.2.2)}

\subsection*{One-to-One Functions (4.2.3)}


\subsection*{One-to-One Functions (4.2.3)}
$f(x)$, must be \emph{One-to-One} and \emph{Onto}



\subsection*{Exponential and Logarithmic Functions (4.3)}

The Laws of Logarithms
\begin{itemize}
\item
\item $log_b(x^y) = y \times log_b(x)$
\item
\item
\end{itemize}

\subsection*{Comparing the size of Functions (4.4)}

Using O-notations

\subsection*{Power Notation (4.4.2)}


\subsection*{Section 8 Exercises}
\begin{itemize}
\item $8^{\frac{1}{3}}$ Recall $a^{\frac{b}{c}} = a^{\frac{b}{c}}$
\item
\item
\end{itemize}


\section{Mathematics for Computing: Onsite Tutorial two}

\textbf{Today's Class}
\begin{itemize}
\item Chapter 3: Logic
\begin{itemize}
\item
\item
\end{itemize}
\item Chapter 4: Functions
\begin{itemize}
\item Inverse of a Function
\item One-to-One and Onto
\item Special Functions
\end{itemize}
\end{itemize}
\newpage
%--------------------------------------------%
\section{Chapter 3: Logic}

\textbf{2003 Question 3}\\
Let p, q be the following propositions:
\begin{itemize}
\item p : this apple is red, 
\item q : this apple is ripe.
\end{itemize}

Express the following statements in words as simply as you can:
\begin{itemize}
\item (i) $p \rightarrow q$
\item (ii) $p \wedge \neg q$.
\end{itemize}

 
Express the following statements symbolically:
\begin{itemize}
\item (iii) This apple is neither red nor ripe.
\item (iv) If this apple is not red it is not ripe.
\end{itemize}



\subsection{Logical Operations}

\begin{itemize}
\item Logical "AND" ($\wedge$)
\item Logical "OR" ($\vee$)
\item Logical "NOT" 
\end{itemize}
%--------------table
%\begin{tabular}{cccc}\hline
%p & q& P \vee q& p \wedge q  \\
%0& 0& 0& 0\\ 
%0& 1& 0& 1 \\ 
%1& 0& 0& 1 \\ 
%1& 1&1 &1  \\  \hline
%\end{tabular}
\newpage

%--------------------------------------------%
\subsection*{Logic Networks}
\begin{itemize}
\item AND Gates
\item OR Gates
\item NOT Gates
\end{itemize}

\newpage

%--------------------------------------------%
\section*{Chapter 4: Functions}

\subsection{Arrow Diagrams}

\begin{itemize}
\item Domain 
\item Co-Domain 
\item Range
\end{itemize}

\newpage
%--------------------------------------------%
\subsection{Boolean Functions and ordered $n-$tuples}


\begin{itemize}
\item Ordered Triples
\item Boolean Fucntions
\item
\end{itemize}

\subsection*{The Asbolute Value, Floor and Ceiling Functions}
\begin{itemize}
\item The Absolute Value Function
\item Floor
\item Ceiling
\end{itemize}



\begin{itemize}
\item
\item
\item
\end{itemize}

\subsection*{Power functions and Polynomials}


Consider the function f: Z-> Z defined by f(n) = 3n-1. Does this function have the onto property?
\[ax^2 + bx + c\]
%--------------------------------------------%
\subsection*{Summer 2003 Question 4}

\begin{tabular}{ccccccc}
 x & & & & & & \\ \hline
 f(x) & & & & & & \\ \hline
 g(x) & & & & & & \\ \hline
\end{tabular}

%--------------------------------------------%

\section*{Functions}
\begin{itemize}
\item Domain of a Function
\item Range of a function
\item Inverse of a function
\end{itemize}
\begin{itemize}
\item one-one (surjective)
\item onto (bijective)
\end{itemize}

%---------------------------------------------- %

% \[P(A |B = \frac{P(A \cap B)}{P(B)})\]

The complement rule in Probability

$P(C^{\prime}) = 1- P(C)$

 

If the probability of C is $70 \%$ then the probability of $C^{\prime}$ is $30\%$
\section{Matrices}

What are the dimensions of the following matrix


\[ \left(
\begin{array}{cc}
a_1 & a_2 \\ 
b_1 & b_2
\end{array} \right)\left(
\begin{array}{cc}
c_1 & d_1 \\ 
c_2 & d_2
\end{array} \right) = \left(
\begin{array}{cc}
(a_1 \times c_1) + (a_2 \times c_2) & (a_1 \times d_1) + (a_2 \times d_2) \\ 
(b_1 \times c_1) + (b_2 \times c_2) & (b_1 \times d_1) + (b_2 \times d_2)
\end{array} \right) \]

\bigskip
\large{
\[ \left(
\begin{array}{cc}
1 & 3 \\ 
0 & 2
\end{array} \right)\left(
\begin{array}{cc}
1 & 2 \\ 
4 & 1
\end{array} \right) = \left(
\begin{array}{cc}
(1 \times 1) + (3 \times 4) & (1 \times 2) + (3 \times 1) \\ 
(0 \times 4) + (2 \times 4) & (0 \times 2) + (2 \times 1)
\end{array} \right) = \left(
\begin{array}{cc}
14 & 5 \\ 
8 & 2
\end{array} \right) \]
}

\[ \left(
\left(
\begin{array}{cc}
1 & 2 \\ 
4 & 1
\end{array} \right)
\begin{array}{cc}
1 & 3 \\ 
0 & 2
\end{array} \right) = ? \]


\end{document}

