\documentclass{beamer}

\usepackage{amsmath}
\usepackage{amssymb}
\usepackage{framed}
\usepackage{graphicx}

\begin{document}

\begin{frame}
\frametitle{Spanning trees}
We say that a graph H is a subgraph of a graph G if

\begin{enumerate}
\item its vertices are
a subset of the vertex set of G, 
\item its edges are a subset of the edge set
of G, 
\item and each edge of H has the same end-vertices in G and H.
\end{enumerate} 
\end{frame}
%--------------------------------------------------------------------------%
\begin{frame}
\begin{description}
%Definition 3.3
\item[Definition]  If H is a subgraph of G such that V (H) = V (G), then H
is called a spanning subgraph of G. If H is a spanning subgraph
which is also a tree, then H is said to be a spanning tree of G.

\item[Example] In Figure 3.2, the graphs T1 and T2 are both spanning
trees of the graph G.
\end{description}
\end{frame}
%--------------------------------------------------------------------------%
\end{document}
