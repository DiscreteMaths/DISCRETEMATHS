\documentclass[12pt]{article}

\usepackage{amsmath}
\usepackage{amssymb}

%opening

\begin{document}

\begin{center}
\huge{Mathematics for Computing}\\
\LARGE{Functions}
\end{center}
\tableofcontents



\end{document}

Both oneone and onto
one one only
neither one ontoone or 
Arrow Diagram
onto functions (4:2:2)
Invertible Functions
one-to-one correspondence
f(x) =3x-5
f-1(x) = x+5/3
Domain and Co-domain
Polynomial Functions (4:1:5)
 Linear quadratic cubic and quintic
 1- x 3 is  best described as which type of function
Absolute Value Function
Laws of Logarithms
Exponentials
O-notation (4:4:1)
Power Functions (4:4:2)
log2(x)  + log2(y) 
options : log2(xy) log2(x+y) log2(xy) 
log2(x/y)
exp(ab)





http://www.analyzemath.com/college_algebra/problems_5.html
Let X = {1, 2, 3, 4} and Y={A,B,C,D}. Determine whether each relation is a function, with X -> Y.
(a) f = {(2, C), (1, D), (2, A), (3,B), (4, D)}

No:  Two different ordered pairs (2, C) and (2, A) in f have the same number 2 as their first coordinate

Let X = {1, 2, 3, 4}. Determine whether the following relation on X is a function.

(b) g = {(3, A), (4, B), (1, C)}

No The element 2 does not appear as the first coordinate in any ordered pair in g.
