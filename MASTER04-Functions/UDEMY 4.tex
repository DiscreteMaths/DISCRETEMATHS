\documentclass[12 pt]{article}

%opening
\title{Hibernia College}
\author{}
\usepackage{amsmath}
\begin{document}

\maketitle

% 1.1 Started 2010 Q3 (Propositions and Sets)
% 1.2 Started (Membership Tables)
% 1.3 READY - Membership Tables for Laws of Logic
% 1.4 ..
% 1.5 ..
% 1.6 ..
% 1.7 READY  - Contrapositives
% 1.8 Logic Networks - Started - Need Graphics

% 2.1 Arrow Diagram
% 2.2 Arrow Diagram 2004 Q3
% 2.3
% 2.4 Inverse of a Function (2010 Q4)
% 2.5 Floor and Ceiling Function
% 2.6 Floor and Ceiling Function 2004 Q3
% 2.7 READY Power Functions
% 2.8
% 2.9
% 2.10
% 2.11
% 2.12 Logarithms 

\newpage
\section*{Part 2 : Functions}







% \[ \mathcal{N} \in  \mathcal{Z} \in \mathcal{Q} \in \mathcal{R}]

\begin{itemize}
\item $\mathcal{Z}$ : Integers $ \{\ldots,-3,-2,-1,0,1,2,3, \ldots\}$
\item $\mathcal{Z}^{-}$ : Negative Integers $\{\ldots,-3,-2,-1\}$
\item $\mathcal{Z}^{+}$ : Positive Integers $\{0,1,2,3, \ldots\}$
\item $\mathcal{R}$ : Real Numbers
\end{itemize}

\begin{tabular}{|c|c|c|c|}
\hline  & Domain & Co-Domain  & Range  \\ 
\hline f(x) & $\mathcal{R}$ & $\mathcal{Z}$ & $\mathcal{Z}$ \\ 
\hline  g(x)& $\mathcal{R}$ & $\mathcal{R}$ & $\mathcal{R}^{+} \cup {0} $\\ 
\hline 
\end{tabular} 
f(x) is not one-to-one $f(2.6) = f(2.7)$.\\
g(x) is not one to one $g(2) =g(0)$\\
Are Co-Domain and Range Equal?\\
%--------------------------------------------------- %

%--------------------------------------------------- %
\subsection*{2.4 2009 Zone Question 4- Contd }

(c) Explain what it means for a function to be $O(x)$?
(c) Justifying your answer, say whether the function $
f(x)$ from part (b) is $O(n)$.

From book page 62:
\[ f(x) \leq 0.5 |x|\] (x$\geq 0)$
\newpage
%--------------------------------------------------- %


%--------------------------------------------------- %
%--------------------------------------------------- %

\subsection*{2.6 Functions}
\begin{itemize}
\item $\lceil x\rceil$ : Ceiling function
\item $\lfloor x\rfloor$ Floor Function
\item $\{x\}$: Fractional Part of a number
\end{itemize}

\begin{tabular}{|c|c|c|c|} \hline
Sample value x & Floor $\lfloor x\rfloor$ & Ceiling  $\lceil x\rceil$ & Fractional part $ \{ x \} $\\ \hline
12/5 = 2.4 &	2	&3&	2/5 = 0.4\\
2.7&	2&	3	&0.7\\
-2.7&	-3&	-2	&0.3\\
-2&	-2&	-2	&0\\\hline
\end{tabular} 



%--------------------------------------------------- %
%--------------------------------------------------- %
\newpage
\subsection*{2.8 Power Functions}
Let s be any positive rational number, then the function $f: \mathcal{R} \rightarrow \mathcal{R}$, which is defined
as \[ f(x) = x^s\] is called the \textbf{power function} of x with \textbf{exponent} s.\\

\bigskip 
\noindent \emph{Page 67 Question 9}
Without using your calculator, express each of the following as an integer or fraction (with exponent 1)
\begin{itemize}
\item $8^{1/3}$ = $(2^3)^{1/3} = 2$
\item $8^{-2/3}$ = $(2^3)^{-2/3} = 2^{-2} = {1 \over 4}$
\item $16^{1.25}$ = $(2^4)^{1.25}  = 2^5 = 32$ 
\item $32^{1.4}$ = $(2^5)^{1.4} = 2^7 = 128$
\item $32^{-0.6}$ = $(2^5)^{-0.6} = 2^{-3} =  {1 \over 8}$
\end{itemize}
\subsection{2.9}
\textbf{2008 Question 5}\\
Let $S$ be the set of names of students on a particular course in a college and
let $f$ be the function that counts the number of letters in a student’s name. So
if X is the name of an individual student then
f(X) = the number of letters in X where $f : X —> \mathcal{Z}$ and $X \in S$.\\
\bigskip

For example if Y is the name \emph{\textbf{Bruce Lee}} then $f(Y) = 8.$\\

(i) Given A is the name \emph{\textbf{Alexander Nevsky}}and B is the name \emph{\textbf{Leo Tolstoy}},
find f(A) and f(B).
\\
(ii) Give two different examples of a possible pre-image or ancestor of 6.
\\
(m) Say under what circumstances you think this function is one to one, justifying your answer.
\\
(iv) Say whether or not this function is onto, justifying your answer. \\

\textbf{comments}\\
This function is only one to one when there are no two or more
candidates with the same number of letters in their name. The function
could not be onto because there are no names with an infinite number (or even a very large number) of letters, whereas the co-domain is all of the
positive integers from 1 up to infinity. Thus the range and the co-domain
are not equal.








%--------------------------------------------------- %
%--------------------------------------------------- %
\subsection*{2.12 Laws of Logaritms}

\subsubsection*{ Laws of Exponents}
\emph{See Section 4.3.1, in particular Rule 4.18}

\[ exp_b(x) = b^x \]


{\Large
\[\mbox{Log}_2(128) = 7\]
\[\mbox{Log}_2(1/8) = -3\]
\[\mbox{Log}_2(\sqrt{2}) = -3\]
}
\begin{itemize}
\item $a^x = y$  $log_a(y) = x$

\item $e^x = y$  $ln(y)=x$

\item $log_a(x\times y) = log_a(x) + log_a(y)$

\item $log_a(\frac{x}{y}) = log_a(x) - log_a(y)$

\item $log_a(\frac{1}{x}) = - log_a(x)$

\item $log_a(a) = 1$

\item $log_a(1) = 0$
\end{itemize}



\end{document}
