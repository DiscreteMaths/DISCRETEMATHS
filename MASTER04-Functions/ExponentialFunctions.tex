
%===================================================================================================== %
\begin{frame}
\frametitle{Taylor series expressions}
It is possible to express the above functions as Taylor series:
\[\sinh x = x + \frac {x^3} {3!} + \frac {x^5} {5!} + \frac {x^7} {7!} +\cdots = \sum_{n=0}^\infty \frac{x^{2n+1}}{(2n+1)!}\]

\end{frame}
%===================================================================================================== %
\begin{frame}
\begin{itemize}
\item The function sinh x has a Taylor series expression with only odd exponents for x. 
\item 
Thus it is an odd function, that is, −sinh x = sinh(−x), and sinh 0 = 0.
\end{itemize}

\[\cosh x = 1 + \frac {x^2} {2!} + \frac {x^4} {4!} + \frac {x^6} {6!} + \cdots = \sum_{n=0}^\infty \frac{x^{2n}}{(2n)!}\]
\end{frame}

\begin{frame}
	\begin{itemize}
		
\item	On the y-axis, the value of $x$ is always 0. hence we would specify this line mathematically as $x=0$.
		
\item		Conversely, on the $x-$axis, the value of y is always 0. We specify this line as $y=0$.
		
\item		The intercept is the point on the plot where the function (i.e. the line or curves) crosses the y-axis.
		
\item		We might be interested in where a function crosses the x-axis. We will term these points as points of intersection ( reserve the word "intercept" for the y-axis, as described previously)
\end{itemize}	
\end{frame}
%===================================================================================================== %
\begin{frame}
		
		Consider the behaviour of the function \[ f(x) = \frac{1}{x} \]
		
		As values for $x$ get larger and larger, the corresponding values of the function get smaller and smaller.
		
\end{frame}
%===================================================================================================== %
\end{document}
