
\documentclass[a4paper,12pt]{article}
%%%%%%%%%%%%%%%%%%%%%%%%%%%%%%%%%%%%%%%%%%%%%%%%%%%%%%%%%%%%%%%%%%%%%%%%%%%%%%%%%%%%%%%%%%%%%%%%%%%%%%%%%%%%%%%%%%%%%%%%%%%%%%%%%%%%%%%%%%%%%%%%%%%%%%%%%%%%%%%%%%%%%%%%%%%%%%%%%%%%%%%%%%%%%%%%%%%%%%%%%%%%%%%%%%%%%%%%%%%%%%%%%%%%%%%%%%%%%%%%%%%%%%%%%%%%
\usepackage{eurosym}
\usepackage{vmargin}
\usepackage{amsmath}
\usepackage{amssymb}
\usepackage{graphics}
\usepackage{epsfig}
\usepackage{subfigure}
\usepackage{fancyhdr}

\setcounter{MaxMatrixCols}{10}
%TCIDATA{OutputFilter=LATEX.DLL}
%TCIDATA{Version=5.00.0.2570}
%TCIDATA{<META NAME="SaveForMode"CONTENT="1">}
%TCIDATA{LastRevised=Wednesday, February 23, 201113:24:34}
%TCIDATA{<META NAME="GraphicsSave" CONTENT="32">}
%TCIDATA{Language=American English}

\pagestyle{fancy}
\setmarginsrb{20mm}{0mm}{20mm}{25mm}{12mm}{11mm}{0mm}{11mm}
\lhead{Discrete Mathematics} \rhead{Functions} \chead{Session 4} %\input{tcilatex}

\begin{document}
\begin{verbatim}
4.1 What is a function?      
4.1.1 Arrow diagram of a function   
4.1.2 Boolean functions and ordered n-tuples   
4.1.3 Absolute value function   
4.1.4 Floor and Ceiling Functions    
4.1.5 Polynomial functions    
4.1.6 Equality of functions  

4.2 Functions with Special Properties    
4.2.1 Encoding and decoding functions    
4.2.2 Onto functions     
4.2.3 One-to-one functions    
4.2.4 Inverse functions     
4.2.5 One-to-one correspondence   

4.3 Exponential and Logarithmic functions    
4.3.1 Laws of exponents     
4.3.2 Logarithmic functions

4.4 Comparing the size of functions    
4.4.1 O-notation     
4.4.2 Power functions    
4.4.3 Orders of polynomial functions    
4.4.4 Comparing the exponential and logarithmic functions with the power functions
4.4.5 Comparison of algorithms 
\end{verbatim}


\newpage


%-------------------------------------------- %

\subsection{Properties of Functions}

%%- %%- \vspace{-1cm}
The function $f : \mathbb{R} \rightarrow \mathbb{Z}$ is defined by the rule:
\[f(x) = \bigg\lfloor  {x \over 2}\bigg\rfloor \]





%-------------------------------------------- %

\subsection{Properties of Functions}

%%- %%- \vspace{-2cm}
The function $f : \mathbb{R} \rightarrow \mathbb{Z}$ is defined by the rule:
\[f(x) = \bigg\lfloor  {x \over 2}\bigg\rfloor \]
What is the range of this function?



%-------------------------------------------- %

\subsection{Properties of Functions}

%%- %%- \vspace{-2cm}
The function $f : \mathbb{R} \rightarrow \mathbb{Z}$ is defined by the rule:
\[f(x) = \bigg\lfloor  {x \over 2}\bigg\rfloor \]
Is this function ``onto" ?



%-------------------------------------------- %

\subsection{Properties of Functions}

%%- %%- \vspace{-2cm}
The function $f : \mathbb{R} \rightarrow \mathbb{Z}$ is defined by the rule:
\[f(x) = \bigg\lfloor  {x \over 2}\bigg\rfloor \]
Is this function ``one-to-one" ?



%%- %%- \vspace{-2cm}
The function $f : \mathbb{R} \rightarrow \mathbb{Z}$ is defined by the rule:
\[f(x) = \bigg\lfloor  {x \over 2}\bigg\rfloor \]
Is this function invertible ?


%====================================================================%

\subsection{Even and Odd functions}

\textbf{Even Functions}\\
Then f is even if the following equation holds for all x and -x in the domain of f:

\[f(x) = f(-x), \,\]

Geometrically speaking, the graph face of an even function is symmetric with respect to the y-axis, meaning that its graph remains unchanged after reflection about the y-axis.
%Examples of even functions are |x|, x2, x4, cos(x), and cosh(x).


%================================================================= %

\subsection{Even and Odd functions}

\textbf{Odd Functions}\\
Let f(x) be a real-valued function of a real variable. Then f is odd if the following equation holds for all x and -x in the domain of f:

\[-f(x) = f(-x), \,\]
or

\[f(x) + f(-x) = 0. \,\]



\subsection{Even and Odd functions}
\begin{itemize}
\item \textbf{Important:} A function may be neither even nor odd.
\item Discussion with examples on Blackboard
\item Examples of Questions from Past Papers done on board
\end{itemize}



%====================================================================%

\subsection{Even and Odd Functions}


$F(x) = x^3 $is an example of an odd function.
Again, let f(x) be a real-valued function of a real variable. Then f is odd if the following equation holds for all x and -x in the domain of f:[2]

\[-f(x) = f(-x), \,\]
or

\[f(x) + f(-x) = 0. \,\]

%====================================================================%

\subsection{Even and Odd Functions}

Geometrically, 
the graph of an odd function has rotational symmetry with respect to the origin, meaning that its graph remains unchanged after rotation of 180 degrees about the origin.
Examples of odd functions are x, x3, sin(x), sinh(x), and erf(x).

%====================================================================%



\newpage
%---------------------------------------%
\subsection*{Absolute Value Function}


%--------------------------------------%
\section*{Logarithms}


%------------------------------------------------------------------------%
\section{Sets of Numbers}

\begin{itemize}
\item $\mathbb{Z}$ Set of all integers
\item $\mathbb{Q}$ Set of all rational numbers
\item $\mathbb{R}$ Set of all real numbers
\end{itemize}


\begin{itemize}
\item $\mathbb{Z}^{+}$ Set of all positive integers
\item $\mathbb{Z}^{-}$ Set of all negative integers
\item $\mathbb{R}^{+}$ Set of all positive real numbers
\item $\mathbb{R}^{-}$ Set of all negative real numbers
\end{itemize}
% Session 4 - Functions

%------------------------------------------------------------------------%
\section{Arrow Diagrams}

\begin{itemize}

\item Domain
\item Co-Domain
\item Range
\end{itemize}
\[  f(x) : \mbox{Domain} \rightarrow \mbox{Co-Domain} \]
\[  f(x) : \mathbb{R} \rightarrow \mathbb{R} \]
%----------------------------------------------------------- %
\newpage
\subsection*{Polynomial Functions (4.1.5)}

\begin{itemize}
\item[Constants] $(P_0)$
\item[Linear Functions] $(P_1)$
\item[Quadratic Functions] $(P_2)$
\item[Cubic Functions] $(P_3)$
\end{itemize}


\subsection*{Equality of Functions (4.1.6)}
\[f(x) = g(x) \]

%------------------------------------------------------------------------%
\section{Special Mathematical Functions}
\subsection{Mathematical Operators}
\begin{itemize}
\item The Square Root function
\item The Floor and Ceiling functions
\item The Absolute Value functions
\end{itemize}


\begin{itemize}
\item Root Functions
\item Absolute Value Function
\item Floor Function
\item Ceiling Function
\end{itemize}
{

\begin{eqnarray}
\lfloor 3.14 \rfloor =& 3 \\
\lceil -4.5 \rceil =& -5 \\
| -4 | =&  4
\end{eqnarray}
}
For this course, only positive numbers have square roots. The square roots are positive numbers. (This statement is not strictly true. The square root of a negative number is called a complex number. However this is not part of the course).

Negative numbers can have cube roots

{
\[ -27 = -3 \times -3 \times -3 \qquad \]

\[ \sqrt[3]{-27} = -3 \]
}
%-------------------------------------------------------------------------%

\newpage
\section{Exponential and Logarithms}
\subsection*{Laws of Logarithms}

\begin{itemize}
\item Law 1 : Multiplcation of Logarithms
\[ Log(a) \times Log(b) = Log(a+b) \]
\item Law 2 : Division of Logarithms
\[ \frac{Log(a)}{Log(b)} = Log(a-b) \]
\item Law 3 : Powers of Logarithms
\[ Log(a^b) = b \times Log(a) \]
\end{itemize}



%------------------------------------------------------%
\subsection{Exercise} 
$h(x): \mathbb{R} \rightarrow \mathbb{R}$ 
$g(x): \mathbb{R} \rightarrow \mathbb{R}$

\[f(x) = sqrt(x)\]
\[g(x) = \sqrt{3}{x+2}\]
\[h(x) = 2^x\]

\begin{itemize}
\item Is the function $h(x)$ an \textit{onto} function?
\item determine the inverse function of $h(x)$ and $g(x)$
\item Simplify the following function.
\[ j(x) = \mbox{log}_4(h(6x))\]
\end{itemize}
%--------------------------------------------%
\subsection{Onto Functions}
Definition: If every element in the co-domain of the function has an ancestor, the function is said to be "onto".
An onto function has the property that the domain is equal to the co-domain.


\textbf{Example 4.26 Page 53}

%--------------------------------------------%
\subsection*{Exponential and Logarithms}

\textbf{Rules}\\
Expontials : Rules 4.18 Page 58 \\
Logarithms : Rules 4.23 Page 61 \\

\[ log_a(a) = 1 \]
\[ log_a(b^c) = c \times log_a(b) \]

\begin{itemize}
\item $log_2(128) = 7$
\item $log_2(1/4) = -2$
\item $log_2(2) = 1$
\end{itemize}

\[ log_a(b) = \frac{log_x(b)}{log_x(a)} \]

%------------------------------------------------------%
\subsection{Logarithms} 
- Laws of Logarithms
- Change of Base

\[ \mbox{Log}_b(x) = a \] \[b^a = x \]
\[ \mbox{Log}_2(8) = 3 \] \[2^3 = 8 \]

\[ \mbox{Log}_b(x) \times \mbox{Log}_b(y) =  \mbox{Log}_b(x+y) \]
\[ \mbox{Log}_b(x^y) =  y \times \mbox{Log}_b(x) \]

\[ \mbox{Log}_y(x)  =  \frac{ \mbox{Log}_b(x) }{ \mbox{Log}_b(y) } \]
%------------------------------------------------------%
\subsection{Exponents}
- Rules of Exponents

\[ (a^b)^c = a^{b \times c}\]

\[ 64^{2/3} =  (4^3)^{2/3} = 4^{3\times2/3} = 4^2 = 16 \]


\[ (a^b) \times (a^c) = a^{b+c}\]
\[ (3^2) \times (3^3) = 3^{2+3} = 3^5  =243 \]

%------------------------------------------------------%
\textbf{Exercises}
\begin{itemize}
\item[(a)] 
Complete the following table for the functions 
\begin{itemize}
\item[i)] $g(x) = \mbox{log}_3x$,
\item[ii)] $h(x) =\sqrt[3]{x}$.
\end{itemize} 
\begin{center}

\begin{tabular}{|c||c|c|c|c|c|c|}
\hline $x$ &  \phantom{p}1\phantom{p}&  &  &  & 81 &  \\ 
\hline \phantom{p} $g(x)$ \phantom{p}&  & \phantom{p}1\phantom{p} & \phantom{p}2\phantom{p} &  &  &  \phantom{p}5\phantom{p} \\ 
\hline \phantom{p}$h(x)$ \phantom{p}&  &  &  &  3.00 & \phantom{p}\phantom{p}\phantom{p}  &  \\ 
\hline 
\end{tabular} 
\end{center}

Express your answers to 2 decimal places only.

\newpage
%------------------------------------------------------------------------%
\section{\textit{One-to-One} Functions and \textit{Onto} Functions}

\subsection{Invertible Functions}
\begin{itemize}
\item One-to-One Function
\item Onto Function
\end{itemize}

Onto Functions : Range and Co-Domain are equivalent

\subsection{Inverting a Function}

\begin{itemize}
\item[$\bullet$] You are given $f(x)$ in terms of $x$
\item[$\bullet$] Re-arrange the equation so that $x$ is given in terms of $f(x)$
\item[$\bullet$] Replace $x$ with $f^{-1}(x)$ and $f(x)$ with $x$
\end{itemize}

\subsubsection{Example}
\begin{itemize}
\item[$\bullet$]Determine the inverse function of $f(x)$. Re-arrange the equation so that $x$ is given in terms of $f(x)$
\[  f(x): \mathbb{R} \rightarrow \mathbb{R}  \mbox{   } f(x)  = \sqrt{x+1} \]
\item[$\bullet$] Square both sides of the equation.
\[[f(x)]^2 = x+1 \]
\item[$\bullet$] Subtract 1 from both sides of the equation. We have the equation written in terms of x.
\[f(x)^2-1 = x \]
\item[$\bullet$] Replace $x$ with $f^{-1}(x)$ and $f(x)$ with $x$
\[x^2-1 = f^{-1}(x) \]
\item[$\bullet$] 
Re-arrange equation and specify domain and co-domain.
\[ f(x): \mathbb{R} \rightarrow \mathbb{R}  \mbox{   }  f^{-1}(x) = x^2-1  \]
\end{itemize}
% \[ f(x)  = \sqrt{x+1} \]
% \[f(x)^2 = x+1 \]
% \[f(x)^2-1 = x \]
% \[x^2-1 = f^{-1}(x) \]
\newpage
\section{Big O-Notation}

%------------------------------------------------------------------------%
% Section 4 Functions
% http://doc.gold.ac.uk/~maa01km/solutions/tut4sol.pdf

\item[(b)] Let $S$ be the set of all 4 bit binary strings. 

The function $f : S \rightarrow \mathbb{Z}$
is defined by the rule:
\[f(x) = \mbox{the number of zeros in x}\]
for each binary string $x \in S$.\\
Find:
\begin{enumerate}
\item the number of elements in the domain
\item $f(1000)$
\item the set of pre-images of 1
\item the range of $f$.
\end{enumerate}
\item[(c)]
\end{itemize}
\newpage
\begin{itemize}
\item[4.a] $ \lfloor x - y \rfloor = \lfloor x \rfloor - \lfloor y \rfloor$
\item[4.b]
\item[4.c]
\end{itemize}
\newpage
%------------------------------------------------------------------------%
\section{Invertible Functions}
A function is invertible if it fulfils two criteria
\begin{itemize}
\item The function is \textbf{\textit{onto}},
\item The function is \textbf{\textit{one-to-one}}.
\end{itemize}

State the conditions to be satisfied by a function
$f : X \leftarrow Y$ for it to have an inverse function
$f^{-1} : Y \leftarrow X$.
%---------------------------------------------------------%

$\lceil \frac{x^2+1}{4} \rceil$
where $f : A \rightarrow \textbf{Z}$
\begin{itemize}
\item[(i)] Find $f(4)$ and the ancestors of 3.
\item[(ii)] Find the range of $f$.
\item[(iii)] Is f invertible? Justify your answer
\end{itemize}

Given $f : \textbf{R} \rightarrow \textbf{R}$ where f(x) =3x-1,define fully
the inverse of the function f ,i.e.$f^{-1}$. 
State the value of $f^{-1}(2)$
%---------------------------------------------------------%
\subsection{Precision Functions}

\begin{itemize}
\item Absolute Value Function $| x |$
\item Ceiling Function $\lceil x \rceil$
\item Floor Function  $\lfloor x \rfloor $
\end{itemize}
%\[ \lfloorx\rfloor\]

\noindent \textbf{Question1.2}: State the range and domain of the following function
\[ F(x) = \lfloor x-1 \rfloor \]
%---------------------------------------------------------%
\subsection{Powers}

\[  2^ 4 = 2 \times 2 \times 2 \times 2 = 16 \]

\[  5^ 3 = 5 \times 5 \times 5 =125 \]

\subsubsection{Special Cases}

Anything to the power of zero is always 1

\[  X^ 0 = 1 \mbox{ for all values of X} \]

Sometimes the power is a negative number.

\[  X^{-Y} = { 1 \over X^Y}  \]

Example 
\[  2^{-3} = { 1 \over 2^3} = { 1 \over 8}  \]



%---------------------------------------------------------%
\subsection{Exponentials Functions}

\[ e^a \times e^b = e^{a+b}\]

\[ (e^a )^b = e^{ab}\]
%---------------------------------------------------------%
\subsection{Logarithmic Functions}

\subsubsection{Laws for Logarithms}
The following laws are very useful for working with logarithms.
\begin{enumerate}
\item $\mbox{log}_b(X)$ + $\mbox{log}_b(Y)$ = $\mbox{log}_b(X\times Y)$
\item $\mbox{log}_b(X)$ - $\mbox{log}_b(Y)$ = $\mbox{log}_b(X / Y)$
\item $\mbox{log}_b(X^Y)$= $Y \mbox{log}_b(X)$
\end{enumerate}

\noindent \textbf{Question1.3} Compute the Logarithm of the following
\begin{itemize}
\item $\mbox{log}_2(8)$
\item $\mbox{log}_2(\sqrt{128})$
\item $\mbox{log}_2(64)$
\item $\mbox{log}_5(125)$ +   $\mbox{log}_3(729)$
\item $\mbox{log}_2(64/4)$
\end{itemize}

%----------------------------------------------------------------%

\begin{itemize}
\item $a^x = y$  $log_a(y) = x$

\item $e^x = y$  $ln(y)=x$

\item $log_a(x\times y) = log_a(x) + log_a(y)$

\item $log_a(\frac{x}{y}) = log_a(x) - log_a(y)$

\item $log_a(\frac{1}{x}) = - log_a(x)$

\item $log_a(a) = 1$

\item $log_a(1) = 0$
\end{itemize}


\begin{itemize}
\item $\lceil x\rceil$

\item $\lfloor x\rfloor$
\end{itemize}

\begin{tabular}{|c|c|c|c|}
\hline Sample value x & Floor $\lfloor x\rfloor$ & Ceiling  $\lceil x\rceil$ & Fractional part $ \{ x \} $\\
12/5 = 2.4 &2&3&2/5 = 0.4\\
2.7&2&3&0.7\\
-2.7&-3&-2&0.3\\
-2&-2&-2&0\\
\hline 
\end{tabular} 


%=========================================================================================== %
\subsection*{Topic 1 : Special Functions}

\subsection{Special Functions}
\begin{itemize}
% \item The factorial Operator
\item Absolute Value Function
\item The Sign Function
\item Floor and Ceiling Functions
\item Hyperbolic Functions
\end{itemize}



%---------------------------------------%

\subsection{Absolute Value Function}

\textbf{Absolute Value Function}
\begin{itemize}
\item The absolute value (or modulus) $|x|$ of a real number x is the non-negative value of x without regard to its sign. 

%\item For example, $|x| = x$ for a positive x, $|x| = -x$ for a negative x (in which case −x is positive), and $|0| = 0$. 
%\item For example, the absolute value of 4 is 4, and the absolute value of $-4$ is also 4. 
%\item The absolute value of a number may be thought of as its distance from zero.
\end{itemize}
\[|x| = \begin{cases} x, & \mbox{if }  x \ge 0  \\ -x,  & \mbox{if } x < 0. \end{cases} \]

%---------------------------------------%

\subsection{Topic 1 : Absolute Value Function}

%%- \vspace{-1cm}
\begin{itemize}

\item For a positive x, $|x| = x$ 
\item For a negative x (in which case −x is positive) $|x| = -x$ 
\item The absolute value of 0 is 0:  $|0| = 0$. 
\item For example, the absolute value of 4 is 4, and the absolute value of $-4$ is also 4. 
\item IMPORTANT:  The input to this function is any real number. The output of this function will always be a positive real numbers.
%\item The absolute value of a number may be thought of as its distance from zero.
\end{itemize}




\subsection{Topic 1 : Sign Function}

\textbf{Sign Function}
\begin{itemize}
\item The sign function $sng(x)$ of a real number x is a signed value of absolute value of 1, dependent on the sign of $x$. 

%\item For example, $|x| = x$ for a positive x, $|x| = -x$ for a negative x (in which case −x is positive), and $|0| = 0$. 
%\item For example, the absolute value of 4 is 4, and the absolute value of $-4$ is also 4. 
%\item The absolute value of a number may be thought of as its distance from zero.
\item IMPORTANT:  The input to this function is any real number. The output of this function will always be either 1 or -1.
\end{itemize}
\[ sgn(x) = \begin{cases} 1, & \mbox{if }  x \ge 0  \\ -1,  & \mbox{if } x < 0. \end{cases} \]

%---------------------------------------%

\subsection{Topic 1 : Floor and Ceiling Functions}

\begin{itemize}
\item The floor and ceiling functions map a real number to the largest previous or the smallest following integer, respectively. \item More precisely, \[floor(x) = \lfloor x\rfloor \] is the largest integer not greater than x and \[ceiling(x) =  \lceil x \rceil \] is the smallest integer not less than x.
\end{itemize}




%---------------------------------------%

\subsection{Topic 1 : Floor and Ceiling Functions}
\textbf{Examples}
{

\begin{eqnarray}
\lfloor 3.14 \rfloor =& 3 \\ \bigskip
\lceil -4.5 \rceil =& -5 \\ \bigskip
| -4 | =&  4
\end{eqnarray}
}
{
\textbf{Remark:}  Input to the floor and ceiling function can be any really number, but outputs are always integers.
}




\subsection{Floor and Ceiling Functions}
The floor and ceiling functions map a real number to the largest previous or the smallest following integer, respectively. More precisely, floor(x) = $\lfloor x\rfloor$ is the largest integer \textbf{not greater} than x and ceiling(x) =  $\lceil x \rceil$ is the smallest integer \textbf{not less} than x.


%=================================================================================== %

\subsection{Floor and Ceiling Functions}

{

\begin{itemize}
\item $\lceil x\rceil$ : Ceiling function
\item $\lfloor x\rfloor$  : Floor Function
\item $\{x\}$ : Fractional Part of a number
\[\{x\} = x- \lfloor x\rfloor \]
\end{itemize}
}


\end{document}
%---------------------------------------------- %

\subsection{Examples}
{   

\begin{tabular}{|c|c|c|c|} \hline
Value x & Floor $\lfloor x\rfloor$ & Ceiling  $\lceil x\rceil$ & Fractional $ \{ x \} $\\ \hline

-1.4  && & \\
2.3&&& \\
7/9&&& \\
-16/3&&& \\
0 & && \\
1 & && \\
\hline
\end{tabular} 

}

%---------------------------------------------- %

\subsection{Examples}
{   

\begin{tabular}{|c|c|c|c|} \hline
Value x & Floor $\lfloor x\rfloor$ & Ceiling  $\lceil x\rceil$ & Fractional $ \{ x \} $\\ \hline
-1.4  &-2&-1& 0.4\\
2.3&&& \\
7/9&&& \\
-16/3&&& \\
0 & && \\
1 & && \\
\hline
\end{tabular} 

}

\end{document}
%---------------------------------------------- %

\subsection{Examples}
{   

\begin{tabular}{|c|c|c|c|} \hline
Value x & Floor $\lfloor x\rfloor$ & Ceiling  $\lceil x\rceil$ & Fractional $ \{ x \} $\\ \hline
-1.4  &-2&-1& 0.4\\
2.3&2&3& 0.3\\
7/9&&& \\
-16/3&&& \\
0 & && \\
1 & && \\
\hline
\end{tabular} 

}

%---------------------------------------------- %

\subsection{Examples}
{   

\begin{tabular}{|c|c|c|c|} \hline
Value x & Floor $\lfloor x\rfloor$ & Ceiling  $\lceil x\rceil$ & Fractional $ \{ x \} $\\ \hline
-1.4  &-2&-1& 0.4\\
2.3&2&3& 0.3\\
7/9&0&1& 0.7777\\
-16/3&&& \\
0 & && \\
1 & && \\
\hline
\end{tabular} 

}

%---------------------------------------------- %

\subsection{Examples}
{   

\begin{tabular}{|c|c|c|c|} \hline
Value x & Floor $\lfloor x\rfloor$ & Ceiling  $\lceil x\rceil$ & Fractional $ \{ x \} $\\ \hline
-1.4  &-2&-1& 0.4\\
2.3&2&3& 0.3\\
7/9&0&1& 0.7777\\
-16/3&-6&-5& 0.3333\\
0 & && \\
1 & && \\
\hline
\end{tabular} 

}

%---------------------------------------------- %

\subsection{Examples}
{   

\begin{tabular}{|c|c|c|c|} \hline
Value x & Floor $\lfloor x\rfloor$ & Ceiling  $\lceil x\rceil$ & Fractional $ \{ x \} $\\ \hline
-1.4  &-2&-1& 0.4\\
2.3&2&3& 0.3\\
7/9&0&1& 0.7777\\
-16/3&-6&-5& 0.3333\\
0 & 0&0 &0\\
1 & && \\
\hline
\end{tabular} 

}

%---------------------------------------------- %

\subsection{Examples}
{   

\begin{tabular}{|c|c|c|c|} \hline
Value x & Floor $\lfloor x\rfloor$ & Ceiling  $\lceil x\rceil$ & Fractional $ \{ x \} $\\ \hline
-1.4  &-2&-1& 0.4\\
2.3&2&3& 0.3\\
7/9&0&1& 0.7777\\
-16/3&-6&-5& 0.3333\\
0 & 0&0 &0\\
1 & 1 &1 &0 \\
\hline
\end{tabular} 

}

%========================================================================= %

\section*{Laws of Logarithms}
\subsection*{Basic Rules}
\begin{itemize}
\item[(1)] Multiplication inside the log can be turned into addition outside the log, and vice versa.
\[\mbox{log}_b(mn) = \mbox{log}_b(m) + \mbox{log}_b(n)\]

\item[(2)]  Division inside the log can be turned into subtraction outside the log, and vice versa.
\[\mbox{log}_b(m/n) = \mbox{log}_b(m) – \mbox{log}_b(n)\]

\item[(3)] An exponent on everything inside a log can be moved out front as a multiplier, and vice versa.
\[mbox{log}_b(m^n) = n \times \mbox{log}_b(m)\]
\end{itemize}

\section*{Encoding and Decoding Functions (4.2)}


\subsection*{Onto Functions (4.2.2)}

\subsection*{One-to-One Functions (4.2.3)}


\subsection*{One-to-One Functions (4.2.3)}
$f(x)$, must be \emph{One-to-One} and \emph{Onto}



\subsection*{Exponential and Logarithmic Functions (4.3)}

The Laws of Logarithms
\begin{itemize}
\item
\item $log_b(x^y) = y \times log_b(x)$
\item
\item
\end{itemize}

\subsection*{Comparing the size of Functions (4.4)}

Using O-notations

\subsection*{Power Notation (4.4.2)}


\subsection*{Section 8 Exercises}
\begin{itemize}
\item $8^{\frac{1}{3}}$ Recall $a^{\frac{b}{c}} = a^{\frac{b}{c}}$
\item
\item
\end{itemize}


\section{Mathematics for Computing: Onsite Tutorial two}

\textbf{Today's Class}
\begin{itemize}
\item Chapter 3: Logic
\begin{itemize}
\item
\item
\end{itemize}
\item Chapter 4: Functions
\begin{itemize}
\item Inverse of a Function
\item One-to-One and Onto
\item Special Functions
\end{itemize}
\end{itemize}
\newpage
%--------------------------------------------%

\section*{Chapter 4: Functions}

\subsection{Arrow Diagrams}

\begin{itemize}
\item Domain 
\item Co-Domain 
\item Range
\end{itemize}

\newpage
%--------------------------------------------%
\subsection{Boolean Functions and ordered $n-$tuples}


\begin{itemize}
\item Ordered Triples
\item Boolean Fucntions
\item
\end{itemize}

\subsection*{The Asbolute Value, Floor and Ceiling Functions}
\begin{itemize}
\item The Absolute Value Function
\item Floor
\item Ceiling
\end{itemize}



\begin{itemize}
\item
\item
\item
\end{itemize}

\subsection*{Power functions and Polynomials}


Consider the function f: Z-> Z defined by f(n) = 3n-1. Does this function have the onto property?
\[ax^2 + bx + c\]
%--------------------------------------------%
\subsection*{Summer 2003 Question 4}

\begin{tabular}{ccccccc}
x & & & & & & \\ \hline
f(x) & & & & & & \\ \hline
g(x) & & & & & & \\ \hline
\end{tabular}

%--------------------------------------------%

\section*{Functions}
\begin{itemize}
\item Domain of a Function
\item Range of a function
\item Inverse of a function
\end{itemize}
\begin{itemize}
\item one-one (surjective)
\item onto (bijective)
\end{itemize}

%---------------------------------------------- %
\end{document}
