\documentclass{beamer}

\usepackage{amsmath}
\usepackage{amssymb}
\usepackage{graphicx}

\begin{document}
	\begin{frame}
		\Large
		\begin{itemize}
			\item Lecturer :  Kevin O'Brien
			\item Email: kevin.obrien@ul.ie
			\item Office Hours : to be confirmed
		\end{itemize}
	\end{frame}
%====================================================================================================== %
\begin{frame}
	\frametitle{MA4702 Tech Maths 2 Syllabus}
	
On successful completion of this module, students should be able to:
\begin{itemize}
\item[1.] Define the domain and range of a function and define and plot simple inverse trigonometric and hyperbolic  functions. 

\item[2.] Sketch curves using properties such as symmetry, intercepts, discontinuities, turning points and asymptotic behaviour. 

\item[3.] Sum arithmetic, geometric and telescoping series; test series for convergence; find the Maclaurin series of a function; manipulate power series; use l'Hopital's rule. 
\end{itemize}
\end{frame}

%======================================================================== %
\begin{frame}
\frametitle{MA4702 Tech Maths 2 Syllabus}
	
\begin{itemize}
	\item[4.] Integrate standard functions using substitution and parts; Apply to calculation of areas and volumes. \\ Integrate numerically using Simpson's rule. 

\item[5.] Find partial derivatives of functions of two variables as well as higher partial derivatives;\\ apply to analysis of small errors.
\end{itemize}
\end{frame}
%======================================================================= %

\begin{frame}
\frametitle{Tutorials}

\begin{itemize}
\item Tutorials start in Week 3.

\item I will use the ``Teaching Week" naming convention, not the ``Timetable Week" convention as in the SAA calendar. Both are the same up until Easter anyway.
%That is to say Week 10 is the easter week break. Week 13 is the 12th teaching Week. Reading Week is Week 14.

\item There is no lectures in Reading Week (Teaching Week 13), but we will probably schedule tutorials.

\end{itemize}
\end{frame}
%======================================================================= % 

\begin{frame}
	\frametitle{ Mid Terms }
	\textbf{Details about Mid-Term Exams}
	\begin{itemize}
		\item Two Mid-Terms each worth 15 \%.
		\item Mid Term 1 will take place on Week 5.
		\item Mid Term 2 will take place on Week 9.
		\item Precise Dates to be confirmed (I want to see which of our rooms is most suitable holding an exam).
		\item End of Year Exam is worth 70\%. 
	\end{itemize}
\end{frame}
%====================================================================================================== %
\begin{frame}
	\frametitle{MA4701 Tech Maths 1 Syllabus}


\begin{itemize}

\item[1.] Define elementary functions including polynomials, exponential logarithms and graph simple examples. 

\item[2.] Define trigonometric functions and use formulas and identities including sine and cosine rules. 

\item[3.] Differentiate elementary functions using the laws of differentiation and apply to curve-sketching. 

\item Other sections of Tech Maths 1 are not relevant for this module
\end{itemize}
\end{frame}

%========================================================= %

\begin{frame}
\frametitle{Quick Check}
\textbf{Quick Check on MA4701}
\begin{itemize}
\item Trigonometric Functions
\item Differentiation
\end{itemize}

\end{frame}

%========================================================== %

\begin{frame}
	\frametitle{Revision and Fundamental Concepts}
	\textbf{Revision and Fundamental Concepts}
	\begin{itemize}
		\item	In this first section we will review various fundamental theorems and concepts that
		will feature in the course.
		\item	Please be mindful of these, and regularly refer back to this section 
		throughout the semester.
		\item Expect some short questions from this section in all of the exams
	\end{itemize}
	
\end{frame}
%======================================================================================================== %
\begin{frame}
	\frametitle{Lecture 1A}
	\textbf{Revision and Fundamentals}
	
	\begin{itemize}
		\item Numbers and Number Sets (notation)
		\item (Quick) Revision of Functions
		\item Exponents and powers
		\item Logarithms
		\item Special functions and operators
		\item Cross multiplication (fractions expansion)
	\end{itemize}
\end{frame}
%========================================================== %

\begin{frame}
	\frametitle{Topic 1 : Sets of Numbers}
	\begin{itemize} 
		\item $\mathbb{N}$ Set of all natural numbers
		\item $\mathbb{Z}$ Set of all integers
		\item $\mathbb{Q}$ Set of all rational numbers
		\item $\mathbb{R}$ Set of all real numbers
	\end{itemize}
	
	\bigskip There are, of course, other numbers sets, but we will not be encountering them on the course.
\end{frame}
%========================================================== %

\begin{frame}
	\frametitle{Topic 1 : Sets of Numbers}
	\begin{itemize}
		\item $\mathbb{Z}^{+}$ Set of all positive integers
		\item $\mathbb{Z}^{-}$ Set of all negative integers
		\item $\mathbb{R}^{+}$ Set of all positive real numbers
		\item $\mathbb{R}^{-}$ Set of all negative real numbers
	\end{itemize}
\end{frame}

%===================================================================================================== %

\begin{frame}
\frametitle{Topic 1 : The $e$ constant}
\begin{itemize}
\item The number $e$ is an important mathematical constant (another is $\pi$) that is the base of the natural logarithm. 
\item It is approximately equal to 2.71828.
\end{itemize}
\[e =  \displaystyle\sum\limits_{n = 0}^{ \infty} \dfrac{1}{n!} = 1 + \frac{1}{1} + \frac{1}{1\cdot 2} + \frac{1}{1\cdot 2\cdot 3} + \cdots  \]
\end{frame}

%========================================================================================== %
\begin{frame}
	\frametitle{Topic 1 : The Factorial Operator}
	\textbf{Topic 1 : Factorial Function}
	The product of the positive integers from 1 to n inclusive is denoted by $n!$, read “n factorial.” Namely:
	\[n! = 1 \times 2 \times 3 \times \ldots \times (n−2) \times(n−1)\times n\] 
	\begin{itemize}
		\item Accordingly, 1! = 1 and $n! = n(n − l)!$. 
		
		\item It is also convenient to define 0! = 1.
	\end{itemize}	
\end{frame}

%========================================================================================== %
\begin{frame}
	\frametitle{Topic 1 : The Factorial Operator}	
	\[ \frac{5!}{3!} = \frac{5 \times 4 \times 3 \times 2 \times 1}{ 3 \times 2 \times 1 }  = 5 \times 4 = 20 \]
	
	\begin{itemize}
		\item Remark :  We will use the factorial Operator frequently in this module. You are also now expected to be familiar with it for future modules.
	\end{itemize}
\end{frame}
%===================================================================================================== %
\begin{frame}
	\large
	Topic 2
	\begin{itemize}
	\item Power (also known as exponents)
	\item Root Functions
	\item Exponentials
	\item Logarithns
\end{itemize}
\end{frame}
%=================================================================================== %
\begin{frame}
	\frametitle{Topic 2 : Revision of Power Rules}
	\LARGE
	\[ (a^b)^c = a^{b \times c}\] \bigskip
	\[ 64^{2/3} =  (4^3)^{2/3} = 4^{3\times2/3} = 4^2 = 16 \] \bigskip
	\[ (a^b) \times (a^c) = a^{b+c}\] \bigskip
	\[ (3^2) \times (3^3) = 3^{2+3} = 3^5  =243 \]
\end{frame}
%===================================================================================================== %

\begin{frame}
	\frametitle{Topic 2 : Revision of Power Rules}
	{
		\Large
		\[ (e^x)^2 = e^{x \times 2}  = e^{2x} \]
	}
	
	{
		\Large
		\[ (e^x) \times (e^{-x}) = e^{x + (-x)}  = e^{0} = 1 \]
	}
\end{frame}
%===================================================================================================== %
\begin{frame}
	\frametitle{Topic 2 : Roots}
	\textbf{Cube Roots}
	
	\[  \sqrt[a]{b} = c \]  necessarily \[ c \times c \times c = b \]
	\[  \sqrt[3]{27} = 3 \]
\end{frame}
%===================================================================================================== %
\begin{frame}
	\frametitle{Topic 2 : Roots}
	\textbf{Sign of Roots}
		
	Remark: In this course, we will assume the positive square root for a function, in the first instance.
	We will only consider the negative root of a function in some special cases.
	
	
	\[ \sqrt[2]{4x^2} \equiv 2x \]
	
	The $\equiv$ symbols is the symbol for equivalence. You would use it to say that two expressions are equivalent, although the \textit{equals} sign is conventionally used also.
	
\end{frame}
%---------------------------------------%
\begin{frame}
	\frametitle{Cube Roots}
		\begin{itemize}
		\item For this course, only positive numbers have square roots. 
		\item The square roots are also positive numbers. 
		\item \textit{(This statement is not strictly true. The square root of a negative number is called a complex number. However this is not part of the course).}
	\end{itemize}
\end{frame}
%---------------------------------------%
\begin{frame}
	Negative numbers can have cube roots
	
	{
		\Large
		\[ -27 = -3 \times -3 \times -3 \qquad \]
		
		\[ \sqrt[3]{-27} = -3 \]
	}
\end{frame}
%--------------------------------------%

\begin{frame}
	\frametitle{ Topic 2 : Laws of Logarithms}
	\Large
	\begin{itemize}
		\item Law 1 : Multiplcation of Logarithms
		\[ Log(a) \times Log(b) = Log(a+b) \]
		\item Law 2 : Division of Logarithms
		\[ \frac{Log(a)}{Log(b)} = Log(a-b) \]
		\item Law 3 : Powers of Logarithms
		\[ Log(a^b) = b \times Log(a) \]
	\end{itemize}
	
	
	
\end{frame}
%===========================================% 

\begin{frame}
	\frametitle{Topic 3 : Functions}
	\textbf{Revision of Functions}
\end{frame}
\begin{frame}
	\frametitle{Example 1: Evaluating a Function}
	Evaluate the following function for x =1,2 and 3 respectively.
	\[ f(x) = \frac{e^x}{x!} \]
\end{frame}
%==================================================================================== %
\begin{frame}
	\frametitle{Example 2: Evaluating a Function}
	Evaluate the following function for x =1,2 and 5 respectively.
	\[ f(x) = \frac{e^x + e^{e^{-x}}}{2} \]
\end{frame}
%================================================================================== %
\begin{frame}

\frametitle{Example 3: Evaluating a Function}
Evaluate the function for each of the following values : 0.5,1,1.25,2.

\[f(x) =  \sqrt{1+e^{x}} dx \]
 Four decimal places will suffice.
\end{frame}
\begin{frame}
	\begin{center}
\begin{tabular}{|c|c|c|c|}
	\hline \rule[-2ex]{0pt}{5.5ex} x & $e^x$ & $1+e^x$ & $\sqrt{1+e^x}$ \\ 
	\hline \rule[-2ex]{0pt}{5.5ex} 0.5 & \phantom{space}  &  &  \\ 
	\hline \rule[-2ex]{0pt}{5.5ex} 1 &  &  &  \\ 
	\hline \rule[-2ex]{0pt}{5.5ex} 1.25 & \phantom{space}  &  &  \\ 
	\hline \rule[-2ex]{0pt}{5.5ex} 2 & \phantom{space}  &  & \phantom{space}  \\ 
	\hline 
\end{tabular} 
 \end{center}
\end{frame}


%=========================================================================================== %
\subsection*{Topic 4 : Special Functions}
\begin{frame}
	\frametitle{Special Functions}
	\begin{itemize}
		% \item The factorial Operator
		\item Absolute Value Function
		\item Floor and Ceiling Functions
		\item Hyperbolic Functions
	\end{itemize}	
\end{frame}


%---------------------------------------%
\begin{frame}
	\frametitle{Absolute Value Function}
	\Large
	\textbf{Absolute Value Function}
	\begin{itemize}
	\item The absolute value (or modulus) $|x|$ of a real number x is the non-negative value of x without regard to its sign. 
	
%	\item For example, $|x| = x$ for a positive x, $|x| = -x$ for a negative x (in which case −x is positive), and $|0| = 0$. 
%	\item For example, the absolute value of 4 is 4, and the absolute value of $-4$ is also 4. 
	%\item The absolute value of a number may be thought of as its distance from zero.
	\end{itemize}
	\[|x| = \begin{cases} x, & \mbox{if }  x \ge 0  \\ -x,  & \mbox{if } x < 0. \end{cases} \]
\end{frame}
%---------------------------------------%
\begin{frame}
	\frametitle{Absolute Value Function}
	\Large
	\vspace{-1cm}
\begin{itemize}

	\item For a positive x$|x| = x$ , for a negative x (in which case −x is positive) $|x| = -x$ , and $|0| = 0$. 
	\item For example, the absolute value of 4 is 4, and the absolute value of $-4$ is also 4. 
	\item IMPORTANT:  The input to this function is an real number. The output of this function will always be a positive real numbers.
	%\item The absolute value of a number may be thought of as its distance from zero.
\end{itemize}


\end{frame}
%---------------------------------------%
\begin{frame}
	\frametitle{Topic 4 : Floor and Ceiling Functions}
	\Large
	\begin{itemize}
	\item The floor and ceiling functions map a real number to the largest previous or the smallest following integer, respectively. \item More precisely, \[floor(x) = \lfloor x\rfloor \] is the largest integer not greater than x and \[ceiling(x) =  \lceil x \rceil \] is the smallest integer not less than x.
	\end{itemize}
	
	
\end{frame}
%---------------------------------------%
\begin{frame}
	\frametitle{Topic 4 : Floor and Ceiling Functions}
	{
		\LARGE
		\begin{eqnarray}
		\lfloor 3.14 \rfloor =& 3 \\
		\lceil -4.5 \rceil =& -5 \\
		| -4 | =&  4
		\end{eqnarray}
	}
\end{frame}
%====================================================%
\begin{frame}
	\frametitle{Tech Maths 2}
	\textbf{Topic 5 : Cross Multiplication}
	\Large 
	\begin{itemize}
		\item Can simplify an expression by multiplying both the numerator and denominator by same term.
		\item This does not change the value of the expression.
		\item Remark
		{
			\LARGE
		\[ \frac{A}{B} + \frac{X}{Y} = \frac{AY}{BY} + \frac{BX}{BY} = \frac{AY+BX}{BY}\]
	}
	\end{itemize}
\end{frame}
%======================================================================= %

\end{document}

\begin{frame}
	\frametitle{Cross Mutliplication}
	\Large
	\[  \frac{p}{x+a} + \frac{q}{x+b}   = \frac{p(x+b) + q(x+a)}{(x+a)(x+b)} = \frac{(p+q)x+ (pb+aq)}{(x+a)(x+b)} \]
	
	\begin{itemize}
		\item $\{p,q,a,b\} \in R$
	\end{itemize}
	
\end{frame}
%======================================================================= 
\begin{frame}
	\frametitle{Cross Mutliplication}
	\Large
	\[  \frac{4}{x+2} + \frac{2}{x-1}   = \frac{4(x-1) + 2(x+2)}{(x+2)(x-1)} \]
	\[ = \frac{(4+2)x + (4(-1)+(2\times 2)}{(x+2)(x-1)} \]
	\[ = \frac{2x}{x^2+x-2}\]
	
	\end{frame}
	%====================================================%
	\begin{frame}
	\frametitle{Tech Maths 2}
	\textbf{Cross-Multiplication: Example 1}
	% \textbf{Lecture 1B}
	\begin{itemize}
	\item Solve the following Equation for A and B
	\item $A,B \in \mathbb{R}$
	\end{itemize}
	
	\[ \frac{2x + 5}{x^2 - 4x - 12} = \frac{A}{x-6} + \frac{B}{x+2}\]
\end{frame}
%====================================================%
\begin{frame}
	\frametitle{Tech Maths 2}
	%\textbf{Lecture 1B}
	\textbf{Cross-Multiplication: Example 2}
	\begin{itemize}
		\item Solve the following Equation for A and B
		\item $A,B \in \mathbb{R}$
	\end{itemize}
	
	\[ \frac{11}{x^2 - 4x - 12} = \frac{A}{x-6} + \frac{B}{x+2}\]
\end{frame}
%====================================================%
\begin{frame}
\frametitle{Tomorrow's Class}

\begin{itemize}
\item Continuation of Revision Materials
\item Summations and "Sigma" Notation
\item Some Worked Examples
\end{itemize}
\end{frame}
%---------------------------------------%
\end{document}
