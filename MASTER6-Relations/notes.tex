\documentclass{beamer}

\usepackage{default}

\begin{document}
\begin{frame}{Relations}
%Definition 1.4 

\Large
Let R be an equivalence relation defined on a set S and let x 2 S. 
Then the equivalence class of x is the subset of S containing all
elements of S which are related to x. 

We denote this by [x]. Thus
\[[x] = {y 2 S : yRx}\].

\end{frame}
%-----------------------------%

\begin{frame}

\Large
Theorem 1.3 
Let R be an equivalence relation on a set S. Then:
\begin{itemize}
\item The set of distinct equivalence classes of R on S is a partition of S.
\item Two elements of S are related if and only if they belong to the same equivalence
class.
\end{itemize}
\end{frame}
%-----------------------------%
\begin{frame}

\Large
Definition 1.6 We say that a relation R on a set S is anti-symmetric if for
all x, y 2 S such that xRy and yRx, we have x = y. 
Thus R is \textbf{anti-symmetric}
if and only if the relationship digraph of R has no directed cycles of length two.
\end{frame}

\end{document}
