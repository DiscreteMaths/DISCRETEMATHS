\documentclass[]{article}
\usepackage{framed}
\usepackage{amsmath}
\usepackage{amssymb}
\usepackage{graphicx}
\usepackage{multicol}
%opening

\begin{document}
\subsection*{Chapter 6: Digraphs and Relations}
%--------------------------------------------- %
\textbf{Example 1.19} Let X be the set of all students registered at a college and Y be
the set or" atl courses being taught at the college. Then the college database keeps
a record of the relationship in which a student z is related to a course y if 1: is
registered for y.
%--------------------------------------------- %
We can model a relation 'R between sets X and Y by a digraph D in much the
same way as we did previously for relations on a. single set: We put l/(D) = X U Y
and draw an arc from a vertex m 5 X to a vertex y E Y if x7?.y.
%---------------- %
Example 1.20 Let $X = \{1, 2,3\}$ and $Y = \{b, c\}$. Define R by l’R.b, 1Rc, 272::, and
3‘I?.c. Then the relationship digraph for 'R is shown in Figure 1.3.
%--------------------------------------------- %
There is at strong similarity between Figure 1.3 and the figures drawn in Volume
1, Chapter 4 to illustrate functions, Indeed we can view a function $f : X \rightarrow Y$ as
being a relation in which each a.- E X is related to a unique element of Y. We write
f(:c) to represent the unique element of Y which is related to m.



%--------------------------------------------- %
\newpage
Exercise 2
For each of the relations given in questions l, 2 and 3:
\begin{itemize}
\item[(a)] Draw the relationship digraph.
\item[(b)] Determine when the relation is either refiexive, symrneric, transitive or antir
symmetric. For the cases when one of these properties does not hold, justify
your answer by giving an example to show that it does not hold.
\item[(c)] Determine which of the rekations is an equivalence relation? For the case
when it is an equivalence relation, calculate the distinct equivalence classes
and verify that they give a partition of S.
\item[(d)] Determine which of the relations is a partial order or an order.
\end{itemize}
%---------------- %
Q1 Let S : {0,1,2,3}. Define a relation R; between the elements of S by ":c is
related to y if the product my is even”.
Q2 Let S : {0,1,2,2}. Define a relation R2 between the elements of S by "z is
related to y if 0: — y E {0,3, -3}".
Q3 Let S : {{l},{1,2}, {I, 2,3}, {1,2,4}}. Denne a relation 728 between the ele-
ments of S by "X is related to Y ifX Q Y ”’.
[B0}
%---------------- %
Q4 Let S be a set and R be a relation on S. Explain what it means to say that R
is
\begin{itemize}
\item[(a)] reflexive,
\item[(b)] symmetric,
\item[(c)] transitive,
\item[(d)] anti-symmetric,
\item[(e)] an equivalence relation,
\item[(f)] a partial order,
\item[(g)] an order. 
\end{itemize}
%---------------- %
 
Q5 Lgt X .: {0,1,2} and $Y = \{3,4\}$. Define a relation 72, from X to Y by “1 is
related to y ifs =y—3",for so EX and y EY.
(a) Draw the relationship digraph corresponding to 72.
(bl Determine the set X >< Y.
Determine the subset of X >< Y corresponding to 72.
[cz} Is 72 a function from X to Y'? Justify your answer.

\end{document}
