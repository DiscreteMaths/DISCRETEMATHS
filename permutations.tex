\documentclass{beamer}

\usepackage{amsmath}
\usepackage{framed}
\usepackage{amssymb}
\usepackage{graphics}

\begin{document}
\begin{frame}
\Huge
\[ \mbox{Mathematics for Computing} \]
\huge
\[ \mbox{Anagrams as Permutations} \]

\Large
\[ \mbox{kobriendublin.wordpress.com} \]
\[ \mbox{Twitter: @StatsLabDublin} \]

\end{frame}
%-------------------------------------%
\begin{frame}
\frametitle{Permutations}
\Large
\vspace{-2cm}
How many anagrams (permutations of the letters) are there of the following words
\begin{enumerate}
\item ANSWER
\item PERMUTE
\item ANAGRAM
\item LITTLE
\end{enumerate}

\end{frame}
%-------------------------------------%
\begin{frame}
\frametitle{Permutations}
\Large
\vspace{-2cm}
\textbf{Part 1 : ANSWER}\\
Examples:
\begin{center}
ASNWRE,\;
SANERW,\;
REWSAN,\;...
\end{center}

Since ANSWER has 6 distinct letters, the number of permutations (anagrams) is
\LARGE
\[6! = 6\times 5 \times 4 \times 3 \times 2\times 1 = \boldsymbol{720} \]
\end{frame}
%-------------------------------------%
\begin{frame}
\frametitle{Permutations}
\Large
\vspace{-0.3cm}
\textbf{Part 2 : PERMUTE}\\
\begin{itemize}
\item The word PERMUTE has 7 letters, but only 6 different letters. 
\item There are 7! ways to arrange 7 letters.
\item However, interchanging the two Es does not result in a new permutation. There would be two identical anagrams.
\end{itemize}

\begin{center}
P\textcolor{red}{E}RMUT\textcolor{blue}{E}, \; MUT\textcolor{red}{E}P\textcolor{blue}{E}R, \; P\textcolor{red}{E}T\textcolor{blue}{E}MUR,\; ..\\
P\textcolor{blue}{E}RMUT\textcolor{red}{E}, \; MUT\textcolor{blue}{E}P\textcolor{red}{E}R, \; P\textcolor{blue}{E}T\textcolor{red}{E}MUR,\; ..
\end{center}
\end{frame}
%-------------------------------------%
\begin{frame}
\frametitle{Permutations}
\Large
\vspace{-0.3cm}
\textbf{Part 2 : PERMUTE}\\
\begin{itemize}
\item The number of permutations (anagrams) is half of 7! .
\LARGE
\[\frac{7!}{2} =  \frac{5040}{2} = \boldsymbol{2520} \]
\end{itemize}
\end{frame}

%-------------------------------------%
\begin{frame}
\frametitle{Permutations}
\Large
\vspace{-0.1cm}
\textbf{Part 3 : ANAGRAM}\\
\begin{itemize}
\item The word ANAGRAM has 7 letters, but there are three As.
\item From before, there are 7! ways to arrange 7 letters.
\item How many new permutations are found by re-arranging the As?
\end{itemize}
\LARGE
\begin{center}
\textcolor{red}{A}N\textcolor{blue}{A}GR\textcolor{purple}{A}M \; 
\textcolor{red}{A}N\textcolor{purple}{A}GR\textcolor{blue}{A}M \; 
\textcolor{blue}{A}N\textcolor{red}{A}GR\textcolor{purple}{A}M \; \\
\textcolor{purple}{A}N\textcolor{red}{A}GR\textcolor{blue}{A}M \; 
\textcolor{blue}{A}N\textcolor{purple}{A}GR\textcolor{red}{A}M \; 
\textcolor{purple}{A}N\textcolor{blue}{A}GR\textcolor{red}{A}M \; 
\end{center}
\end{frame}

%-------------------------------------%
\begin{frame}
\frametitle{Permutations}
\Large
\vspace{-0.1cm}
\textbf{Part 3 : ANAGRAM}\\
\begin{itemize}
\item We divide 7! by 3! to account for the identical anagrams.
\LARGE
\[\frac{7!}{3!} =  \frac{5040}{6} = \boldsymbol{840} \]
\end{itemize}
\end{frame}
%-------------------------------------%
\begin{frame}
\frametitle{Permutations}
\Large
\vspace{-2.3cm}
\textbf{Part 2 : PERMUTE}\\
\begin{itemize}
\item We re-express the answer from part 2 as follows:
\LARGE
\[\frac{7!}{2!} =  \frac{5040}{2} = \boldsymbol{2520} \]
\end{itemize}
\end{frame}
%-------------------------------------%
\begin{frame}
\frametitle{Permutations}
\Large
\vspace{-0.3cm}
\textbf{Part 4 : LITTLE}\\
\begin{itemize}
\item The word LITTLE has 6 letters, but there are two Ls and two Ts.
\item From before, there are 6! ways to arrange 6 letters.
\item Again, interchanging the two Ls and Ts does not result in a new permutation. 
\LARGE
\[\frac{6!}{2!\times 2!} =  \frac{720}{4} = \boldsymbol{180} \]
\end{itemize}
\end{frame}
\end{document}
