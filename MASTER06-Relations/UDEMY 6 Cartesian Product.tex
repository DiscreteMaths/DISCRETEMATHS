

\subsection*{Cartesian Product}
We close this chapter by introducing one more mathematical structure which
can be used to define a relation.

%--------------------------------------------- %
\textbf{Definition 1.21} Let X and Y be sets. Then the \textbf{Cartesian product} $X \times Y$ is
the set whoose elements are all ordered pairs of elements (1, y) where $x \in X$ and
$y \in Y$.
%--------------------------------------------- %
\textbf{Example 1.22} Let $X : \{1,2,3\}$ and $Y : \{b, c\}$.
%Then X >< Y : {(1,12), (1,:;), (2,b), (2,c), (3,b), (3, c)}.
%--------------------------------------------- %
%10 CHAPTER 1. DIGRAPHS AND RELATIONS
Given a relationship $\mathcal{R}$ between X and Y, we can use the cartesian product of
X and Y to help us define $\mathcal{R}$: we simply give the subset of X x Y containing all
ordered pairs (m,y) for which z7{y. This is equivalent. so listing the ordered pairs
corresponding to the arcs in the relationship digraph of $\mathcal{R}$.
%--------------------------------------------- %
Example 1.23 The relation 72 in Example 1.20 is defined by the subset
$\{(1,b), (1,c),(2,c),(3,c)\}$ of $X \times Y$.
%---------- %
Example 1.24 In Example 1.3 we have $X = \{l, 2, 3\}$ = Y, and $\mathcal{R}$ is defined by
the subset
\[  \{ (1,1), (1.2), (2,1), (2,2), (2,3), (3,2), (3,3)\}  \]
of $X \times X$.
%---------------- %
Definition 1.25 We can generalize the idea of an ordered pair to an ordered n-tuple, $(x_1, x_2, x_3, \ldots, x_n)$, and the cartesian product of two sets to the Cartesian
product of $n$ sets, for any $n \in \mathbb{Z}^{+}$.
%--------------------------------------------- %
When each of the sets in a cartesian product is the same, as in Example 1.24, we
often use an abbreviated notation: we denote $X \times X$ by X2 and in general, we
denote the set $X \times X$ x . .. x X, by $X^n$, where there are ri X’s altogether in the
product.
%---------------- %
Example 1.26 An n-bit binary string is an example of an ordered n—tuple, even
though we write it without the brackets and without commas between the entries.
Lei B : {0.1} {the set of bits). 

%--------------------------------------------- %
Then the set of all n»bit binary strings is {cz;a2...<2,r ::1] E B,ag€B..,..a,, E B},
and this set. can be denoted by $B^n$.
.