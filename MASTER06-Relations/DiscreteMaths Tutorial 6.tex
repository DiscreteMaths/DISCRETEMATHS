\documentclass[12pt]{article}

\usepackage{amsmath}
\usepackage{amssymb}

%opening
\title{Mathematics for Computing}
%\author{Hibernia College}

\begin{document}

\begin{center}
\huge{Mathematics for Computing}\\
\LARGE{Digraphs and Relations}
\end{center}

\section{Digraphs and Relations}
% 2007 Q8
Given a flock of chickens, between any two chickens one of them is
dominant. A relation, R, is defined between chicken x and chicken y as xRy if x is
dominant over y. This gives what is known as a pecking order to the flock. Home
Farm has 5 chickens: Amy, Beth, Carol, Daisy and Eve, with the following relations:

\begin{itemize}
\item Amy is dominant over Beth and Carol
\item Beth is dominant over Eve and Carol
\item Carol is dominant over Eve and Daisy
\item Daisy is dominant over Eve, Amy and Beth
\item Eve is dominant over Amy.
\end{itemize}

\newpage


Types of Relations

\begin{itemize}
\item Antisymmetric
\item Symmetric
\item Reflexive 
\item Transitive
\end{itemize}
%---------------------------------



\end{document}