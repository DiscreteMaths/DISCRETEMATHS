\documentclass[a4paper,12pt]{article}
%%%%%%%%%%%%%%%%%%%%%%%%%%%%%%%%%%%%%%%%%%%%%%%%%%%%%%%%%%%%%%%%%%%%%%%%%%%%%%%%%%%%%%%%%%%%%%%%%%%%%%%%%%\usepackage{eurosym}
\usepackage{vmargin}
\usepackage{amsmath}
\usepackage{graphics}
\usepackage{epsfig}
\usepackage{subfigure}
%\usepackage{fancyhdr}
\usepackage{subfiles}

\begin{document}
%---------------------------------------------------------%

\section{Fundamental of Mathematics}

\subsection{Powers}

\[  2^ 4 = 2 \times 2 \times 2 \times 2 = 16 \]

\[  5^ 3 = 5 \times 5 \times 5 =125 \]

\subsubsection{Special Cases}

Anything to the power of zero is always 1

\[  X^ 0 = 1 \mbox{ for all values of X} \]

Sometimes the power is a negative number.

\[  X^{-Y} = { 1 \over X^Y}  \]

Example 
\[  2^{-3} = { 1 \over 2^3} = { 1 \over 8}  \]


%====================================== %
\newpage
\begin{center}
\huge{Mathematics for Computing}\\
{\LARGE Session 2 : Set Theory}
\end{center}


\subfile{session02a.tex}
\newpage
%====================================== %
\newpage
\begin{center}
\huge{Mathematics for Computing}\\
{\LARGE Session 3 : Logic}
\end{center}
\subfile{session03a.tex}
\subfile{session03d.tex}
\subfile{DM0305DeMorgans.tex}

%============================================ 
\newpage
\begin{center}
\huge{Mathematics for Computing}\\
\LARGE{Digraphs and Relations}
\end{center}
\subfile{session06a.tex}
\subfile{session06e-antisymmetricrelations.tex}
\subfile{session06f.tex}


\end{document}