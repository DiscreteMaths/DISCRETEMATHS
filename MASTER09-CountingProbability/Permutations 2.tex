\documentclass{beamer}

\usepackage{graphics}
\usepackage{framed}
\usepackage{amsmath}


\begin{document}

\begin{frame}



\begin{center}
{\Huge Permutations : }\\\vspace{0.3cm} {\Huge  Ordered Subsets} \\\bigskip{\Large kobriendublin.wordpress.com}\\ \vspace{0.4cm}
{\Large Twitter: @statslabdublin}
\end{center}


\end{frame}
%-----------------------------------------------------------------------------%
\begin{frame}

\frametitle{Permutations}
{\Large
\begin{itemize}

\item Suppose we have a set of \textbf{n} items.
\item From that set, we create a subset of \textbf{k} items.
\item The \textbf{order} in which items are selected is recorded. (The ordering of selected items is very important.) 
\item The total number of \textbf{ordered subsets} of \textbf{k} items chosen from a set of \textbf{n} items is

\[\frac{n!}{n-k!}\]
\end{itemize}
}
%-----------------------------------------------------------------------------%
\end{frame}
\end{document}
