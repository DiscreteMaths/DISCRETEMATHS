\documentclass[a4paper,12pt]{article}
%%%%%%%%%%%%%%%%%%%%%%%%%%%%%%%%%%%%%%%%%%%%%%%%%%%%%%%%%%%%%%%%%%%%%%%%%%%%%%%%%%%%%%%%%%%%%%%%%%%%%%%%%%%%%%%%%%%%%%%%%%%%%%%%%%%%%%%%%%%%%%%%%%%%%%%%%%%%%%%%%%%%%%%%%%%%%%%%%%%%%%%%%%%%%%%%%%%%%%%%%%%%%%%%%%%%%%%%%%%%%%%%%%%%%%%%%%%%%%%%%%%%%%%%%%%%
\usepackage{eurosym}
\usepackage{vmargin}
\usepackage{amsmath}
\usepackage{graphics}
\usepackage{epsfig}
\usepackage{subfigure}
\usepackage{fancyhdr}
%\usepackage{listings}
\usepackage{framed}
\usepackage{graphicx}
\usepackage{amsmath}
\usepackage{chngpage}
%\usepackage{bigints}


\setcounter{MaxMatrixCols}{10}
%TCIDATA{OutputFilter=LATEX.DLL}
%TCIDATA{Version=5.00.0.2570}
%TCIDATA{<META NAME="SaveForMode" CONTENT="1">}
%TCIDATA{LastRevised=Wednesday, February 23, 2011 13:24:34}
%TCIDATA{<META NAME="GraphicsSave" CONTENT="32">}
%TCIDATA{Language=American English}

%\pagestyle{fancy}
%\setmarginsrb{20mm}{0mm}{20mm}{25mm}{12mm}{11mm}{0mm}{11mm}
%\lhead{MA4413} \rhead{Mr. Kevin O'Brien}
%\chead{Statistics For Computing}
%\input{tcilatex}

\begin{document}

% http://www.mathsisfun.com/combinatorics/combinations-permutations.html


%-----------------------------------------------------%
\section*{Session 09: Probability}
\begin{itemize}
\item[9A.1] Counting Methods
\item[9A.2] Counting using Sets
\item[9A.3] Probability
\item[9A.4] Independent Events
\end{itemize}
\begin{itemize}
\item[9B.1] Permutation

\[ {n \choose r} = \frac{n!}{(n-r)! r!} \]


\[ {6 \choose 3} = \frac{6!}{(6-3)! 3!} = \frac{6!}{3! \times 3!}\]


\[ \frac{6!}{3! \times 3!} = \frac{6 \times 5 \times 4 \times 3!}{3! \times 3!} = \frac{120}{6} = 120\]
\end{itemize}

%\begin{multicol}{2}
\begin{itemize}
\item ${6 \choose 2} = 15$
\item ${5 \choose 2} = 10$  
\item ${4 \choose 0} = 1$  
\item ${4 \choose 3} = 4$  
\end{itemize}
%\end{multicol}

\begin{itemize}
\item pairwise disjoint sets
\item The addition principle
\end{itemize}
\subsection*{Theorem}
\[ |A \cup B| = |A| + |B| - |A \cap B|  \]

\subsection*{Probability}
\begin{itemize}
\item[9B.2] The sample space of an experiment ($S$)
\item[9B.3] The size of a sample space
\item[9B.4] Indepedent Events (9.3.1)
\end{itemize}


\section*{Combinations}

\subsection*{Formula}
\[ \binom nk  = \frac{n!}{k!(n-k)!} = \frac{n(n-1)\ldots(n-k+1)}{k(k-1)\dots 1},\]
which can be written using factorials as  whenever $k\leq n$

\subsection*{Example 1}

\[ \binom 5 2  = \frac{5!}{2!(5-2)!} = \frac{5.4.3!}{2! .3!} = \frac{5.4}{2.1} = 10\]

\subsection*{Example 2}

\[ \binom 5 0  = \frac{5!}{0!(5-0)!} = \frac{5!}{0! .5!} = \frac{5!}{2!} = 1\]
Recall $0! =1$

\section{Permutations}

\begin{itemize}
\item The notion of permutation relates to the act of permuting (rearranging) objects or values. 
\item Informally, a permutation of a set of objects is an arrangement of those objects into a particular order. 

\item For example, there are six permutations of the set $\{1,2,3\}$, namely (1,2,3), (1,3,2), (2,1,3), (2,3,1), (3,1,2), and (3,2,1). 
\item As another example, an anagram of a word is a permutation of its letters. 

\end{itemize}


%-----------------------------------------------------%



\begin{itemize}
\item Permutations where repetition is allowed: 
\[ n! \]
\item Permutations where repetition is not allowed
\[ \frac{n!}{(n-k)!} \]
\end{itemize}


%-----------------------------------------------------%

%------------------------------------------------%
\section*{Session 9 Probability}
\subsection*{Binomial Coefficients}
\begin{itemize}
\item factorials 
\[ n! = (n)\times (n-1)\times(n-2) \times \ldots \times 1 \]
\begin{itemize}
\item $5! = 5 \times 4 \times 3 \times 2 \times 1 = 120 $
\item $3! = 3 \times 2 \times 1$
\end{itemize}
\item Zero factorial
\[ 0! =  1 \]
\end{itemize}
%---------------------------------------------- %

% \[P(A |B = \frac{P(A \cap B)}{P(B)})\]

The complement rule in Probability

$P(C^{\prime}) = 1- P(C)$

 

If the probability of C is $70 \%$ then the probability of $C^{\prime}$ is $30\%$

\section*{Combinations and Permutations }
\subsection*{The factorial function}
The factorial function (symbol: !) just means to multiply a series of descending natural numbers. Examples:

\begin{itemize}
\item $4! = 4 \times 3 \times 2 \times 1 = 24$
\item $7! = 7 \times 6 \times 5 \times 4 \times 3 \times 2 \times 1 = 5,040$
\item $1! = 1$
\item $0! = 1 $
\end{itemize}
Importantly 
\[n! = n \times (n-1)!  = n \times (n-1) \times (n-2)! \]
For Example
\[6! = 6 \times 5!  = 6 \times 5 \times 4! \]
\section*{Combinations}
In mathematical terms, a combination is an subset of items from a larger set such that the order of the items does not matter.

\section*{Permutations}
There are two types of permutation:
\begin{enumerate}
\item Repetition is Allowed: such as the lock above. It could be "333".
\item No Repetition: for example the first three people in a running race. You can't be first and second.
\end{enumerate}

\subsection*{Summary}
\begin{itemize}
\item If the order doesn't matter, it is a Combination.
\item If the order does matter it is a Permutation.
\end{itemize}


\begin{itemize}

\item Permutations where repetition is allowed: 
\[ n! \]
\item Permutations where repetition is not allowed
\[ \frac{n!}{(n-k)!} /]
\end{itemize}

\section{Choose Operator}
%----------------------------------------------------------%
\begin{itemize}
%----------------------------------------------------------%


\item[1] Choose Operator
\[ {n \choose k} = \frac{n!}{k! \times (n-k)!} \]
Evaluate the following:
\begin{multicols}{3}
    \begin{enumerate}
    \item[1] ${5 \choose 2}$
    \item[2] ${5 \choose 0}$
    \item[3] ${6 \choose 3}$
    \item[4] ${6 \choose 6}$
    \item[5] ${10 \choose 1}$
    \item[6] ${10 \choose 9}$
    \end{enumerate}        
  \end{multicols}
%----------------------------------------------------------%
%Page 29
\item[2] In how many ways can a group of four people be selected from three men and four women?
In how many of these groups are there more women than men?
%----------------------------------------------------------%
%PAGE 66
\item[3] In how many ways can a group of five be selected from ten people
How many groups can be selected if two particular people from the ten can not be selected in the same group?\\
\\
\textbf{Counting Sets using Venn Diagrams}
\item[4] 
%http://www.mathsireland.com/LCHGeneralNotes/PermCombProb/5_5_Prob_MultAnd/Q_5_5_Prob_MultAnd.html
The Venn Diagram shows the number of elements in each subset of set $S$.
If $P(A) = 3/10$ and $P(B) = 1/2$, find the values of $x$ and $y$

%Page 27
\item[5] How many different four digit numners greater than 5000 can be formed from the digits \textbf{2,4,5,8,9} if each digit can only be used once in any given number. How many of these numbers are odd?
\end{itemize}

\newpage


%------------------------------------------------------ %

\subsection{Permutations}

\begin{itemize}
\item In how many permutations are there of counting a subset of k elements, when there are $n$ elements in total.

\item The number of permutations of a set of n elements is denoted n! (pronounced n factorial.)
\end{itemize}

%------------------------------------------------------ %

\subsection{Permutation Formula}

A formula for the number of possible permutations of k objects from a set of n. This is usually written $^nP_k$ .

\bigskip
\textbf{Formula:}	
\[ ^nP_k = \frac{n!}{(n-k)!} =  n.(n-1).(n-2).\ldots(n-k+1) \]


%------------------------------------------------------ %

\subsection{Permutation Formula}

\textbf{Example:}\\	
How many ways can 4 students from a group of 15 be lined up for a photograph?\\
\bigskip
%--------------- %
\textbf{Answer:	}\\
There are $^{15}P_4$ possible permutations of 4 students from a group of 15.
\[ ^{15}P_4 = \frac{15!}{11!} = 15\times 14\times 13\times 12 = 32760 \]
There are 32760 different lineups.

%------------------------------------------------------ %


%------------------------------------------------------------------%

\subsection{ Permutations }

\begin{itemize}
\item Given a set of distinct numbers, $\{1, 2, 3, 4, 5, 6\}$, find all permutations containing 3 numbers. All the permutations have to be in ascending order.

\item For example, some correct permutations would be $\{1, 2, 3\}$, $\{2, 4, 6\}$, etc. 
\item $\{2, 3, 1\}$ would not be acceptable because it is not in ascending order.
\end{itemize}

%------------------------------------------------------------------%

\subsection{ Permutations}


Given a set of distinct numbers, $\{1, 2, 3, 4 \}$, find all permutations containing 2 numbers. 
 \[\{1,2\},\{1,3\},\{1,4\}, \{2,1\}, \{2,3\}, \{2,4\}, \]
 \[\{3,1\},\{3,2\},\{3,4\}, \{4,1\}, \{4,2\}, \{4,3\}, \]
\bigskip

When all the permutations have to be in ascending order.
 \[\{1,2\},\{1,3\},\{1,4\}, \phantom{\{2,1\}}, \{2,3\}, \{2,4\}, \]
 

 \[\phantom{\{3,1\},\{3,2\}},\{3,4\}, \phantom{\{4,1\}, \{4,2\}, \{4,3\}}\]



%------------------------------------------------------------------%
%

\subsection{ Permutations}
\[\{1,2\},\{1,3\},\{1,4\}, \{2,1\}, \{2,3\}, \{2,4\}, \]
\[\{3,1\},\{3,2\},\{3,4\}, \{4,1\}, \{4,2\}, \{4,3\}, \]
\[ ^4P_2 = \frac{4!}{2!} = 12 \; \]
\bigskip

\[\{1,2\},\{1,3\},\{1,4\}, \phantom{\{2,1\}}, \{2,3\}, \{2,4\}, \]
\[\phantom{\{3,1\},\{3,2\}},\{3,4\}, \phantom{\{4,1\}, \{4,2\}, \{4,3\}}\]

\[ ^4C_2 = \frac{4!}{2!\times 2!} = 6 \]





\subsection{ Permutations}

Given a set of distinct numbers, $\{1, 2, 3, 4 \}$, find all permutations containing 3 numbers. (Strictly ascending permutations in red). 
 \[\boldsymbol{	extbf{\{1,2,3\}}}, \boldsymbol{	extbf{\{1,2,4\}}}, \{1,3,2\},\boldsymbol{	extbf{\{1,3,4\}}}, \{1,4,2\},\{1,4,3\},  \]
 \[\{2,1,3\}, \{2,1,4\}, \{2,3,1\},\boldsymbol{	extbf{\{2,3,4\}}}, \{2,4,1\},\{2,4.3\},  \]
 \[\{3,1,2\}, \{3,1,4\}, \{3,2,1\},\{3,2,4\}, \{3,4,1\},\{3,4,2\},  \]
 \[\{4,1,2\}, \{4,1,3\}, \{4,2,1\},\{4,3,1\}, \{4,3,1\},\{4,3,2\}   \]  


\[ ^4P_3 = \frac{4!}{1!} = 24 \]
\[ ^4C_3 = \frac{4!}{3!\times 1!} = 4 \]



%------------------------------------------------------------------%

\subsection{ Permutations}

\begin{itemize}
\item In general, when you must select $k$ numbers from $n$, and those $k$ numbers must be in ascending order, then there are $^nC_k$ ways permutations.
\end{itemize}
\[ ^C_k = {n \choose k} = \frac{n!}{k!(n-k)!} \]




%-----------------------------------------------------------------------------%


\section{Permutations}
{
\begin{itemize}

\item Suppose we have a set of \textbf{n} items.
\item From that set, we create a subset of \textbf{k} items.
\item The \textbf{order} in which items are selected is recorded. (The ordering of selected items is very important.) 
\item The total number of \textbf{ordered subsets} of \textbf{k} items chosen from a set of \textbf{n} items is

\[\frac{n!}{n-k!}\]
\end{itemize}
}
%-----------------------------------------------------------------------------%


\end{document}
