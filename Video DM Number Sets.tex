\documentclass{beamer}

\usepackage{amsmath}
\usepackage{framed}
\usepackage{amssymb}
\usepackage{graphics}

\begin{document}
%----------------------------------------------------------------%
\begin{frame}

{
\Huge
\[\mbox{Mathematics for Computing}\]
\LARGE
\[\mbox{Number Sets}\]
}
{
\Large
\[\mbox{kobriendublin.wordpress.com}\]
\[\mbox{Twitter: @StatsLabDublin}\]
}
\end{frame}
%----------------------------------------------------------------%
\begin{frame}
\frametitle{Number Sets}
\Large
\vspace{-1cm}
\textbf{Blackboard Bold Typeface}

\begin{itemize}
\item Conventionally the symbols for numbers sets are written in a special typeface, known as \textbf{blackboard bold}.
\item Examples : $\mathbb{N}$, $\mathbb{Z}$ and $\mathbb{R}$.

\end{itemize}
\end{frame}
%----------------------------------------------------------------%
\begin{frame}
\frametitle{Number Sets}
\Large
\vspace{-1cm}
\textbf{Natural Numbers} ($\mathbb{N}$)
\begin{itemize}
\item The whole numbers from 1 upwards. 

\item The set of natural numbers is 
\[\{1,2,3,4,5,6,\ldots\} \]
\item In some branches of mathematics, $0$ might be counted as a natural number.
\[\{0,1,2,3,4,5,6,\ldots\} \]
\end{itemize}
 \end{frame}
%----------------------------------------------------------------%

 \begin{frame}
 \frametitle{Number Sets}
 
 \Large
 \vspace{-0.5cm}
 \textbf{Integers} ($\mathbb{Z}$)
 \begin{itemize}
 \item The integers are all the whole numbers, all the negative whole numbers and zero.
 
 \item The set of integers is 
 \[\{\ldots,-4,-3,-2,-1,\;0,\;1,\;2,\;3,\ldots\} \]
 \item The notation $\mathbb{Z}$ is from the German word for numbers: \textit{Zahlen}. 
 \item All natural numbers are integers.
 \[ \mathbb{Q}  \subset \mathbb{Z}\]
 \end{itemize}
  \end{frame}
 %----------------------------------------------------------------%
 
  \begin{frame}
  \frametitle{Number Sets}
    \Large
  \vspace{-1.8cm}
  \textbf{Integers} ($\mathbb{Z}$)
  \begin{itemize}
 \item Natural numbers may also be referred to as positive integers, denoted $\mathbb{Z}^{+}$. \\(note the superscript)
 \item Negative integers are denoted $\mathbb{Z}^{-}$.
 \[\{\ldots,-4,-3,-2,-1\}\]
  \end{itemize}
   \end{frame}
  %----------------------------------------------------------------%
  
   \begin{frame}
   \frametitle{Number Sets}
     \Large
   \vspace{-1.8cm}
   \textbf{Integers} ($\mathbb{Z}$)
   \begin{itemize}
 \item 0 is neither positive nor negative. The following set of non-negative numbers \[\{0,1,2,3,4,5,6,\ldots\} \] might be denoted $0 \cup \mathbb{Z}^{+}$
 \item $\cup$ is the mathematical symbol for \textbf{union}.
 \end{itemize}
  \end{frame}

\begin{frame}
\frametitle{Number Sets}
\Large
\vspace{-1cm}
\textbf{Rational Numbers} ($\mathbb{Q}$)
\begin{itemize}
\item Rational numbers, also known as quotients, are numbers you can make by dividing one integer by another (but not dividing by zero). 
\item If a number can be expressed as one integer divided by another, it is a rational number.
\[ \mathbb{Q} = \left\{\; \frac{p}{q} \;\bigg| p \in \mathbb{Z},\; q \in \mathbb{Z},\; q \neq 0  \;   \right\}   \]
\end{itemize}
\end{frame}
%-------------------------------------------------------------- %   
\begin{frame}
\frametitle{Number Sets}
\Large
\vspace{-1cm}
\textbf{Rational Numbers} ($\mathbb{Q}$)
   \begin{itemize}
   \item All integers are rational numbers 
   \[ \mathbb{Z}  \subset \mathbb{Q}\]
   (and by extension all natural numbers are rational numbers too)
   \item Examples of rational numbers
   \[ 9500,\;7,\; \frac{1}{2} ,\; \frac{3}{7},\; -2.6 ,\; 0.001\] 
   \end{itemize}
   \end{frame}
   %-------------------------------------------------------------- %   
   \begin{frame}
   \frametitle{Number Sets}
   \Large
   \vspace{-1cm}
   \textbf{Irrational Numbers} 
      \begin{itemize}
   \item A number that can not be written as the ratio of two integers is known as an irrational number.
   \item Two famous examples of irrational numbers are $\pi$ and $\sqrt{2}$. 
   \[\pi = 3.141592\ldots\]
   \[\sqrt{2} = 1.41421\ldots\]
   \end{itemize}
    \end{frame}
\begin{frame}
\frametitle{Number Sets}
\Large
\vspace{-1cm}
\textbf{Real Numbers} ($\mathbb{R}$)
\begin{itemize}
\item Irrational numbers are types of real numbers.
\item Rational numbers are real numbers too.
\[ \mathbb{Q}  \subset \mathbb{R}\]

\item A real number is simply any point anywhere on the number line.
   \end{itemize}
    \end{frame}
\begin{frame}
\frametitle{Number Sets}
\Large
\vspace{-1cm}
\textbf{Real Numbers} ($\mathbb{R}$)
\begin{itemize}
\item There are numbers that are not real numbers, for example \textbf{imaginary numbers}, but we will not cover them in this presentation.
\end{itemize}

\end{frame}    
 \end{document}   
    