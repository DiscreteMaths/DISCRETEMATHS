\documentclass{beamer}

\usepackage{graphics}
\usepackage{amsmath}
\usepackage{amssymb}

\begin{document}

%Slide 1 Title Slide
\begin{frame}
\bigskip
{
\Huge
\[ \mbox{Logic Proposition}\]
}
{
\Large
\[ \mbox{kobriendublin.wordpress.com}\]
\[ \mbox{Youtube: StatsLabDublin}\]
}
\end{frame}





%--------------------------------------%
\begin{frame}
\frametitle{Logic Proposition}
Let $p$, $Q$ and $r$ be the following propositions concerning integers $n$:

\begin{itemize}
\item $p$ : n is a factor of 36 (2)
\item $q$ : n is a factor of 4 (2)
\item $r$ : n is a factor of 9 (3)
\end{itemize}
\end{frame}


\begin{frame}
\Large
\begin{center}
\begin{tabular}{|c||c|c|c|}
\hline 
\phantom{spa} \textbf{n} \phantom{spa}	& \phantom{spa}	\textbf{p} \phantom{spa}	& \phantom{spa}	\textbf{q} \phantom{spa}	& \phantom{spa}	\textbf{r} \phantom{spa}	\\ \hline \hline
1	&	1	&	1	&	1	\\ \hline
2	&	1	&	0	&	1	\\ \hline
3	&	0	&	1	&	1	\\ \hline
4	&	1	&	0	&	1	\\ \hline
6	&	0	&	0	&	1	\\ \hline
9	&	0	&	1	&	1	\\ \hline
12	&	0	&	0	&	1	\\ \hline
18	&	0	&	0	&	1	\\ \hline
36	&	0	&	0	&	1	\\
\hline 
\end{tabular} 
\end{center}
\end{frame}
%--------------------------------------%
\begin{frame}
\frametitle{Logic Proposition}
For each of the following compound statements, express it using the propositions P q and r, andng logical symbols, then given the truth table for it,

\begin{itemize}
\item[1)] If n is a factor of 36, then n is a factor of 4 or n is a factor of 9
\item[2)] If n is a factor of 4 or n is a factor of 9 then  n is a factor of 36
\end{itemize}
\end{frame}
%--------------------------------------%

\end{document}