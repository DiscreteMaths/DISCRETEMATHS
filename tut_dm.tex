\documentclass[]{article}
\voffset=-1.5cm
\oddsidemargin=0.0cm
\textwidth = 470pt
\usepackage[utf8]{inputenc}
\usepackage[english]{babel}
\usepackage{framed}

\usepackage{multicol}
\usepackage{amsmath}
\usepackage{amssymb}
\usepackage{enumerate}
\usepackage{multicol}

\begin{document}
\section*{Counting}



%%%%%%%%%%%%%%%%%%%%%%%%%%%%%%%%%%%%%%%%%%%%
\begin{enumerate}
\item
Let n be an element of the set $\{10, 11, 12, 13, 14, 15, 16, 17, 18, 19\}$,
and p and q be the propositions:
$p : n is even$, $q : n > 15$.
Draw up truth tables for the following statements and find the values of n for
which they are true:

\begin{enumerate}[(a)]
\item $p \vee \neg q$
\item $\neg p \wedge q$
\end{enumerate}
\end{enumerate}
%%%%%%%%%%%%%%%%%%%%%%%%%%%%%%%%%%%%%%%%%%
\section*{Logarithms and Exponentials}
\begin{enumerate}
    \item Evaluate the following expression:
\[ \mbox{Log}_4 64 + \mbox{Log}_5 625 + \mbox{Log}_9 3 \] 
\end{enumerate}


\section*{Mathematical Functions Exercise}

\begin{enumerate}
\item Complete the following table for the functions 
\begin{itemize}
\item $g(x) = \mbox{log}_3x$,
\item $h(x) =\sqrt[3]{x}$.
\end{itemize} 
\begin{center}

\begin{tabular}{|c||c|c|c|c|c|c|}
\hline $x$ &  \phantom{p}1\phantom{p}&  &  &  & 81 &  \\ 
\hline \phantom{p} $g(x)$ \phantom{p}&  & \phantom{p}1\phantom{p} & \phantom{p}2\phantom{p} &  &  &  \phantom{p}5\phantom{p} \\ 
\hline \phantom{p}$h(x)$ \phantom{p}&  &  &  &  3.00 & \phantom{p}\phantom{p}\phantom{p}  &  \\ 
\hline 
\end{tabular} 
\end{center}

Express your answers to 2 decimal places only.

%------------------------------------------ %


\item 
Complete the following table for the functions 
\begin{itemize}
\item $g(x) = \mbox{log}_3x$,
\item $h(x) =\sqrt[3]{x}$.
\end{itemize} 
\begin{center}

\begin{tabular}{|c||c|c|c|c|c|c|}
\hline $x$ & 1	&	3	&	9	&	27	&	81	&	243	\\ \hline
\phantom{p} $g(x)$ \phantom{p}&0	&	1	&	2	&	3	&	4	&	5	\\ \hline
\phantom{p}$h(x)$ \phantom{p}&  1.00	&	1.44	&	2.08	&	3.00	&	4.33	&	6.24	\\ \hline
\end{tabular} 
\end{center}

%Express your answers to 2 decimal places only.

\end{enumerate}
%%%%%%%%%%%%%%%%%%%%%%%%%%%%%%%%%%
\section*{Trees}
\begin{enumerate}
\item Draw two non-isomorphic graphs with the following degree sequence.
\[ 4,3,3,2,2,2,2,1,1\]
\end{enumerate}
%%%%%%%%%%%%%%%%%%%%%%%%%%%%%%%%%%

%---------------------------------------------------%
\begin{enumerate}[(a)]
\item 	Show that
	\[ ^{n}C_0  = 1 \]
	
	\textbf{Solution: }
	\[ ^{n}C_0  = {n! \over 0!  \times (n-0)!} =  {n! \over n!} = 1 \]
	

%---------------------------------------------------%
\item 
	Show that
	\[ ^{n}C_1  = n \]
	
	\textbf{Solution: }
	\[ ^{n}C_1  = {n! \over 1!  \times (n-1)!} =  {n \times (n-1)! \over (n-1)!} = n \]
	


%---------------------------------------------------%
\item 	Compute $ ^{7}C_2  $\\
	
	\textbf{Solution: }
	\[ ^{7}C_2  = {7! \over 2!  \times (7-2)!} =  {7 \times 6 \times 5! \over 2! \times 5!} = {42 \over 2} =21  \]
	

%---------------------------------------------------%
\item Compute $ ^{11}C_1  $\\
	
	\textbf{Solution: }
	\[ ^{11}C_1  = {11! \over 1!  \times 10!} =  {11 \times 10! \over 1 \times 10!} = 11 \]
	
\end{enumerate}	
	
\section*{Permutations}

\begin{enumerate}
    \item How many anagrams (permutations of the letters) are there of the following words
\begin{enumerate}[(a)]
\item ANSWER
\item PERMUTE
\item ANAGRAM
\item LITTLE
\end{enumerate}

\end{enumerate}
%---------------------------------------%
\section*{Relations and Digraphs}
%2002 Question 7
\begin{enumerate}
\item 
%2002 Question 7
Let S be a set and let R be a relation on S
Explain what it means to say thet $\mathcal{R}$ is

\begin{itemize}
\item[(i)] reflexive
\item[(ii)] symmetrix
\item[(iii)] anti-symmetric
\item[(iv)] Transitive
\end{itemize}



\item  Let $\mathcal{S}$ be the set $\{2,3,4,5,6,7\}$ and a relation $\mathcal{R}$ is defined between the elements of $\mathcal{S}$ by
    \begin{center}
\begin{quote}
x is related to y if $|x - y| \in \{0,3\}$.
\end{quote}
\end{center}
\begin{itemize}
\item[(i)] Draw the relationship digraph. [3 Marks]
\item[(ii)] Determine whether or not $\mathcal{R}$ is reflexive, symmetric or transitive. In cases
where one of these properties does not hold give an example to show that
it does not hold.n 
\end{itemize}


\item Let $A=\{0,1,2\}$ and $R=\{ (0,0),(0,1),(0,2),(1,1), (1,2), (2,2)\}$
and $S=\{(0,0),(1,1),(2,2)\}$ be 2 relations on A. Show that

\begin{itemize}
\item[(i)] R is a partial order relation.
\item[(ii)] S is an equivalence relation.
\end{itemize}

\item 
%2002 Question 7
Let S be a set and let R be a relation on S
Explain what it means to say that $\mathcal{R}$ is

\begin{itemize}
	\item[(i)] reflexive
	\item[(ii)] symmetrix
	\item[(iii)] anti-symmetric
	\item[(iv)] Transitive
\end{itemize}

\end{enumerate}
%%%%%%%%%%%%%%%%%%%%%%%%%%%%%%%%%%
\section*{Sequences and Series}

\begin{enumerate}
\item Compute the following summation

\[ \sum^{i=100}_{i=25} i^2 + 3i -5)\]

\end{enumerate}

\end{document}