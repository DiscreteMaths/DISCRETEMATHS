\documentclass[]{article}
\usepackage{framed}
\usepackage{amsmath}
\usepackage{amssymb}
\usepackage{graphicx}
\usepackage{multicol}
%opening

\begin{document}
\subsection*{Chapter 6: Digraphs and Relations}
%--------------------------------------------- %
%------------------------------------------ %
Definition 1.8 If a relation $\mathcal{R}$ cle5necl on a set S is reflexive, symmetric and transitive, 
then we say that $\mathcal{R}$ is an equivalence relation on S.
The relation ":" defined on any set S is an example of en equivalence relation
on S. 

Indeed equivalence relations can be seen as generalizations of the "equality"
relation, lf two elements of a set are related by an equivalence relation, then they
are in some sense equivalent.
%------------------------------------------------------



\noindent \textbf{Example 1.9} Let S be the set of all 3»bit binary strings. Define a relation $\mathcal{R}$ on
S by saying that two binary strings are related if they contain the same number
of ones. Thus $100 R 010$ but 100 is not related to 101. Then $\mathcal{R}$ is an equivalence
relation on S.
%------------------------------------------------------


\noindent \textbf{Definition 1.10} Let $\mathcal{R}$ be an equivalence relation defined on a. set S and x E S.
Then the equivalence class of m is the subset of S containing all elements of S
which are related to x. We denote this by [:c]. Thus
{2*] : {y E S : 34722:}.
%------------------------------------------------------ 
Example 1.11 In Example 1.9, the equivalence class of a 3-bit binary string n: is
the set of all 3-bit binary strings which contain the same number of ones as ac. This
gives us four distinct eqivalence classes: [000] : {000}, [100] : {100,010, 001] :
[010} = {001}, [110] : {110, 011, 101} : {011] : [101}, and {111} : {lll}.
%------------------------------------------------------ 
Example 1.11 has the following nice structural properties.

\begin{itemize}
\item Every element of S belongs to exactly one of the four distinct equivalence
classes.

\item Two elements ofS are related if and only if they belong Lo the same equivalence
class.
\end{itemize}
%------------------------------------------------------ 
We shall see that these properties hold for all equivalence relations. We first need
one more definition.
%------------------------------------------------------ 


\noindent \textbf{Definition 1.12} Let S be a set and T : {5],5;, .. ,5,,,} be a set of uomempty
subsets of S. Then T is said to be a \textbf{partition} of S if every element of 5 belongs
to exactly one element of T,


%------------------------------------------------------ 
\noindent \textbf{Example 1.13} In Example 1.9, let T be the set whose elements are the four distinct
equivalence classes. Thus
\[T : {{000},{00,010,001},{110,011,101},{111}}\]
Then T is a partition of S.
Our promised result on equivalence relations is:
%------------------------------------------------------ 
Theorem 1.14 Let 72, be an equivalence relation on a set S. Then:

\begin{itemize}
\item The set of distinct equivalence classes of 72 in S is a partition of S.
\item Two elements ofS are related if and only lf they belong to the same equivalence
class.
\end{itemize}
%----------------------- %
The proof of this result is beyond the scope of this course Theorem 1.14 tells us
that if we have an equivalence relation R defined on a set S, then we can think
of the equivalence classes as the subsets of S containing all the elements which are
“\textit{equivalent}" to each under this relation. 
%---------------- %
The relationship digraph corresponding
to the equivalence relation falls into distinct components, one component for each
equivalence class. Inside each component every vertex is incident to a directed loop
and every pair of vertices are joined by a directed cycle of length two.
%--------------------------------------------- %
\subsubsection*{Partial orders}
In the previous subsection we looked an equivalence relations as a generalisation of
the relation "=”. In this subsection we will consider relations which gerneralise the
relation “S”.
%--------------------------------------------- %
Definition 1.15 We say that a relation 72 on a. set S is \textbf{\textit{anti-symmetric}} if for all
z,y 6 S such that 1/R.y and yR.z:, we have z : y. 

Thus 72, is anti-symmetric if and
only if the relationship digraph of 'R has no directed cycles of length two,
lt follows that the relations described in Examples 1.3, 1.5 and 1.9 are not antl-
symmetric. 
%---------------- %
Examples of relations which are antisymmetric are "g" on any set of
numbers, and "Q’ on the set of all subsets of a given set,
%--------------------------------------------- %
Definition 1.16 
We say that a relation $\mathcal{R}$ on a set S is a \textit{\textbf{partial order}}- if it is
reflexive, anti-symmetric and transitive. 

We say further that $\mathcal{R}$ is an order if it is a partial order with the additional property that for any two elements m, y E S,
either z'Ry or y'R,x.
%--------------------------------------------- %
It follows that S is an example of an order on any set of numbers. Au example
of a partial Order which is not an order is the following.
%--------------------------------------------- %
\textbf{Example 1.17} Let $U : \{1, 2,3\}$ and S : ’P[U) be the set of all subsets of U. Then
Q is a partial order on S. It is not an order, however, since if we let X : {1,*2} and
Y : {1,3} then X Q Y and Y Q X. Thus X and Y are two elements of S such
that X is not related to Y, aud Y is not related no X.

\begin{figure}
\centering
\includegraphics[width=0.7\linewidth]{./Digraph3}
\caption{}
\label{fig:Digraph3}
\end{figure}

Figure 1.3: The relationship digraph of Example 1.20
\newpage
%--------------------------------------------- %
\subsection*{Relations and Cartesian products}
ln the previous subsections we considered relations between the elements of a single
set S. We now extend this concept to define relations between the elements of two
sets.
%--------------------------------------------- %
\textbf{Defiition 1.18} Let X and Y be sets. Then a relation 72 from X to Y is a rule
which compares any two elements $x \in X$ and $y in Y$, and tells us either that r is
related to y or that z is not related to y.
%--------------------------------------------- %
Relations between two sets are fundamental to databases, as can be seen from
the following example.


\end{document}