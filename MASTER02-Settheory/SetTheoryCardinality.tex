\documentclass{beamer}

\usepackage{amsmath}
\usepackage{amssymb}

\begin{document}
\begin{frame}
\huge
\[ \mbox{Discrete Mathematics: Set Theory}\]
\[ \mbox{Cardinality of a Set}\]
\Large
\[ \mbox{www.Stats-Lab.com}\]
\[ \mbox{Twitter: StatsLabDublin}\]
\end{frame}
%---------------------------------------------------- %
\begin{frame}
\frametitle{Cardinality of a Set}
\Large
\vspace{-1cm}
\begin{itemize}
\item The cardinality of a set A is the number of elements in A, which is written as $|A|$.
\item An element of a set is any one of the \textbf{distinct} objects that make up that set.

\end{itemize}
\end{frame}
%---------------------------------------------------- %
%---------------------------------------------------- %
\begin{frame}
\frametitle{Cardinality of a Set}
\Large
\vspace{-1cm}
\begin{itemize}

\item Note that this vertical-bar notation looks the same as absolute value notation, 
but the meaning of cardinality is different from absolute value.

\item In particular, absolute value operates on numbers (e.g., $|-4| = 4$) 
while cardinality operates on sets (e.g., $|\{-4\}| = 1$).
\end{itemize}
\end{frame}
%---------------------------------------------------- %
%---------------------------------------------------- %
\begin{frame}
\frametitle{Examples of cardinality}
\LARGE
\vspace{-1cm}
\textbf{Examples}
\begin{itemize}
\item[(i)] $|\{2,6,7\}|$
\item[(ii)] $|\{5,6,5,2,2,6,5,1,1,1\}|$
\item[(iii)] $|\{ \;\}| = 0$. %The empty set has no elements.
\item[(iv)] $|\{\{1,2\},\{3,4\}\}| = 2$. 
%In this case the two elements of $\{\{1,2\},\{3,4\}\}$ are themselves sets: $\{1,2\}$ and $\{3,4\}$.
\end{itemize}
\end{frame}

%---------------------------------------------------- %
%---------------------------------------------------- %
\begin{frame}
\frametitle{Examples of cardinality}
\LARGE
\vspace{-1cm}
\textbf{Examples}
\begin{itemize}
\item[(i)] $|\{2,6,7\}| $
\vspace{2cm}
\item[(ii)] $|\{5,6,5,2,2,6,5,1,1,1\}| $
\end{itemize}
\end{frame}

%---------------------------------------------------- %
%---------------------------------------------------- %
\begin{frame}
\frametitle{Examples of cardinality}
\LARGE
\vspace{-1cm}
\textbf{Examples}
\begin{itemize}
\item[(i)] $|\{2,6,7\}| = 3 $
\vspace{2cm}
\item[(ii)] $|\{5,6,5,2,2,6,5,1,1,1\}| = |\{1,2,5,6\}| $
\end{itemize}
\end{frame}

%---------------------------------------------------- %
%---------------------------------------------------- %
\begin{frame}
\frametitle{Examples of cardinality}
\LARGE
\vspace{-1cm}
\textbf{Examples}
\begin{itemize}
\item[(i)] $|\{2,6,7\}| $
\vspace{2cm}
\item[(ii)] $|\{5,6,5,2,2,6,5,1,1,1\}| = |\{1,2,5,6\}| $
\end{itemize}
\end{frame}

%---------------------------------------------------- %
%---------------------------------------------------- %
\begin{frame}
\frametitle{Examples of cardinality}
\LARGE
\vspace{-1cm}
\textbf{Examples}
\begin{itemize}
\item[(i)] $|\{2,6,7\}| = 3$
\vspace{2cm}
\item[(ii)] $|\{5,6,5,2,2,6,5,1,1,1\}| = |\{1,2,5,6\}| = 4$
\end{itemize}
\end{frame}
%---------------------------------------------------- %
%---------------------------------------------------- %
\begin{frame}
\frametitle{Examples of cardinality}
\LARGE
\vspace{-1cm}
\textbf{Examples}
\begin{itemize}
\item[(iii)] $|\{ \; \}| $
\vspace{2cm}
\item[(iv)] $|\{\{1,2\},\{3,4\}\}| $.
\end{itemize}
\end{frame}

%---------------------------------------------------- %
%---------------------------------------------------- %
\begin{frame}
\frametitle{Examples of cardinality}
\LARGE
\vspace{-1cm}
\textbf{Examples}
\begin{itemize}
\item[(iii)] $|\{ \; \}| = 0$. \\ The empty set has no elements.
\vspace{0.4cm} 
\item[(iv)] $|\{\{1,2\},\{3,4\}\}| = 2$. \\ \vspace{0.4cm} In this case the two elements of $\{\{1,2\},\{3,4\}\}$ are themselves sets: $\{1,2\}$ and $\{3,4\}$.
\end{itemize}
\end{frame}
\begin{frame}
END
\end{frame}
\end{document}
