%------------------------------------------------------------------------%



\section{Set Theory Operations}
\begin{enumerate}
\item The Universal Set $\mathcal{U}$
\item Union
\item Intersection
\item Set Difference
\item Relative Difference
\end{enumerate}

\section*{Set Operations}
\begin{itemize}
\item Union ($\cup$) - also known as the \textbf{OR} operator. A union signifies a bringing together. The union of the sets A and B consists of the elements that are in either A or B.
\item Intersection ($\cap$) - also known as the \textbf{AND} operator. An intersection is where two things meet. The intersection of the sets A and B consists of the elements that in both A and B.
\item Complement ($A^{\prime}$ or $A^{c}$) - The complement of the set A consists of all of the elements in the universal set that are not elements of A.
\end{itemize}

%--------------------------------------------------------%
\subsubsection*{Complements}
\begin{itemize}

\item The Complements of A and B are the elements of the universal set not contained in A and B.

\item The complements are denoted $\mathcal{A}^{\prime}$ and $\mathcal{B}^{\prime}$
\[ \mathcal{A}^{\prime} = \{4,6,8,9\}, \]
\[ \mathcal{B}^{\prime} = \{1,3,5,7,9\}, \]

\end{itemize}
\section*{Complement of a Set}
%(2.3.1) 
Consider the universal set $U$ such that
\[U=\{2,4,6,8,10,12,15\} \]
For each of the sets $A$,$B$,$C$ and $D$, specify the complement sets.
{

\begin{center}
\begin{tabular}{|c|c|}
\hline
Set &\phantom{sp} Complement \phantom{sp}\\
\hline \phantom{sp} $A=\{4,6,12,15\}$ \phantom{sp} &
$A^{\prime}=\{2,8,10\}$ \\ \hline $B=\{4,8,10,15\}$ & \\ \hline
$C=\{2,6,12,15\}$ & \\ \hline $D=\{8,10,15\}$ & \\ \hline

\end{tabular}
\end{center}
}
%--------------------------------------------------------%

\subsubsection*{Intersection}
\begin{itemize}

\item Intersection of two sets describes the elements that are members of both the specified Sets

\item The intersection is denoted $\mathcal{A\cap B}$ 
\[ \mathcal{A\cap B} = \{2\}\]

\item only one element is a member of both A and B.
\end{itemize}
%--------------------------------------------------------%

\subsubsection*{Set Difference}
\begin{itemize}

\item The Set Difference of A with regard to B are list of elements of A not contained by B.

\item The complements are denoted $\mathcal{A-B}$ and $\mathcal{B-A}$
\[ \mathcal{A-B} = \{1,3,5,7\}, \]

\[ \mathcal{B-A} = \{4,6,8\}, \]
\end{itemize}





%--------------------------------------------------------%
\subsubsection*{Relative Difference}
\begin{itemize}
\item $ A \oplus B$
\end{itemize}
%--------------------------------------------------------%


