\documentclass[MASTER.tex]{subfiles}
\begin{document}

\end{document}



   




%-------------------------------------% Ellipsis

When using Ellipsis, it should be clear what the pattern is

%-------------------------------------%


%-----------Reference to section 2.2.3 Power Sets




%-------------------------------------%

 %----(Reference to Section 2.2.2 Cardinality)



%-------------------------------------% % Complement of a set


%--------------------------------------%
\subsection*{ Three Sets }

%- Section 2.3.5 %- Associative Law %- Distribution Law





%-------------------------------------% %- Section 3
Propositional Logic A statement is a declarative sentence that
is either true or false.
\begin{itemize}
\item $\tilde q$ not q \item $p \vee q$ \item $p \wedge \tilde
q$
\end{itemize}

%-------------------------------%
------------------------------------------------
Question 1


$\{ s :  s is an odd integer and 2 \leq s \leq 10 \}$
$\{ 2m :  m \element Z and 5 \leq m \leq 10 \}$
$\{ 2^t :  t \element Z and 0 \leq t \leq 5 \}$

------------------------------------------------
Question 2

\{12,13,14,15,16,17\}
{0,5,-5,10,-10,15,-15,.....\}
{6,8,10,12,14,16,18\}
Question 5

-----------------------------------------------------


--------------------------------------------

Let A, B be subsets of the universal set \mathcal{U}.

Use membership tables to prove De Morgan's Laws.

---------------------------------------------
Draw a venn diagram to show three subsets A,B and C of a universal set U intersecting in
the most general way?
How are sets $X$ and $Z$ related?
Can you describe each of the subsets X,Y and Z in terms  of the
sets A,B,C using the operations union intersection and set complement.

Construct Membership tables for each of the sets
(A-B) - C
A-(B- C)

(A-B) -C = A-(B-C)
A



\begin{itemize}
\item[a.] (1 mark) Write out the sample space for the outcomes for both players A and B.
\item[b.] (1 mark) Write out the sample space for the outcomes of C, where C is the difference of the two scores (i.e. B-A)
\item[c.] (1 mark) Are the sample points for the sample space of C equally probable? Provide a brief justification for your answer.
\end{itemize}

%----------------------------------------------------------%
\newpage
\section*{Section B: Set Operations}
\begin{itemize}
\item[B.1] complement of A $A^{\prime}$
\item[B.2] Union $A \cup B$
\item[B.3] Intersection $A \cap B$
\item[B.4] Relative Difference $A \otimes B$
\item[A.5]
\item[A.6]
\item[A.7]
\item[A.8]
\end{itemize}
\newpage


\begin{itemize}
\item Specifying Sets
\item Listing Method
\item Rules of Inclusion method
\end{itemize}


\begin{itemize}
\item Subsets Notation of a subset
\item Cardinality of a set
\item Power of a set
\end{itemize}

\subsection*{Operation on Sets}

\begin{itemize}
\item The complement of Set
\item Binary Operations
\begin{itemize}
\item Union
\item Intersection
\end{itemize}
\item Membership tables
\item Laws for Combining Sets
\end{itemize}
%----------------------------------------------------------%
\subsection*{Membership Tables}
Using membership tables
\begin{tabular}{|ccc|c|c|c|}
  \hline
  % after \\: \hline or \cline{col1-col2} \cline{col3-col4} ...
  A & B & C & x & y & z \\\hline
  0 & 0 & 0 & 1 & 1 & 1 \\
  0 & 0 & 1 & 0 & 0 & 1 \\
  0 & 1 & 0 & 0 & 0 & 1 \\
  0 & 1 & 1 & 0 & 0 & 1 \\
  1 & 0 & 0 & 1 & 0 & 1 \\
  1 & 0 & 1 & 1 & 0 & 1 \\
  1 & 1 & 0 & 0 & 0 & 1 \\
  1 & 1 & 1 & 1 & 0 & 1 \\
  \hline
\end{tabular}

%----------------------------------------------------------%
\section*{Section B: Hexadecimal Numbers}
\begin{itemize}
\item[B.1]
\item[B.2]
\item[B.3]
\item[B.4]
\item[B.5]
\item[B.6]
\item[B.7]
\item[B.8]
\end{itemize}
\newpage


\subsection*{Associative Laws}
\[ (A \cup B) \cup C =  A \cup (B \cup C)  \]
\[ (A \cap B) \cap C =  A \cap (B \cap C)  \]

\subsection*{Distributive Laws}
\[ (A \cup B) \cap C =  (A \cup B) \cap (A \cup C)  \]
\[ (A \cap B) \cup C =  (A \cap B) \cup (A \cap C)  \]


\[ (A \cup B) \cap B^{\prime} \]
%----------------------------------------------------------%
\section*{Section C: Real and Rational Numbers}
\begin{itemize}
\item[C.1]
\item[C.2]
\item[C.3]
\item[C.4]
\item[C.5]
\item[C.6]
\item[C.7]
\item[C.8]
\end{itemize}
\newpage
%----------------------------------------------------------%
\newpage
\section*{Formulae}
\begin{itemize}




\item Bayes' Theorem:
\begin{equation*}
P(B|A)=\frac{P\left(A|B\right) \times P(B) }{P\left( A\right) }.
\end{equation*}



\item Binomial probability distribution:
\begin{equation*}
P(X = k) = ^{n}C_{k} \times p^{k} \times \left( 1-p\right) ^{n-k}\qquad \left( \text{where}\qquad
^{n}C_{k} =\frac{n!}{k!\left(n-k\right) !}. \right)
\end{equation*}

\item Poisson probability distribution:
\begin{equation*}
P(X = k) =\frac{m^{k}\mathrm{e}^{-m}}{k!}.
\end{equation*}
\end{itemize}

\end{document} 
