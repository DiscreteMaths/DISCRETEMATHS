\documentclass{beamer}

\usepackage{graphicx}
\usepackage{framed}
\begin{document}


% What are Binary Numbers ?
% What are Hexadecimal Numbers?
% What are primes numbers?
% Floating Point Notation
%--------------------------------%
\begin{frame}
\frametitle{Binary Numbers}
\Large
\begin{itemize}
\item The \textbf{binary} number system is a way of representing numbers using only $0$ and $1$ as symbols. Numbers from this system are called binary numbers.
\item Often numbers are constructed using a combination of the following ten symbols.
\[ 0,1,2,3,4,5,6,7,8,9\]. We call these numbers decimal numbers.
\item It is possible to represent a decimal number as a binary numbers, and vice versa.
\end{itemize}
\end{frame}
%--------------------------------%
\begin{frame}
\frametitle{Binary Numbers in Computing}


For computers, binary numbers are great stuff because:
\begin{itemize}
\item They are simple to work with -- no big addition tables and multiplication tables to learn, just do the same things over and over, very fast.
\item They just use two values of voltage, magnetism, or other signal, which makes the hardware easier to design and less prone to mechanical errors.
\end{itemize}


\end{frame}


\begin{frame}
\frametitle{Base of a number System}
\Large
\textbf{Base of a Number System}
\begin{itemize}
\item The base of a number system is the number of symbols that that system uses. The binary number system has a base of 2. The decimal number system has a base of 10.
\item There are other number systems. Two systems commonly used in computer sciences it the octal system (base 8)
, and the hexadecimal number sytem (base 16).
\end{itemize}

\end{frame}
%--------------------------------%
\begin{frame}
\frametitle{Base of a number System}
\Large
\begin{itemize}
\item To clarify which number system is being used, the convention is to write the base as a subscript.
\item The decimal number $5$ is represented as $101$ in the binary system.
\[5_{10} = 101_2\]
\item We will show how to determine the binary equivalents of decimal numbers shortly.
\end{itemize}


\end{frame}
%--------------------------------%
\begin{frame}



\end{frame}
%--------------------------------%
\begin{frame}



\end{frame}
%--------------------------------%
\end{document}
