\documentclass[]{article}
\voffset=-1.5cm
\oddsidemargin=0.0cm
\textwidth = 470pt
\usepackage[utf8]{inputenc}
\usepackage[english]{babel}
\usepackage{framed}

\usepackage{multicol}
\usepackage{amsmath}
\usepackage{amssymb}
\usepackage{enumerate}
\usepackage{multicol}




%opening
\title{Binary Arithmetic - Tutorial Sheet}
\author{www.MathsResource.com}

\begin{document}

\maketitle

\Large

\section*{Binary Arithmetic : addition, multiplication and bit-borrowing}
%http://www.csgnetwork.com/binaddsubcalc.html
%(See section 1.1.3 of the text)
\begin{enumerate}
	\item Perform the following binary additions.
	\begin{multicols}{2}
		\begin{itemize}
			\item[a)] $(110101)_{2}$ + $(1010111)_{2}$
			\item[b)] $(1010101)_{2}$ + $(101010)_{2}$
			\item[c)] $(11001010)_{2}$ + $(10110101)_{2}$
			\item[d)] $(1011001)_{2}$ + $(111010)_{2}$
		\end{itemize}
	\end{multicols}
	\item Perform the following binary multiplications.
	\begin{multicols}{2}
		\begin{itemize}
			\item[a)] $(1001)_{2}\times( 1000)_{2}$  % 9 by 8
			\item[b)] $(101)_{2}\times(1101)_{2}$ % 5 by 11
			\item[c)] $(111)_{2}\times(1111)_{2}$ % 7 by 15
			\item[d)] $(10000)_{2}\times(11001)_{2}$    %16 by 25
		\end{itemize}
		\end{multicols}		
	\item Perform the following binary subtractions (using bit-borrowing).
	\begin{multicols}{2}
		\begin{itemize}
			\item[a)] $(110101)_{2}$ - $(1010111)_{2}$
			\item[b)] $(1010101)_{2}$ - $(101010)_{2}$
			\item[c)] $(11001010)_{2}$ - $(10110101)_{2}$
			\item[d)] $(1011001)_{2}$ - $(111010)_{2}$
		\end{itemize}

	
	

	\end{multicols}
	
	\newpage
	
	\item Perform the following binary multiplications.
	%\begin{multicols}{2}
	%\begin{itemize}
	%\item[a)] $(1001000)_{2} \div ( 1000)_{2}$
	%\item[b)] $(101101)_{2} \div (1001)_{2}$
	%\item[c)] $(1001011000)_{2} \div (101000)_{2}$
	%\item[d)] $(1100000)_{2} \div (10000)_{2}$
	%\end{itemize}
	%\end{multicols}
	
	\begin{enumerate}
		\item Which of the following binary numbers is the result of this binary division: $(10)_{2} \times ( 1101)_{2}$. % % (2) /  (13)
		\begin{multicols}{2}
			\begin{itemize}
				\item[a)] $(11010)_{2}$ %26
				\item[b)] $(11100)_{2}$ %28
				\item[c)] $(10101)_{2}$ %21
				\item[d)] $(11011)_2$ %27
			\end{itemize}
		\end{multicols}
		\item Which of the following binary numbers is the result of this binary division: $(101010)_{2} \times( 111 )_{2}$. % (4) /  (6)
		\begin{multicols}{2}
			\begin{itemize}
				\item[a)] $(11000)_{2}$ %24
				\item[b)] $(11001)_{2}$ %25
				\item[c)] $(10101)_{2}$ %21
				\item[d)] $(11011)_2$ %27
			\end{itemize}
		\end{multicols}
		\item Which of the following binary numbers is the result of this binary division: $(1001110)_{2}\times ( 1101 )_{2}$. % (9) /  (3)
		\begin{multicols}{2}
			\begin{itemize}
				\item[a)] $(11000)_{2}$ %24
				\item[b)] $(11001)_{2}$ %25
				\item[c)] $(10101)_{2}$ %21
				\item[d)] $(11011)_2$ %27
			\end{itemize}
		\end{multicols}
	\end{enumerate}
	
	\newpage
	%----------------------------------------------------------------%
	
	\item Perform the following binary division exercises.
	%\begin{multicols}{2}
	%\begin{itemize}
	%\item[a)] $(1001000)_{2} \div ( 1000)_{2}$
	%\item[b)] $(101101)_{2} \div (1001)_{2}$
	%\item[c)] $(1001011000)_{2} \div (101000)_{2}$
	%\item[d)] $(1100000)_{2} \div (10000)_{2}$
	%\end{itemize}
	%\end{multicols}
	
	\begin{enumerate}
		\item Which of the following binary numbers is the result of this binary division: $(111001)_{2} \div ( 10011)_{2}$. % (57) /  (19)
		\begin{multicols}{2}
			\begin{itemize}
				\item[a)] $(10)_2$ %2
				\item[b)] $(11)_{2}$ %3
				\item[c)] $(100)_{2}$ %4
				\item[d)] $(101)_{2}$ %5
			\end{itemize}
		\end{multicols}
		\item Which of the following binary numbers is the result of this binary division: $(101010)_{2} \div ( 111 )_{2}$. % (42) /  (7)
		\begin{multicols}{2}
			\begin{itemize}
				\item[a)] $(11)_2$ %3
				\item[b)] $(100)_{2}$ %4
				\item[c)] $(101)_{2}$ %5
				\item[d)] $(110)_{2}$ %6
			\end{itemize}
		\end{multicols}
		\item Which of the following binary numbers is the result of this binary division: $(1001110)_{2} \div ( 1101 )_{2}$. % (78) /  (13)
		\begin{multicols}{2}
			\begin{itemize}
				
				\item[a)] $(100)_{2}$ %4
				\item[b)] $(110)_{2}$ %6
				\item[c)] $(111)_{2}$ %7
				\item[d)] $(1001)_2$ %9
			\end{itemize}
		\end{multicols}
	\end{enumerate}
	
	
\end{enumerate}
\end{document}