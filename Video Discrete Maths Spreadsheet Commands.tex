\documentclass{beamer}

\usepackage{amsmath}
\usepackage{framed}
\usepackage{amssymb}
\usepackage{graphics}

\begin{document}
%----------------------------------------------------------------%
\begin{frame}

{
\Huge
\[\mbox{Mathematics for Computing}\]
\LARGE
\[\mbox{Spreadsheet Commands for Number Systems}\]
}
{
\Large
\[\mbox{kobriendublin.wordpress.com}\]
\[\mbox{Twitter: @StatsLabDublin}\]
}
\end{frame}
%----------------------------------------------------------------%
%----------------------------------------------------------------%
\begin{frame}
\frametitle{Number Systems}

\Large
Simple spreadsheet commands that can used to determine the equivalent decimal numbers for binary or hexadecimal numbers, and vice versa.


\bigskip
\begin{itemize}
\item \texttt{bin2dec()} : Binary to Decimal 
\item \texttt{dec2bin()} : Decimal to Binary 
\item \texttt{hex2dec()} : Hexadecimal to Decimal 
\item \texttt{dec2hex()} : Decimal to Hexadecimal
\item \texttt{hex2bin()} : Hexadecimal to Binary 
\item \texttt{bin2hex()} : Binary to Hexadecimal
\end{itemize}

 \end{frame}
 %----------------------------------------------------------------%
 %----------------------------------------------------------------%
 \begin{frame}
 \frametitle{Number Systems}
 \Large
 \textbf{Constraints}:
 \begin{itemize}
 \item The commands related to binary numbers do not work for any number greater than 511$_{(10)}$
 \item That is to say, the maximum number of binary digits is 9.
 \item (You can specify negative numbers also, but not relevant to most college courses)
 \end{itemize}

\end{frame}
%----------------------------------------------------------------%
%----------------------------------------------------------------%
\begin{frame}
\frametitle{Number Systems}
\Large
\textbf{Exercises}:
\begin{itemize}
\item \textbf{1101011}$_{2}$ :  Find decimal and hexadecimal equivalents,
\item \textbf{234}$_{10}$ : Find binary and hexadecimal equivalent,
\item \textbf{A5E}$_{16}$ : Find binary and decimal equivalent.
\item \textbf{250000}$_{10}$ : Find hexadecimal equivalent.
\end{itemize}
\end{frame}
%----------------------------------------------------------------%
%----------------------------------------------------------------%
\begin{frame}
\frametitle{Number Systems}

\end{frame}
%----------------------------------------------------------------%
%----------------------------------------------------------------%
\begin{frame}
\frametitle{Financial Mathematics}

\end{frame}
%----------------------------------------------------------------%
\end{document}