\documentclass[12 pt]{article}

%opening
\title{Hibernia College}
\author{}
\usepackage{amsmath}
\begin{document}

\maketitle

% 1.1 Started 2010 Q3 (Propositions and Sets)
% 1.2 Started (Membership Tables)
% 1.3 READY - Membership Tables for Laws of Logic
% 1.4 ..
% 1.5 ..
% 1.6 ..
% 1.7 READY  - Contrapositives
% 1.8 Logic Networks - Started - Need Graphics

% 2.1 Arrow Diagram
% 2.2 Arrow Diagram 2004 Q3
% 2.3
% 2.4 Inverse of a Function (2010 Q4)
% 2.5 Floor and Ceiling Function
% 2.6 Floor and Ceiling Function 2004 Q3
% 2.7 READY Power Functions
% 2.8
% 2.9
% 2.10
% 2.11
% 2.12 Logarithms 

\newpage
\section*{Part 2 : Functions}







%--------------------------------------------------- %
%--------------------------------------------------- %
\subsection*{2.1 2010 Question 4 part A}

A function $f : X \rightarrow Y$, where $X = \{p,q,r,s\}$ and $Y=\{1,2,3,4,5\}$ is given by the subset of $ X \times Y$ ,
i.e. $\{(q,3),(r,3),(p,5),(s,2)\}$.

\begin{itemize}
\item[i.] Show $f$ as an arrow diagram.
\item[ii.] State the domain, the co-domain and range of $f$.
\item[iii.] Say why $f$ does not have the \textbf{one-to-one} property and why $f$ does not have the \textbf{onto} property, giving a specific counter example in each case.
\end{itemize}

\textbf{Solutions}
\begin{itemize}
\item[i.] (Done on whiteboard)
\item[ii.] Domain, Co-Domain and Range
\begin{itemize}
\item[Domain] $\{a,b,c,d\}$,
\item[Co-Domain] $\{1,2,3,4,5\}$
\item[Range] $\{2,3,5\}$
\end{itemize}
\end{itemize}

\begin{itemize}
\item Onto - Range is equivalent to Co-domain.
\item no element of co-domain unused.
\item one to one - Each element of domain has one image in the co-domain. Each image has only one ancestor.
\end{itemize}

%--------------------------------------------------- %
%--------------------------------------------------- %
\subsection*{2.2 2010 Question 4 part A}


Consider the functions $f: \mathcal{R} \rightarrow \mathcal{Z}$ and $g: \mathcal{r} \rightarrow \mathcal{R}$ given by
\[f(X) = \rfloor x-1 \lfloor \]
\[g(X) = | x-1 | \]

\begin{itemize}
\item[i.] Write down the domain, co-domain and range of f and g(x). 
\item[ii.] For each function, say whether or not it is one to one, justifying your
answer. 
\item[iii.] For each function, say whether or not it is onto, justifying your answer.
\end{itemize}

% \[ \mathcal{N} \in  \mathcal{Z} \in \mathcal{Q} \in \mathcal{R}]

\begin{itemize}
\item $\mathcal{Z}$ : Integers $ \{\ldots,-3,-2,-1,0,1,2,3, \ldots\}$
\item $\mathcal{Z}^{-}$ : Negative Integers $\{\ldots,-3,-2,-1\}$
\item $\mathcal{Z}^{+}$ : Positive Integers $\{0,1,2,3, \ldots\}$
\item $\mathcal{R}$ : Real Numbers
\end{itemize}

\begin{tabular}{|c|c|c|c|}
\hline  & Domain & Co-Domain  & Range  \\ 
\hline f(x) & $\mathcal{R}$ & $\mathcal{Z}$ & $\mathcal{Z}$ \\ 
\hline  g(x)& $\mathcal{R}$ & $\mathcal{R}$ & $\mathcal{R}^{+} \cup {0} $\\ 
\hline 
\end{tabular} 
f(x) is not one-to-one $f(2.6) = f(2.7)$.\\
g(x) is not one to one $g(2) =g(0)$\\
Are Co-Domain and Range Equal?\\
%--------------------------------------------------- %
%--------------------------------------------------- %
\subsection*{2.3 2011 Zone Question 4 }



(a) Given a real number x, say how $\lfloor x \rfloor$, the floor of $x$, is defined.
(b) The function $f : \mathcal{R} —> \mathcal{R}$ is given by the rule
\[f(x) \lfloor x/2 \rfloor\]

\begin{itemize}
\item[ i.] Find $f(-3)$ and $f(3)$ 
\item[ii.] Justifying your answer, say whether f is one-to-one.
\item[iii.], Justifying your answer, say whether f is onto.
\end{itemize}
\textbf{Solutions}
f(3) = 0 and f(-3) = -1\\
No, Different members of Domain can take the same value.\\
No, $f$ takes on integer values only while the codomain
is specified as $\mathcal{R}$.\\
%--------------------------------------------------- %
%--------------------------------------------------- %
\subsection*{2.4 2009 Zone Question 4- Contd }

(c) Explain what it means for a function to be $O(x)$?
(c) Justifying your answer, say whether the function $
f(x)$ from part (b) is $O(n)$.

From book page 62:
\[ f(x) \leq 0.5 |x|\] (x$\geq 0)$
\newpage
%--------------------------------------------------- %
%--------------------------------------------------- %
\subsection*{2.5 Inverse of a Function}

\begin{itemize}
\item [i.] State the condition to be satisified in order for a function to have an inverse.
\item [ii.] Given $ f : \mathcal{R} \rightarrow \mathcal{R} $ where $f(x) = 2x-1$, define fully the inverse function $f^{-1}$ 
and state the values of $f^{-1}(1)$
\item [ii.] Given $ g : \mathcal{R} \rightarrow \mathcal{R} $ where $g(x) = 3^x$, define fully the inverse function $g^{-1}$ 
and state the values of $g^{-1}(1)$
\end{itemize}
This requires giving the inverse function in algebraic terms and its domain and co-domain.
\newpage

%--------------------------------------------------- %
%--------------------------------------------------- %

\subsection*{2.6 Functions}
\begin{itemize}
\item $\lceil x\rceil$ : Ceiling function
\item $\lfloor x\rfloor$ Floor Function
\item $\{x\}$: Fractional Part of a number
\end{itemize}

\begin{tabular}{|c|c|c|c|} \hline
Sample value x & Floor $\lfloor x\rfloor$ & Ceiling  $\lceil x\rceil$ & Fractional part $ \{ x \} $\\ \hline
12/5 = 2.4 &	2	&3&	2/5 = 0.4\\
2.7&	2&	3	&0.7\\
-2.7&	-3&	-2	&0.3\\
-2&	-2&	-2	&0\\\hline
\end{tabular} 

%--------------------------------------------------- %
%--------------------------------------------------- %
\subsection*{2.7 2004 Question 3 parts A and B}
Given any number $x in \mathcal{R}$ the floor value is denoted $\rfloor x \lfloor$ and the absolute value is denoted by $|x|$.

\begin{itemize}
\item[a] Find $\rfloor \sqrt{2} \lfloor$ and $|-2|$.
\item[b] Find the set of values of a such that $\rfloor a \lfloor$ = 1, and the set of values $|b| = 1$
\item[c] (Discussed Separately)
\end{itemize}

\textbf{Part A}: 
\begin{itemize}
\item $\sqrt{2} = 1.414214 \ldots$
\item Floor function of x is the integer that precedes x.
\[\rfloor \sqrt{2} \lfloor = 1\]
\item The absolute value of -2 is simply 2.
\item The set of values of $a$ for which $\rfloor a \lfloor$ = 1 is all real numbers between 1 and 2. $a$ may take the value 1, but not the value $2$.
\[ 1 \leq a <2 , a \in \mathcal{R}\]
\item The set of values of $b$ for which $|b| = 1$ are simply the values -1 and 1.
\[ b= \{-1,1\}  b \in \mathcal{Z}\]
\end{itemize}

%--------------------------------------------------- %
%--------------------------------------------------- %
\newpage
\subsection*{2.8 Power Functions}
Let s be any positive rational number, then the function $f: \mathcal{R} \rightarrow \mathcal{R}$, which is defined
as \[ f(x) = x^s\] is called the \textbf{power function} of x with \textbf{exponent} s.\\

\bigskip 
\noindent \emph{Page 67 Question 9}
Without using your calculator, express each of the following as an integer or fraction (with exponent 1)
\begin{itemize}
\item $8^{1/3}$ = $(2^3)^{1/3} = 2$
\item $8^{-2/3}$ = $(2^3)^{-2/3} = 2^{-2} = {1 \over 4}$
\item $16^{1.25}$ = $(2^4)^{1.25}  = 2^5 = 32$ 
\item $32^{1.4}$ = $(2^5)^{1.4} = 2^7 = 128$
\item $32^{-0.6}$ = $(2^5)^{-0.6} = 2^{-3} =  {1 \over 8}$
\end{itemize}
\subsection{2.9}
\textbf{2008 Question 5}\\
Let $S$ be the set of names of students on a particular course in a college and
let $f$ be the function that counts the number of letters in a student’s name. So
if X is the name of an individual student then
f(X) = the number of letters in X where $f : X —> \mathcal{Z}$ and $X \in S$.\\
\bigskip

For example if Y is the name \emph{\textbf{Bruce Lee}} then $f(Y) = 8.$\\

(i) Given A is the name \emph{\textbf{Alexander Nevsky}}and B is the name \emph{\textbf{Leo Tolstoy}},
find f(A) and f(B).
\\
(ii) Give two different examples of a possible pre-image or ancestor of 6.
\\
(m) Say under what circumstances you think this function is one to one, justifying your answer.
\\
(iv) Say whether or not this function is onto, justifying your answer. \\

\textbf{comments}\\
This function is only one to one when there are no two or more
candidates with the same number of letters in their name. The function
could not be onto because there are no names with an infinite number (or even a very large number) of letters, whereas the co-domain is all of the
positive integers from 1 up to infinity. Thus the range and the co-domain
are not equal.

%--------------------------------------------------- %
%--------------------------------------------------- %
\subsection*{2.10 Floor and Ceiling Functions}


For each of the following equations, give \textbf{two} different examples of a real number $x$ which 
satisfies the equations:
\begin{itemize}
\item $\lfloor x \rfloor = 3 $
\item $\lceil x \rceil = -1 $
\item $| x-5 | =12 $
\end{itemize}

\textbf{Solutions}
\begin{itemize}
\item $\lfloor x \rfloor = 3 $ : two examples $\pi$ and $3.5$
\item $\lceil x \rceil = -1 $ : two examples -1.2 , -1.6
\item $| x-5 | =12 $  Answers: -7 and 17
\end{itemize}



%--------------------------------------------------- %
%--------------------------------------------------- %
\subsection{2.11}
\textbf{2008 Question 5}

State whether or not each of the following functions has an inverse, justifying your answer. In the cases Where there is an inverse define it fully.

\begin{itemize}
\item[(i)] $f : S -> \mathcal{Z}^{+} $ defined in part (a) (see Section 2.9).
\item[(ii)] $g : \mathcal{R} —> \mathcal{R} $ defined by $g(x) = x^2$.
\item[(iii)] $h : \mathcal{R} —> \mathcal{R} $ defined by $h(x) = 4x — 1. $
\end{itemize}

\textbf{Solutions}
\begin{itemize}
\item No Inverse. Function is not onto, only one-to-one . Each name has only one image But each number can have more than one ancestor. Also the Co-domain and Range must be assumed to be not equal. Even very very long names do not exceed 200 letters.
\item No Inverse
\item Inverse Exists 
\[h^{-1}(x) = \frac{x+1}{4}  \]

\end{itemize}




%--------------------------------------------------- %
%--------------------------------------------------- %
\subsection*{2.12 Laws of Logaritms}

\subsubsection*{ Laws of Exponents}
\emph{See Section 4.3.1, in particular Rule 4.18}

\[ exp_b(x) = b^x \]


{\Large
\[\mbox{Log}_2(128) = 7\]
\[\mbox{Log}_2(1/8) = -3\]
\[\mbox{Log}_2(\sqrt{2}) = -3\]
}
\begin{itemize}
\item $a^x = y$  $log_a(y) = x$

\item $e^x = y$  $ln(y)=x$

\item $log_a(x\times y) = log_a(x) + log_a(y)$

\item $log_a(\frac{x}{y}) = log_a(x) - log_a(y)$

\item $log_a(\frac{1}{x}) = - log_a(x)$

\item $log_a(a) = 1$

\item $log_a(1) = 0$
\end{itemize}



\end{document}
