\documentclass{article}
\usepackage{amsmath}
\usepackage{amssymb}
\usepackage{graphicx}
\begin{document}
\section{Functions}
\newpage

\section*{Functions}
\begin{itemize}
\item Domain of a Function
\item Range of a function
\item Inverse of a function
\end{itemize}
\begin{itemize}
\item one-one (surjective)
\item onto (bijective)
\end{itemize}
%------------------------------------------------%
\section*{Probability}
\subsection*{Binomial Coefficients}
\begin{itemize}
\item factorials 
\[ n! = (n)\times (n-1)\times(n-2) \times \ldots \times 1 \]
\begin{itemize}
\item $5! = 5 \times 4 \times 3 \times 2 \times 1 = 120 $
\item $3! = 3 \times 2 \times 1$
\end{itemize}
\item Zero factorial
\[ 0! =  1 \]
\end{itemize}
%---------------------------------------------- %

% \[P(A |B = \frac{P(A \cap B)}{P(B)})\]

The complement rule in Probability

$P(C^{\prime}) = 1- P(C)$

 

If the probability of C is $70 \%$ then the probability of $C^{\prime}$ is $30\%$


\end{document}
%---------------------------------------%
\subsection*{Absolute Value Function}


%--------------------------------------%
\subsection*{Logarithms}

%----------------------------------------------------------- %
\newpage
\subsection*{Polynomial Functions (4.1.5)}

\begin{itemize}
\item[Constants] $(P_0)$
\item[Linear Functions] $(P_1)$
\item[Quadratic Functions] $(P_2)$
\item[Cubic Functions] $(P_3)$
\end{itemize}


\subsection*{Equality of Functions (4.1.6)}
\[f(x) = g(x) \]


\subsection*{Encoding and Decoding Functions (4.2)}


\subsection*{Onto Functions (4.2.2)}

\subsection*{One-to-One Functions (4.2.3)}


\subsection*{One-to-One Functions (4.2.3)}
$f(x)$, must be \emph{One-to-One} and \emph{Onto}



\subsection*{Exponential and Logarithmic Functions (4.3)}

The Laws of Logarithms
\begin{itemize}
\item
\item $log_b(x^y) = y \times log_b(x)$
\item
\item
\end{itemize}

\subsection*{Comparing the size of Functions (4.4)}

Using O-notations

\subsection*{Power Notation (4.4.2)}


\subsection*{Section 8 Exercises}
\begin{itemize}
\item $8^{\frac{1}{3}}$ Recall $a^{\frac{b}{c}} = a^{\frac{b}{c}}$
\item
\item
\end{itemize}
\end{document}

\section{Mathematics for Computing: Onsite Tutorial two}

\textbf{Today's Class}
\begin{itemize}
\item Chapter 3: Logic
\begin{itemize}
\item
\item
\end{itemize}
\item Chapter 4: Functions
\begin{itemize}
\item Inverse of a Function
\item One-to-One and Onto
\item Special Functions
\end{itemize}
\end{itemize}
\newpage


%--------------------------------------------%
\section*{Chapter 4: Functions}

\subsection{Arrow Diagrams}

\begin{itemize}
\item Domain 
\item Co-Domain 
\item Range
\end{itemize}

\newpage
%--------------------------------------------%
\subsection{Boolean Functions and ordered $n-$tuples}


\begin{itemize}
\item Ordered Triples
\item Boolean Fucntions
\item
\end{itemize}

\subsection*{The Asbolute Value, Floor and Ceiling Functions}
\begin{itemize}
\item The Absolute Value Function
\item Floor
\item Ceiling
\end{itemize}


%--------------------------------------------%
\subsection{Onto Functions}
Definition: If every element in the co-domain of the function has an ancestor, the function is said to be "onto".
An onto function has the property that the domain is equal to the co-domain.


\textbf{Example 4.26 Page 53}

%--------------------------------------------%
\subsection*{Exponential and Logarithms}

\textbf{Rules}\\
Expontials : Rules 4.18 Page 58 \\
Logarithms : Rules 4.23 Page 61 \\

\[ log_a(a) = 1 \]
\[ log_a(b^c) = c \times log_a(b) \]

\begin{itemize}
\item $log_2(128) = 7$
\item $log_2(1/4) = -2$
\item $log_2(2) = 1$
\end{itemize}

\[ log_a(b) = \frac{log_x(b)}{log_x(a)} \]

\begin{itemize}
\item
\item
\item
\end{itemize}

\subsection*{Power functions and Polynomials}


Consider the function f: Z-> Z defined by f(n) = 3n-1. Does this function have the onto property?
\[ax^2 + bx + c\]
%--------------------------------------------%
\subsection*{Summer 2003 Question 4}

\begin{tabular}{ccccccc}
 x & & & & & & \\ \hline
 f(x) & & & & & & \\ \hline
 g(x) & & & & & & \\ \hline
\end{tabular}

%--------------------------------------------%

\section*{Functions}
\begin{itemize}
\item Domain of a Function
\item Range of a function
\item Inverse of a function
\end{itemize}
\begin{itemize}
\item one-one (surjective)
\item onto (bijective)
\end{itemize}

%---------------------------------------------- %

% \[P(A |B = \frac{P(A \cap B)}{P(B)})\]

The complement rule in Probability

$P(C^{\prime}) = 1- P(C)$

 

If the probability of C is $70 \%$ then the probability of $C^{\prime}$ is $30\%$
x\end{document}

