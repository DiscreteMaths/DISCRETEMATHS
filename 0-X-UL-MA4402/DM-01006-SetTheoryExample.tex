%========================================================================================= %
\newpage
\section*{Rules of Inclusion, Listing and Cardinality}
For each of the following sets, a set is specified by the rules of inclusion method and listing method respectively. Also stated is the cardinality of that data set.
\subsection*{Worked example 1}
\begin{itemize}
\item $\{ x : x $ is an odd integer $ 5 \leq x \leq 17 \}$
\item $x = \{5,7,9,11,13,15,17\}$
\item The cardinality of set $x$ is 7.
\end{itemize}

\subsection*{Worked example 2}
\begin{itemize}
\item $\{ y : y $ is an even integer $ 6 \leq y < 18 \}$
\item $y = \{6,8,10,12,14,16\}$
\item The cardinality of set $y$ is 6.
\end{itemize}

\subsection*{Worked example 3}
A perfect square is a number that has a integer value as a
square root. 4 and 9 are perfect squares ($\sqrt{4} = 2$,
$\sqrt{9} = 3$).
\begin{itemize}
\item $\{ z : z $ is an perfect square $ 1 < z < 100 \}$
\item $z = \{4,9,16,25,36,49,64,81\}$
\item The cardinality of set $z$ is 8.
\end{itemize}