\newpage
\subsection*{1.6 Biconditional}
\emph{See Section 3.2.1}.\\
Use truth tables to prove that $ \neg p \leftrightarrow \neg q $ is equivalent to  $ p \leftrightarrow q $

\begin{tabular}{|c|c|c|}
\hline  p& q & $p \leftrightarrow q$ \\ 
\hline  0& 0 &  1\\ 
\hline  0& 1 &  0\\ 
\hline  1& 0 &  0\\ 
\hline  1& 1 &  1\\ 
\hline 
\end{tabular} 


\begin{tabular}{|c|c|c|c|c|}
\hline  p& q & $\neg p$ & $\neg q$ & $p \leftrightarrow q$ \\ 
\hline  0& 0 & 1& 1 & 1\\ 
\hline  0& 1 &  1& 0& 0\\ 
\hline  1& 0 &  0& 1& 0\\ 
\hline  1& 1 &  0 & 0& 1\\ 
\hline 
\end{tabular} 
\newpage
%--------------------------------------------------- %
%--------------------------------------------------- %
\subsection*{1.7 2008 Q3b Logic Networks }

Construct a logic network that accepts as input p and q, which may independently have the value 0 or 1, and
gives as final input $\neg(p \wedge \not q)$ (i.e. $\equiv p \rightarrow q$).\\
\bigskip

\textbf{Logic Gates}
\begin{itemize}
\item AND
\item OR
\item NOT
\end{itemize}
\bigskip

\emph{\textbf{Examiner's Comments:}Many
diagrams were carefully and clearly drawn and well labelled, gaining full
marks. The logic table was also well done by most, but there were a few marks
lost in the final part by failing to deduce that ‘since the columns of the table are
identical the expressions are equivalent’.}

%--------------------------------------------------- %