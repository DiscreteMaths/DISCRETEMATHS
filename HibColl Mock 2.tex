

\documentclass[12pt]{article} % use larger type; default would be 10pt


\usepackage{graphicx} 
\usepackage{framed}



\title{Hibernia College}
\author{Kevin O'Brien}

\begin{document}
\maketitle
\tableofcontents
% Joanne O'Hagan
% The Exchange
% June 13th to 20th
% Pub Standards

% https://onlinecourses.science.psu.edu/stat505/node/78

% http://rer.sagepub.com/content/45/4/543.full.pdf


%-----------%


%http://www.uvm.edu/~dhowell/StatPages/More_Stuff/Missing_Data/Missing.html

% http://www.uvm.edu/~dhowell/StatPages/More_Stuff/Missing_Data/Missing.html
% http://www.stat.columbia.edu/~gelman/arm/missing.pdf
%---------------------------------------------------------------------------------------------------%
\section{Dublin's Data Community}

CeADAR
Dublin Data Visualization
Dublin Riak
Coding Grace
Dublin MongoDB
Dublin Cassandra
Python Ireland
IRlogi
Irish Ecological Modelling
ie-pug Irish PostGresQL user group
Open Data Ireland

EngineYard

\begin{itemize}
\item \texttt{roots()} roots of a polynomial
\item \texttt{poly()} characteristic equation
\item \texttt{primes()} generate a sequence of prime numbers
\item \texttt{isprime()} check for prime numbers
\item \texttt{any()} check if any value fulfils a logical condition
\item \texttt{all()} check if all value fulfils a logical condition
\end{itemize}

 \newpage

\textbf{Grubbs Test}

p: value



The null hypothesis is that both sampmes are drawn from the same population of values.

The test is a non parametric test. Non Parametric tests are a family of inference tests that do not require the assumption of normality.
\subsection{Grubbs's Test for Outliers}
Grubbs Testnis is used to determine whether or not a particular value in the data set is an \textbf{\textit{outlier}}. There is no standard definition of what defines an outlier. The definition of an outlier used for this procedure is a value that is numerically distance from the rest of the values of the data set. (We will refer to such an outlier as a \textbf{\textit{Grubbs outlier}}.


Another type of outlier is a point identified as such by a boxplot. We will refer to this type of outlier as a \textbf{\textit{boxplot outlier}}.  An outlier may be either one or the other type, but not necessarily both.

Some outliers may be due to incorrectly recorded data. Other outliers are correctly recorded, but very unusual, values.

null $H_0$ : There are no outliers present in the data set.

alt $H_1$ : There is an outlier present in the data set.

\subsubsection{Implementation using \texttt{R}}
To implement the Grubbs test, we require the package \textbf{\textit{outliers}}.

%--------------------------------------------------------------------------------%
\subsection{Kolmorgorov Smirnov Test}
Two sample KS test

The Kolmorogorov Smirnov Test (aka The KS test) is used to determine whether or not two data sets are from the same distribution.
The test is implemented in \texttt{R}n using the command \texttt{ks.test()}. No packages are required to run this procedure.




\begin{framed}
\begin{verbatim}
x=c(3,4,5,1,6)
library(outliers)
grubbs.test(X,two.tailed=T)
\end{verbatim}
\end{framed}


%-------------------------------------------------------------------------%
\newpage

%-------------------------------------------------------------------------%
\newpage
\newpage
\begin{verbatim}
MONDAY 15 April
1. MS4024 MATLAB (Project Euler) - DONE
2. MS4024 R (Complete Course) - Week 12 DONE
3. MA4128 - Paper Submit (Joe Lynch) - DONE
4. MA4128 - Practical Exam Sample Paper FORWARD
5. MA4704 11A
           Update Sample Papers(DONE)
           Re-publish 10B/C (DONE)
           HTs and CIs for Proportions (DONE)
           sample size estimation for proportions (Corrections)
6. HibColl Review Sessions 3 and 4
7. Corrections for MTs

TUESDAY 16 April
1. MA4128 Discriminant Analysis + Missing Data 
2. HibColl Review Sessions 5 and 6
3. Update Sample Questions for MA4128

WEDNESDAY 17 April
1. Dentist (09:30) DONE
2. Dublin R (17:30) and Python Ireland (18:30) DONE
3. MA4704 Tutorials Prep (Proportions + p-value prcoedures)
4. MT2 MA4704 Corrections (Some Done)
5. MS4024 Payments
6. Method Comparison Studies
FRIDAY 19 April
1. Kubrick Night (Cant Make It)
2.Julia Language - Learn
3. Coursera Downloads
4. Prof. Nial Friel (4pm)
\end{verbatim}
\newpage
%---------------------------------------------%
\section{Useful MATLAB Functions}

\begin{itemize}
\item \texttt{unique()}
\item \texttt{primes()}
\item \texttt{size()}
\item \texttt{factor()}
\end{itemize}
\begin{framed}
\begin{verbatim}

factor(223)

\end{verbatim}
\end{framed}


\subsection{For Loops in MATLAB}
\begin{framed}
\begin{verbatim}
for( )

\end{verbatim}
\end{framed}
\subsection{While Loops in MATLAB}
\texttt{while} operator allows a set of commands to be repeated until a specific logical condition is met.

\begin{framed}
\begin{verbatim}
while( )

\end{verbatim}
\end{framed}
\end{document}
