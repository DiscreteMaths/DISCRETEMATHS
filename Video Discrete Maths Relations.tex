\documentclass{beamer}

\usepackage{amsmath}
\usepackage{amssymb}

\begin{document}

% http://www.textbooksonline.tn.nic.in/books/12/std12-bm-em-1.pdf

\begin{frame}
\frametitle{Discrete Maths : Relations}
\Large
\begin{itemize}
\item A relation $R$ from a set A to a set B is a subset of the
\textbf{cartesian product} A x B. 
\item Thus R is a set of \textbf{ordered pairs} where
the first element comes from A and the second element comes
from B i.e. $(a, b)$
\end{itemize}
\end{frame}
%-----------------------------------%
\begin{frame}
\frametitle{Discrete Maths : Relations}
\Large
\begin{itemize}

\item  If $(a, b) \in R$ we say that $a$ is related to $b$ and write $aRb$.
\item If $(a, b) \notin R$, we say that $a$ is not related to $b$ and write $aRb$. CHECK
\item If
$R$ is a relation from a set $A$ to itself then we say that ``$R$ is a relation
on $A$".
\end{itemize}
\end{frame}
%-----------------------------------%
\begin{frame}
\frametitle{Discrete Maths : Relations}
\Large
\vspace{-2cm}
\textbf{Example}
\begin{itemize}
\item Let $A = \{2, 3, 4, 6\}$ and $B = \{4, 6, 9\}$
\item Let R be the relation from A to B defined by \textit{\textbf{xRy}} if $x$
divides $y$ exactly.
\end{itemize}

\end{frame}
%-----------------------------------%
\begin{frame}
\frametitle{Discrete Maths : Relations}
\Large
\vspace{-0.7cm}
\textbf{Example}
\begin{itemize}

\item Let $A = \{2, 3, 4, 6\}$ and $B = \{4, 6, 9\}$
\item Let R be the relation from A to B defined by \textit{\textbf{xRy}} if $x$
divides $y$ exactly.
\item  Then
\[R = {(2, 4), (2, 6), (3, 6), (3, 9), (4, 4), (6, 6)}\]
\end{itemize}

\end{frame}
\end{document}