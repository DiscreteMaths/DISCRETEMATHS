\documentclass[12pt]{article}

\usepackage{amsmath}
\usepackage{amssymb}

\begin{document}

\begin{center}
\huge{Mathematics for Computing}\\
\LARGE{Numbers and Number Systems}
\end{center}



%------------------------------------------%

\section{Hexadecimal Numbers}
\Large{
	Hex Characters: 0,1,2,3,4,5,6,7,8,9,A,B,C,D,E,F}

%------------------------------------------%

\section{Types of Numbers}
\Large{
	\begin{itemize}
		\item Natural Numbers (1,2,3)
		\item Integers (..-3,-2-1,0,1,2,3..)
		\item Rational Numbers (e.g 4/7, 12/3)
		\item Real Numbers (3.14151)
	\end{itemize}
}
\section*{Binary and Hex}
\begin{itemize}
	\item[1A.1] Coverting from Binomial to Decimal
	\item[1A.2] Converting to Decimal
	\item[1A.3] Priority of Operation
	\item[1A.4] 
\end{itemize}
\newpage
%---------------------------------------%
\section*{Numbers}
\begin{itemize}
	\item[1B.1] Real Numbers
	\item[1B.2] Rational Numbers
	\item[1B.3] Floating Point Aritmetic
	\item[1B.4] 
\end{itemize}
%----------------------------------------------------%
\section*{Binary and Hex}
\begin{itemize}
	\item[1A.1] Coverting from Binomial to Decimal
	\item[1A.2] Converting to Decimal
	\item[1A.3] Priority of Operation
	\item[1A.4] 
\end{itemize}

\section*{Numbers}
\begin{itemize}
	\item[1B.1] Real Numbers
	\item[1B.2] Rational Numbers
	\item[1B.3] Floating Point Aritmetic
	\item[1B.4] 
\end{itemize}
\section*{Binary and Hex}
\begin{itemize}
	\item[1A.1] Coverting from Binomial to Decimal
	\item[1A.2] Converting to Decimal
	\item[1A.3] Priority of Operation
	\item[1A.4] 
\end{itemize}
%---------------------------------------%
\section*{Numbers}
\begin{itemize}
	\item[1B.1] Real Numbers
	\item[1B.2] Rational Numbers
	\item[1B.3] Floating Point Aritmetic
	\item[1B.4] 
\end{itemize}
\section*{Binary and Hex}
\begin{itemize}
	\item[1A.1] Coverting from Binomial to Decimal
	\item[1A.2] Converting to Decimal
	\item[1A.3] Priority of Operation
	\item[1A.4] 
\end{itemize}
%----------------------------------------%
\newpage
%---------------------------------------%
\subsection*{Adding Binary Numbers}


%--------------------------------------%

%--------------------------------------------------------%
\subsubsection*{Complements}

%------------------------------------------------------------------------%

\subsection*{Part 1 : Binary numbers}
\begin{itemize}
%---------------------------%
\item[(a)] Express the following binary numbers as decimal numbers
\begin{itemize}
\item[(i)] $11011$
\item[(ii)] $100101$
\end{itemize}
%---------------------------%
\item[(b)] Express the following decimal numbers as binary numbers
\begin{itemize}
\item[(i)] 6
\item[(ii)] 15
\item[(iii)] 37
\end{itemize}
%---------------------------%
\item[(c)] Perform the following binary additions
\begin{itemize}
\item[(i)] $1011+ 1111$
\item[(ii)] $10101  + 10011$
\item[(iii)] $1010 + 11010$
\end{itemize}
%---------------------------%

\end{itemize}
% End of Subsection

\subsection*{Part 2 : Hexadecimal numbers}
%------------------------------------------------------------------------%
\begin{itemize}
\item[(i)] Calculate the decimal equivalent of the hexadecimal number $(A2F.D)_{16}$
\item[(ii)] Working in base 2, compute the following binary additions, showing all you workings
\[(1110)_2 + (11011)_2 + (1101)_2 \]
\item[(iv)] Express the recurring decimal $0.727272\ldots$ as a rational number in its simplest form.
\end{itemize}


\subsection*{Part 3 : Base 5 and Base 8 numbers}


\subsection*{Part 4 : Real and Rational Numbers}


\newpage






\end{document}


\item Express the following decimal number as a hexadecimal number

\[44321\]

%-----------------------------------


%-----------------------------------

\item Multiply the following Hexadecimal numbers

\[AA3 \times F\]

0.126
%Worksheet Question 13

Express 42900 as a product of its prime factors, using index notation for repeated factors
\[
2 \times 2 \times 3 \times 5 \times 5 \times 11 \times 13 \]

Using index notations

\[2^2 \times 3 \times 5^2 \times 11 \times 13 \]



%--------------------------------------------------------------------%
IS the number 89 a prime?
\[  \sqrt{89} = 9.4340 \]
Primes that are less than this number
2,3,5,7
%--------------------------------------------------------------------%

Decide whether 899 is a prime number?
(No $899 = 29 \times 31$)


%-------------------------------------------------------------------------%

%Decimal:	0	1	2	3	4	5	6	7	8	9	10	11	12	13	14	15
%Hexadecimal:	0	1	2	3	4	5	6	7	8	9	A	B	C	D	E	F
%-------------------------------------------------------------------------%


\end{itemize}
% End of Subsection