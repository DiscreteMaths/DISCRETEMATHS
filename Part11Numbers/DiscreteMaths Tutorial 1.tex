\documentclass[12pt]{article}

\usepackage{amsmath}
\usepackage{amssymb}

\begin{document}

\begin{center}
\huge{Mathematics for Computing}\\
\LARGE{Numbers and Number Systems}
\end{center}
%------------------------------------------------------------------------%

\subsection*{Part 1 : Binary numbers}
\begin{itemize}
%---------------------------%
\item[(a)] Express the following binary numbers as decimal numbers
\begin{itemize}
\item[(i)] $11011$
\item[(ii)] $100101$
\end{itemize}
%---------------------------%
\item[(b)] Express the following decimal numbers as binary numbers
\begin{itemize}
\item[(i)] 6
\item[(ii)] 15
\item[(iii)] 37
\end{itemize}
%---------------------------%
\item[(c)] Perform the following binary additions
\begin{itemize}
\item[(i)] $1011+ 1111$
\item[(ii)] $10101  + 10011$
\item[(iii)] $1010 + 11010$
\end{itemize}
%---------------------------%

\end{itemize}
% End of Subsection

\subsection*{Part 2 : Hexadecimal numbers}
%------------------------------------------------------------------------%
\begin{itemize}
\item[(i)] Calculate the decimal equivalent of the hexadecimal number $(A2F.D)_{16}$
\item[(ii)] Working in base 2, compute the following binary additions, showing all you workings
\[(1110)_2 + (11011)_2 + (1101)_2 \]
\item[(iv)] Express the recurring decimal $0.727272\ldots$ as a rational number in its simplest form.
\end{itemize}


\subsection*{Part 3 : Base 5 and Base 8 numbers}
\begin{itemize}
%---------------------------%
\item[(a)] Suppose 2341 is a base-5 number
Compute the equivalent in each of the following forms:
\begin{itemize}
\item[(i)] decimal number
\item[(ii)] hexadecimal number
\item[(iii)] binary number
\end{itemize}
%---------------------------%
\item[(b)] Perform the following binary additions
\begin{itemize}
\item[(i)] $1011+ 1111$
\item[(ii)] $10101  + 10011$
\item[(iii)] $1010 + 11010$
\end{itemize}
%---------------------------%
\end{itemize}

\subsection*{Part 4 : Real and Rational Numbers}
\begin{itemize}
\item[(i)] Express the recurring decimal $0.727272\ldots$ as a rational number in its simplest form.
\end{itemize}
%-------------------------------------------------------- %
\subsection*{Miscellaneous Questions}
\begin{itemize}
\item[(i)] Given x is the irrational positive number $\sqrt{2}$, express $x^8$ in binary notation\\
\item[(ii)] From part (i), is $x^8$ a rational number?
\end{itemize}
\newpage






\end{document}
\begin{itemize}
\item Which of the following are not valid hex numbers?

a) A5G 	\\
b) 73\\
c) EEF\\	
d) 101\\
%-----------------------------------

\item Express the following decimal number as a hexadecimal number

\[44321\]

%-----------------------------------

\item What is highest Hexadecimal number that can be written with two characters, and what is it's equivalent in decimal form?
What is the next highest hexadecimal number?

\[FF = 255\] % Remark 

%-----------------------------------

\item Multiply the following Hexadecimal numbers

\[AA3 \times F\]

0.126
%Worksheet Question 13

Express 42900 as a product of its prime factors, using index notation for repeated factors
\[
2 \times 2 \times 3 \times 5 \times 5 \times 11 \times 13 \]

Using index notations

\[2^2 \times 3 \times 5^2 \times 11 \times 13 \]



%--------------------------------------------------------------------%
IS the number 89 a prime?
\[  \sqrt{89} = 9.4340 \]
Primes that are less than this number
2,3,5,7
%--------------------------------------------------------------------%

Decide whether 899 is a prime number?
(No $899 = 29 \times 31$)


%-------------------------------------------------------------------------%

%Decimal:	0	1	2	3	4	5	6	7	8	9	10	11	12	13	14	15
%Hexadecimal:	0	1	2	3	4	5	6	7	8	9	A	B	C	D	E	F
%-------------------------------------------------------------------------%


\end{itemize}
% End of Subsection