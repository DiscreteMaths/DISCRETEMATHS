
\documentclass{beamer}

\usepackage{amsmath}

\begin{document}
\begin{frame}
\begin{center}
\Huge
Properties of Functions\\
\LARGE
Example 1\\
\bigskip
kobriendublin.wordpress.com\\
Twitter: \texttt{@}statslabdublin


\end{center}

\end{frame}

%-------------------------------------------- %
\begin{frame}
\frametitle{Properties of Functions}
\huge
\vspace{-1cm}
The function $f : \mathbb{R} \rightarrow \mathbb{Z}$ is defined by the rule:
\[f(x) = \bigg\lfloor  {x \over 2}\bigg\rfloor \]



\end{frame}

%-------------------------------------------- %
\begin{frame}
\frametitle{Properties of Functions}
\Large
\vspace{-2cm}
The function $f : \mathbb{R} \rightarrow \mathbb{Z}$ is defined by the rule:
\[f(x) = \bigg\lfloor  {x \over 2}\bigg\rfloor \]
What is the range of this function?


\end{frame}
%-------------------------------------------- %
\begin{frame}
\frametitle{Properties of Functions}
\Large
\vspace{-2cm}
The function $f : \mathbb{R} \rightarrow \mathbb{Z}$ is defined by the rule:
\[f(x) = \bigg\lfloor  {x \over 2}\bigg\rfloor \]
Is this function ``onto" ?


\end{frame}
%-------------------------------------------- %
\begin{frame}
\frametitle{Properties of Functions}
\Large
\vspace{-2cm}
The function $f : \mathbb{R} \rightarrow \mathbb{Z}$ is defined by the rule:
\[f(x) = \bigg\lfloor  {x \over 2}\bigg\rfloor \]
Is this function ``one-to-one" ?


\end{frame}
%-------------------------------------------- %
\begin{frame}
\frametitle{Properties of Functions}
\Large
\vspace{-2cm}
The function $f : \mathbb{R} \rightarrow \mathbb{Z}$ is defined by the rule:
\[f(x) = \bigg\lfloor  {x \over 2}\bigg\rfloor \]
Is this function invertible ?


\end{frame}
\end{document}
