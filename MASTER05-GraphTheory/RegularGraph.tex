
\documentclass{beamer}

\usepackage{amsmath}

\begin{document}
%--------------------------------------------------%
\begin{frame}
\begin{center}
{ \Huge
Regular Graphs}
\\
\bigskip
{ \Large
youtube.com/StatsLabDublin \\ \vspace{0.2cm} Twitter: @statslabdublin
}
\end{center}
\end{frame}

%--------------------------------------------------%

\begin{frame}
\frametitle{Regular Graphs}
{
\LARGE
What is meant by an \textbf{n-regular} graph?\\
\bigskip
Each vertex in the graph has the same degree, a degree of n.
\\
\bigskip
In a 3-regular graph, each vertex has degree of three.

}
\end{frame}

%--------------------------------------------------%
\begin{frame}
\frametitle{Regular Graphs}
{\huge
\vspace{-3cm}
A 3-regular graph with six vertices.
}
\end{frame}
%--------------------------------------------------%
\begin{frame}
\frametitle{Regular Graphs}
{\huge 
\vspace{-3cm}
A 3-regular graph with seven vertices.
}
\end{frame}
%--------------------------------------------------%
\begin{frame}
\frametitle{Regular Graphs}
{\huge 
\vspace{-3cm}
A 3-regular graph with seven vertices.
}
\end{frame}
\end{document}