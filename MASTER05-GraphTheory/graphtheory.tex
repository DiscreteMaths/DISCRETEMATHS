% %- https://www.tutorialspoint.com/graph_theory/index.htm


\section{Graph Theory - Fundamentals}

A graph is a diagram of points and lines connected to the points. It has at least one line joining a set of two vertices with no vertex connecting itself. The concept of graphs in graph theory stands up on some basic terms such as point, line, vertex, edge, degree of vertices, properties of graphs, etc. Here, in this chapter, we will cover these fundamentals of graph theory.

\subsection{Point}
A point is a particular position in a one-dimensional, two-dimensional, or three-dimensional space. For better understanding, a point can be denoted by an alphabet. It can be represented with a dot.

Example
point
Here, the dot is a point named ‘a’.

Line
A Line is a connection between two points. It can be represented with a solid line.

Example
line
Here, ‘a’ and ‘b’ are the points. The link between these two points is called a line.

Vertex
A vertex is a point where multiple lines meet. It is also called a node. Similar to points, a vertex is also denoted by an alphabet.

Example
Vertex
Here, the vertex is named with an alphabet ‘a’.

\subsection{Edge}
An edge is the mathematical term for a line that connects two vertices. Many edges can be formed from a single vertex. Without a vertex, an edge cannot be formed. There must be a starting vertex and an ending vertex for an edge.

Example
Edge
Here, ‘a’ and ‘b’ are the two vertices and the link between them is called an edge.

\subsection{Graph}
A graph ‘G’ is defined as G = (V, E) Where V is a set of all vertices and E is a set of all edges in the graph.

Example 1
graph
In the above example, ab, ac, cd, and bd are the edges of the graph. Similarly, a, b, c, and d are the vertices of the graph.

Example 2
graph1
In this graph, there are four vertices a, b, c, and d, and four edges ab, ac, ad, and cd.

\subsection{Loop}
In a graph, if an edge is drawn from vertex to itself, it is called a loop.

Example 1
Loop
In the above graph, V is a vertex for which it has an edge (V, V) forming a loop.

Example 2
Loop 1
In this graph, there are two loops which are formed at vertex a, and vertex b.

Degree of Vertex
It is the number of vertices incident with the vertex V.

Notation − deg(V).

In a simple graph with n number of vertices, the degree of any vertices is −

\[deg(v) ≤ n – 1 ∀ v \in G \]
A vertex can form an edge with all other vertices except by itself. So the degree of a vertex will be up to the number of vertices in the graph minus 1. This 1 is for the self-vertex as it cannot form a loop by itself. If there is a loop at any of the vertices, then it is not a Simple Graph.

Degree of vertex can be considered under two cases of graphs −

Undirected Graph
Directed Graph
Degree of Vertex in an Undirected Graph
An undirected graph has no directed edges. Consider the following examples.

Example 1
Take a look at the following graph −

\subsection{Undirected Graph}
In the above Undirected Graph,

deg(a) = 2, as there are 2 edges meeting at vertex ‘a’.

deg(b) = 3, as there are 3 edges meeting at vertex ‘b’.

deg(c) = 1, as there is 1 edge formed at vertex ‘c’

So ‘c’ is a pendent vertex.

deg(d) = 2, as there are 2 edges meeting at vertex ‘d’.

deg(e) = 0, as there are 0 edges formed at vertex ‘e’.

So ‘e’ is an isolated vertex.

Example 2
Take a look at the following graph −

Undirected Graph 1
In the above graph,

deg(a) = 2, deg(b) = 2, deg(c) = 2, deg(d) = 2, and deg(e) = 0.

The vertex ‘e’ is an isolated vertex. The graph does not have any pendent vertex.

Degree of Vertex in a Directed Graph
In a directed graph, each vertex has an indegree and an outdegree.

\subsection{Indegree of a Graph}
Indegree of vertex V is the number of edges which are coming into the vertex V.

Notation − deg+(V).

Outdegree of a Graph
Outdegree of vertex V is the number of edges which are going out from the vertex V.

Notation − deg-(V).

Consider the following examples.

Example 1
Take a look at the following directed graph. Vertex ‘a’ has two edges, ‘ad’ and ‘ab’, which are going outwards. Hence its outdegree is 2. Similarly, there is an edge ‘ga’, coming towards vertex ‘a’. Hence the indegree of ‘a’ is 1.

\subsection{Directed Graph}
The indegree and outdegree of other vertices are shown in the following table −

Vertex	Indegree	Outdegree
a	1	2
b	2	0
c	2	1
d	1	1
e	1	1
f	1	1
g	0	2

\subsection{Example 2}
Take a look at the following directed graph. Vertex ‘a’ has an edge ‘ae’ going outwards from vertex ‘a’. Hence its outdegree is 1. Similarly, the graph has an edge ‘ba’ coming towards vertex ‘a’. Hence the indegree of ‘a’ is 1.

Directed Graph 1
The indegree and outdegree of other vertices are shown in the following table −

Vertex	Indegree	Outdegree
a	1	1
b	0	2
c	2	0
d	1	1
e	1	1
Pendent Vertex
By using degree of a vertex, we have a two special types of vertices. A vertex with degree one is called a pendent vertex.

Example
Pendent Vertex
Here, in this example, vertex ‘a’ and vertex ‘b’ have a connected edge ‘ab’. So with respect to the vertex ‘a’, there is only one edge towards vertex ‘b’ and similarly with respect to the vertex ‘b’, there is only one edge towards vertex ‘a’. Finally, vertex ‘a’ and vertex ‘b’ has degree as one which are also called as the pendent vertex.

\subsection{Isolated Vertex}
A vertex with degree zero is called an isolated vertex.

Example
Isolated Vertex
Here, the vertex ‘a’ and vertex ‘b’ has a no connectivity between each other and also to any other vertices. So the degree of both the vertices ‘a’ and ‘b’ are zero. These are also called as isolated vertices.

Adjacency
Here are the norms of adjacency −

In a graph, two vertices are said to be adjacent, if there is an edge between the two vertices. Here, the adjacency of vertices is maintained by the single edge that is connecting those two vertices.

In a graph, two edges are said to be adjacent, if there is a common vertex between the two edges. Here, the adjacency of edges is maintained by the single vertex that is connecting two edges.

Example 1
Adjacency
In the above graph −

‘a’ and ‘b’ are the adjacent vertices, as there is a common edge ‘ab’ between them.

‘a’ and ‘d’ are the adjacent vertices, as there is a common edge ‘ad’ between them.

ab’ and ‘be’ are the adjacent edges, as there is a common vertex ‘b’ between them.

be’ and ‘de’ are the adjacent edges, as there is a common vertex ‘e’ between them.

Example 2
Adjacency 1
In the above graph −

a’ and ‘d’ are the adjacent vertices, as there is a common edge ‘ad’ between them.

‘c’ and ‘b’ are the adjacent vertices, as there is a common edge ‘cb’ between them.

‘ad’ and ‘cd’ are the adjacent edges, as there is a common vertex ‘d’ between them.

ac’ and ‘cd’ are the adjacent edges, as there is a common vertex ‘c’ between them.

Parallel Edges
In a graph, if a pair of vertices is connected by more than one edge, then those edges are called parallel edges.

Parallel Edges
In the above graph, ‘a’ and ‘b’ are the two vertices which are connected by two edges ‘ab’ and ‘ab’ between them. So it is called as a parallel edge.

Multi Graph
A graph having parallel edges is known as a Multigraph.

Example 1
Multigraph
In the above graph, there are five edges ‘ab’, ‘ac’, ‘cd’, ‘cd’, and ‘bd’. Since ‘c’ and ‘d’ have two parallel edges between them, it a Multigraph.

Example 2
Multigraph 1
In the above graph, the vertices ‘b’ and ‘c’ have two edges. The vertices ‘e’ and ‘d’ also have two edges between them. Hence it is a Multigraph.

Degree Sequence of a Graph
If the degrees of all vertices in a graph are arranged in descending or ascending order, then the sequence obtained is known as the degree sequence of the graph.

Example 1
Degree Sequence of Graph
Vertex	A	b	c	d	e
Connecting to	b,c	a,d	a,d	c,b,e	d
Degree	2	2	2	3	1
In the above graph, for the vertices {d, a, b, c, e}, the degree sequence is {3, 2, 2, 2, 1}.

\subsection{Example 2}
Degree Sequence of Graph 1
Vertex	A	b	c	d	e	f
Connecting to	b,e	a,c	b,d	c,e	a,d	-
Degree	2	2	2	2	2	0
In the above graph, for the vertices {a, b, c, d, e, f}, the degree sequence is {2, 2, 2, 2, 2, 0}.

% % % % % % % % % % % % % % % % % % % % % % % % % % % % % %
\newpage
\section{Graph Theory - Basic Properties}

 
Graphs come with various properties which are used for characterization of graphs depending on their structures. These properties are defined in specific terms pertaining to the domain of graph theory. In this chapter, we will discuss a few basic properties that are common in all graphs.

\subsection{Distance between Two Vertices}
It is number of edges in a shortest path between Vertex U and Vertex V. If there are multiple paths connecting two vertices, then the shortest path is considered as the distance between the two vertices.

Notation − d(U,V)

There can be any number of paths present from one vertex to other. Among those, you need to choose only the shortest one.

\subsection{Example}
Take a look at the following graph −

Distance between two Vertices
Here, the distance from vertex ‘d’ to vertex ‘e’ or simply ‘de’ is 1 as there is one edge between them. There are many paths from vertex ‘d’ to vertex ‘e’ −

da, ab, be
df, fg, ge
de (It is considered for distance between the vertices)
df, fc, ca, ab, be
da, ac, cf, fg, ge
Eccentricity of a Vertex
The maximum distance between a vertex to all other vertices is considered as the eccentricity of vertex.

Notation − e(V)

The distance from a particular vertex to all other vertices in the graph is taken and among those distances, the eccentricity is the highest of distances.

Example
In the above graph, the eccentricity of ‘a’ is 3.

The distance from ‘a’ to ‘b’ is 1 (‘ab’),

from ‘a’ to ‘c’ is 1 (‘ac’),

from ‘a’ to ‘d’ is 1 (‘ad’),

from ‘a’ to ‘e’ is 2 (‘ab’-‘be’) or (‘ad’-‘de’),

from ‘a’ to ‘f’ is 2 (‘ac’-‘cf’) or (‘ad’-‘df’),

from ‘a’ to ‘g’ is 3 (‘ac’-‘cf’-‘fg’) or (‘ad’-‘df’-‘fg’).

So the eccentricity is 3, which is a maximum from vertex ‘a’ from the distance between ‘ag’ which is maximum.

In other words,

e(b) = 3

e(c) = 3

e(d) = 2

e(e) = 3

e(f) = 3

e(g) = 3

Radius of a Connected Graph
The minimum eccentricity from all the vertices is considered as the radius of the Graph G. The minimum among all the maximum distances between a vertex to all other vertices is considered as the radius of the Graph G.

Notation − r(G)

From all the eccentricities of the vertices in a graph, the radius of the connected graph is the minimum of all those eccentricities.

Example − In the above graph r(G) = 2, which is the minimum eccentricity for ‘d’.

Diameter of a Graph
The maximum eccentricity from all the vertices is considered as the diameter of the Graph G. The maximum among all the distances between a vertex to all other vertices is considered as the diameter of the Graph G.

Notation − d(G)

From all the eccentricities of the vertices in a graph, the diameter of the connected graph is the maximum of all those eccentricities.

Example − In the above graph, d(G) = 3; which is the maximum eccentricity.

Central Point
If the eccentricity of a graph is equal to its radius, then it is known as the central point of the graph. If

e(V) = r(V),

then ‘V’ is the central point of the Graph ’G’.

Example − In the example graph, ‘d’ is the central point of the graph.

e(d) = r(d) = 2

Centre
The set of all central points of ‘G’ is called the centre of the Graph.

Example − In the example graph, {‘d’} is the centre of the Graph.

Circumference
The number of edges in the longest cycle of ‘G’ is called as the circumference of ‘G’.

Example − In the example graph, the circumference is 6, which we derived from the longest cycle a-c-f-g-e-b-a or a-c-f-d-e-b-a.

Girth
The number of edges in the shortest cycle of ‘G’ is called its Girth.

Notation − g(G).

Example − In the example graph, the Girth of the graph is 4, which we derived from the shortest cycle a-c-f-d-a or d-f-g-e-d or a-b-e-d-a.

Sum of Degrees of Vertices Theorem
If G = (V, E) be a non-directed graph with vertices V = {V1, V2,…Vn} then

n
∑
i=1
 deg(Vi) = 2|E|
Corollary 1
If G = (V, E) be a directed graph with vertices V = {V1, V2,…Vn}, then

n
∑
i=1
 deg+(Vi) = |E| = 
n
∑
i=1
 deg−(Vi)
Corollary 2
In any non-directed graph, the number of vertices with Odd degree is Even.

Corollary 3
In a non-directed graph, if the degree of each vertex is k, then

k|V| = 2|E|

\textbf{Corollary 4}\\
In a non-directed graph, if the degree of each vertex is at least k, then

\[k|V| ≤ 2|E|\]

Corollary 5
In a non-directed graph, if the degree of each vertex is at most k, then

k|V| ≥ 2|E|


% % % % % % % % % % % % % % % % % % % %

% %- https://www.tutorialspoint.com/graph_theory/types_of_graphs.htm
\section{Graph Theory - Types of Graphs}

 
There are various types of graphs depending upon the number of vertices, number of edges, interconnectivity, and their overall structure. We will discuss only a certain few important types of graphs in this chapter.

\subsection{Null Graph}
A graph having no edges is called a Null Graph.

Example
Null Graph
In the above graph, there are three vertices named ‘a’, ‘b’, and ‘c’, but there are no edges among them. Hence it is a Null Graph.

\subsection{Trivial Graph}
A graph with only one vertex is called a Trivial Graph.

Example
Trivial
In the above shown graph, there is only one vertex ‘a’ with no other edges. Hence it is a Trivial graph.

Non-Directed Graph
A non-directed graph contains edges but the edges are not directed ones.

Example
Non-Directed
In this graph, ‘a’, ‘b’, ‘c’, ‘d’, ‘e’, ‘f’, ‘g’ are the vertices, and ‘ab’, ‘bc’, ‘cd’, ‘da’, ‘ag’, ‘gf’, ‘ef’ are the edges of the graph. Since it is a non-directed graph, the edges ‘ab’ and ‘ba’ are same. Similarly other edges also considered in the same way.

\subsection{Directed Graph}
In a directed graph, each edge has a direction.

Example
Directed Graph
In the above graph, we have seven vertices ‘a’, ‘b’, ‘c’, ‘d’, ‘e’, ‘f’, and ‘g’, and eight edges ‘ab’, ‘cb’, ‘dc’, ‘ad’, ‘ec’, ‘fe’, ‘gf’, and ‘ga’. As it is a directed graph, each edge bears an arrow mark that shows its direction. Note that in a directed graph, ‘ab’ is different from ‘ba’.

\subsection{Simple Graph}
A graph with no loops and no parallel edges is called a simple graph.

The maximum number of edges possible in a single graph with ‘n’ vertices is nC2 where nC2 = n(n – 1)/2.

The number of simple graphs possible with ‘n’ vertices = 2nc2 = 2n(n-1)/2.

Example
In the following graph, there are 3 vertices with 3 edges which is maximum excluding the parallel edges and loops. This can be proved by using the above formulae.

Simple Graph
The maximum number of edges with n=3 vertices −

nC2 = n(n–1)/2
   = 3(3–1)/2
   = 6/2
   = 3 edges
The maximum number of simple graphs with n=3 vertices −

2nC2 = 2n(n-1)/2
   = 23(3-1)/2
   = 23
   = 8
These 8 graphs are as shown below −

Eight Graphs
Connected Graph
A graph G is said to be connected if there exists a path between every pair of vertices. There should be at least one edge for every vertex in the graph. So that we can say that it is connected to some other vertex at the other side of the edge.

Example
In the following graph, each vertex has its own edge connected to other edge. Hence it is a connected graph.

\subsection{Connected Graph/Disconnected Graph}
A graph G is disconnected, if it does not contain at least two connected vertices.

Example 1
The following graph is an example of a Disconnected Graph, where there are two components, one with ‘a’, ‘b’, ‘c’, ‘d’ vertices and another with ‘e’, ’f’, ‘g’, ‘h’ vertices.

Disconnected Graph
The two components are independent and not connected to each other. Hence it is called disconnected graph.

Example 2
Disconnected Graph 1
In this example, there are two independent components, a-b-f-e and c-d, which are not connected to each other. Hence this is a disconnected graph.

\subsection{Regular Graph}
A graph G is said to be regular, if all its vertices have the same degree. In a graph, if the degree of each vertex is ‘k’, then the graph is called a ‘k-regular graph’.

Example
In the following graphs, all the vertices have the same degree. So these graphs are called regular graphs.

Regular Graph
In both the graphs, all the vertices have degree 2. They are called 2-Regular Graphs.

Complete Graph
A simple graph with ‘n’ mutual vertices is called a complete graph and it is denoted by ‘Kn’. In the graph, a vertex should have edges with all other vertices, then it called a complete graph.

In other words, if a vertex is connected to all other vertices in a graph, then it is called a complete graph.

Example
In the following graphs, each vertex in the graph is connected with all the remaining vertices in the graph except by itself.

Complete Graph
In graph I,

a	b	c
a	Not Connected	Connected	Connected
b	Connected	Not Connected	Connected
c	Connected	Connected	Not Connected
In graph II,

p	q	r	s
p	Not Connected	Connected	Connected	Connected
q	Connected	Not Connected	Connected	Connected
r	Connected	Connected	Not Connected	Connected
s	Connected	Connected	Connected	Not Connected
Cycle Graph
A simple graph with ‘n’ vertices (n >= 3) and ‘n’ edges is called a cycle graph if all its edges form a cycle of length ‘n’.

If the degree of each vertex in the graph is two, then it is called a Cycle Graph.

Notation − Cn

Example
Take a look at the following graphs −

Graph I has 3 vertices with 3 edges which is forming a cycle ‘ab-bc-ca’.

Graph II has 4 vertices with 4 edges which is forming a cycle ‘pq-qs-sr-rp’.

Graph III has 5 vertices with 5 edges which is forming a cycle ‘ik-km-ml-lj-ji’.

Cycle Graph
Hence all the given graphs are cycle graphs.

\subsection{Wheel Graph}
A wheel graph is obtained from a cycle graph Cn-1 by adding a new vertex. That new vertex is called a Hub which is connected to all the vertices of Cn.

Notation − Wn

No. of edges in Wn = No. of edges from hub to all other vertices +
                     No. of edges from all other nodes in cycle graph without a hub.
                     = (n–1) + (n–1)
                     = 2(n–1)
Example
Take a look at the following graphs. They are all wheel graphs.

Wheel Graph
In graph I, it is obtained from C3 by adding an vertex at the middle named as ‘d’. It is denoted as W4.

Number of edges in W4 = 2(n-1) = 2(3) = 6
In graph II, it is obtained from C4 by adding a vertex at the middle named as ‘t’. It is denoted as W5.

Number of edges in W5 = 2(n-1) = 2(4) = 8
In graph III, it is obtained from C6 by adding a vertex at the middle named as ‘o’. It is denoted as W7.

Number of edges in W4 = 2(n-1) = 2(6) = 12
Cyclic Graph
A graph with at least one cycle is called a cyclic graph.

Example
Cyclic Graph
In the above example graph, we have two cycles a-b-c-d-a and c-f-g-e-c. Hence it is called a cyclic graph.

\subsection{Acyclic Graph}
A graph with no cycles is called an acyclic graph.

Example
Acyclic Graph
In the above example graph, we do not have any cycles. Hence it is a non-cyclic graph.

Bipartite Graph
A simple graph G = (V, E) with vertex partition V = {V1, V2} is called a bipartite graph if every edge of E joins a vertex in V1 to a vertex in V2.

In general, a Bipertite graph has two sets of vertices, let us say, V1 and V2, and if an edge is drawn, it should connect any vertex in set V1 to any vertex in set V2.

Example
Bipartite Graph
In this graph, you can observe two sets of vertices − V1 and V2. Here, two edges named ‘ae’ and ‘bd’ are connecting the vertices of two sets V1 and V2.

\subsection{Complete Bipartite Graph}
A bipartite graph ‘G’, G = (V, E) with partition V = {V1, V2} is said to be a complete bipartite graph if every vertex in V1 is connected to every vertex of V2.

In general, a complete bipartite graph connects each vertex from set V1 to each vertex from set V2.

\subsection{Example}
The following graph is a complete bipartite graph because it has edges connecting each vertex from set V1 to each vertex from set V2.

Complete Bipartite Graph
If |V1| = m and |V2| = n, then the complete bipartite graph is denoted by Km, n.

Km,n has (m+n) vertices and (mn) edges.

Km,n is a regular graph if m=n.

In general, a complete bipartite graph is not a complete graph.

Km,n is a complete graph if m=n=1.

The maximum number of edges in a bipartite graph with n vertices is

[
n2
4
]
If n=10, k5, 5= ⌊
n2
4
⌋ = ⌊
102
4
⌋ = 25

Similarly K6, 4=24

K7, 3=21

K8, 2=16

K9, 1=9

If n=9, k5, 4 = ⌊
n2
4
⌋ = ⌊
92
4
⌋ = 20

Similarly K6, 3=18

K7, 2=14

K8, 1=8

‘G’ is a bipartite graph if ‘G’ has no cycles of odd length. A special case of bipartite graph is a star graph.

Star Graph
A complete bipartite graph of the form K1, n-1 is a star graph with n-vertices. A star graph is a complete bipartite graph if a single vertex belongs to one set and all the remaining vertices belong to the other set.

Example
Star Graph
In the above graphs, out of ‘n’ vertices, all the ‘n–1’ vertices are connected to a single vertex. Hence it is in the form of K1, n-1 which are star graphs.

\subsection{Complement of a Graph}
Let 'G−' be a simple graph with some vertices as that of ‘G’ and an edge {U, V} is present in 'G−', if the edge is not present in G. It means, two vertices are adjacent in 'G−' if the two vertices are not adjacent in G.

If the edges that exist in graph I are absent in another graph II, and if both graph I and graph II are combined together to form a complete graph, then graph I and graph II are called complements of each other.

Example
In the following example, graph-I has two edges ‘cd’ and ‘bd’. Its complement graph-II has four edges.

Complement of Graph
Note that the edges in graph-I are not present in graph-II and vice versa. Hence, the combination of both the graphs gives a complete graph of ‘n’ vertices.

Note − A combination of two complementary graphs gives a complete graph.

If ‘G’ is any simple graph, then

|E(G)| + |E('G-')| = |E(Kn)|, where n = number of vertices in the graph.

Example
Let ‘G’ be a simple graph with nine vertices and twelve edges, find the number of edges in 'G-'.

You have, |E(G)| + |E('G-')| = |E(Kn)|

12 + |E('G-')| = 

9(9-1) / 2 = 9C2
12 + |E('G-')| = 36

|E('G-')| = 24

‘G’ is a simple graph with 40 edges and its complement 'G−' has 38 edges. Find the number of vertices in the graph G or 'G−'.

Let the number of vertices in the graph be ‘n’.

We have, |E(G)| + |E('G-')| = |E(Kn)|

40 + 38 = 
n(n-1)
2

156 = n(n-1)

13(12) = n(n-1)

n = 13

\newpage
% https://www.tutorialspoint.com/graph_theory/graph_theory_trees.htm

Trees are graphs that do not contain even a single cycle. They represent hierarchical structure in a graphical form. Trees belong to the simplest class of graphs. Despite their simplicity, they have a rich structure.

Trees provide a range of useful applications as simple as a family tree to as complex as trees in data structures of computer science.

\subsection{Tree}
A connected acyclic graph is called a tree. In other words, a connected graph with no cycles is called a tree.

The edges of a tree are known as branches. Elements of trees are called their nodes. The nodes without child nodes are called leaf nodes.

A tree with ‘n’ vertices has ‘n-1’ edges. If it has one more edge extra than ‘n-1’, then the extra edge should obviously has to pair up with two vertices which leads to form a cycle. Then, it becomes a cyclic graph which is a violation for the tree graph.

Example 1
The graph shown here is a tree because it has no cycles and it is connected. It has four vertices and three edges, i.e., for ‘n’ vertices ‘n-1’ edges as mentioned in the definition.

Tree
Note − Every tree has at least two vertices of degree one.

Example 2
Tree 1
In the above example, the vertices ‘a’ and ‘d’ has degree one. And the other two vertices ‘b’ and ‘c’ has degree two. This is possible because for not forming a cycle, there should be at least two single edges anywhere in the graph. It is nothing but two edges with a degree of one.

Forest
A disconnected acyclic graph is called a forest. In other words, a disjoint collection of trees is called a forest.

Example
The following graph looks like two sub-graphs; but it is a single disconnected graph. There are no cycles in this graph. Hence, clearly it is a forest.

Forest
Spanning Trees
Let G be a connected graph, then the sub-graph H of G is called a spanning tree of G if −

H is a tree
H contains all vertices of G.
A spanning tree T of an undirected graph G is a subgraph that includes all of the vertices of G.

Example
Spanning Trees
In the above example, G is a connected graph and H is a sub-graph of G.

Clearly, the graph H has no cycles, it is a tree with six edges which is one less than the total number of vertices. Hence H is the Spanning tree of G.

\subsection{Circuit Rank}
Let ‘G’ be a connected graph with ‘n’ vertices and ‘m’ edges. A spanning tree ‘T’ of G contains (n-1) edges.

Therefore, the number of edges you need to delete from ‘G’ in order to get a spanning tree = m-(n-1), which is called the circuit rank of G.

This formula is true, because in a spanning tree you need to have ‘n-1’ edges. Out of ‘m’ edges, you need to keep ‘n–1’ edges in the graph.

Hence, deleting ‘n–1’ edges from ‘m’ gives the edges to be removed from the graph in order to get a spanning tree, which should not form a cycle.

Example
Take a look at the following graph −

Circuit Rank
For the graph given in the above example, you have m=7 edges and n=5 vertices.

Then the circuit rank is

\begin{eqnarray}
G = m – (n – 1)\\
  = 7 – (5 – 1)\\
  = 3\\
\end{eqnarray}

Example
Let ‘G’ be a connected graph with six vertices and the degree of each vertex is three. Find the circuit rank of ‘G’.

By the sum of degree of vertices theorem,

n
∑
i=1
 deg(Vi) = 2|E|
6 × 3 = 2|E|

|E| = 9

Circuit rank = |E| – (|V| – 1)

= 9 – (6 – 1) = 4

\subsection{Kirchoff’s Theorem}
Kirchoff’s theorem is useful in finding the number of spanning trees that can be formed from a connected graph.

Example
Kirchoff’s Theorem
The matrix ‘A’ be filled as, if there is an edge between two vertices, then it should be given as ‘1’, else ‘0’.

Kirchoff’s Theorem Example


\newpage


\section{Graph Theory - Connectivity}

% %- https://www.tutorialspoint.com/graph_theory/graph_theory_connectivity.htm
 
 Previous Page Next Page  
Whether it is possible to traverse a graph from one vertex to another is determined by how a graph is connected. Connectivity is a basic concept in Graph Theory. Connectivity defines whether a graph is connected or disconnected. It has subtopics based on edge and vertex, known as edge connectivity and vertex connectivity. Let us discuss them in detail.

\subsection{Connectivity}
A graph is said to be connected if there is a path between every pair of vertex. From every vertex to any other vertex, there should be some path to traverse. That is called the connectivity of a graph. A graph with multiple disconnected vertices and edges is said to be disconnected.

Example 1
In the following graph, it is possible to travel from one vertex to any other vertex. For example, one can traverse from vertex ‘a’ to vertex ‘e’ using the path ‘a-b-e’.

Connectivity
Example 2
In the following example, traversing from vertex ‘a’ to vertex ‘f’ is not possible because there is no path between them directly or indirectly. Hence it is a disconnected graph.

Connectivity 1
Cut Vertex
Let ‘G’ be a connected graph. A vertex V ∈ G is called a cut vertex of ‘G’, if ‘G-V’ (Delete ‘V’ from ‘G’) results in a disconnected graph. Removing a cut vertex from a graph breaks it in to two or more graphs.

Note − Removing a cut vertex may render a graph disconnected.

A connected graph ‘G’ may have at most (n–2) cut vertices.

Example
In the following graph, vertices ‘e’ and ‘c’ are the cut vertices.

Cut Vertex
By removing ‘e’ or ‘c’, the graph will become a disconnected graph.

Cut Vertices with eCut Vertices without e
Without ‘g’, there is no path between vertex ‘c’ and vertex ‘h’ and many other. Hence it is a disconnected graph with cut vertex as ‘e’. Similarly, ‘c’ is also a cut vertex for the above graph.

\subsection{Cut Edge (Bridge)}
Let ‘G’ be a connected graph. An edge ‘e’ ∈ G is called a cut edge if ‘G-e’ results in a disconnected graph.

If removing an edge in a graph results in to two or more graphs, then that edge is called a Cut Edge.

Example
In the following graph, the cut edge is [(c, e)]

Cut Edge
By removing the edge (c, e) from the graph, it becomes a disconnected graph.

Cut with EdgeCut without Edge
In the above graph, removing the edge (c, e) breaks the graph into two which is nothing but a disconnected graph. Hence, the edge (c, e) is a cut edge of the graph.

Note − Let ‘G’ be a connected graph with ‘n’ vertices, then

a cut edge e ∈ G if and only if the edge ‘e’ is not a part of any cycle in G.

the maximum number of cut edges possible is ‘n-1’.

whenever cut edges exist, cut vertices also exist because at least one vertex of a cut edge is a cut vertex.

if a cut vertex exists, then a cut edge may or may not exist.

\subsection{Cut Set of a Graph}
Let ‘G’= (V, E) be a connected graph. A subset E’ of E is called a cut set of G if deletion of all the edges of E’ from G makes G disconnect.

If deleting a certain number of edges from a graph makes it disconnected, then those deleted edges are called the cut set of the graph.

Example
Take a look at the following graph. Its cut set is E1 = {e1, e3, e5, e8}.

Cut Set of a Graph
After removing the cut set E1 from the graph, it would appear as follows −

Cut Set of a Graph 1
Similarly there are other cut sets that can disconnect the graph −

E3 = {e9} – Smallest cut set of the graph.
E4 = {e3, e4, e5}
Edge Connectivity
Let ‘G’ be a connected graph. The minimum number of edges whose removal makes ‘G’ disconnected is called edge connectivity of G.

Notation − λ(G)

In other words, the number of edges in a smallest cut set of G is called the edge connectivity of G.

If ‘G’ has a cut edge, then λ(G) is 1. (edge connectivity of G.)

Example
Take a look at the following graph. By removing two minimum edges, the connected graph becomes disconnected. Hence, its edge connectivity (λ(G)) is 2.

Edge Connectivity
Here are the four ways to disconnect the graph by removing two edges −

Edge Connectivity 1
Vertex Connectivity
Let ‘G’ be a connected graph. The minimum number of vertices whose removal makes ‘G’ either disconnected or reduces ‘G’ in to a trivial graph is called its vertex connectivity.

Notation − K(G)

Example
In the above graph, removing the vertices ‘e’ and ‘i’ makes the graph disconnected.

Vertex Connectivity
If G has a cut vertex, then K(G) = 1.

Notation − For any connected graph G,

K(G) ≤ λ(G) ≤ δ(G)

Vertex connectivity (K(G)), edge connectivity (λ(G)), minimum number of degrees of G(δ(G)).

Example
Calculate λ(G) and K(G) for the following graph −

Vertex Connectivity Example
Solution

From the graph,

δ(G) = 3

K(G) ≤ λ(G) ≤ δ(G) = 3 (1)

K(G) ≥ 2 (2)

Deleting the edges {d, e} and {b, h}, we can disconnect G.

Therefore,

λ(G) = 2

2 ≤ λ(G) ≤ δ(G) = 2 (3)

From (2) and (3), vertex connectivity K(G) = 2

\end{document}
