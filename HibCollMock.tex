\documentclass[a4paper,12pt]{article}
%%%%%%%%%%%%%%%%%%%%%%%%%%%%%%%%%%%%%%%%%%%%%%%%%%%%%%%%%%%%%%%%%%%%%%%%%%%%%%%%%%%%%%%%%%%%%%%%%%%%%%%%%%%%%%%%%%%%%%%%%%%%%%%%%%%%%%%%%%%%%%%%%%%%%%%%%%%%%%%%%%%%%%%%%%%%%%%%%%%%%%%%%%%%%%%%%%%%%%%%%%%%%%%%%%%%%%%%%%%%%%%%%%%%%%%%%%%%%%%%%%%%%%%%%%%%
\usepackage{eurosym}
\usepackage{vmargin}
\usepackage{amsmath}
\usepackage{fancyhdr}
\usepackage{listings}
\usepackage{framed}
\usepackage{graphics}
\usepackage{epsfig}
\usepackage{subfigure}
\usepackage{fancyhdr}

\setcounter{MaxMatrixCols}{10}
%TCIDATA{OutputFilter=LATEX.DLL}
%TCIDATA{Version=5.00.0.2570}
%TCIDATA{<META NAME="SaveForMode" CONTENT="1">}
%TCIDATA{LastRevised=Wednesday, February 23, 2011 13:24:34}
%TCIDATA{<META NAME="GraphicsSave" CONTENT="32">}
%TCIDATA{Language=American English}

\pagestyle{fancy}
\setmarginsrb{20mm}{0mm}{20mm}{25mm}{12mm}{11mm}{0mm}{11mm}
\lhead{Hibernia College} \rhead{Mr. Kevin O'Brien}
\chead{Mathematics for Computing}
%\input{tcilatex}

\begin{document}

% Question 1 - numbers - Started
% Question 2 - Sets - Not Started
% Question 3 - Logic - Not Started
% Question 4 - Functions - Started
% Question 5 - Graphs
% Question 6 - Digraphs and Relations
% Question 7 - 
% Question 8 -
% Question 9
% Question 10 - Matrices - STARTED

%---------------------------------------%
\subsection*{Question 1}
% 2003 Question 1
\begin{itemize}
\item[(a)] Working in base 2 perform the following calculation, showing all your working. [3 Marks]
\[110101_2 + 10111_2 - 100001_2\]
\item[(b)] Express the following hexadecimal number as a decimal number: $(A32.C)_{16}$.
[3 Marks]
\item[(c)] Convert the following decimal number into base 2, showing all your working:
$(253)_{10}$. \newline [2 Marks]
\item[(d)]  Express the recurring decimal $0.4242424\ldots$
 as a rational number in its simplest
form.\newline [2 Marks]
\end{itemize}



%---------------------------------------%
\subsection*{Question 5}
%2002 Question 7
\begin{itemize}
\item[(a)] Let $\mathcal{S}$ be a set and let $\mathcal{R}$ be a relation on $\mathcal{S}$
Explain what it means to say that $\mathcal{R}$ is

\begin{itemize}
\item[(i)] reflexive
\item[(ii)] symmetrix
\item[(iii)] anti-symmetric
\item[(iv)] transitive [4 Marks]
\end{itemize}

\item[(b)]  Let $\mathcal{S}$ be the set $\{2,3,4,5,6,7\}$ and a relation $\mathcal{R}$ is defined between the elements of $\mathcal{S}$ by
    \begin{center}
\begin{quote}
x is related to y if $|x - y| \in \{0,3\}$.
\end{quote}
\end{center}
\begin{itemize}
\item[(i)] Draw the relationship digraph. [3 Marks]
\item[(ii)] Determine whether or not $\mathcal{R}$ is reflexive, symmetric or transitive. In cases
where one of these properties does not hold give an example to show that
it does not hold.n \newline [3 Marks]
\end{itemize}
\end{itemize}
%--------------------------------------------%
