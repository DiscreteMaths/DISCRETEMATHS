\documentclass{beamer}

\usepackage{graphics}
\usepackage{amsmath}
\usepackage{amssymb}


\begin{document}

%Slide 1 Title Slide
\begin{frame}
\bigskip
{
\Huge
\[ \mbox{Proof By Induction}\]
}
{
\Large
\[ \mbox{kobriendublin.wordpress.com}\]
\[ \mbox{Youtube: StatsLabDublin}\]
}
\end{frame}
%------------------------------------------------ %
\begin{frame}
\frametitle{Proof By Induction}
\Large
Prove by Induction the following expression
\[ \sum_{r=1}^{r=n} r^2 = \frac{n(n+1)(2n+1)}{6} \]
for all $n \in N_0$.
\end{frame}
%------------------------------------------------ %
\begin{frame}
\frametitle{Proof By Induction}
\vspace{-1cm}
\Large
\begin{itemize}
\item \textbf{Step 1} : Is the proposition true for $n = 1$?
\item \textbf{Step 2} : Show that if the proposition is true for $n=k$, then it is also true for $n=k+1$.
\item \textbf{Step 3} : Conclude that the proposition is true for all natural numbers greater than or equal to one.
\end{itemize}
\end{frame}
%------------------------------------------------ %
\begin{frame}
\frametitle{Proof By Induction}
\Large
Prove by induction the following expression
\[ \sum_{r=1}^{r=n} r^2 = \frac{n(n+1)(2n+1)}{6} \]
for all $n \in N_0$.
\begin{itemize}
\item \textbf{Step 1} : Is the proposition true for $n = 1$? 
\end{itemize}
\textbf{Left-hand side}
\[ \sum_{r=1}^{r=1} r^2 = 1^2 = 1 \]


\end{frame}
%------------------------------------------------ %
\begin{frame}
\frametitle{Proof By Induction}
\Large
\textbf{Left-hand side}
\[ \sum_{r=1}^{r=1} r^2 = 1^2 = \boldsymbol{1} \]
\textbf{Right-hand side}

\[ \frac{n(n+1)(2n+1)}{6}  = \frac{1(1+1)(2\cdot 1+1)}{6} \]
\[  = \frac{1\times 2 \times 3}{6} = \frac{6}{6}=\boldsymbol{1} \]
\end{frame}
%------------------------------------------------ %
\begin{frame}
\frametitle{Proof By Induction}
\Large
\begin{itemize}
\item \textbf{Step 1} : Is the proposition true for $n = 1$? 
\end{itemize}
\bigskip
Yes: The left-hand side and right-hand side of the equation yield the same value.

\end{frame}
%------------------------------------------------ %

\begin{frame}
\frametitle{Proof By Induction}
\vspace{-1cm}
\Large
\begin{itemize}
\item \textbf{Step 2} : Show that if the proposition is true for $n=k$, then it is also true for $n=k+1$.
\end{itemize}
Given: \[ \sum_{r=1}^{r=k} r^2 = \frac{k(k+1)(2k+1)}{6} \]

To Prove: \[ \sum_{r=1}^{r=k+1} r^2 = \frac{(k+1)(k+2)(2k+3)}{6} \]
\end{frame}

%------------------------------------------------ %
\begin{frame}
\frametitle{Proof By Induction}
\vspace{-1cm}
\Large


\[ \sum_{r=1}^{r=k+1} r^2 = 1^2 + 2^2 + 3^2 + \ldots + k^2 + (k+1)^2 \]



\end{frame}
%------------------------------------------------ %
\begin{frame}
\frametitle{Proof By Induction}
\vspace{-1cm}
\Large


\[ \sum_{r=1}^{r=k+1} r^2 = 1^2 + 2^2 + 3^2 + \ldots + k^2 + (k+1)^2 \]

\[ \sum_{r=1}^{r=k+1} r^2 = \boldsymbol{\sum_{r=1}^{r=k} r^2} + (k+1)^2 \]

\end{frame}
%------------------------------------------------ %
\begin{frame}
\frametitle{Proof By Induction}
\vspace{-1cm}
\Large
\[ \sum_{r=1}^{r=k+1} r^2 = \boldsymbol{\sum_{r=1}^{r=k} r^2} + (k+1)^2 \]

\[ \sum_{r=1}^{r=k+1} r^2 = \frac{k(k+1(2k+1
))}{6} + (k+1)^2 \]
\end{frame}
%------------------------------------------------ %
\begin{frame}
\frametitle{Proof By Induction}
\vspace{-1cm}
\Large


\end{frame}
%------------------------------------------------ %
\end{document}