\documentclass[]{article}
\voffset=-1.5cm
\oddsidemargin=0.0cm
\textwidth = 470pt
\usepackage[utf8]{inputenc}
\usepackage[english]{babel}
\usepackage{framed}

\usepackage{multicol}
\usepackage{amsmath}
\usepackage{amssymb}
\usepackage{enumerate}
\usepackage{multicol}




%opening
\title{Number Systems - Tutorial Sheet}
\author{www.MathsResource.com}

\begin{document}
\maketitle
\Large

\subsection*{Question 1 : Binary numbers}
\begin{itemize}
	%---------------------------%
	\item[(a)] Express the following binary numbers as decimal numbers
	\begin{multicols}{2}
		\begin{itemize}
			\item[(i)] $101$
			\item[(ii)] $1101$
			\item[(iii)] $11011$
			\item[(iv)] $100101$
		\end{itemize}
	\end{multicols}

	%---------------------------%
	\item[(b)] Express the following decimal numbers as binary numbers
	\begin{multicols}{2}
		\begin{itemize}
			\item[(i)] 6
			\item[(ii)] 15
			\item[(iii)] 37
			\item[(iv)] 77
		\end{itemize}
	\end{multicols}

\end{itemize}
%-------------%-------------%-------------%----------------
\subsection*{Question 2}
A number is expressed in base 5 as $(234)_5$. What is it as decimal number?
Suppose you multiply $(234)_5$ by 5. what would be the answer in base 5.

%-------------%-------------%-------------%----------------
\subsection*{Question 3}

 Perform the following binary additions
 \begin{multicols}{2}
  \begin{itemize}
  	\item[(i)] $1011+ 1111$
  	\item[(ii)] $10101  + 10011$
  	\item[(iii)] $1010 + 11010$
  	\item[(iv)] $101010 + 10101 + 101$
  \end{itemize}
 \end{multicols}


%-------------%-------------%-------------%----------------
\subsection*{Question 4}
Perform the binary additions

\begin{itemize}
\item $(10111)_2 +(111010)_2$

\item $(1101)_2 + (1011)_2 + (1111)_2$
\end{itemize}

%-------------%-------------%-------------%----------------
\subsection*{Question 5}

Perform the binary subtractions using both the bit-borrowing method and the two's complement method.
\begin{itemize}
\item $(1001)_2 -(111)_2$
\item $(110000)_2 -(10111)_2$
\end{itemize}


%-------------%-------------%-------------%----------------
\subsection*{Question 6}

Perform the binary multiplications
\begin{itemize}
\item $(1101)_2 \times (101)_2$
\item $(1101)_2 \times (1101)_2$
\end{itemize}


%-------------%-------------%-------------%----------------
\subsection*{Question 7}
\begin{itemize}
	

\item[(a)] What is highest Hexadecimal number that can be written with two characters, and what is it's equivalent in decimal form?
What is the next highest hexadecimal number?

% \[FF = 255\] % Remark 
\item[(b)] Which of the following are not valid hexadecimal numbers?

\begin{multicols}{2}
\begin{itemize}
\item[(i)] A5G 	
\item[(ii)] 73 
\item[(iii)] EEF	
\item[(iv)] 101
\end{itemize}
\end{multicols}
\end{itemize}	



%-------------%-------------%-------------%----------------
\subsection*{Question 8 : Binary Substraction}
\textbf{Exercises:}

\begin{multicols}{2}
	\begin{itemize}
		\item[(i)] 110 - 10	
		\item[(ii)] 101 - 11  
		\item[(iii)] 1001 - 11	
		\item[(iv)] 10001 - 100 
		\item[(v)] 101001 - 1101
		\item[(vi)] 11010101-1101
	\end{itemize}
\end{multicols}



%-------------%-------------%-------------%----------------
\subsection*{Question 9}

\begin{itemize}
	%---------------------------%
	\item[(a)] Suppose 2341 is a base-5 number
	Compute the equivalent in each of the following forms:
	\begin{itemize}
		\item[(i)] decimal number
		\item[(ii)] hexadecimal number
		\item[(iii)] binary number
	\end{itemize}
	%---------------------------%
	\item[(b)] Perform the following binary additions
	\begin{itemize}
		\item[(i)] $1011+ 1111$
		\item[(ii)] $10101  + 10011$
		\item[(iii)] $1010 + 11010$
	\end{itemize}
	%---------------------------%
\end{itemize}

%-------------%-------------%-------------%----------------
\subsection*{Question 10}

Calculate working in hexadecimal
\begin{itemize}
\item $(BBB)_{16} + (A56)_{16}$
\item $(BBB)_{16} - (A56)_{16}$
\end{itemize} 


%-------------%-------------%-------------%----------------
\subsection*{Question 11}

Write the hex number $(EC4)_{16}$ in binary.
Write the binary number $(11110110101|)_2$ in hex.
%-------------%-------------%-------------%----------------
%-------------%-------------%-------------%----------------
\subsection*{Question 12}

Express the decimal number 753 in binary , base 5 and hexadecimal.

%-------------%-------------%-------------%----------------
%-------------%-------------%-------------%----------------
\subsection*{Question 13}

Express 42900 as a product of its prime factors, using index notation for repeated factors.

%-------------%-------------%-------------%----------------
%-------------%-------------%-------------%----------------
\subsection*{Question 14}

Expresse the recuring decimals 
\begin{itemize}
	\item[(i)] $0.727272\ldots$
	\item[(ii)] $0.126126126....$
	\item[(iii)] $0.7545454545...$
\end{itemize} 
as rational numbers in its simplest form.

%-------------%-------------%-------------%----------------
%-------------%-------------%-------------%----------------
\subsection*{Question 15}
Given that $\pi$ is an irrational number, can you say whether $\frac{\pi}{2}$ is rational or irrational.
or is it impossible to tell?

%-------------------------------------------------------- %
\subsection*{Question 16}
\begin{itemize}
	\item[(i)] Given x is the irrational positive number $\sqrt{2}$, express $x^8$ in binary notation\\
	\item[(ii)] From part (i), is $x^8$ a rational number?
\end{itemize}

%-------------%-------------%-------------%----------------
%-------------%-------------%-------------%----------------
\subsection*{Question 17}
\begin{enumerate}
\item[(i)] 5/7 lies between 0.714 and 0.715.
\item[(ii)] $\sqrt(2)$ is at least 1.41.
\item[(iii)] $\sqrt(3)$  9s at lrast 1.732 and at most 1.7322.
\end{enumerate}


%-------------%-------------%-------------%----------------
%-------------%-------------%-------------%----------------
\subsection*{Question 18}
\begin{itemize}
\item[(i)] Write down the numbers 0.0000526 in floating point form.
\item[(ii)] How is the number 1 expressed in floating point form.
\end{itemize}


%-------------%-------------%-------------%----------------
%-------------%-------------%-------------%----------------
\subsection*{Question 19}
\begin{itemize}
\item Deduce that every composite integer $n$ has a prime factor such that $p \leq \sqrt{n}$.
\item Decide whether 899 is a prime.
\end{itemize}

%-------------%-------------%-------------%----------------
%-------------%-------------%-------------%----------------
\subsection*{Question 20}

\begin{itemize}
\item What would be the maximum numbber of digits that a decimal fraction with denominator 13 
could have in a recurring block in theory?

\item Can you predict which other fractions with denominator 13 will have the same digits as 1/13 in their recurring block?
\end{itemize}





\end{document}

