\documentclass{beamer}

\usepackage{amsmath}
\usepackage{framed}
\usepackage{amssymb}
\usepackage{graphics}

\begin{document}
\begin{frame}
\Huge
\[ \mbox{Mathematics for Computing} \]
\huge
\[ \mbox{Binary Strings and Functions} \]

\Large
\[ \mbox{kobriendublin.wordpress.com} \]
\[ \mbox{Twitter: @StatsLabDublin} \]

\end{frame}
%--------------------------------------------------- %
\begin{frame}
\frametitle{Binary Strings and Functions}
\Large
Let $S$ be the set of all 4 bit binary strings. 

The function $f : S \rightarrow \mathbb{Z}$
is defined by the rule:
\[f(x) = \mbox{the number of zeros in x}\]
for each binary string $x \in S$.\\
Find:
\begin{enumerate}
\item the number of elements in the domain
\item $f(1000)$
\item the set of pre-images of 1
\item the range of $f$.
\end{enumerate}

\end{frame}
\begin{frame}
\frametitle{Binary Strings and Functions}
\Large
\vspace{-1cm}
\textbf{Part 1}: Find the number of all elements in the domain.\\


\begin{itemize}
\item In this question, the domain is $S$
\item $S$ is the set of all 4 bit binary strings. 
\begin{center}
{
\LARGE
1010 \; 1101 \; 0101\; ... \\
0010 \; 1001 \; 1110\; ... \\
}
\end{center}
\item There are $2^4$ 4-bit binary strings.
\[ 2^4 = \boldsymbol{16}\]
\end{itemize}
\end{frame}
%-------------------------------------- %
\begin{frame}
\frametitle{Binary Strings and Functions}
\Large
\vspace{-1cm}
\textbf{Part 2}: Find $f(1000)$

\begin{itemize}
\item Recall
\[f(x) = \mbox{the number of zeros in x}\]
\item How many zeros in the binary string 1000? 
\[f(1000) =3\]
\end{itemize}


\end{frame}
%-------------------------------------- %
\begin{frame}
\frametitle{Binary Strings and Functions}
\Large
\vspace{-1cm}
\textbf{Part 3}: Find the set of pre-images of 1.

\begin{itemize}
\item Find all the 4-bit binary strings with only one zero.
\LARGE
\[\{1110, \; 1101, \; 1011,\; 0111\; \}\] 

\end{itemize}
\end{frame}
%-------------------------------------- %
\begin{frame}
\frametitle{Binary Strings and Functions}
\Large
\vspace{-1cm}
\textbf{Part 4}: Find the range of $f$.

\begin{itemize}
\item What is minimum and maximum numbers of 0s in a 4-bit binary string
\item There can be no 0s or as many as 4.
\item $f(1111) = 0$ and $f(0000) = 4$
\item The range of $f$ is therefore
\LARGE 
\[\{0,1,2,3,4\}\] 

\end{itemize}
\end{frame}
%-------------------------------------- %
\end{document}