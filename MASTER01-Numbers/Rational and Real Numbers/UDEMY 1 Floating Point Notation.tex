\documentclass[12pt]{beamer} % use larger type; default would be 10pt


\usepackage{graphicx} 
\usepackage{framed}

\begin{document}


\begin{frame}
\frametitle{Floating Point Numbers}

In computing, floating point describes a method of 
representing an approximation of a real number in a 
way that can support a wide range of values. 
The numbers are, in general, represented approximately 
to a fixed number of significant digits (the mantissa) and scaled using an exponent. 
\end{frame}
%---------------------------------------------------------------%
\begin{frame}
\frametitle{Floating Point Numbers}
In essence, computers are integer machines and are capable of representing real numbers only by using complex codes. The most popular code for representing real numbers is called the IEEE Floating-Point Standard .
The term floating point is derived from the fact that there is no fixed number of digits before and after the decimal point; that is, the decimal point can float. 
\end{frame}
%---------------------------------------------------------------%
\begin{frame}
\frametitle{Floating Point Numbers}
There are also representations in 
which the number of digits before and after the decimal point is set, called fixed-point representations. In general, floating-point representations are slower and less accurate than fixed-point representations, but they can handle a larger range of numbers.
\end{frame}
%---------------------------------------------------------------%
\end{document}
