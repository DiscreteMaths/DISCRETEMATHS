\section*{Question 1}
\subsection*{Part A :  Binary numbers}
\begin{itemize}
%---------------------------%
\item[(a)] Express the following binary numbers as decimal numbers
\begin{itemize}
\item[(i)] $11011$
\item[(ii)] $100101$
\end{itemize}
%---------------------------%
\item[(b)] Express the following decimal numbers as binary numbers
\begin{itemize}
\item[(i)] 6
\item[(ii)] 15
\item[(iii)] 37
\end{itemize}
%---------------------------%
\item[(c)] Perform the following binary additions
\begin{itemize}
\item[(i)] $1011+ 1111$
\item[(ii)] $10101  + 10011$
\item[(iii)] $1010 + 11010$
\end{itemize}

\end{itemize}
\subsection*{Part 2 : Hexadecimal numbers}
%------------------------------------------------------------------------%
\begin{itemize}
	\item[(i)] Calculate the decimal equivalent of the hexadecimal number $(A2F.D)_{16}$
	\item[(ii)] Working in base 2, compute the following binary additions, showing all you workings
	\[(1110)_2 + (11011)_2 + (1101)_2 \]
	\item[(iv)] Express the recurring decimal $0.727272\ldots$ as a rational number in its simplest form.
\end{itemize}


\subsection*{Part 3 : Base 5 and Base 8 numbers}
\begin{itemize}
	%---------------------------%
	\item[(a)] Suppose 2341 is a base-5 number
	Compute the equivalent in each of the following forms:
	\begin{itemize}
		\item[(i)] decimal number
		\item[(ii)] hexadecimal number
		\item[(iii)] binary number
	\end{itemize}
	%---------------------------%
	\item[(b)] Perform the following binary additions
	\begin{itemize}
		\item[(i)] $1011+ 1111$
		\item[(ii)] $10101  + 10011$
		\item[(iii)] $1010 + 11010$
	\end{itemize}
	%---------------------------%
\end{itemize}

\subsection*{Part 4 : Real and Rational Numbers}
\begin{itemize}
	\item[(i)] Express the recurring decimal $0.727272\ldots$ as a rational number in its simplest form.
\end{itemize}
%-------------------------------------------------------- %
\subsection*{Miscellaneous Questions}
\begin{itemize}
	\item[(i)] Given x is the irrational positive number $\sqrt{2}$, express $x^8$ in binary notation\\
	\item[(ii)] From part (i), is $x^8$ a rational number?
\end{itemize}
\newpage

\subsection*{Part B :  Hexadecimal numbers}
%------------------------------------------------------------------------%
\begin{itemize}
\item[(i)] Calculate the decimal equivalent of the hexadecimal number $(A2F.D)_{16}$
\item[(ii)] Working in base 2, compute the following binary additions, showing all you workings
\[(1110)_2 + (11011)_2 + (1101)_2 \]
\item[(iv)] Express the recurring decimal $0.727272\ldots$ as a rational number in its simplest form.
\end{itemize}


\subsection*{Part C : Base 5 and Base 8 numbers}
\begin{itemize}
%---------------------------%
\item[(a)] Suppose 2341 is a base-5 number
Compute the equivalent in each of the following forms:
\begin{itemize}
\item[(i)] decimal number
\item[(ii)] hexadecimal number
\item[(iii)] binary number
\end{itemize}
%---------------------------%

%---------------------------%
\end{itemize}
\subsection*{Part D : Real and Rational Numbers}
\begin{itemize}
\item[(i)] Express the recurring decimal $0.727272\ldots$ as a rational number in its simplest form.
\end{itemize}
%-------------------------------------------------------- %
\subsection*{Part E : Real and Rational Numbers}
\begin{itemize}
\item[(i)] Given x is the irrational positive number $\sqrt{2}$, express $x^8$ in binary notation.
\item[(ii)] From part (i), is $x^8$ a rational number?
\end{itemize}
%========================================================================%

\subsection*{Part 1 : Binary numbers}
\begin{itemize}
	%---------------------------%
	\item[(a)] Express the following binary numbers as decimal numbers
	\begin{multicols}{2}
		\begin{itemize}
			\item[(i)] $11011$
			\item[(ii)] $100101$
		\end{itemize}
	\end{multicols}
	%---------------------------%
	\item[(b)] Express the following decimal numbers as binary numbers
	\begin{multicols}{2}
		\begin{itemize}
			\item[(i)] 6
			\item[(ii)] 15
			\item[(iii)] 37
		\end{itemize}
	\end{multicols}
	%---------------------------%
	\item[(c)] Perform the following binary additions
	\begin{itemize}
		\item[(i)] $1011+ 1111$
		\item[(ii)] $10101  + 10011$
		\item[(iii)] $1010 + 11010$
	\end{itemize}
	%---------------------------%
	
\end{itemize}
% End of Subsection
%=========================================================== %
\end{document}