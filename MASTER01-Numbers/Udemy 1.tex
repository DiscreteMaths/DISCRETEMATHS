
\subsection*{Hexadecimal}
\begin{itemize}
\item Hexadecimal basic concepts:	
\item Hexadecimal is base 16.  
\item There are 16 digits in counting (0, 1, 2, 3, 4, 5, 6, 7, 8, 9, A, B, C, D, E, F)
\item When you reach 16, you carry a “1” over to the next column
\item The number after F (decimal 15) is 10 in hex (or 16 in decimal)
\end{itemize}
%---------------------------------------------------------

\begin{tabular}{|c||c|c|c|c|c|c|}
\hline Hex & A & B  & C & D & E & F \\  \hline
\hline Dec & 10 & 11 & 12 & 13  &14  &15  \\  \hline
\hline 
\end{tabular} 

\textbf{Conversion to Decimal}
\[\textbf{5} \times 16^2 + \textbf{A} \times 16^1  + \textbf{9} \times 16^0\]
\[=5 \times 16^2 + 10 \times 16^1  + 9 \times 16^0\]
\[= 1280 + 160 + 9\]


% Use the following steps to perform hexadecimal addition:

\subsection*{Hexadecimal Addition}
\begin{itemize}
\item		Add one column at a time.

\item		Convert to decimal and add the numbers.

\item	If the result of step two is 16 or larger subtract the result from 16 and carry 1 to the next column.

\item	If the result of step two is less than 16, convert the number to hexadecimal.

\end{itemize}

\[\begin{array}{cccc}
	&		&		&		\\	
	&	5	&	A	&	9	\\	
	-&	6	&	9	&	4	\\	\hline
	&		&		&		\\	
\end{array}\]

\subsection*{Hexadecimal Addition}
\begin{itemize}
\item		Add one column at a time.

\item		Convert to decimal and add the numbers.

\item	If the result of step two is 16 or larger subtract the result from 16 and carry 1 to the next column.

\item	If the result of step two is less than 16, convert the number to hexadecimal.

\end{itemize}
\[\begin{array}{cccc}
	&		&		&		\\	
	&	5	&	A	&	9	\\	
	+&	6	&	9	&	4	\\	\hline
	&		&		&		\\	
\end{array} \]

\noindent 1. Add one column at a time\\
2. Convert to decimal and add (9 + 4 = 13)\\
3. Decimal 13 is hexadecimal D\\
4. Next column :Convert to decimal and add (10 + 9 = 19)\\
5. (19 larger than 16) 14 in Hexadecimal \\
6. Leave 4 carry to one\\
7.  Next Colulm add 1 + 5 + 6 = 12)\\
8. Decimal 12 is hexadecimal C\\

\[\begin{array}{cccc}
	&		&		&		\\	
	&	5	&	A	&	9	\\	
	+&	6	&	9	&	4	\\	\hline
	&	C	&	4	&	D	\\	
\end{array} \]

\subsection*{Hexadecimal Subtraction}
\[\begin{array}{cccc}
	&		&		&		\\	
	&	B	&	B	&	B	\\	
	-&	A	&	5	&	D	\\	\hline
	&		&		&		\\	
\end{array} \]


\[\begin{array}{cccc}
	&		&		&		\\	
	&	B	&	\textbf{\emph{A}}	&	\textbf{\emph{1B}}	\\	
	&	A	&	5	&	D	\\	\hline
	&		&		&		\\	
\end{array} \]


\begin{itemize}
\item Binary to Decimal conversion
\item Decimal to Binary conversion
\item Decimal to Hexadecimal conversion
\item Hexadecimal to Decimal conversion
\item Subtraction and Addition
\item Subtraction and Addition
\end{itemize}


2008 Zone A Question 1 A
Find the Hexadecimal equivalent pf the binary number 1101101.01

Re-express the number in terms of 4 digit blocks, starting from the decimal point:

$\emph{0}1101101.01\emph{00}$


\begin{itemize}
\item The first 4 digits 0110 correspond to the Hex integer 6
\item The second 4 digits 1101 correspond to the Hex integer D
\item The last four digits (i.e after the decimal place) 0100 correspond to the Hex integer 4
\item So the Hexadecimal equivalent is $6D.4$
\end{itemize}



Find the binary equivalent of  $A9.8$

\begin{itemize}
\item A: 1010
\item 9: 1001
\item B: 1011
\end{itemize}

10101001.1011
%---------------------------------------%
Question 2B 2009 Zone A

Convert the decimal number 299 to a hexadecimal number.
\begin{array}{|c|c|c|c|c|}
	&	Division	& Quotient & Remainder (Dec)	&Remainder (Hex)	\\
299	&	18.6875	&	18	&	11	&	B	\\
18	&	1.125	&	1	&	2	&	2	\\
1	&	0.0625	&	0	&	1	&	1	\\
\end{array}

The answer: $(299)_{10} = (12B)_{16}$




Floating Point Notation
Membership Tables



\begin{tabular}{|c|c|c|c|}
	&		&	Quotient	&	Remainder	\\
507	&\emph{	253.5	}&	253	&	1	\\
253	&\emph{	126.5	}&	126	&	1	\\
126	&\emph{	63	}&	63	&	0	\\
63	&\emph{	31.5	}&	31	&	1	\\
31	&\emph{	15.5	}&	15	&	1	\\
15	&\emph{	7.5	}&	7	&	1	\\
7	&\emph{	3.5	}&	3	&	1	\\
3	&\emph{	1.5	}&	1	&	1	\\
1	&\emph{	0.5	}&	0	&	1	\\
\end{tabular} 

Correct Answer : 111111011

%------------------------------------------------------------------------------------%

\begin{tabular}{|c||c|c|c|c|c|c|}
\hline Hex & A & B  & C & D & E & F \\  \hline
\hline Dec & 10 & 11 & 12 & 13  &14  &15  \\  \hline
\hline 
\end{tabular} 

ABA+EFA

Hexadecimal basic concepts:	
Hexadecimal is base 16.  
There are 16 digits in counting (0, 1, 2, 3, 4, 5, 6, 7, 8, 9, A, B, C, D, E, F)
When you reach 16, you carry a “1” over to the next column
The number after F (decimal 15) is 10 in hex (or 16 in decimal)

Use the following steps to perform hexadecimal addition:


\textbf{5} \times 16^2 + \textbf{A} \times 16^1  + \textbf{9} \times 16^0
=5 \times 16^2 + 10 \times 16^1  + 9 \times 16^0
= 1280 + 160 + 9

\subsection*{Hexadecimal Addition}
\begin{itemize}
\item		Add one column at a time.

\item		Convert to decimal and add the numbers.

\item	If the result of step two is 16 or larger subtract the result from 16 and carry 1 to the next column.

\item	If the result of step two is less than 16, convert the number to hexadecimal.

\end{itemize}
\[\begin{array}{cc:c:c}
	&		&		&		\\	
	&	5	&	A	&	9	\\	
	-&	6	&	9	&	4	\\	\hline
	&		&		&		\\	
\end{array} \]

1. Add one column at a time
2. Convert to decimal and add (9 + 4 = 13)
3. Decimal 13 is hexadecimal D
4. Add next column
2.	Convert to decimal and add (10 + 9 = 19)
3.	Follow 16 or larger than 16 rule 

2.	Convert to decimal and add (1 + 5 + 6 = 12)
3.	Follow less than 16 rule, convert to hex
	Decimal 12 is hexadecimal C




\subsection*{Hexadecimal Subtraction}
\[\begin{array}{cccc}
	&		&		&		\\	
	&	B	&	B	&	B	\\	
	-&	A	&	5	&	D	\\	\hline
	&		&		&		\\	
\end{array} \]


\[\]\begin{array}{cccc}
	&		&		&		\\	
	&	B	&	\textbf{\emph{A}}	&	\textbf{\emph{1B}}	\\	
	&	A	&	5	&	D	\\	\hline
	&		&		&		\\	
\end{array} \]

\begin{itemize}
\item Binary to Decimal conversion
\item Decimal to Binary conversion
\item Decimal to Hexadecimal conversion
\item Hexadecimal to Decimal conversion
\item Subtraction and Addition
\item Subtraction and Addition
\end{itemize}

Floating Point Notation
Membership Tables



\begin{tabular}{|c|c|c|c|}
	&		&	Quotient	&	Remainder	\\
507	&\emph{	253.5	}&	253	&	1	\\
253	&\emph{	126.5	}&	126	&	1	\\
126	&\emph{	63	}&	63	&	0	\\
63	&\emph{	31.5	}&	31	&	1	\\
31	&\emph{	15.5	}&	15	&	1	\\
15	&\emph{	7.5	}&	7	&	1	\\
7	&\emph{	3.5	}&	3	&	1	\\
3	&\emph{	1.5	}&	1	&	1	\\
1	&\emph{	0.5	}&	0	&	1	\\
\end{tabular}

Correct Answer : 111111011

%------------------------------------------------------------------------------------%

\begin{tabular}{|c||c|c|c|c|c|c|}
\hline Hex & A & B  & C & D & E & F \\  \hline
\hline Dec & 10 & 11 & 12 & 13  &14  &15  \\  \hline
\hline
\end{tabular}

ABA+EFA

Hexadecimal basic concepts:	
Hexadecimal is base 16.
There are 16 digits in counting (0, 1, 2, 3, 4, 5, 6, 7, 8, 9, A, B, C, D, E, F)
When you reach 16, you carry a “1” over to the next column
The number after F (decimal 15) is 10 in hex (or 16 in decimal)

Use the following steps to perform hexadecimal addition:


\textbf{5} \times 16^2 + \textbf{A} \times 16^1  + \textbf{9} \times 16^0
=5 \times 16^2 + 10 \times 16^1  + 9 \times 16^0
= 1280 + 160 + 9

\subsection*{Hexadecimal Addition}
\begin{itemize}
\item		Add one column at a time.

\item		Convert to decimal and add the numbers.

\item	If the result of step two is 16 or larger subtract the result from 16 and carry 1 to the next column.

\item	If the result of step two is less than 16, convert the number to hexadecimal.

\end{itemize}
\[\begin{array}{cc:c:c}
	&		&		&		\\	
	&	5	&	A	&	9	\\	
	-&	6	&	9	&	4	\\	\hline
	&		&		&		\\	
\end{array} \]

1. Add one column at a time
2. Convert to decimal and add (9 + 4 = 13)
3. Decimal 13 is hexadecimal D
4. Add next column
2.	Convert to decimal and add (10 + 9 = 19)
3.	Follow 16 or larger than 16 rule

2.	Convert to decimal and add (1 + 5 + 6 = 12)
3.	Follow less than 16 rule, convert to hex
	Decimal 12 is hexadecimal C




\subsection*{Hexadecimal Subtraction}
\[\begin{array}{cccc}
	&		&		&		\\	
	&	B	&	B	&	B	\\	
	-&	A	&	5	&	D	\\	\hline
	&		&		&		\\	
\end{array} \]


\[\]\begin{array}{cccc}
	&		&		&		\\	
	&	B	&	\textbf{\emph{A}}	&	\textbf{\emph{1B}}	\\	
	&	A	&	5	&	D	\\	\hline
	&		&		&		\\	
\end{array} \]
----------------------------------------------------------%
\section*{Section B: Hexadecimal Numbers}
\begin{itemize}
\item[B.1]
\item[B.2]
\item[B.3]
\item[B.4]
\item[B.5]
\item[B.6]
\item[B.7]
\item[B.8]
\end{itemize}
\newpage


\subsection*{Associative Laws}
\[ (A \cup B) \cup C =  A \cup (B \cup C)  \]
\[ (A \cap B) \cap C =  A \cap (B \cap C)  \]

\subsection*{Distributive Laws}
\[ (A \cup B) \cap C =  (A \cup B) \cap (A \cup C)  \]
\[ (A \cap B) \cup C =  (A \cap B) \cup (A \cap C)  \]


\[ (A \cup B) \cap B^{\prime} \]
%----------------------------------------------------------%
\section*{Section C: Real and Rational Numbers}
\begin{itemize}
\item[C.1]
\item[C.2]
\item[C.3]
\item[C.4]
\item[C.5]
\item[C.6]
\item[C.7]
\item[C.8]
\end{itemize}
\newpage
%----------------------------------------------------------%
\newpage
\section*{Formulae}
\begin{itemize}




\item Bayes' Theorem:
\begin{equation*}
P(B|A)=\frac{P\left(A|B\right) \times P(B) }{P\left( A\right) }.
\end{equation*}



\item Binomial probability distribution:
\begin{equation*}
P(X = k) = ^{n}C_{k} \times p^{k} \times \left( 1-p\right) ^{n-k}\qquad \left( \text{where}\qquad
^{n}C_{k} =\frac{n!}{k!\left(n-k\right) !}. \right)
\end{equation*}

\item Poisson probability distribution:
\begin{equation*}
P(X = k) =\frac{m^{k}\mathrm{e}^{-m}}{k!}.
\end{equation*}
\end{itemize}

\end{document} 