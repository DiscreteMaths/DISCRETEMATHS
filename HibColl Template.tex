\documentclass[a4]{beamer}
\usepackage{amssymb}
\usepackage{graphicx}
\usepackage{subfigure}
\usepackage{newlfont}
\usepackage{eurosym}
\usepackage{amsmath,amsthm,amsfonts}
%\usepackage{beamerthemesplit}
\usepackage{pgf,pgfarrows,pgfnodes,pgfautomata,pgfheaps,pgfshade}
\usepackage{mathptmx}  % Font Family
\usepackage{helvet}   % Font Family
\usepackage{color}

\mode<presentation> {
 \usetheme{Default} % was Frankfurt
 \useinnertheme{rounded}
 \useoutertheme{infolines}
 \usefonttheme{serif}
 %\usecolortheme{wolverine}
% \usecolortheme{rose}
\usefonttheme{structurebold}
}

\setbeamercovered{dynamic}

\title[MathsCast]{MathsCast Presentations \\ {\normalsize Coefficient of Variation }}
\author[Kevin O'Brien]{Kevin O'Brien \\ {\scriptsize kevin.obrien@ul.ie}}
\date{Summer 2011}
\institute[Maths \& Stats]{Dept. of Mathematics \& Statistics, \\ University \textit{of} Limerick}

\renewcommand{\arraystretch}{1.5}


%------------------------------------------------------------------------%
\begin{document}

\begin{frame}
\titlepage
\end{frame}

%------------------------------------------------------------------------%
\section{Coefficient of Variation}
%------------------------------------------------------------------------%

%---------------------------------------------
\frame{
\frametitle{The Coefficient of Variation}
\vspace{-2.3cm}
The coefficient of variation is the ratio of the standard deviation to the mean, and is a useful statistic for comparing the degree of variation from one data series to another, particularly  if the means are drastically different from each other. \\\bigskip When comparing different distributions are compared, the distribution with the highest CV value has the greatest dispersion.
}
\frame{
\frametitle{Computing The Coefficient of Variation}
\vspace{-2.3cm}
The coefficient of variation ($CV$) is calculated by dividing the sample standard deviation ($s$) by the absolute value of the sample mean ($|\bar{x}|$). The coefficient of variation is normally expressed as a percentage. \\ \bigskip The formula is as follows:\\


\[
CV = {s \over |\bar{x}|} \times 100\%
\]

\bigskip
N.B. Remember that the standard deviation is the square root of the variance.
}

%---------------------------------------------
\frame{
\frametitle{Example 1}
\vspace{-2cm}
Data on the delivery time (in hours) of purchases were collected for a random sample of similar orders sent to a building supplies company. The following summary statistics are available:
\[
\begin{tabular}{|c|c|c|c|c|c|}
  \hline
  % after \\: \hline or \cline{col1-col2} \cline{col3-col4} ...
  mean & median & variance  & maximum & minimum & IQR \\ \hline
  12 & 13 & 16 & 20 & 2 & 8\\
  \hline
\end{tabular}
\]
Use the data to compute the coefficient of variation.
}


%---------------------------------------------
\frame{
\frametitle{Example 2}
\vspace{-1cm}
In 2000, the average donation to a charity was \euro 225 with a standard deviation of \euro 45.\\
In 2001, the average donation to the same charity was \euro400 with a standard deviation of \euro 60.\\
\bigskip
What are the coefficients of variation for both years?\\
\vspace{2.5cm}
In which year do the donations show a more dispersed distribution?\\
}


%------------------------------------------------------------------------%
\end{document}





