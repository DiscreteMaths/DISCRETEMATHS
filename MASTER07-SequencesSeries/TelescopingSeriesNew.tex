In mathematics, a telescoping series is a series whose partial sums eventually only have a fixed number of terms after cancellation.[1][2] Such a technique is also known as the method of differences.
For example, the series
\sum_{n=1}^\infty\frac{1}{n(n+1)}
simplifies as
\begin{align}
\sum_{n=1}^\infty \frac{1}{n(n+1)} & {} = \sum_{n=1}^\infty \left( \frac{1}{n} - \frac{1}{n+1} \right) \\
{} & {} = \lim_{N\to\infty} \sum_{n=1}^N \left( \frac{1}{n} - \frac{1}{n+1} \right) \\
{} & {} = \lim_{N\to\infty} \left\lbrack {\left(1 - \frac{1}{2}\right) + \left(\frac{1}{2} - \frac{1}{3}\right) + \cdots + \left(\frac{1}{N} - \frac{1}{N+1}\right) } \right\rbrack  \\
{} & {} = \lim_{N\to\infty} \left\lbrack {  1 + \left( - \frac{1}{2} + \frac{1}{2}\right) + \left( - \frac{1}{3} + \frac{1}{3}\right) + \cdots + \left( - \frac{1}{N} + \frac{1}{N}\right) - \frac{1}{N+1} } \right\rbrack = 1.
\end{align}
