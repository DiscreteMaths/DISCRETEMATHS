

\documentclass{beamer}

\usepackage[english]{babel}
% or whatever

\usepackage[utf8]{inputenc}
% or whatever

\usepackage{times}
\usepackage[T1]{fontenc}
% Or whatever. Note that the encoding and the font should match. If T1
\begin{document}
%---------------------------------------%
\section{Sequences}

%---------------------------------------%
\begin{frame}
\frametitle{Sequences}
\large
\begin{itemize}
\item A sequence is any succession of numbers. 
\item A general sequence is denoted by
\[ u_1, u_2, \ldots , u_n, \ldots \]
in which $u_1$ is the first term, $u_2$ is the second term and $u_n$ is the $n$-th 
term in the sequence.
\item If the sequence goes on forever it is called an \textbf{infinite sequence},
otherwise it is called a \textbf{finite sequence}.
\item A sequence usually has a rule, which is a way to find the value of each term.
\end{itemize}
\end{frame}
%--------------------------------------%
\begin{frame}
\frametitle{Sequences}
\large
\vspace{-1.5cm}
\textbf{Examples of Sequences}
\begin{itemize}
\item $\{1, 2, 3, 4 ,\ldots\}$ is a very simple sequence (and it is an infinite sequence)
\item $\{20, 25, 30, 35, \ldots \}$ is also an infinite sequence
\item $\{1, 3, 5, 7\}$ is the sequence of the first 4 odd numbers (and is a finite sequence)
\end{itemize}
\end{frame}
%--------------------------------------%
\begin{frame}
\frametitle{Sequences: Recursive Formulas}
\begin{itemize}
\item Often the rule for evaluating the current term in the sequence depends on the values of one or more previous terms.
\item In such cases, these rules are called \textbf{recursive formulas}.
\item Recursive Rules also have initial values that allow the terms to be evaluated.
\item The rule defining the \textit{Fibonacci} sequence is a recursive formula.
\end{itemize}
\end{frame}

%---------------------------------------%
\subsection{Fibonnacci Sequences}
\begin{frame}
\frametitle{Fibonnacci Sequence}

\[ u_n = u_{n-1} + u_{n-2} \qquad \mbox{ for } n \geq 3 ,\; u_1=0,\;u_2=1 \]


The first few terms of the Fibonnaci Sequence looks like this:
\[ 1, 1, 2, 3, 5, 8, \ldots\]


\end{frame}

%---------------------------------------%
\section{Series}
\begin{frame}
\frametitle{Series}
\large
\begin{itemize}
\item A series is the sum of the terms of a sequence. 
\[ S_n = u_1 + u_2 + u_3+ \ldots +u_n\]
\item A series is usuall expressed in terms pf \textbf{sigma notation}.
\item It is useful to remember the following, particularl in the context of proof by induction.
\[S_1 = u_1\]
\[S_{n+1} = S_n + u_{n+1}\]

\end{itemize}

\end{frame}

%---------------------------------------%
\end{document}
