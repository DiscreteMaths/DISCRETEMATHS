

\documentclass{beamer}

\usepackage[english]{babel}
% or whatever

\usepackage[utf8]{inputenc}
% or whatever

\usepackage{times}
\usepackage[T1]{fontenc}
% Or whatever. Note that the encoding and the font should match. If T1
\begin{document}
%---------------------------------------------%

\begin{frame}

\frametitle{Proof by Induction}

Mathematical Induction is a special way of proving things. It has only 3 steps:

\begin{description}
\item[Step 1.] Show it is true for the first one
\item[Step 2.]
\item[Step 3.] Show that if any one is true then the next one is true
Then all will be true
\end{description}

\end{frame}
%---------------------------------------------%
\begin{frame}
\frametitle{Proof by Induction : Example 1}
\large
\begin{itemize}
\item Use proof by induction to show that $3^n-1$ is a multiple of 2, for all values of the integer $n$.
\item Is that true? Let us find out.
\end{itemize} 
\end{frame}
%---------------------------------------------%
\begin{frame}
\frametitle{Proof by Induction : Example 1}
Step 1. Show it is true for n=1

\begin{itemize}
\item $3^1-1 = 3-1 = 2$
\item Yes 2 is a multiple of 2. 
\item $3^1-1$ is true
\end{itemize}

\end{frame}
%---------------------------------------------%
\begin{frame}
\frametitle{Example 1}

Step 2. Assume it is true for $n=k$ i.e. assume that $3^k-1$ is true

This is an assumption that we treat 
as a fact for the rest of this exercise.


\end{frame}
%---------------------------------------------%
\begin{frame}
\frametitle{Example 1}
\large
Now, prove that $3^{k+1}-1$ is a multiple of 2
 
 
$3^{k+1}$ is also $3\times3^{k}$
And each of these are multiples of 2
\end{frame}
%---------------------------------------------%
\begin{frame}
\frametitle{Example 1}
\large
Because:
2.3k is a multiple of 2 (you are multiplying by 2)
3k-1 is true (we said that in the assumption above)
So:
3k+1-1 is true

\end{frame}
%---------------------------------------------%
\begin{frame}
\frametitle{Example 2}
\large
Example: Adding up Odd Numbers
\[1 + 3 + 5 + \ldots + (2n-1) = n^2\]
1. Show it is true for n=1
1 = $1^2$ is True
 
2. Assume it is true for n=k
\[1 + 3 + 5 + \ldots + (2k-1) = k^2\]

\end{frame}
%---------------------------------------------%
\begin{frame}
\frametitle{Example 2}
\large 
Now, prove it is true for "k+1"
\[ 1 + 3 + 5 + \ldots + (2k-1) + (2(k+1)-1) = (k+1)^2  \]
\end{frame}
%---------------------------------------------%
\begin{frame}
\frametitle{Example 2}
 
We know that $1 + 3 + 5 + \ldots + (2k-1) = k^2$ 

(the assumption above), so we can do a replacement for all but the last term:
\[k^2 + (2(k+1)-1) = (k+1)^2\]

\end{frame}
%---------------------------------------------%
\begin{frame}
\frametitle{Example 2}

\begin{itemize}
\item Now expand all terms:
\[ k^2 + 2k + 2 - 1 = k^2 + 2k+1 \]
\item And simplify:
\[ k^2 + 2k + 1 = k^2 + 2k + 1 \]
\end{itemize}
They are the same! So it is true.

\end{frame}
%---------------------------------------------%
\begin{frame}
\frametitle{Example 2}

So the following expression is true. 
\[ 1 + 3 + 5 + \ldots + (2(k+1)-1) = (k+1)^2 \]
\end{frame}
%---------------------------------------------%

%------------------------------------------------------------------------------%
\begin{frame}
\frametitle{Proof by Induction}
Prove by induction that the series $3 + 7 + 11 + \ldots$ has the sum to $r$
terms given by $S_r$, where
\[S_r = 2r^2 + r.\]

\end{frame}
%-------------------------------------------------------------------------------%
\begin{frame}
\frametitle{Proof by Induction}
Step 1

Demonstrate for $r=1$
\begin{itemize}
\item We know that first term is 3 
\item $S_1  = 2(1)^2 + 1 = 3$
\end{itemize}

\end{frame}

%-------------------------------------------------------------------------------%
\begin{frame}
\frametitle{Proof by Induction}
Step 2  : Make statement for $r=k$
\begin{itemize}
\item 
\end{itemize}
\end{frame}
%-------------------------------------------------------------------------------%
\begin{frame}
\frametitle{Proof by Induction}
Step 2  : Make statement for $r=k+1$
\[S_{k+1} = 2(k+1)^2 + (k+1)\]
\begin{itemize}
\item $(k+1)^2$ = $k^2 + 2k + 1$
\item $2(k+1)^2$ = $2k^2 + 4k + 2$
\end{itemize}
\[S_{k+1} = 2k^2 + 4k + 2+ (k+1) = 2k^2 + 5k + 3 \]
\end{frame}
%-------------------------------------------------------------------------------%
\begin{frame}
\frametitle{Proof by Induction}

\[S_{k} = 2k^2 + k\]

\[S_{k+1} = 2k^2 + 4k + 2+ (k+1) = 2k^2 + 5k + 3 \]

Difference is $4k+3$ which is also expressed as $4(k+1)-1$

\end{frame}
%--------------------------------------------------------------------------------%
%--------------------------------% 
\begin{frame}
\frametitle{Proof By Induction}
\large

A sequence is defined by the recurrence relation
\[ x_{n+2} = 3x_{n+1} - 2x_{n} \]

The initial terms are $x_1$=1 and $x_2$ =3.

Find the values of $x_3$ and $x_4$ showing your workings.

\[ x_{3} = 3x_{2} - 2x_{1} = 3(3) - 2(1) = 7 \]

\[ x_{4} = 3x_{3} - 2x_{2} = 3(7) - 2(3) = 15 \]


\end{frame}
%--------------------------------% 
\begin{frame}
\frametitle{Proof By Induction}
\large
Prove by induction that
\[ x_n = 2^n - 1 \qquad \mbox{ for } n \geq 1 \]

\end{frame}

%--------------------------------%
\begin{frame}
\frametitle{Proof By Induction}
Step 1 Show that statement is true for $n=1$.
(N.B. $x_{1} = 1$).

\[ x_n = 2^n - 1 \]
\[ x_1 = 2^1 - 1 = 2 - 1 = 1\]
\end{frame}
%--------------------------------%
%--------------------------------%
\begin{frame}
\frametitle{Proof By Induction}
Step 2 Assume that statement is true for $n=k$.

\[ x_k = 2^k - 1 \]
Similarly we will assume it for $n=k-1$.The reason for this will become obvious later on.)
\[ x_{k-1} = 2^{k-1} - 1 \]
\end{frame}
%--------------------------------%
%--------------------------------%
\begin{frame}
\frametitle{Proof By Induction}
\textbf{Step 3} Show that statement is true for $n=k+1$.

\[ x_{k+1} = 2^{k+1} - 1 \]

Re-expressing this

\[ x_{k+1} = 2.2^{k} - 1 \]
\end{frame}
%--------------------------------%
%--------------------------------%
\begin{frame}
\frametitle{Proof By Induction}
\large
Recall
\Large 
\[ x_{k+1} = 3x_{k} - 2x_{k-1} \]

\vspace{0.5cm}
\large
Looking at right hand side 
\begin{itemize}
\item First Term: $\mbox{       }  3x_{k}\; = \;3(2^{k} - 1) \;=\; 3.2^{k} - 3$
\item Second Term: $ \mbox{       }2x_{k-1} \;= \;2(2^{k-1} - 1) \;= \;2^{k} - 2$
\end{itemize}
\vspace{0.5cm}
\Large
\[ x_{k+1} = (3.2^{k} - 3) - (2^{k} - 2) = 2.2^{k} + 1 \]
\[ x_{k+1} = 2.2^{k} + 1 = 2^{k+1} + 1 \]
\end{frame}
%--------------------------------%
 \end{document}
 





