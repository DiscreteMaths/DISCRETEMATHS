
\documentclass{beamer}

\usepackage[english]{babel}
% or whatever

\usepackage[utf8]{inputenc}
% or whatever

\usepackage{times}
\usepackage[T1]{fontenc}
% Or whatever. Note that the encoding and the font should match. If T1
\begin{document}

%--------------------------------------------------%
\begin{frame}
\frametitle{Recurrence Relations}
\textit{CIS102 2004 Question 7 2 Marks}
\Large

\begin{itemize}
\item Consider the sequence given by \[ 1, 4, 7, 10, 13, \ldots\]
\item State a recurrence relation which expresses the nth term, $u_n$
, in terms of the$(n - 1)$th term, $u_{n-1}$, 
\item State a recurrence relation which expresses the nth term, $u_n$
, in terms of the first term $u_1$.
\end{itemize}

\end{frame}
%--------------------------------------------------%
\begin{frame}
\frametitle{Recurrence Relations}
\Large
\begin{itemize}
\item $u_1$ = 1 , $u_2$ = 4, $u_3$ = 7 etc 
\item Difference in successive terms is 3.
\item Therefore we can say 
\[ u_n = u_{n-1} + 3 \]
\end{itemize}
\end{frame}
%--------------------------------------------------%
\begin{frame}
\frametitle{Recurrence Relations}
\large
\begin{itemize}
\item Difference between $u_2$ and $u_1$ is 3 (i.e. $1 \times 3$).
\item Difference between $u_3$ and $u_1$ is 6 (i.e. $2 \times 3$)
\item Difference between $u_4$ and $u_1$ is 9 (i.e. $3 \times 3$)
\item In general the difference between $u_n$ and $u_1$ is $(n-1)\times 3$.
\[ u_{n} = u_1 + 3 \times (n-1) \]
\[ u_{n} = 1 + (3n-3) = 3n-2\]
\item Equivalently
\[ u_{n+1} = u_1 + 3n = 3n+1\]
\end{itemize}
\end{frame}
%--------------------------------------------------%
\begin{frame}
\frametitle{Recurrence Relations}
\Large
\begin{itemize}
\item Another sequence is defined by the recurrence relation 
\[ u_n = u_{n-1} + 2n-1 \] and
$u_1$ = 1.
\item Calculate $u_2$ , $u_3$ , $u_4$  and $u_5$ .
\item (Answers 1,4,9,16,25)
\end{itemize}
\end{frame}
\end{document}
%---------------------------------------------------%
\begin{frame}

\begin{array}{|c|c|c|c|}
$n$    & $u_{n-1}$ & 2n-1 & $u_n$ \\ \hline
2      & 1         &  3   & $u_2$ = 4 \\ \hline
3      & 4         &  5   & $u_3$ = 9 \\ \hline
4      & 9         &  7   & $u_4$ = 16 \\ \hline
5      & 16        &  9   & $u_5$ = 25 \\ \hline
\end{array}

\end{frame}
%---------------------------------------------------%
\end{document}




 