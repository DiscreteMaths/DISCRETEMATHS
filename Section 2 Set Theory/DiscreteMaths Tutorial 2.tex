\documentclass[12pt]{article}

\usepackage{amsmath}
\usepackage{amssymb}

\begin{document}
\begin{center}
\huge{Mathematics for Computing}\\
\LARGE{Set Theory}
\end{center}

\section{Set Theory}
DRAFT

%---------------------------------%

\subsection{Important Operations in Set Theory}

\begin{itemize}
\item Union ($\cup$) - also known as the OR operator. A union signifies a bringing together. The union of the sets A and B consists of the elements that are in either A or B.
\item Intersection ($\cap$) - also known as the AND operator. An intersection is where two things meet. The intersection of the sets A and B consists of the elements that in both A and B.
\item Complement ($^{c}$) - The complement of the set A consists of all of the elements in the universal set that are not elements of A.
\end{itemize}

%------------------------------------------------------------------------%
\begin{itemize}
\item[2.a] Describe the following set by the listing method
\[ \{ 2r+1 : r \in Z^{+} and r \leq 5  \} \]
\item[2.b] Let A,B be subsets of the universal set U.


\end{itemize}
%------------------------------------------------------------------------%
\begin{itemize}
\item[3.a]
Let n be an element of the set $\{10, 11, 12, 13, 14, 15, 16, 17, 18, 19\}$,
and p and q be the propositions:
p : n is even, $q : n > 15$.
Draw up truth tables for the following statements and find the values of n for
which they are true:
(i) $p \vee \neg q$
(ii) $\neg p \wedge q$
\end{itemize}


%--------------------------------- %
\subsection{Universal Set and the Empty Set}
The first is the \textbf{\textit{universal set}}, typically denoted $U$. This set is all of the elements that we may choose from. This set may be different from one setting to the next. 

For example one universal set may be the set of real numbers whereas for another problem the universal set may be the whole numbers $\{0, 1, 2,\ldots\}$.

The other set that requires consideration is called the \textit{\textbf{empty set}}. The empty set is the unique set is the set with no elements. We write this as $\{ \}$ and denote this set by $\emptyset$.


%---------------------------------%
\section{Number Sets}
The font that the symbols are written in (i.e. $\mathbb{N}$, $\mathbb{R}$) is known as \textit{\textbf{blackboard font}}.
\begin{itemize}
\item $\mathbb{N}$ Natural Numbers ($0,1,2,3$) 
(Not used in the CIS102 Syllabus)
\item $\mathbb{Z}$ Integers ($-3,-2,-1,0,1,2,3, \ldots$)
\begin{itemize}
\item[$\ast$] $\mathbb{Z}^{+}$ Positive Integers
\item[$\ast$] $\mathbb{Z}^{-}$ Negative Integers
\end{itemize}
\item $\mathbb{Q}$ Rational Numbers
\item $\mathbb{R}$ Real Numbers
\end{itemize}

\begin{itemize}

\item[(a)] 

\item[(b)]

\item[(c)]

\end{itemize}

\end{document}