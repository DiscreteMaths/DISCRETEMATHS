\documentclass{beamer}

\usepackage{amsmath}
\usepackage{amssymb}


\begin{document}
%---------------------------------------------------%
%SLIDE SET 1
%---------------------------------------------------%
%---------------------------------------------------%
% Opening Slide 1

\begin{frame}
\Huge
\[\mbox{Discrete Mathematics}\]
\Huge
\[\mbox{Set Theory}\]

\Large
\[\mbox{www.Stats-Lab.com}\]
\Large
\[\mbox{Twitter: @StatsLabDublin}\]

\end{frame}
%---------------------------------------------------%

\begin{frame}
\frametitle{Equal and Equivalent Sets}
\Large
\vspace{-1cm}
Difference between equal sets and equivalent sets

\begin{itemize}
\item Consider the sets \textbf{A} and \textbf{B}
\[ \boldsymbol{A} = \{ 1,2,3,4,5,6 \} \] 
\[ \boldsymbol{B} = \{1,2,3,4,5,6 \} \]
\vspace{0.2cm}
\item \textbf{A} and \textbf{B} are \textit{\textbf{equal}} sets because \textit{\textbf{all}} their
elements are precisely the \textit{\textbf{same}}.
\end{itemize}


\end{frame}
%---------------------------------------------------%

\begin{frame}
\frametitle{Equal and Equivalent Sets}
\Large
\vspace{-0.7cm}
Difference between equal sets and equivalent sets

\begin{itemize}
\item Consider the sets \textbf{C} and \textbf{D}
\[ \boldsymbol{C} = \{a,b,c,d,e,f\} \]  \[ \boldsymbol{D} = \{3,4,5,6,7,8\} \]
\vspace{0.2cm}
\item \textbf{C} and \textbf{D} are \textit{\textbf{equivalent}} sets
because the cardinality of both the sets is the same (i.e. 6.)
\item However \textbf{C} and \textbf{D} are not equal, as they are comprised of different elements.
\end{itemize}





\end{frame}
%---------------------------------------------------%

\begin{frame}
\frametitle{Equal and Equivalent Sets}
\Large
\vspace{-1cm}

\begin{itemize}
\item Necessarily all equal sets are equivalent sets.
\item But are equivalent sets equal sets?

\item No, because equivalent sets are sets that have the \textit{\textbf{same}} cardinality but equal sets are sets that all
their elements are precisely the \textit{\textbf{same}}. 
\end{itemize}

\textbf{Example}:
\begin{itemize} \item \textbf{X}=\{p,q,r\} and \textbf{Y}=\{1,2,3\} are equivalent sets 
\item \textbf{E} =\{m,n,o,p\}
and \textbf{F}=\{m,n,o,p\} are equal and equivalent sets
\end{itemize}
\end{frame}
%---------------------------------------------------%
%---------------------------------------------------%
%SLIDE SET 2
%---------------------------------------------------%
%---------------------------------------------------%
% Opening Slide 2

\begin{frame}
\Huge
\[\mbox{Discrete Mathematics}\]
\Huge
\[\mbox{Set Theory}\]

\Large
\[\mbox{www.Stats-Lab.com}\]
\Large
\[\mbox{Twitter: @StatsLabDublin}\]

\end{frame}
%---------------------------------------------------%
%---------------------------------------------------%
% Schaum 1:4
\begin{frame}
\Large
\frametitle{Set Operations}
Let $U = \{1,2,\ldots, 9\}$ be the universal set, and let
\begin{itemize}
\item A = $\{1, 2, 3, 4, 5\}$,  
\item B = $\{4, 5, 6, 7\}$,  
\item C = $\{5, 6, 7, 8, 9\}$
\item D = $\{1, 3, 5, 7, 9\}$,
\item E = $\{2, 4, 6, 8\}$,
\item F = $\{1, 5, 9\}$.
\end{itemize}
\end{frame}
%---------------------------------------------------%
%---------------------------------------------------%
% Schaum 1:4
\begin{frame}
\frametitle{Set Operations}
\Large
\vspace{-3cm}
Find: 
\begin{itemize}
\item[(a)] A $\cup$ B and A $\cap$ B, 
\item[(b)] C $\cup$ D and C $\cap$ D, 
\item[(c)] E $\cup$ F and E $\cap$ F.
\end{itemize}

\end{frame}
%---------------------------------------------------%
%---------------------------------------------------%
% Schaum 1:4
\begin{frame}
\frametitle{Set Operations}
\Large
\vspace{-3cm}
Find: 
\begin{itemize}
\item[(a)] A $\cup$ B and A $\cap$ B, \bigskip
\item A = $\{1, 2, 3, 4, 5\}$  
\item B = $\{4, 5, 6, 7\}$  
\end{itemize}

\end{frame}
%---------------------------------------------------%
%---------------------------------------------------%
% Schaum 1:4
\begin{frame}
\frametitle{Set Operations}
\Large
\vspace{-3cm}
Find: 
\begin{itemize}
\item[(b)] C $\cup$ D and C $\cap$ D,  \bigskip
\item C = $\{5, 6, 7, 8, 9\}$
\item D = $\{1, 3, 5, 7, 9\}$
\end{itemize}

\end{frame}
%---------------------------------------------------%
%---------------------------------------------------%
% Schaum 1:4
\begin{frame}
\frametitle{Set Operations}
\vspace{-3cm}
\Large
Find: 
\begin{itemize}
\item[(c)] E $\cup$ F and E $\cap$ F. \bigskip
\item E = $\{2, 4, 6, 8\}$

\item F = $\{1, 5, 9\}$
\end{itemize}

\end{frame}
%---------------------------------------------------%
%---------------------------------------------------%
%SLIDE SET 3 : FINITE SETS
%---------------------------------------------------%
%---------------------------------------------------%
% Opening Slide 3

\begin{frame}
\Huge
\[\mbox{Discrete Mathematics}\]
\Huge
\[\mbox{Set Theory : Finite Sets}\]

\Large
\[\mbox{www.Stats-Lab.com}\]
\Large
\[\mbox{Twitter: @StatsLabDublin}\]

\end{frame}
%---------------------------------------------------%
\begin{frame}
\frametitle{Finite Sets}
\Large
\vspace{-1.2cm}
Determine which of the following sets are finite:
\begin{itemize}
\item[(a)] Set of Prime numbers %infinite.
\vspace{0.4cm}
\item[(b)] Set of two digit Prime numbers %infinite.
%\item[(a)] Lines parallel to the x axis. 
\vspace{0.4cm}
\item[(c)] Letters in the English alphabet.
\vspace{0.4cm}
\item[(d)] Integers which are multiples of 5.
\vspace{0.4cm}
%\item[(c)] Letters of the English alphabet
\item[(e)] Days of the week % finite sets.
%\item[(e)] The number of grains of rice in a ton of the grain
\end{itemize}
\end{frame}




%---------------------------------------------------%
%---------------------------------------------------%
%SLIDE SET 4
%---------------------------------------------------%
%---------------------------------------------------%
% Opening Slide 4

\begin{frame}
\Huge
\[\mbox{Discrete Mathematics}\]
\Huge
\[\mbox{Set Theory}\]

\Large
\[\mbox{www.Stats-Lab.com}\]
\Large
\[\mbox{Twitter: @StatsLabDublin}\]

\end{frame}

%---------------------------------------------------%
\begin{frame}
\frametitle{Set Theory}
\Large
\vspace{-1.8cm}
Given the set \textbf{A} is contructed as follows 
\[ [\{a, b\}, \{c\}, \{d, e, f \} ]. \]

\begin{itemize}
\item[(a)] List the elements of \textbf{A}. 
\item[(b)] Find the cardinality of \textbf{A} : $n(\boldsymbol{A})$. 
\item[(c)] Find the power set of \textbf{A}.
\end{itemize}

\end{frame}

%---------------------------------------------------%
\begin{frame}
\frametitle{Set Theory}
\Large
\vspace{-3.8cm}
\[\boldsymbol{A} = [\{a, b\}, \{c\},\{d, e, f \}]. \]
\begin{itemize}
\item[(a)] List the elements of \textbf{A}. 
\end{itemize}

\end{frame}

%---------------------------------------------------%
%---------------------------------------------------%
\begin{frame}
\frametitle{Set Theory}
\Large
\vspace{-3.8cm}
\[\boldsymbol{A} = [\{a, b\}, \{c\},\{d, e, f \}]. \]
\begin{itemize}
\item[(b)] Find the cardinality of \textbf{A} : $n(\boldsymbol{A})$. 
\end{itemize}

\end{frame}

%---------------------------------------------------%
%---------------------------------------------------%
\begin{frame}
\frametitle{Set Theory}
\Large
\vspace{-3.8cm}
\[\boldsymbol{A} = [\{a, b\}, \{c\},\{d, e, f \}]. \]
\begin{itemize}
\item[(c)] Find the power set of \textbf{A}. 
\end{itemize}

\end{frame}

%---------------------------------------------------%
%---------------------------------------------------%
%SLIDE SET 5
%---------------------------------------------------%
%---------------------------------------------------%
% Opening Slide 5

\begin{frame}
\Huge
\[\mbox{Discrete Mathematics}\]
\Huge
\[\mbox{Set Theory}\]

\Large
\[\mbox{www.Stats-Lab.com}\]
\Large
\[\mbox{Twitter: @StatsLabDublin}\]

\end{frame}
\begin{frame}
\Large
Consider the set \textbf{A}, which is a subset of the the universal set of real numbers $\mathbb{R}$
 \[\boldsymbol{A} = \{4, \sqrt{2}, 2/3, -2.5, -5, 33, \sqrt{9}, \pi \}\]
Using formal set notation, write the sets of:
\begin{itemize}
\item[(a)] natural numbers in \textbf{A}
\item[(b)] integers in \textbf{A}
\item[(c)] rational numbers in \textbf{A}
\item[(d)] irrational numbers in \textbf{A}
\end{itemize}
\end{frame}
%---------------------------------------------------------- %
\begin{frame}
%0 Reveals
\Large
\vspace{-0.5cm}
\[\boldsymbol{A} = \{4,\; \sqrt{2},\; 2/3,\; -2.5,\; -5,\; 33,\; \sqrt{9},\; \pi \}\]
\vspace{-0.5cm}
\begin{itemize}
\item[(a)] natural numbers in \textbf{A}\\
\phantom{Answer: $\{4,\; 33,\; \sqrt{9}\}$}
\item[(b)] integers in \textbf{A}\\
\phantom{Answer: $\{4,\; -5, 33,\; \sqrt{9}\}$}
\item[(c)] rational numbers in \textbf{A}\\
\phantom{Answer: $\{4,\; 2/3,\; -2.5,\; -5,\; 33,\; \sqrt{9}\}$}
\item[(d)] irrational numbers in \textbf{A}\\
\phantom{Answer: $\{\sqrt{2},\; \pi\}$}
\end{itemize}
\end{frame}

%---------------------------------------------------------- %
\begin{frame}
%1 Reveals
\Large
\vspace{-0.5cm}
\[\boldsymbol{A} = \{4,\; \sqrt{2},\; 2/3,\; -2.5,\; -5,\; 33,\; \sqrt{9},\; \pi \}\]
\vspace{-0.5cm}
\begin{itemize}
\item[(a)] natural numbers in \textbf{A}\\
\hspace{1cm} \textit{Answer}: $\{4,\; 33,\; \sqrt{9}\}$
\item[(b)] integers in \textbf{A}\\
\phantom{Answer: $\{4,\; -5, 33,\; \sqrt{9}\}$}
\item[(c)] rational numbers in \textbf{A}\\
\phantom{Answer: $\{4,\; 2/3,\; -2.5,\; -5,\; 33,\; \sqrt{9}\}$}
\item[(d)] irrational numbers in \textbf{A}\\
\phantom{Answer: $\{\sqrt{2},\; \pi\}$}
\end{itemize}
\end{frame}

%---------------------------------------------------------- %
\begin{frame}
%2 Reveals
\Large
\vspace{-0.5cm}
\[\boldsymbol{A} = \{4,\; \sqrt{2},\; 2/3,\; -2.5,\; -5,\; 33,\; \sqrt{9},\; \pi \}\]
\vspace{-0.5cm}
\begin{itemize}
\item[(a)] natural numbers in \textbf{A}\\
\hspace{1cm} \textit{Answer}: $\{4,\; 33,\; \sqrt{9}\}$
\item[(b)] integers in \textbf{A}\\
\hspace{1cm} \textit{Answer}: $\{4,\; -5,\; 33,\; \sqrt{9}\}$
\item[(c)] rational numbers in \textbf{A}\\
\phantom{Answer: $\{4,\; 2/3,\; -2.5,\; -5,\; 33,\; \sqrt{9}\}$}
\item[(d)] irrational numbers in \textbf{A}\\
\phantom{Answer: $\{\sqrt{2},\; \pi\}$}
\end{itemize}
\end{frame}

%---------------------------------------------------------- %
\begin{frame}
%3 Reveals
\Large
\vspace{-0.5cm}
\[\boldsymbol{A} = \{4,\; \sqrt{2},\; 2/3,\; -2.5,\; -5,\; 33,\; \sqrt{9},\; \pi \}\]
\vspace{-0.5cm}
\begin{itemize}
\item[(a)] natural numbers in \textbf{A}\\
\hspace{1cm} \textit{Answer}: $\{4,\; 33,\; \sqrt{9}\}$
\item[(b)] integers in \textbf{A}\\
\hspace{1cm} \textit{Answer}: $\{4,\; -5,\; 33,\; \sqrt{9}\}$
\item[(c)] rational numbers in \textbf{A}\\
\hspace{1cm} \textit{Answer}: $\{4,\; 2/3,\; -2.5,\; -5,\; 33,\; \sqrt{9}\}$
\item[(d)] irrational numbers in \textbf{A}\\
\phantom{Answer: $\{\sqrt{2},\; \pi\}$}
\end{itemize}
\end{frame}
%---------------------------------------------------------- %
\begin{frame}
%4 Reveals
\Large
\vspace{-0.5cm}
\[\boldsymbol{A} = \{4,\; \sqrt{2},\; 2/3,\; -2.5,\; -5,\; 33,\; \sqrt{9},\; \pi \}\]
\vspace{-0.5cm}
\begin{itemize}
\item[(a)] natural numbers in \textbf{A}\\
\hspace{1cm} \textit{Answer}: $\{4,\; 33,\; \sqrt{9}\}$
\item[(b)] integers in \textbf{A}\\
\hspace{1cm} \textit{Answer}: $\{4,\; -5,\; 33,\; \sqrt{9}\}$
\item[(c)] rational numbers in \textbf{A}\\
\hspace{1cm} \textit{Answer}: $\{4,\; 2/3,\; -2.5,\; -5,\; 33,\; \sqrt{9}\}$
\item[(d)] irrational numbers in \textbf{A}\\
\hspace{1cm} \textit{Answer}: $\{\sqrt{2},\; \pi\}$
\end{itemize}
\end{frame}



%---------------------------------------------------%
%---------------------------------------------------%
%END OF SLIDES
%---------------------------------------------------%
%---------------------------------------------------%
\begin{frame}

End of Slide Set
\end{frame}
\end{document}

