% http://www.sosmath.com/matrix/markov/markov.html
% http://www.dartmouth.edu/~chance/teaching_aids/books_articles/probability_book/Chapter11.pdf
%------------------------------------------------------------------------------------------%
\begin{frame}{MathsResource.com}{Markov Chains}
\Large
\begin{itemize}
\item Ireland never has two nice days in a row. 

\item If they have a nice day, they are just as likely to have light showers as heavy rain the
next day. 

\item If they have light showers or heavy rain, they have an even chance of having the same
the next day. 

\item If there is change from light showers or heavy rain, only half of the time is this a
change to a nice day
\end{itemize}
\end{frame}
%------------------------------------------------------------------------------------------%
\begin{frame}{MathsResource.com}{Markov Chains}
\Large
\textbf{Transition Matrix}
 
% R  1/2 1/4 1/4
% N  1/2  0    1/2
% S  1/4  1/4 1/2


\end{frame}
%------------------------------------------------------------------------------------------%

\begin{frame}{MathsResource.com}{Markov Chains}
\Large
\textbf{Absorbing States}
\begin{itemize}
\item A state $S_i$ of a Markov chain is called \textbf{absorbing} if it is impossible
to leave it (i.e. $p_{ii} = 1$). 
\item A Markov chain is absorbing if it has at least one absorbing
state, and if from every state it is possible to go to an absorbing state (not necessarily
in one step).
\item In a Markov chain that contains an absorbing state, a state which is not absorbing is
called a \textbf{transient} state.
\end{itemize}
\end{frame}
%---------------------------------------------------------%
\section{Canonical Form}
\begin{frame}{MathsResource.com}{Markov Chains}
\Large
\textbf{
Canonical Form}
\begin{itemize}
\item Consider an arbitrary absorbing Markov chain. \item Renumber the states so that the
transient states come first. \item If there are r absorbing states and t transient states,
the transition matrix will have the following canonical form
\end{itemize}

%----------------------------------------------------------------------------%
\end{document}