\documentclass{beamer}

\usepackage{amsmath}
\usepackage{amssymb}

\begin{document}

\begin{frame}
\frametitle{Permutations}
\Large
\begin{itemize}
\item The notion of permutation relates to the act of permuting (rearranging) objects or values. 
\item Informally, a permutation of a set of objects is an arrangement of those objects into a particular order. 

\item For example, there are six permutations of the set $\{1,2,3\}$, namely (1,2,3), (1,3,2), (2,1,3), (2,3,1), (3,1,2), and (3,2,1). 
\item As another example, an anagram of a word is a permutation of its letters. 

\end{itemize}

\end{frame}
%-----------------------------------------------------%
\begin{frame}
\frametitle{Permutations}
\Large
\begin{itemize}
\item Permutations where repetition is allowed: 
\[ n! \]
\item Permutations where repetition is not allowed
\[ \frac{n!}{(n-k)!} \]
\end{itemize}

\end{frame}
%-----------------------------------------------------%

\end{document}