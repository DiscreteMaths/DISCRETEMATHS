
\documentclass{beamer}

\usepackage{amsmath}

\begin{document}
\begin{frame}
\tableofcontents
\end{frame}
\section{Counting methods and probability}

Chapter 5
Systems of linear equations and
matrices
Essential reading
M&B Sections 6.1, 6.2. Only part of this chapter is covered in Epp (Section 11.8)
Keywords

\end{frame}
\begin{frame}
Systems of linear equations; Gaussian elimination. Matrix algebra.
It often happens that information has to be stored in a list or a table.
Matrices are a convenient way of representing such data, as you
have already experienced with the incidence matrix of graphs and
digraphs. We thus study the basics of matrix algebra in this chapter.

\end{frame}
\begin{frame}
We shall also see that a system of 7L linear equations in 7L unknowns
can be conveniently expressed in the form of a matrix equation. The
equations in this form can be solved in an algorithmic manner. The
method is suitable for programming into a computer, which
distinguishes it from the methods for solving equations you learnt in
school. The algorithm for solving systems of linear equations is
known as Gaussian Elimination.

\end{frame}
\begin{frame}
Systems of linear equations
Learning objectives for this section
I Recognizing linear and non-linear equations
I Writing down the augmented matrix of a system of linear equations
I Using Gaussian elimination to reduce a matrix to row echelon form
I Using Gaussian elimination and back substitution to lind all solutions to a system
of linear equations
Definition 5.1 A linear equation in n. unknowns 1-1,1»; . . . ,1.-,L is an
equation of the form
(1111 + agar; + . . . + awmn = b


\end{frame}
\begin{frame}

ClS102 Mathematics ior computing volume 2
where (11,112. . . . , an and b are given real numbers.
Example 5.1 The equation
, ,, _ 1
.£1+2£F2’l.3 — E
is a linear equation.
The equations $112 = 3 or  + 1:2 = 4 or log(r1) + .'I?2 = 7 are not
linear.
Definition 5.2 A system of linear equations in n. unknowns
r1‘w2_ . . . ,1~,, is a collection of one or more linear equations involving
the same unknowns. A solutionfor this system is an assignment of
real numbers .11 : kl, ac; = k2, . . . ,;1c,,, = kn which satisfies each of the
equations in the system. To solve the system we want to find all the
solutions to the system.

\end{frame}
\begin{frame}
Example 5.2 A system of linear equations and its solution
Consider the following system of two equations in two unknowns.
3.r1 + 612 = 60 (5.1)
211 — 1'2 = 10. (5.2)
Then %>< Equation (5.1) is
001 + 2172 = Z0. (5.3)
Now Equation (5.2) — 2>< Equation (5.3) gives
—5.rZ = -30.
Therefore
.rg = 6.
Substituting back into Equation (5.3) gives
av1=2O—2av2 =20—12=8.
So the only solution to this system is $1 = 8 and 12 = 6.
We can solve a system of m equations with n unknowns using a
method similar to the one of Example 5.2 by first reducing to m — 1
equations with n — 1 unknowns, then m — 2 equations with n — 2
unknowns, and so on. To describe accurately the general procedure
we use matrices.
Definition 5.3 An m. >< n matrix is a rectangular array of real numbers
with m rows and n columns.
01 -2
A1< >
_§ 3 5

\end{frame}
\begin{frame}
We shall usually use upper case letters such as A, B, X to represent
matrices. The only exception being for matrices with just one row or
one column which we represent by bold face letters, such as b. Note
that other books may use a different notation. Some authors use
overlined or underlined letters, such as h or Q, or even underlined
bold face letters like I3 for matrices with one row/column.
74
Example 5.3 The matrix
is a 2 >< 3 matrix.

\end{frame}
\begin{frame}


We write A = (aw-) to mean that the entry in the ith row and the jth
column of A is labelled a.,_,. Sometimes we also use the notation
A(i,j) for the entry in the ith row and the jth column of A.
Example 5.3 (cont.) In this matrix 0.17; = 1 and 11.2,] = —%.
Definition 5.4 We can write a system of m equations in n unknowns as:
M1961 + ll1,2Jv2 +  + 111.1%" = bi
(Z2111 + (Lgyglljg +  + (L2Yn.’l‘n = 122
um.1-T1 + amzwz +  + ammwn = bm
where am and b,-I are given real numbersfor 1 § 2} § m and 1 § j § 11..

\end{frame}
\begin{frame}
The matrix of coeflicients for this system is the m >< n matrix A in
which the entry in the ith row and jth column is the coeflicient of the
unknown rj in the ith equation of the system. Thus, using the above
representation for the system, we have A = (aw-), that is
111.1 (11.2  I11,"
@2.1 112.2  dz,"
A = _
am am;  am.
The augmented matrix for this system is the m >< (n + 1) matrix
(A : b) which we get by adding the extra column
“(ii
to the right of A, that is the augmented matrixfor the system is
(11.1 111,2  aim I bl
112.1 <l2,2  I12.” I 52
(AIb)= . .
u’m,1 am;  a’m_7l I bm

\end{frame}
\begin{frame}
Example 5.4 The augmented matrix of a system oi linear equations
For the system of equations of Example 5.2 we have the matrix of
coefficients
3 6
A e ( 2 -1 >
and the augmented matrix
 _i z :2)
We shall solve the system of equations of Example 5.2 by using the
following three operations on the rows of the augmented matrix
(A : b) rather than working with the equations (5.1) and (5.2).

\begin{itemize}
\item Interchange two rows.
\item Multiply a row by a non-Zero real number.
\item Add multiples of one row to another row.


\end{frame}
\begin{frame}

C|S102 Mathematics ior computing volume 2
To describe which operation we are using we denote the first and
second rows of (A : b) by R1 and R2 respectively We first multiply
R1 by 1/3, so R1 := %R1:
1 2 : 20
2 —1 : 10 '
We next subtract twice R1 from R2, so R2 := R2 — 2R1:
1 2 : 20
O —5 : ~30
From the final augmented matrix we can read off the solution as
follows:
R2 :> —5a?2 = -30 :> r2 = 6.
Now substituting this into the equation corresponding to R1 we get
R1 =>x1+2m2=20=>m1 =20—2x2=20—(2><6)=8.
You can check that these values for $1, r2 satisfy both the equations
of Example 5.2 by substituting into the equations:
3 - 8 + 6 - 6 = 60
2 1 8 — 6 = 1O
\end{frame}
\begin{frame}

Example 5.5 We now solve the system
1'2 + 2x3 = 3
ZT1 + 3.102 : 1
11 + $2 — av; = 1.

\end{frame}
\begin{frame}
The augmented matrix is
O l 2 : 3
(A:b)= Z 3 O : 1
1 1 *1 : 1
Like the previous example we number the rows of the augmented
matrix R1, R2 and R3. We first interchange the first row (R1) with
the third row (R3) to bring a one into the upper left hand corner of
the augmented matrix, that is R1 <-> R3:
1 1 —1
O 1 2

\end{frame}
\begin{frame}
We next subtract twice the first row from the second row so that all
entries in the first column, below the first entry, become equal to
zero, that is R2 z: R2 — 2R1:
< _1 )
Finally, we subtract the second row from the third row so that the
entry in the second column, below the second entry, becomes equal
to zero. R3 2: R3 — R2:
co»; N
)—\P—\Pi D)
NN O
www “H”
V
@@>—\
Ow»-I
ON
>J>>-
Li
—1: 1
76


\end{frame}
\begin{frame}

Systems of linear equations
The system of equations corresponding to the final augmented
matrix is
1'1 + at; — x3 = 1
O11 + 1'2 + 213 = -1
OT] + Om; + 0.733 : 4.
Now R3 :> Om + 0.12 + O13 = 4. Since we cannot have
O11 + Or; + OJE3 = 4, there are no solutions to this system of
equations.

\end{frame}
\begin{frame}
Example 5.6 We solve a system with the same matrix of coefficients as
in Example 5.5, but a different column b.
IE2 + 2173 = 3
211 + 31"; = 1
11 + 1'2 — 13 = -1
The augmented matrix is
>—\I\7©
>—\ua>~
>—l(Dl\7
>—\>—\UJ
(A:b)=
Proceeding as in Example 5.5:
R1 <—> R3
1 1 —1 : —
@l\J
wt»
NO
UJ>—\>—\
R2:=R2—2R1
@@>4
»-1»-I
NN
wo-vii
1 -1: —
R3:=R3—R;
@@»—\
Qvd
QN
©UJ>—\
1—1:—
\end{frame}
\begin{frame}
The system of equations corresponding to the final augmented
matrix is
1'1 + 1'; — 1'3 = —1
$2 + 2.13 = 3
0 = O.
Note that the third equation holds trivially for all values of J?1,.7?2. $3,
so we have found that the original system is equivalent to a system
of two linear equations in three unknowns. When there are less
equations than unknowns, the system has either infinitely many
solutions or no solution at all. This system has infinitely many
solutions.

\end{frame}
\begin{frame}
We can write down the general solution by working our way
backwards through the system of equations corresponding to the
final augmented matrix:
R2:m2+2x3=3.
So
T2 = 3 — ZI3.
We let  = r where 1- is any real number. Then $2 = 3 — Zr. Now
R1¢z1+.r;—av3=—1.
77



ClS102 Mathematics lor computing volume 2
So
$1=*1~m2+:n3=~1~(3~Zr)+7‘=~4-l-31".
Thus the general solution is 1:1 = —4 + 3r, 1:2 = 3 — 21" and $3 = r,
where r E R. We have infinitely many solutions as we get a different
solution for ml , 1;, 41:3 for each real number r.
Gaussian elimination
The algorithm we used to solve the systems of linear equations
given in Examples 5.4, 5.5 and 5.6 is called Gaussian elimination.
It gives a general method for solving all such systems.

\end{frame}
\begin{frame}
The Gaussian elimination algorithm
We start with the augmented matrix (A : b) for the system. We then
reduce (A : b) to a ‘simpler’ augmented matrix (A* : b*) by using the
following steps:
Step1 Choose the leftmost column of (A : b) which contains a
non-zero entry. Call it the pivot column.
Step 2 Interchange the first row with another row, if necessary,
so that the top entry in the pivot column is non-Zero. Call
this entry the pivot.
\end{frame}
\begin{frame}
Step 3 Multiply the first row by a suitable real number so that
the pivot becomes equal to 1.
Step 4 Subtract multiples of the first row from the other rows so
that all entries in the pivot column, below the pivot entry,
become equal to zero.

\end{frame}
\begin{frame}
Step 5 Now disregard the first row and return to step 1 with the
resulting smaller matrix.
Step 6 Stop either when there are no more rows or all columns
contain only zeros.

\end{frame}
\begin{frame}
The output of this algorithm will be an augmented matrix with a
very simple type of structure. To describe this structure precisely we
need one further definition.
Definition 5.5 Let M be an m >< n matrix Then an entry of M is said to
be a leading entry if it is the first non-zero entvfy in some row.
Example 5.7 The following matrix M has three leading entries, which
are given in bold face.
Ill = (

\end{frame}
\begin{frame}
Row 4 has no leading entry as it consists entirely of entries zero.
<3<3Or—l
@I§©©
CDIQUOIQ
<3O‘\U‘|(.|J
The output from Gaussian elimination is an augmented matrix with
the four properties listed below:
P1 All rows which consist entirely of zeros occur at the bottom.
P2 All leading entries are equal to 1.
78



P3 If a column contains a leading entry then all entries in that
column below the leading entry are zero.
P4 In any two consecutive non-zero rows, the leading entry in the
upper row occurs to the left of the leading entry in the lower
row.

\end{frame}
\begin{frame}
Definition 5.6 A matrix which satisfies properties P1 to P4 is said to be
in row echelon form.
Given an augmented matrix in row echelon form (A* : b*), we can
easily write down the solution to the corresponding system of
equations. There are three possible cases:
Case 1: Some leading entry occurs in the final column b*.
In this case we deduce that there is no solution. For example:
©®>—\
@@|\)
@@>—\
O>—\v|>
R; => Owl + 0372 + 0703 = 1 => No solution.
Case 2: The final column does not contain a leading entry but all
other columns do contain a leading entry.

\end{frame}
\begin{frame}
In this case the system has a unique solution. We obtain this
solution by the process of back substitution: we use the final
non-zero row to determine the unknown, av”, corresponding to
the last column of A*, then use the second to last row and
substitute for ac” to determine .1:,,_1 and so on. For example:
@@>—\
@>dI\J
>-I >—\
Ull\7>—\
_1-
R3:>.r3:5.
R2=>z2=2+.1:3=2+5=7.
R1=>m1=—1—Zx2—x3=—1—l4—5=—2O.
So the unique solution is 11 = —2O,z2 = 7,13 =5.
Case 3: The final column and some other columns do not contain
leading entries.

\end{frame}
\begin{frame}
In this case the system has infinitely many solutions. We set the
unknowns corresponding to the columns of A* which do not
contain leading entries equal to arbitrary real numbers, then use
back substitution to solve for the remaining unknowns. For
example:
@@>—'
@©l\l
QQUJ
Qré
(DUl>—\
-1
R1 Q 1'4 = 3.
We can choose .203 and 12 freely because columns 3 and 2 of the
augmented matrix in row echelon form contain no leading
entries, so set 11:3 = r where r E R and ac; = s where s E IR.
R1¢11=1—2m2 +ar4=1—2s—3r+3=4—2s—3r.
So the general solution is ml = 4 — 25 — 3r, 1:2 = s, 1:3 = r, $4 = 3,
forvgs E R.
Systems of linear equations
79