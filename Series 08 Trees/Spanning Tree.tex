\documentclass{beamer}

\usepackage{graphics}
\usepackage{framed}
\usepackage{amsmath}
\begin{document}

\begin{frame}



\begin{center}
{\Huge Spanning Tree  } \\

\bigskip{\Large kobriendublin.wordpress.com}\\ \vspace{0.4cm}
{\Large Twitter: @statslabdublin}
\end{center}
\end{frame}

\begin{frame}
\Large
\begin{itemize}
\item A \textbf{spanning tree}, which spans the graph \textbf{G}, is a tree that connects to each vertex of the graph \textbf{G}.

\item The spanning tree has all the same properties of other trees, i.e simple graph with no cycles and no loops.

\item A tree that contains only vertices of degree one or two is called a \textbf{path graph}. \\If two path graphs contain the same number of vertices, then they are \textbf{isomorphic}. 
\end{itemize}
\end{frame}

\begin{frame}
\Large
\begin{itemize}
\item Find ALL non-isomorphic spanning trees of the graph \textbf{G}.
\end{itemize}
\end{frame}

\end{document}