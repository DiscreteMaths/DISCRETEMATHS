\documentclass{article}
\usepackage{amsmath}
\usepackage{amssymb}
\usepackage{graphicx}
\begin{document}

%--------------------------------------%

\section{Logic Proposition}
Let $p$, $Q$ and $r$ be the following propositions concerning integers $n$:

\begin{itemize}
\item $p$ : n is a factor of 36 (2)
\item $q$ : n is a factor of 4 (2)
\item $r$ : n is a factor of 9 (3)
\end{itemize}


\begin{center}
\begin{tabular}{|c||c|c|c|}
\hline 
\phantom{spa} \textbf{n} \phantom{spa}	& \phantom{spa}	\textbf{p} \phantom{spa}	& \phantom{spa}	\textbf{q} \phantom{spa}	& \phantom{spa}	\textbf{r} \phantom{spa}	\\ \hline \hline
1	&	1	&	1	&	1	\\ \hline
2	&	1	&	0	&	1	\\ \hline
3	&	0	&	1	&	1	\\ \hline
4	&	1	&	0	&	1	\\ \hline
6	&	0	&	0	&	1	\\ \hline
9	&	0	&	1	&	1	\\ \hline
12	&	0	&	0	&	1	\\ \hline
18	&	0	&	0	&	1	\\ \hline
36	&	0	&	0	&	1	\\
\hline 
\end{tabular} 

For each of the following compound statements, express it using the propositions P q and r, andng logical symbols, then given the truth table for it,

\begin{itemize}
\item[1)] If n is a factor of 36, then n is a factor of 4 or n is a factor of 9
\item[2)] If n is a factor of 4 or n is a factor of 9 then  n is a factor of 36
\end{itemize}

%--------------------------------------%

\newpage

\section*{Part 1 : Logic}


\subsection*{1.1 2010 Question 3}
Let $S = \{10,11,12,13,14,15,16,17,18,19\}$ and let p, q be the following propositions concerning the integer $n \in S$.

\begin{itemize}
\item p: n is a multiple of two. (i.,18e. $\{10,12,14,16,18\}$)
\item q: n is a multiple of three. {i.e. $\{12,15,18\}$}
\end{itemize}

For each of the following compound statements find the sets of values n for which it is true. 

\begin{itemize}
\item $p \vee q$ : (p or q :  10 12 14 15 16 18) 
\item $p \wedge q$: (p and q: 12 18)
\item $ \neg p \oplus q$: (not-p or q, but not both)
\begin{itemize}
\item $\neg p $ not-p = $\{ 11 13 15 17 19\}$
\item $\neg p \vee q$ not-p or q $\{11 12 13 15 17 18 19\}$
\item $\neg p \wedge q$ not-p and q $\{15\} $
\item $ \neg p \oplus q$ = $\{11, 12, 13, 17, 18, 19\}$
\end{itemize}
\end{itemize}

%--------------------------------------------------- %
%--------------------------------------------------- %
\subsection*{1.2 2010 Question 3}

Let p and q be propositions. Use Truth Tables to prove that

\[ p \rightarrow q \equiv \neg q \rightarrow \neg\]
\textbf{Important} Remember to make a comment at the end to say why the table proves that the two statements are logically equivalent. e.g. \emph{‘since the columns are identical both sides of the equation are equivalent’}.
{ \large
\begin{tabular}{|c|c||c|}
\hline  p&  q& $p \rightarrow q$ \\ 
\hline  0&  0&  1\\ 
\hline  0&  1&  1\\ 
\hline  1&  0&  0\\ 
\hline  1&  1&  1\\ 
\hline 
\end{tabular} \hspace{0.5cm} \begin{tabular}{|c|c||c|c|c|}
\hline  p&  q& $\neg q$ & $\neg p$ & $\neg q \rightarrow \neg p$ \\ 
\hline  0&  0& 1& 1& 1\\ 
\hline  0&  1& 0& 1& 1\\ 
\hline  1&  0& 1& 0& 0\\ 
\hline  1&  1& 0& 0& 1\\ 
\hline 
\end{tabular}
} 
(Key ``difference" is first and last rows)
%--------------------------------------------------- %
%--------------------------------------------------- %
\subsection*{1.3 Membership Tables for Laws}
\emph{Page 44 (Volume 1) Q8.
Also see Section 3.3 Laws of Logic.}\\

Construct a truth table for each of the following compound statement and hence find simpler propositions to which it is equivalent.


\begin{itemize}
\item $p \vee F$
\item $p \wedge T$
\end{itemize}
\textbf{Solutions}
\begin{center}
{\large
\begin{tabular}{|c|c||c|c|}
\hline  p & T & $p \vee T$ & $ p \wedge T$ \\ \hline
\hline  0 & 1 & 1 & 0 \\ 
\hline  1 &  1 & 1 & 1 \\ 
\hline 
\end{tabular} 
}
\end{center}
\begin{itemize}
\item Logical OR: $p \vee T = T $
\item Logical AND: $p \wedge T = p  $
\end{itemize}

%-------------------------------------%
\begin{center}
{\large
\begin{tabular}{|c|c||c|c|}
\hline  p & F & $p \vee F$ & $ p \wedge F$ \\ \hline
\hline  0 & 0 & 0 & 0 \\ 
\hline  1 &  0 & 1 & 0 \\ 
\hline 
\end{tabular} 
}
\end{center}
\begin{itemize}
\item Logical OR:  $p \vee F = p $
\item Logical AND: $p \wedge F = F $
\end{itemize}
%--------------------------------------------------- %
%--------------------------------------------------- %
\subsection*{1.4 Propositions}
\textbf{Page 67 Question 9}
Write the contrapositive of each of the following statements:

\begin{itemize}
\item If n= 12, then n is divisible by 3.
\item If n=5, then n is positive.
\item If the quadrilateral is square, then four sides are equal.
\end{itemize}

\textbf{Solutions}
\begin{itemize}
\item If n is not divisible by 3, then n is not equal to 12.
\item If n is not positive, then n is not equal to 5.
\item If the four sides are not equal, then the quadrilateral is not a square.
\end{itemize}


%--------------------------------------------------- %
%--------------------------------------------------- %
\subsection*{1.5 Truth Sets}
\textbf{2009} 

Let $n = \{1, 2,3,4, 5,6,7, 8, 9\}$ and let p, q be the following propositions concerning the integer $n$.
\begin{itemize}
\item p: n is even, 
\item q: $n\geq 5$.
\end{itemize}
By drawing up the appropriate truth table find the truth set for each of the
propositions $p \vee \neg q$ and $ \neg q \rightarrow p$

\begin{tabular}{|c|c|c|c|c|}
\hline n & p & q & $\neg q$ & $p \vee \neg q$ \\ 
\hline 1 & 0 & 0 & 1 & 1\\ 
\hline 2 & 1 & 0 & 1 & 0\\ 
\hline 3 & 0 & 0 & 1 & 1\\ 
\hline 4 & 1 & 0 & 1 & 0\\ 
\hline 5 & 0 & 1 & 0 & 1\\ 
\hline 6 & 1 & 1 & 0 & 1\\ 
\hline 7 & 0 & 1 & 0 & 1\\ 
\hline 8 & 1 & 1 & 0 & 1\\ 
\hline 9 & 0 & 1 & 0 & 1\\ 
\hline 
\end{tabular} 
\[\mbox{Truth Set} = \{1,3,5,6,7,8,9\}\]

\begin{tabular}{|c|c|c|c|c|}
\hline n & p & q & $  q \rightarrow p$ & $  q \rightarrow p$ \\ 
\hline 1 & 0 & 0 & 1 & 0\\ 
\hline 2 & 1 & 0 & 1 & 0\\ 
\hline 3 & 0 & 0 & 1 & 0\\ 
\hline 4 & 1 & 0 & 1 & 0\\ 
\hline 5 & 0 & 1 & 0 & 1\\ 
\hline 6 & 1 & 1 & 1 & 0\\ 
\hline 7 & 0 & 1 & 0 & 1\\ 
\hline 8 & 1 & 1 & 1 & 0\\ 
\hline 9 & 0 & 1 & 0 & 1\\ 
\hline 
\end{tabular} 
\[\mbox{Truth Set} = \{5,7,9\}\]
%---------------------------------------
%--------------------------------------------------- %
%--------------------------------------------------- %
\newpage
\subsection*{1.6 Biconditional}
\emph{See Section 3.2.1}.\\
Use truth tables to prove that $ \neg p \leftrightarrow \neg q $ is equivalent to  $ p \leftrightarrow q $

\begin{tabular}{|c|c|c|}
\hline  p& q & $p \leftrightarrow q$ \\ 
\hline  0& 0 &  1\\ 
\hline  0& 1 &  0\\ 
\hline  1& 0 &  0\\ 
\hline  1& 1 &  1\\ 
\hline 
\end{tabular} 


\begin{tabular}{|c|c|c|c|c|}
\hline  p& q & $\neg p$ & $\neg q$ & $p \leftrightarrow q$ \\ 
\hline  0& 0 & 1& 1 & 1\\ 
\hline  0& 1 &  1& 0& 0\\ 
\hline  1& 0 &  0& 1& 0\\ 
\hline  1& 1 &  0 & 0& 1\\ 
\hline 
\end{tabular} 
\newpage
%--------------------------------------------------- %
%--------------------------------------------------- %
\subsection*{1.7 2008 Q3b Logic Networks }

Construct a logic network that accepts as input p and q, which may independently have the value 0 or 1, and
gives as final input $\neg(p \wedge \not q)$ (i.e. $\equiv p \rightarrow q$).\\
\bigskip

\textbf{Logic Gates}
\begin{itemize}
\item AND
\item OR
\item NOT
\end{itemize}
\bigskip

\emph{\textbf{Examiner's Comments:}Many
diagrams were carefully and clearly drawn and well labelled, gaining full
marks. The logic table was also well done by most, but there were a few marks
lost in the final part by failing to deduce that ‘since the columns of the table are
identical the expressions are equivalent’.}

%--------------------------------------------------- %
%--------------------------------------------------- %
\subsection*{1.8 2008 Q3b Logic Networks }
Construct a logic network that accepts as input p and q, which may independently have the value 0 or 1, and
gives as final input $(p \wedge  q) \vee \neg q$ (i.e. $\equiv p \rightarrow q$).



\textbf{Important} Label each of the gates appropriately and label the diagram with a symblic expression for the output after each gate.


%---------------------------------------%
\subsection*{Dice Rolls}
Consider rolls of a die. What is the universal set?

\[ \mathcal{U} = \{1,2,3,4,5,6\} \]

%--------------------------------------%
\subsection*{Worked Example}

Suppose that the Universal Set $\mathcal{U}$ is the set of integers from 1 to 9.
\[ \mathcal{U} = \{1,2,3,4,5,6,7,8,9\}, \]

and that the set $\mathcal{A}$ contains the prime numbers between 1 to 9 inclusive.

\[ \mathcal{A} = \{1,2,3,5,7\}, \]

and that the set $\mathcal{B}$ contains the even numbers between 1 to 9 inclusive.

\[ \mathcal{B} = \{2,4,6,8\}. \]

%--------------------------------------------------------%
\subsubsection*{Complements}
\begin{itemize}

\item The Complements of A and B are the elements of the universal set not contained in A and B.

\item The complements are denoted $\mathcal{A}^{\prime}$ and $\mathcal{B}^{\prime}$
\[ \mathcal{A}^{\prime} = \{4,6,8,9\}, \]
\[ \mathcal{B}^{\prime} = \{1,3,5,7,9\}, \]

\end{itemize}


%--------------------------------------------------------%

\subsubsection*{Intersection}
\begin{itemize}

\item Intersection of two sets describes the elements that are members of both the specified Sets

\item The intersection is denoted $\mathcal{A\cap B}$ 
\[ \mathcal{A\cap B} = \{2\}\]

\item only one element is a member of both A and B.
\end{itemize}
%--------------------------------------------------------%

\subsubsection*{Set Difference}
\begin{itemize}

\item The Set Difference of A with regard to B are list of elements of A not contained by B.

\item The complements are denoted $\mathcal{A-B}$ and $\mathcal{B-A}$
\[ \mathcal{A-B} = \{1,3,5,7\}, \]

\[ \mathcal{B-A} = \{4,6,8\}, \]
\end{itemize}
\subsection*{symbols}
$\varnothing$,
$\forall$,
$\in$,
$\notin$,
$\cup$
%----------------------------------------------------------- %
\newpage

\section*{Prepositional Logic}

\subsection{five basic connectives}


\end{document}

