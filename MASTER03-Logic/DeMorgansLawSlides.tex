\documentclass{beamer}

\usepackage{amsmath}

\begin{document}
\section{De Morgan's Laws}

%============================================================= %
\begin{frame}
	\frametitle{De Morgan's Laws}
\Large
The De Morgan's Laws allow the expression of conjunctions and disjunctions purely in terms of each other via negation.


The laws can be verbalized as:
\begin{itemize}
\item The negation of a conjunction is the disjunction of the negations.
\item The negation of a disjunction is the conjunction of the negations.
\end{itemize}
\end{frame}
%============================================================= %
\begin{frame}
	\frametitle{De Morgan's Laws}
\Large
The Rules can be verbalized as 

\begin{itemize}
\item[(i)]"not (A and B)" is the same as "(not A) or (not B)"

\item[(ii)] "\textit{not (A or B)}" is the same as "\textit{(not A) and (not B)}"
\end{itemize}
\end{frame}
%============================================================= %
\begin{frame}
	\frametitle{De Morgan's Laws}
\Large
Use Truth Tables to prove De Morgan's Laws (see page 40).
\[  \neg (p \vee q) = \neg p \wedge \neg q\]
Looking at the lefthand side of equation
\begin{center}
\begin{tabular}{|c|c||c|c|c|c|}
  \hline
  % after \\: \hline or \cline{col1-col2} \cline{col3-col4} ...
p	&	q	&	$ p \vee q$	&	$ p \wedge q$&	$\neg (p \vee q)$	&	$\neg (p \wedge q)$\\
	&		&	(1)	&	(2)	&	(3)	&	(4)	\\ \hline
0	&	0	&	0	&	0	&	1	&	1 \\
0	&	1	&	1	&	0	&	0	&	1\\
1	&	0	&	1	&	0	&	0	&	1\\
1	&	1	&	1	&	1	&	0	&	0\\
  \hline
\end{tabular}
\end{center}
\end{frame}
%============================================================= %
\begin{frame}
\frametitle{De Morgan's Laws}
\Large
Looking at the righthand side of equation
\[  \neg (p \vee q) = \neg p \wedge \neg q\]
\begin{center}
\begin{tabular}{|c|c||c|c|c|c|}
  \hline
  % after \\: \hline or \cline{col1-col2} \cline{col3-col4} ...
p	&	q	&	$\neg$p	&	$\neg$q	&	$\neg p \wedge \neg q$	&	$\neg p \vee \neg q$ \\ 
	&		&	(5)	&	(6)	&	(7)	&	(8)	\\
\hline
0	&	0	&	1	&	1	&	1	&	1	\\
0	&	1	&	1	&	0	&	0	&	1	\\
1	&	0	&	0	&	1	&	0	&	1	\\
1	&	1	&	0	&	0	&	0	&	0	\\
  \hline
\end{tabular}
\end{center}
\end{frame}
%------------------------------------------------------------ %
\end{document}