\documentclass[]{report}

\voffset=-1.5cm
\oddsidemargin=0.0cm
\textwidth = 480pt

\usepackage{framed}
\usepackage{subfiles}
\usepackage{enumerate}
\usepackage{graphics}
\usepackage{newlfont}
\usepackage{eurosym}
\usepackage{amsmath,amsthm,amsfonts}
\usepackage{amsmath}
\usepackage{color}
\usepackage{amssymb}
\usepackage{multicol}
\usepackage[dvipsnames]{xcolor}
\usepackage{graphicx}
\begin{document}
%------------------------------------------------------------------------%
\section{Special Mathematical Functions}
\subsection{Mathematical Operators}
\begin{itemize}
\item The Square Root function
\item The Floor and Ceiling functions
\item The Absolute Value functions
\end{itemize}


\begin{itemize}
\item Root Functions
\item Absolute Value Function
\item Floor Function
\item Ceiling Function
\end{itemize}
{

\begin{eqnarray}
\lfloor 3.14 \rfloor =& 3 \\
\lceil -4.5 \rceil =& -5 \\
| -4 | =&  4
\end{eqnarray}
}
For this course, only positive numbers have square roots. The square roots are positive numbers. (This statement is not strictly true. The square root of a negative number is called a complex number. However this is not part of the course).

Negative numbers can have cube roots

{
\[ -27 = -3 \times -3 \times -3 \qquad \]
\LARGE
\[ \sqrt[3]{-27} = -3 \]
}



%--------------------------------------------%
\subsection*{Exponential and Logarithms}

\textbf{Rules}\\
Expontials : Rules 4.18 Page 58 \\
Logarithms : Rules 4.23 Page 61 \\

\[ log_a(a) = 1 \]
\[ log_a(b^c) = c \times log_a(b) \]

\begin{itemize}
\item $log_2(128) = 7$
\item $log_2(1/4) = -2$
\item $log_2(2) = 1$
\end{itemize}

\[ log_a(b) = \frac{log_x(b)}{log_x(a)} \]

%------------------------------------------------------%
\subsection{Logarithms} 
- Laws of Logarithms
- Change of Base

\[ \mbox{Log}_b(x) = a \] \[b^a = x \]
\[ \mbox{Log}_2(8) = 3 \] \[2^3 = 8 \]

\[ \mbox{Log}_b(x) \times \mbox{Log}_b(y) =  \mbox{Log}_b(x+y) \]
\[ \mbox{Log}_b(x^y) =  y \times \mbox{Log}_b(x) \]

\[ \mbox{Log}_y(x)  =  \frac{ \mbox{Log}_b(x) }{ \mbox{Log}_b(y) } \]
%------------------------------------------------------%
\subsection{Exponents}
- Rules of Exponents

\[ (a^b)^c = a^{b \times c}\]

\[ 64^{2/3} =  (4^3)^{2/3} = 4^{3\times2/3} = 4^2 = 16 \]


\[ (a^b) \times (a^c) = a^{b+c}\]
\[ (3^2) \times (3^3) = 3^{2+3} = 3^5  =243 \]

%---------------------------------------------------------%
\subsection{Exponentials Functions}

\[ e^a \times e^b = e^{a+b}\]

\[ (e^a )^b = e^{ab}\]
%---------------------------------------------------------%
\subsection{Logarithmic Functions}

\subsubsection{Laws for Logarithms}
The following laws are very useful for working with logarithms.
\begin{enumerate}
\item $\mbox{log}_b(X)$ + $\mbox{log}_b(Y)$ = $\mbox{log}_b(X\times Y)$
\item $\mbox{log}_b(X)$ - $\mbox{log}_b(Y)$ = $\mbox{log}_b(X / Y)$
\item $\mbox{log}_b(X^Y)$= $Y \mbox{log}_b(X)$
\end{enumerate}

\noindent \textbf{Question1.3} Compute the Logarithm of the following
\begin{itemize}
\item $\mbox{log}_2(8)$
\item $\mbox{log}_2(\sqrt{128})$
\item $\mbox{log}_2(64)$
\item $\mbox{log}_5(125)$ +   $\mbox{log}_3(729)$
\item $\mbox{log}_2(64/4)$
\end{itemize}

%----------------------------------------------------------------%

\begin{itemize}
\item $a^x = y$  $log_a(y) = x$

\item $e^x = y$  $ln(y)=x$

\item $log_a(x\times y) = log_a(x) + log_a(y)$

\item $log_a(\frac{x}{y}) = log_a(x) - log_a(y)$

\item $log_a(\frac{1}{x}) = - log_a(x)$

\item $log_a(a) = 1$

\item $log_a(1) = 0$
\end{itemize}


\begin{itemize}
\item $\lceil x\rceil$

\item $\lfloor x\rfloor$
\end{itemize}

\begin{tabular}{|c|c|c|c|}
\hline Sample value x & Floor $\lfloor x\rfloor$ & Ceiling  $\lceil x\rceil$ & Fractional part $ \{ x \} $\\
12/5 = 2.4 &2&3&2/5 = 0.4\\
2.7&2&3&0.7\\
-2.7&-3&-2&0.3\\
-2&-2&-2&0\\
\hline 
\end{tabular} 

%---------------------------------------------------------%
\subsection{Precision Functions}

\begin{itemize}
\item Absolute Value Function $| x |$
\item Ceiling Function $\lceil x \rceil$
\item Floor Function  $\lfloor x \rfloor $
\end{itemize}
%\[ \lfloorx\rfloor\]

\noindent \textbf{Question1.2}: State the range and domain of the following function
\[ F(x) = \lfloor x-1 \rfloor \]
%---------------------------------------------------------%
\subsection{Powers}

\[  2^ 4 = 2 \times 2 \times 2 \times 2 = 16 \]

\[  5^ 3 = 5 \times 5 \times 5 =125 \]

\subsubsection{Special Cases}

Anything to the power of zero is always 1

\[  X^ 0 = 1 \mbox{ for all values of X} \]

Sometimes the power is a negative number.

\[  X^{-Y} = { 1 \over X^Y}  \]

Example 
\[  2^{-3} = { 1 \over 2^3} = { 1 \over 8}  \]

%---------------------------------------------------------%
\subsection{Precision Functions}

\begin{itemize}
\item Absolute Value Function $| x |$
\item Ceiling Function $\lceil x \rceil$
\item Floor Function  $\lfloor x \rfloor $
\end{itemize}
%\[ \lfloorx\rfloor\]

\noindent \textbf{Question1.2}: State the range and domain of the following function
\[ F(x) = \lfloor x-1 \rfloor \]
%---------------------------------------------------------%
\subsection{Powers}

\[  2^ 4 = 2 \times 2 \times 2 \times 2 = 16 \]

\[  5^ 3 = 5 \times 5 \times 5 =125 \]

\subsubsection{Special Cases}

Anything to the power of zero is always 1

\[  X^ 0 = 1 \mbox{ for all values of X} \]

Sometimes the power is a negative number.

\[  X^{-Y} = { 1 \over X^Y}  \]

Example 
\[  2^{-3} = { 1 \over 2^3} = { 1 \over 8}  \]



%---------------------------------------------------------%
\subsection{Exponentials Functions}

\[ e^a \times e^b = e^{a+b}\]

\[ (e^a )^b = e^{ab}\]


\section{Logarithmic Functions}

\subsubsection{Laws for Logarithms}
The following laws are very useful for working with logarithms.
\begin{enumerate}
\item $\mbox{log}_b(X)$ + $\mbox{log}_b(Y)$ = $\mbox{log}_b(X\times Y)$
\item $\mbox{log}_b(X)$ - $\mbox{log}_b(Y)$ = $\mbox{log}_b(X / Y)$
\item $\mbox{log}_b(X^Y)$= $Y \mbox{log}_b(X)$
\end{enumerate}

\noindent \textbf{Question1.3} Compute the Logarithm of the following
\begin{itemize}
\item $\mbox{log}_2(8)$
\item $\mbox{log}_2(\sqrt{128})$
\item $\mbox{log}_2(64)$
\item $\mbox{log}_5(125)$ +   $\mbox{log}_3(729)$
\item $\mbox{log}_2(64/4)$
\end{itemize}


\[ Log_b(x) = \frac{1}{Log_x(b)}  \]

\[ Log_b(x) = \frac{Log_a(b)}{Log_a(b)}  \]


Example 1
\[ log_3(x) + 3 log_x(3) = 4  \]

\[ \left(log_3(x)\right)^2 + 3  = 4 log_3(x)  \]

Example 1
\[ log_3(x) + 3 log_x(3) = 4  \]

\[ \left(log_3(x)\right)^2 - 4 log_3(x) + 3  = 0  \]
%---------------------------------------------------------%
\subsection{Logarithmic Functions}

\end{document}
%------------------------------------%
\subsection*{Part B : Logarithms}
Evaluate the following expression.
\[ \mbox{Log}_4 64 + \mbox{Log}_5 625 + \mbox{Log}_9 3 \] 

\[ \mbox{Log}_4 64 + \mbox{Log}_5 625 + \mbox{Log}_9 3 \] 


\subsubsection{Laws for Logarithms}
The following laws are very useful for working with logarithms.
\begin{enumerate}
\item $\mbox{log}_b(X)$ + $\mbox{log}_b(Y)$ = $\mbox{log}_b(X\times Y)$
\item $\mbox{log}_b(X)$ - $\mbox{log}_b(Y)$ = $\mbox{log}_b(X / Y)$
\item $\mbox{log}_b(X^Y)$= $Y \mbox{log}_b(X)$
\end{enumerate}

\noindent \textbf{Question1.3} Compute the Logarithm of the following
\begin{itemize}
\item $\mbox{log}_2(8)$
\item $\mbox{log}_2(\sqrt{128})$
\item $\mbox{log}_2(64)$
\item $\mbox{log}_5(125)$ +   $\mbox{log}_3(729)$
\item $\mbox{log}_2(64/4)$
\end{itemize}


%---------------------------------------------------------%
\subsection{Exponentials Functions}

\[ e^a \times e^b = e^{a+b}\]

\[ (e^a )^b = e^{ab}\]
%---------------------------------------------------------%


%---------------------------------------------------------%
\subsection{Precision Functions}

\begin{itemize}
\item Absolute Value Function $| x |$
\item Ceiling Function $\lceil x \rceil$
\item Floor Function  $\lfloor x \rfloor $
\end{itemize}
%\[ \lfloorx\rfloor\]

\noindent \textbf{Question1.2}: State the range and domain of the following function
\[ F(x) = \lfloor x-1 \rfloor \]
%---------------------------------------------------------%
\subsection{Powers}

\[  2^ 4 = 2 \times 2 \times 2 \times 2 = 16 \]

\[  5^ 3 = 5 \times 5 \times 5 =125 \]

\subsubsection{Special Cases}

Anything to the power of zero is always 1

\[  X^ 0 = 1 \mbox{ for all values of X} \]

Sometimes the power is a negative number.

\[  X^{-Y} = { 1 \over X^Y}  \]

Example 
\[  2^{-3} = { 1 \over 2^3} = { 1 \over 8}  \]


%------------------------------------------------------%
\textbf{Exercises}
\begin{itemize}
\item[(a)] 
Complete the following table for the functions 
\begin{itemize}
\item[i)] $g(x) = \mbox{log}_3x$,
\item[ii)] $h(x) =\sqrt[3]{x}$.
\end{itemize} 
\begin{center}

\begin{tabular}{|c||c|c|c|c|c|c|}
\hline $x$ &  \phantom{p}1\phantom{p}&  &  &  & 81 &  \\ 
\hline \phantom{p} $g(x)$ \phantom{p}&  & \phantom{p}1\phantom{p} & \phantom{p}2\phantom{p} &  &  &  \phantom{p}5\phantom{p} \\ 
\hline \phantom{p}$h(x)$ \phantom{p}&  &  &  &  3.00 & \phantom{p}\phantom{p}\phantom{p}  &  \\ 
\hline 
\end{tabular} 
\end{center}

Express your answers to 2 decimal places only.

\newpage

%-------------------------------------------------------------------------%


\section{Exponential and Logarithms}
\subsection*{Laws of Logarithms}

\begin{itemize}
\item Law 1 : Multiplcation of Logarithms
\[ Log(a) \times Log(b) = Log(a+b) \]
\item Law 2 : Division of Logarithms
\[ \frac{Log(a)}{Log(b)} = Log(a-b) \]
\item Law 3 : Powers of Logarithms
\[ Log(a^b) = b \times Log(a) \]
\end{itemize}

