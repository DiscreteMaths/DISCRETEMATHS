\documentclass[]{report}

\voffset=-1.5cm
\oddsidemargin=0.0cm
\textwidth = 480pt

\usepackage{framed}
\usepackage{subfiles}
\usepackage{enumerate}
\usepackage{graphics}
\usepackage{newlfont}
\usepackage{eurosym}
\usepackage{amsmath,amsthm,amsfonts}
\usepackage{amsmath}
\usepackage{color}
\usepackage{amssymb}
\usepackage{multicol}
\usepackage[dvipsnames]{xcolor}
\usepackage{graphicx}
\begin{document}

/


\subsection*{Adjacency Lists}
\begin{itemize}
\item[u]: $\{v\}$
\item[v]: $\{w,x\}$
\item[w]: $\{v,x\}$
\item[z]: $\{v,w\}$
\end{itemize}




\begin{itemize}
\item Spanning Subgraphs of G.

\item a vertex is said to be an \textbf{emph{ isolated vertex}} if it has a degree of zero.
\item a vertex is said to be an \textbf{emph{ end-vertex}} if it has a degree of one.
\item a vertex is said to be an \textbf{emph{ even vertex}} if it has a degree of an even number.
\item a vertex is said to be an \textbf{emph{ odd vertex}} if it has a degree of an odd number.


\item A graph is said to be \textbf{emph{k-regular}} if the degree of each vertex is $k$. 
\item Every Graph has an even number of odd vertices.
\item A cubic graph is a graph where every vertex has degree three.
\end{itemize}
%------------------------------------------------------------------------- %
\section{Graph Theory - Isomorphic Graphs}

% http://www3.ul.ie/cemtl/pdf%20files/cm2/IsomorphicGraph.pdf
%\frametitle{Isomorphic Graphs}
\begin{itemize}
\item If the graphs are not simple, we need more sophisticated methods to check for when two graphs are isomorphic. 
\item However, it is often straightforward to show that two graphs are not isomorphic. 
\item You can do this by showing any of the following seven conditions are true.
\end{itemize}

%-------------------------------------------------------------------------- %




% Ex 8 



\section*{Session 05 Graph Theory}
\begin{itemize}
\item Eulerian Path
\item Isomorphism
\item Adjacency matrices
\end{itemize}
Adjacency Matrices
\[ \left( \begin{matrix}
o & 1 & 0 & 1 & 1 \\ 
1 & 0 & 1 & 0 & 1 \\ 
1 & 1 & 0 & 1 & 1 \\ 
0 & 1 & 1 & 1 & 1 \\ 
1 & 1 & 0 & 1 & 0
\end{matrix} \right) \]


\section*{Session 05 Graph Theory}
\begin{itemize}
\item Eulerian Path
\item Isomorphism
\item Adjacency matrices
\end{itemize}
Adjacency Matrices
\[ \left( \begin{matrix}
o & 1 & 0 & 1 & 1 \\ 
1 & 0 & 1 & 0 & 1 \\ 
1 & 1 & 0 & 1 & 1 \\ 
0 & 1 & 1 & 1 & 1 \\ 
1 & 1 & 0 & 1 & 0
\end{matrix} \right) \]




\end{document}
