




%opening
\title{Number Systems - Tutorial Sheet}
\author{www.MathsResource.com}

\begin{document}
\maketitle
\Large

\section{Fundamental of Mathematics}

\subsection{Powers}

\[  2^ 4 = 2 \times 2 \times 2 \times 2 = 16 \]

\[  5^ 3 = 5 \times 5 \times 5 =125 \]

\subsubsection{Special Cases}

Anything to the power of zero is always 1

\[  X^ 0 = 1 \mbox{ for all values of X} \]

Sometimes the power is a negative number.

\[  X^{-Y} = { 1 \over X^Y}  \]

Example 
\[  2^{-3} = { 1 \over 2^3} = { 1 \over 8}  \]


%====================================== %
\newpage
\begin{center}
	\huge{Mathematics for Computing}\\
	{\LARGE Session 2 : Set Theory}
\end{center}


% % \subfile{session02a.tex}
\newpage
%====================================== %
\newpage
\begin{center}
	\huge{Mathematics for Computing}\\
	{\LARGE Session 3 : Logic}
\end{center}
% %\subfile{session03a.tex}
% % \subfile{session03d.tex}
% % \subfile{DM0305DeMorgans.tex}

%============================================ 
\newpage
\begin{center}
	\huge{Mathematics for Computing}\\
	\LARGE{Digraphs and Relations}
\end{center}
% % \subfile{session06a.tex}
% % \subfile{session06e-antisymmetricrelations.tex}
% % \subfile{session06f.tex}


\subsection*{Question 1 : Binary numbers}
\begin{itemize}
	%---------------------------%
	\item[(a)] Express the following binary numbers as decimal numbers
	\begin{multicols}{2}
		\begin{itemize}
			\item[(i)] $101$
			\item[(ii)] $1101$
			\item[(iii)] $11011$
			\item[(iv)] $100101$
		\end{itemize}
	\end{multicols}

	%---------------------------%
	\item[(b)] Express the following decimal numbers as binary numbers
	\begin{multicols}{2}
		\begin{itemize}
			\item[(i)] 6
			\item[(ii)] 15
			\item[(iii)] 37
			\item[(iv)] 77
		\end{itemize}
	\end{multicols}

\end{itemize}
%-------------%-------------%-------------%----------------
\subsection*{Question 2}
A number is expressed in base 5 as $(234)_5$. What is it as decimal number?
Suppose you multiply $(234)_5$ by 5. what would be the answer in base 5.

%-------------%-------------%-------------%----------------
\subsection*{Question 3}

 Perform the following binary additions
 \begin{multicols}{2}
  \begin{itemize}
  	\item[(i)] $1011+ 1111$
  	\item[(ii)] $10101  + 10011$
  	\item[(iii)] $1010 + 11010$
  	\item[(iv)] $101010 + 10101 + 101$
  \end{itemize}
 \end{multicols}


%-------------%-------------%-------------%----------------
\subsection*{Question 4}
Perform the binary additions

\begin{itemize}
\item $(10111)_2 +(111010)_2$

\item $(1101)_2 + (1011)_2 + (1111)_2$
\end{itemize}

%-------------%-------------%-------------%----------------
\subsection*{Question 5}

Perform the binary subtractions using both the bit-borrowing method and the two's complement method.
\begin{itemize}
\item $(1001)_2 -(111)_2$
\item $(110000)_2 -(10111)_2$
\end{itemize}


%-------------%-------------%-------------%----------------
\subsection*{Question 6}

Perform the binary multiplications
\begin{itemize}
\item $(1101)_2 \times (101)_2$
\item $(1101)_2 \times (1101)_2$
\end{itemize}


%-------------%-------------%-------------%----------------
\subsection*{Question 7}
\begin{itemize}
	

\item[(a)] What is highest Hexadecimal number that can be written with two characters, and what is it's equivalent in decimal form?
What is the next highest hexadecimal number?

% \[FF = 255\] % Remark 
\item[(b)] Which of the following are not valid hexadecimal numbers?

\begin{multicols}{2}
\begin{itemize}
\item[(i)] A5G 	
\item[(ii)] 73 
\item[(iii)] EEF	
\item[(iv)] 101
\end{itemize}
\end{multicols}
\end{itemize}	



%-------------%-------------%-------------%----------------
\subsection*{Question 8 : Binary Substraction}
\textbf{Exercises:}

\begin{multicols}{2}
	\begin{itemize}
		\item[(i)] 110 - 10	
		\item[(ii)] 101 - 11  
		\item[(iii)] 1001 - 11	
		\item[(iv)] 10001 - 100 
		\item[(v)] 101001 - 1101
		\item[(vi)] 11010101-1101
	\end{itemize}
\end{multicols}



%-------------%-------------%-------------%----------------
\subsection*{Question 9}

\begin{itemize}
	%---------------------------%
	\item[(a)] Suppose 2341 is a base-5 number
	Compute the equivalent in each of the following forms:
	\begin{itemize}
		\item[(i)] decimal number
		\item[(ii)] hexadecimal number
		\item[(iii)] binary number
	\end{itemize}
	%---------------------------%
	\item[(b)] Perform the following binary additions
	\begin{itemize}
		\item[(i)] $1011+ 1111$
		\item[(ii)] $10101  + 10011$
		\item[(iii)] $1010 + 11010$
	\end{itemize}
	%---------------------------%
\end{itemize}

%-------------%-------------%-------------%----------------
\subsection*{Question 10}

Calculate working in hexadecimal
\begin{itemize}
\item $(BBB)_{16} + (A56)_{16}$
\item $(BBB)_{16} - (A56)_{16}$
\end{itemize} 


%-------------%-------------%-------------%----------------
\subsection*{Question 11}

Write the hex number $(EC4)_{16}$ in binary.
Write the binary number $(11110110101|)_2$ in hex.
%-------------%-------------%-------------%----------------
%-------------%-------------%-------------%----------------
\subsection*{Question 12}

Express the decimal number 753 in binary , base 5 and hexadecimal.

%-------------%-------------%-------------%----------------
%-------------%-------------%-------------%----------------
\subsection*{Question 13}

Express 42900 as a product of its prime factors, using index notation for repeated factors.

%-------------%-------------%-------------%----------------
%-------------%-------------%-------------%----------------
\subsection*{Question 14}

Expresse the recuring decimals 
\begin{itemize}
	\item[(i)] $0.727272\ldots$
	\item[(ii)] $0.126126126....$
	\item[(iii)] $0.7545454545...$
\end{itemize} 
as rational numbers in its simplest form.

%-------------%-------------%-------------%----------------
%-------------%-------------%-------------%----------------
\subsection*{Question 15}
Given that $\pi$ is an irrational number, can you say whether $\frac{\pi}{2}$ is rational or irrational.
or is it impossible to tell?

%-------------------------------------------------------- %
\subsection*{Question 16}
\begin{itemize}
	\item[(i)] Given x is the irrational positive number $\sqrt{2}$, express $x^8$ in binary notation\\
	\item[(ii)] From part (i), is $x^8$ a rational number?
\end{itemize}

%-------------%-------------%-------------%----------------
%-------------%-------------%-------------%----------------
\subsection*{Question 17}
\begin{enumerate}
\item[(i)] 5/7 lies between 0.714 and 0.715.
\item[(ii)] $\sqrt(2)$ is at least 1.41.
\item[(iii)] $\sqrt(3)$  9s at lrast 1.732 and at most 1.7322.
\end{enumerate}


%-------------%-------------%-------------%----------------
%-------------%-------------%-------------%----------------
\subsection*{Question 18}
\begin{itemize}
\item[(i)] Write down the numbers 0.0000526 in floating point form.
\item[(ii)] How is the number 1 expressed in floating point form.
\end{itemize}


%-------------%-------------%-------------%----------------
%-------------%-------------%-------------%----------------
\subsection*{Question 19}
\begin{itemize}
\item Deduce that every composite integer $n$ has a prime factor such that $p \leq \sqrt{n}$.
\item Decide whether 899 is a prime.
\end{itemize}

%-------------%-------------%-------------%----------------
%-------------%-------------%-------------%----------------
\subsection*{Question 20}

\begin{itemize}
\item What would be the maximum numbber of digits that a decimal fraction with denominator 13 
could have in a recurring block in theory?

\item Can you predict which other fractions with denominator 13 will have the same digits as 1/13 in their recurring block?
\end{itemize}




\section*{Part A: Number Systems - Binary Numbers}
\begin{enumerate}
	\item Express the following decimal numbers as binary numbers.
	\begin{multicols}{4}
		\begin{itemize}
			\item[i)] $(73)_{10}$
			\item[ii)] $(15)_{10}$
			\item[iii)] $(22)_{10}$
		\end{itemize}
	\end{multicols}
	
	All three answers are among the following options.
	\begin{multicols}{4}
		\begin{itemize}
			\item[a)] $(10110)_{2}$ %22
			\item[b)] $(1111)_{2}$ %15
			\item[c)] $(1001001)_{2}$ %73
			\item[d)] $(1000010)_{2}$ %64
		\end{itemize}
	\end{multicols}
	
	\item Express the following binary numbers as decimal numbers.
	\begin{multicols}{4}
		\begin{itemize}
			\item[a)] $(101010)_{2}$
			\item[b)] $(10101)_{2}$
			\item[c)] $(111010)_{2}$
			\item[d)] $(11010)_{2}$
		\end{itemize}
	\end{multicols}
	\item Express the following binary numbers as decimal numbers.
	\begin{multicols}{4}
		\begin{itemize}
			\item[a)] $(110.10101)_{2}$
			\item[b)] $(101.0111)_{2}$
			\item[c)] $(111.01)_{2}$
			\item[d)] $(110.1101)_{2}$
		\end{itemize}
	\end{multicols}
	\item Express the following decimal numbers as binary numbers.
	\begin{multicols}{4}
		\begin{itemize}
			\item[a)] $(27.4375)_{10}$  %
			\item[b)] $(5.625)_{10}$
			\item[c)] $(13.125)_{10}$
			\item[d)] $(11.1875)_{10}$
		\end{itemize}
	\end{multicols}
\end{enumerate}
\newpage
\section*{Part B: Number Systems - Binary Arithmetic}
%http://www.csgnetwork.com/binaddsubcalc.html
%(See section 1.1.3 of the text)
\begin{enumerate}
	\item Perform the following binary additions.
	\begin{multicols}{2}
		\begin{itemize}
			\item[a)] $(110101)_{2}$ + $(1010111)_{2}$
			\item[b)] $(1010101)_{2}$ + $(101010)_{2}$
			\item[c)] $(11001010)_{2}$ + $(10110101)_{2}$
			\item[d)] $(1011001)_{2}$ + $(111010)_{2}$
		\end{itemize}
	\end{multicols}
	
	\item Perform the following binary subtractions.
	\begin{multicols}{2}
		\begin{itemize}
			\item[a)] $(110101)_{2}$ - $(1010111)_{2}$
			\item[b)] $(1010101)_{2}$ - $(101010)_{2}$
			\item[c)] $(11001010)_{2}$ - $(10110101)_{2}$
			\item[d)] $(1011001)_{2}$ - $(111010)_{2}$
		\end{itemize}
	\end{multicols}
	
	
	\item Perform the following binary multiplications.
	\begin{multicols}{2}
		\begin{itemize}
			\item[a)] $(1001)_{2}\times( 1000)_{2}$  % 9 by 8
			\item[b)] $(101)_{2}\times(1101)_{2}$ % 5 by 11
			\item[c)] $(111)_{2}\times(1111)_{2}$ % 7 by 15
			\item[d)] $(10000)_{2}\times(11001)_{2}$    %16 by 25
		\end{itemize}
	\end{multicols}
	
	
	
	\item Perform the following binary multiplications.
	%\begin{multicols}{2}
	%\begin{itemize}
	%\item[a)] $(1001000)_{2} \div ( 1000)_{2}$
	%\item[b)] $(101101)_{2} \div (1001)_{2}$
	%\item[c)] $(1001011000)_{2} \div (101000)_{2}$
	%\item[d)] $(1100000)_{2} \div (10000)_{2}$
	%\end{itemize}
	%\end{multicols}
	
	\begin{enumerate}
		\item Which of the following binary numbers is the result of this binary division: $(10)_{2} \times ( 1101)_{2}$. % % (2) /  (13)
		\begin{multicols}{2}
			\begin{itemize}
				\item[a)] $(11010)_{2}$ %26
				\item[b)] $(11100)_{2}$ %28
				\item[c)] $(10101)_{2}$ %21
				\item[d)] $(11011)_2$ %27
			\end{itemize}
		\end{multicols}
		\item Which of the following binary numbers is the result of this binary division: $(101010)_{2} \times( 111 )_{2}$. % (4) /  (6)
		\begin{multicols}{2}
			\begin{itemize}
				\item[a)] $(11000)_{2}$ %24
				\item[b)] $(11001)_{2}$ %25
				\item[c)] $(10101)_{2}$ %21
				\item[d)] $(11011)_2$ %27
			\end{itemize}
		\end{multicols}
		\item Which of the following binary numbers is the result of this binary division: $(1001110)_{2}\times ( 1101 )_{2}$. % (9) /  (3)
		\begin{multicols}{2}
			\begin{itemize}
				\item[a)] $(11000)_{2}$ %24
				\item[b)] $(11001)_{2}$ %25
				\item[c)] $(10101)_{2}$ %21
				\item[d)] $(11011)_2$ %27
			\end{itemize}
		\end{multicols}
	\end{enumerate}
	
	%----------------------------------------------------------------%
	
	\item Perform the following binary divisions.
	%\begin{multicols}{2}
	%\begin{itemize}
	%\item[a)] $(1001000)_{2} \div ( 1000)_{2}$
	%\item[b)] $(101101)_{2} \div (1001)_{2}$
	%\item[c)] $(1001011000)_{2} \div (101000)_{2}$
	%\item[d)] $(1100000)_{2} \div (10000)_{2}$
	%\end{itemize}
	%\end{multicols}
	
	\begin{enumerate}
		\item Which of the following binary numbers is the result of this binary division: $(111001)_{2} \div ( 10011)_{2}$. % (57) /  (19)
		\begin{multicols}{2}
			\begin{itemize}
				\item[a)] $(10)_2$ %2
				\item[b)] $(11)_{2}$ %3
				\item[c)] $(100)_{2}$ %4
				\item[d)] $(101)_{2}$ %5
			\end{itemize}
		\end{multicols}
		\item Which of the following binary numbers is the result of this binary division: $(101010)_{2} \div ( 111 )_{2}$. % (42) /  (7)
		\begin{multicols}{2}
			\begin{itemize}
				\item[a)] $(11)_2$ %3
				\item[b)] $(100)_{2}$ %4
				\item[c)] $(101)_{2}$ %5
				\item[d)] $(110)_{2}$ %6
			\end{itemize}
		\end{multicols}
		\item Which of the following binary numbers is the result of this binary division: $(1001110)_{2} \div ( 1101 )_{2}$. % (78) /  (13)
		\begin{multicols}{2}
			\begin{itemize}
				
				\item[a)] $(100)_{2}$ %4
				\item[b)] $(110)_{2}$ %6
				\item[c)] $(111)_{2}$ %7
				\item[d)] $(1001)_2$ %9
			\end{itemize}
		\end{multicols}
	\end{enumerate}
	
	
\end{enumerate}

\newpage
\section*{Part C: Number Bases - Hexadecimal}
\begin{enumerate}
	
	\item Answer the following questions about the hexadecimal number systems
	\begin{itemize}
		\item[a)] How many characters are used in the hexadecimal system?
		\item[b)] What is highest hexadecimal number that can be written with two characters? \item[c)] What is the equivalent number in decimal form?
		\item[d)] What is the next highest hexadecimal number?
	\end{itemize}
	
	
	\item Which of the following are not valid hexadecimal numbers?
	\begin{multicols}{4}
		\begin{itemize}
			\item[a)] $73$
			\item[b)] $A5G$
			\item[c)] $11011$
			\item[d)] $EEF	$
		\end{itemize}
	\end{multicols}
	
	\item Express the following decimal numbers as a hexadecimal number.
	\begin{multicols}{4}
		\begin{itemize}
			\item[a)] $(73)_{10}$
			\item[b)] $(15)_{10}$
			\item[c)] $(22)_{10}$
			\item[d)] $(121)_{10}$
		\end{itemize}
	\end{multicols}
	
	
	\item Compute the following hexadecimal calculations.
	\begin{multicols}{4}
		\begin{itemize}
			\item[a)] $5D2+A30$
			\item[b)] $702+ABA$
			\item[c)] $101+111$
			\item[d)] $210+2A1$
		\end{itemize}
	\end{multicols}
\end{enumerate}


\newpage





% \newpage
\section*{Part D: Natural, Rational and Real Numbers}
\begin{framed}
	\begin{itemize}
		\item $\mb{N}$ : natural numbers (or positive integers) $\{1,2,3,\ldots\}$
		\item $\mb{Z}$ : integers $\{-3,-2,-1,0,1,2,3,\ldots\}$
		\begin{itemize}
			\item[$\ast$] (The letter $\mb{Z}$ comes from the word \emph{Zahlen} which means ``numbers" in German.)
		\end{itemize}
		\item $\mb{Q}$ : rational numbers
		\item $\mb{R}$ : real numbers
		\item $\mb{N} \subseteq \mb{Z } \subseteq \mb{Q} \subseteq \mb{R}$
		\begin{itemize}
			\item[$\ast$] (All natural numbers are integers. All integers are rational numbers. All rational numbers are real numbers.)
		\end{itemize}
	\end{itemize}
\end{framed}

\begin{enumerate}
	\item State which of the following sets the following numbers belong to. 
	\begin{multicols}{4}
		\begin{itemize}
			\item[1)] $18$
			\item[2)] $8.2347\ldots$
			\item[3)] $\pi$
			\item[4)] $1.33333\ldots$
			\item[5)] $17/4$
			\item[6)] $4.25$
			\item[7)] $\sqrt{\pi}$
			\item[8)] $\sqrt{25}$
			%\item[i)]
			%\item[j)] $(15)_{10}$
			%\item[k)] $\pi$
			%\item[l)] $11.132$
		\end{itemize}
	\end{multicols}
	\bigskip
	The possible answers are
	
	\begin{itemize}
		\item[a)] Natural number : $\mb{N} \subseteq \mb{Z } \subseteq \mb{Q} \subseteq \mb{R}$
		\item[b)] Integer : $ \mb{Z } \subseteq \mb{Q} \subseteq \mb{R}$
		\item[c)] Rational Number : $ \mb{Q} \subseteq \mb{R}$
		\item[d)] Real Number $\mb{R}$
		
		%\item[i)]
		%\item[j)] $(15)_{10}$
		%\item[k)] $\pi$
		%\item[l)] $11.132$
	\end{itemize}
	
\end{enumerate}



\end{document}

% \item The Fibonacci sequence $f_n$ is defined recursively by the rule
%   \begin{equation*}
%     \begin{cases}
%       f_0&=0\\
%       f_1&=1\\
%       f_n&=a_{n-1}+a_{n-2}
%     \end{cases}
%   \end{equation*}

%   \begin{enumerate}
%   \item
%     Write a program to evaluate the Fibonacci sequence and hence evaluate $f_{50}$.
%   \item
%     Let the sequence $g_n$ be defined as the ratio
%     \begin{equation*}
%       g_n = \frac{f_{n+1}}{f_n}
%     \end{equation*}
%     Write a program to evaluate the first $50$ terms of the sequence $g_n$.
%   \item
%     Assumming that the sequence $g_n$ has a limit $\phi$, find this limit.
%   \end{enumerate}



% \pagebreak
%   \begin{center}
%     \textbf{Assignment 1}
%     Due Tuesday Week 4 (Sept $28^th$)
%   \end{center}

%   Write an Octave function which computes square roots using Heron's method to as high an accuraccy as possible. The function should be called ``sqr\_root'' and must take the following form
% \lstset{language=Octave}
% \begin{lstlisting}[title=sqr\_root.m,frame=single]
% function X=sqr_root(D)
%     ....
% endfunction
% \end{lstlisting}

% In particular, the function must be able to compute the square roots of your UL ID number, (i.e numbers of the form 10234567), as well as 0.01,0.1,10,100 and 1000 times this number; so test the function on these and many other numbers before submitting.

% Note that the stopping condition shown in the lectures will not suffice as a general stopping condition\footnote{If Octave hangs during calculation, use ``Ctrl-C'' to try to cancel the operation and return to the prompt.}. You must develop and program an alternative strategy for detecting when Heron's method has converged.

% Note also that the square of the number $X$ returned by your function must be as close to D as is possible for $X^2$ to be; that is, $X$ must be as close as it possibly can be to the true value of $\sqrt{D}$.

% Save your solutions source code in a file called ``sqr\_root.m''. To submit the assigment, you must email this solution from your UL student email to the address \emph{niall.ryan@ul.ie}. The email subject line must be ``MA4402 ASSIGNMENT SOLUTION'' and the file ``sqr\_root.m'' must be an attachment to this email.

% The email must be sent \textbf{from your UL student address}. \emph{Emails from other addresses will not be accepted}.



