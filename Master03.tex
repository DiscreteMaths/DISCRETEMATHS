\documentclass[]{report}

\voffset=-1.5cm
\oddsidemargin=0.0cm
\textwidth = 480pt

\usepackage{framed}
\usepackage{subfiles}
\usepackage{enumerate}
\usepackage{graphics}
\usepackage{newlfont}
\usepackage{eurosym}
\usepackage{amsmath,amsthm,amsfonts}
\usepackage{amsmath}
\usepackage{color}
\usepackage{amssymb}
\usepackage{multicol}
\usepackage[dvipsnames]{xcolor}
\usepackage{graphicx}
\begin{document}



\chapter{Logic}

\chapter{Session 3}


\subsection*{Part B : Logical Operations}
Let $n \in \{1,2,3,4,5,6,7,8,9\}$ and let p and q be the following propostions concerning 
the integers $n$.

\begin{itemize}
	\item[p] $n$ is event
	\item[q] $n<5$
\end{itemize}

Find the values of $n$ for which each of the following compound statement is true,

\begin{itemize}
	\item[(i)] $\neg p$
	\item[(ii)] $p \wedge q$
	\item[(iii)] $\neg p \vee q$ 
	\item[(iv)] $p \oplus q$
\end{itemize}

\section*{Question 3}

Let $\mathcal{S} = \{10,11,12,13,14,15,16,17,18,19,20\}$
$n  \in \mathcal{S}$
p : n is a multiple of two
q : n is a multiple of five

Express the following statement using logic symbols
n is not a multiple of either 2 or 5.

List the elements of S which are in the truth set for the statement in (ii).

Write the contrapositive of the following statement concerning an integer $n$.

If the last digit of n is 4, then n is divisible by 3

% %------------------------------------------------------
\section{Conditional Connectives}
Construct the truth table for the proposition $p \rightarrow q$.

\begin{center}
	\begin{tabular}{|c|c|c|c|}
		\hline
		p & q & $p \rightarrow q$ & $q \rightarrow p$ \\
		\hline
		0 & 0 & 1& 1 \\
		0 & 1 & 1 & 0 \\
		1 & 0 & 0 & 1 \\
		1 & 1 & 1 & 1 \\
		\hline
	\end{tabular}
\end{center}

\section{Tautologies and Truth Tables}
\textbf{Truth Table for the Biconditional Connective.} \bigskip
\begin{center}
	\begin{tabular}{|c|c|c|}
		\hline $P$  & $Q$ & $P \leftrightarrow Q$ \\ \hline
		\hline T & T &   \\ 
		\hline T & F &    \\ 
		\hline F & T &    \\ 
		\hline \phantom{sp}F \phantom{sp} & \phantom{sp}F \phantom{sp} & \phantom{sp}T \phantom{sp} \\
		\hline 
	\end{tabular} 
\end{center}

\begin{tabular}{|c|c|c|c|c|}
	\hline $P$ & $Q$ & $P \vee Q$ &  &  \\ \hline
	\hline T & T &  &  &  \\ 
	\hline T & F &  &  &  \\ 
	\hline F & T &  &  &  \\ 
	\hline \phantom{sp}F \phantom{sp} & \phantom{sp}F \phantom{sp} & \phantom{spacespa} & \phantom{spacespa}  & \phantom{spacespa} \\ 
	
	\hline 
\end{tabular} 

%-------------------------------------------------------------------------%
\newpage



\section*{Question 3}
\subsection*{Part A : Propositions}
Let p, q be the following propositions:
\begin{itemize}
	\item p : this apple is red, 
	\item q : this apple is ripe.
\end{itemize}

\noindent Express the following statements in words as simply as you can:
\begin{itemize}
	\item[(i)] $p \rightarrow q$
	\item[(ii)] $p \wedge \neg q$.
\end{itemize}

\noindent Express the following statements symbolically:
\begin{itemize}
	\item[(iii)] This apple is neither red nor ripe.
	\item[(iv)] If this apple is not red it is not ripe.
\end{itemize}



%------------------------------------%
\newpage
\section*{Question 3}
% Logical Proofs



% %------------------------------------------------------


\section*{Question 6}
%Relations and Digraphs

%------------------------------------%
\subsection{Question 6}
Say with reason whether or not $\mathcal{R}$ is
\begin{itemize}
	\item reflexive
	\item symmetric
	\item transitive
\end{itemize}

In the cases where the given property does not hold provide a counter example to justify this.
% %------------------------------------------------------


\subsection*{Question 6 Part A : Digraphs}

Suppose $A = $$\{1,2,3,4\}$. Consider the following relation in A

\[ \{  (1,1),(2,2),(2,3),(3,2),(4,2),(4,4)\} \]

Draw the direct graph of $A$. Based on the Digraph of $A$ discuss whether or not a relation that could be depicted by the digraph could be described as the following, justifying your answer.


\begin{itemize}
	\item[(i)] Symmetric
	\item[(ii)] Reflexive 
	\item[(iii)] Transitive
	\item[(iv)] Antisymmetric
\end{itemize}

\subsection*{Part B : Relations}
Determine which of the following relations $ x R y$ are reflexive, transitive, symmetric, or antisymmetric on the following - there may be more than one characteristic.  if

\begin{itemize} 
	\item[(i)] $x = y$
	\item[(ii)] $x < y$
	\item[(iii)] $x^2 = y^2$
	\item[(iv)] $x \geq y$
\end{itemize}
%---------------------------------------%
\subsection*{Question 2}

% http://staff.scem.uws.edu.au/cgi-bin/cgiwrap/zhuhan/dmath/dm_readall.cgi?page=20

Let $A=\{0,1,2\}$ and $R=\{ (0,0),(0,1),(0,2),(1,1), (1,2), (2,2)\}$
and $S=\{(0,0),(1,1),(2,2)\}$ be 2 relations on A. Show that

\begin{itemize}
	\item[(i)] R is a partial order relation.
	\item[(ii)] S is an equivalence relation.
\end{itemize}
%------------------------------------%

\subsection*{Part C : Partial Orders}
% http://staff.scem.uws.edu.au/cgi-bin/cgiwrap/zhuhan/dmath/dm_readall.cgi?page=20

Let $A=\{0,1,2\}$ and $R=\{ (0,0),(0,1),(0,2),(1,1), (1,2), (2,2)\}$
and $S=\{(0,0),(1,1),(2,2)\}$ be 2 relations on A. Show that

\begin{itemize}
	\item[(i)] R is a partial order relation.
	\item[(ii)] S is an equivalence relation.
\end{itemize}



%------------------------------------%

\section*{Biconnective Operators}

\[ p \leftarrow \rightarrow q\]
We could verbalize this as ``$p$ implies $q$ and $q$ implies $p$".

%-------------------------- %
%-------------------------- %
%Section 4 Invertible Functions



Maths for Computing Sprint

\section*{Part 0. - Numeracy}
\subsection{Factorials}
Evaluate the following
\begin{itemize}
	\item $6!$
	\item $3!$
	\item $1!$
	\item $0!$
\end{itemize}
Evaluate the following expressions
\[  \frac{5!}{3!}  \mbox{   and   } \frac{6!}{2!\times 4!}  \]
\subsection*{Laws of Logarithms}
\begin{itemize}
	\item Addition of Logarithms
	\item Subtraction of Logaritms
	\item Powers of Logarithms
\end{itemize}
%----------------------------------------- %
%----------------------------------------- %


\section*{Section 3. Logic}

\subsection*{Proofs with Truth Tables}

\[ \neg(p \vee q) \wedge p \equiv q \]
\begin{center}
	\begin{tabular}{|c|c||c|c||c|c|}
		\hline $p$ & $q$ &  &  &  &  \\ 
		\hline  &  &  &  &  &  \\ 
		\hline 
	\end{tabular} 
\end{center}

%----------------------------------------- %
%----------------------------------------- %


\section{Section 3 Logic}
\subsection{Logical Operations}
\begin{itemize}
	\item $\neg p$ the negation of proposition $p$.
	\item $p \wedge q$ Both propositions p and q are simultaneously true (Logical State AND)
	\item $p \vee q $ One of the propositions is true, or both (Logical State : OR)
	\item $p \otimes q$ Only one of the propositions is true (Logical State : exclusive OR (i.e XOR)
\end{itemize}
\begin{center}
	\begin{tabular}{|c|c|c|c|c|}
		\hline
		p & q & $p \vee q$ & $q \wedge p$ & $p \otimes q$ \\
		\hline
		0 & 0 & 0 & 0 & 0 \\
		0 & 1 & 1 & 0 & 1\\
		1 & 0 & 1 & 0 & 1 \\
		1 & 1 & 1 & 1 & 0\\
		\hline
	\end{tabular}
\end{center}
%---------------------------------------------------------%


%--------------------------------------%

\section{Logic Proposition}
Let $p$, $Q$ and $r$ be the following propositions concerning integers $n$:

\begin{itemize}
\item $p$ : n is a factor of 36 (2)
\item $q$ : n is a factor of 4 (2)
\item $r$ : n is a factor of 9 (3)
\end{itemize}

{

\begin{center}
\begin{tabular}{|c||c|c|c|}
\hline 
\phantom{spa} \textbf{n} \phantom{spa}& \phantom{spa}\textbf{p} \phantom{spa}& \phantom{spa}\textbf{q} \phantom{spa}& \phantom{spa}\textbf{r} \phantom{spa}\\ \hline \hline
1&1&1&1\\ \hline
2&1&0&1\\ \hline
3&0&1&1\\ \hline
4&1&0&1\\ \hline
6&0&0&1\\ \hline
9&0&1&1\\ \hline
12&0&0&1\\ \hline
18&0&0&1\\ \hline
36&0&0&1\\
\hline 
\end{tabular} 
\end{center}
}
For each of the following compound statements, express it using the propositions P q and r, andng logical symbols, then given the truth table for it,

\begin{itemize}
\item[1)] If n is a factor of 36, then n is a factor of 4 or n is a factor of 9
\item[2)] If n is a factor of 4 or n is a factor of 9 then  n is a factor of 36
\end{itemize}

%------------------------------------------------%
\newpage
\begin{itemize}
	\item $ p \rightarrow q$  p implies q
	\item $p \lg q $
\end{itemize}
\newpage
\section*{Part 1 : Logic}


\subsection*{1.1 2010 Question 3}
Let $S = \{10,11,12,13,14,15,16,17,18,19\}$ and let p, q be the following propositions concerning the integer $n \in S$.

\begin{itemize}
\item p: n is a multiple of two. (i.,18e. $\{10,12,14,16,18\}$)
\item q: n is a multiple of three. {i.e. $\{12,15,18\}$}
\end{itemize}

For each of the following compound statements find the sets of values n for which it is true. 

\begin{itemize}
\item $p \vee q$ : (p or q :  10 12 14 15 16 18) 
\item $p \wedge q$: (p and q: 12 18)
\item $ \neg p \oplus q$: (not-p or q, but not both)
\begin{itemize}
\item $\neg p $ not-p = $\{ 11 13 15 17 19\}$
\item $\neg p \vee q$ not-p or q $\{11 12 13 15 17 18 19\}$
\item $\neg p \wedge q$ not-p and q $\{15\} $
\item $ \neg p \oplus q$ = $\{11, 12, 13, 17, 18, 19\}$
\end{itemize}
\end{itemize}

%--------------------------------------------------- %

%--------------------------------------------------- %
\subsection*{1.2 2010 Question 3}

Let p and q be propositions. Use Truth Tables to prove that

\[ p \rightarrow q \equiv \neg q \rightarrow \neg\]
\textbf{Important} Remember to make a comment at the end to say why the table proves that the two statements are logically equivalent. e.g. \emph{‘since the columns are identical both sides of the equation are equivalent’}.
{ 
\begin{tabular}{|c|c||c|}
\hline  p&  q& $p \rightarrow q$ \\ 
\hline  0&  0&  1\\ 
\hline  0&  1&  1\\ 
\hline  1&  0&  0\\ 
\hline  1&  1&  1\\ 
\hline 
\end{tabular} \hspace{0.5cm} \begin{tabular}{|c|c||c|c|c|}
\hline  p&  q& $\neg q$ & $\neg p$ & $\neg q \rightarrow \neg p$ \\ 
\hline  0&  0& 1& 1& 1\\ 
\hline  0&  1& 0& 1& 1\\ 
\hline  1&  0& 1& 0& 0\\ 
\hline  1&  1& 0& 0& 1\\ 
\hline 
\end{tabular}
} 
(Key ``difference" is first and last rows)
%--------------------------------------------------- %
%--------------------------------------------------- %
\subsection*{1.3 Membership Tables for Laws}
\emph{Page 44 (Volume 1) Q8.
Also see Section 3.3 Laws of Logic.}\\

Construct a truth table for each of the following compound statement and hence find simpler propositions to which it is equivalent.


\begin{itemize}
\item $p \vee F$
\item $p \wedge T$
\end{itemize}
\textbf{Solutions}
\begin{center}
{
\begin{tabular}{|c|c||c|c|}
\hline  p & T & $p \vee T$ & $ p \wedge T$ \\ \hline
\hline  0 & 1 & 1 & 0 \\ 
\hline  1 &  1 & 1 & 1 \\ 
\hline 
\end{tabular} 
}
\end{center}
\begin{itemize}
\item Logical OR: $p \vee T = T $
\item Logical AND: $p \wedge T = p  $
\end{itemize}

%-------------------------------------%
\begin{center}
{
\begin{tabular}{|c|c||c|c|}
\hline  p & F & $p \vee F$ & $ p \wedge F$ \\ \hline
\hline  0 & 0 & 0 & 0 \\ 
\hline  1 &  0 & 1 & 0 \\ 
\hline 
\end{tabular} 
}
\end{center}
\begin{itemize}
\item Logical OR:  $p \vee F = p $
\item Logical AND: $p \wedge F = F $
\end{itemize}
%--------------------------------------------------- %
%--------------------------------------------------- %
\subsection*{1.4 Propositions}
\textbf{Page 67 Question 9}
Write the contrapositive of each of the following statements:

\begin{itemize}
\item If n= 12, then n is divisible by 3.
\item If n=5, then n is positive.
\item If the quadrilateral is square, then four sides are equal.
\end{itemize}

\textbf{Solutions}
\begin{itemize}
\item If n is not divisible by 3, then n is not equal to 12.
\item If n is not positive, then n is not equal to 5.
\item If the four sides are not equal, then the quadrilateral is not a square.
\end{itemize}

\section*{Prepositional Logic}

\subsection{five basic connectives}


%--------------------------------------------------%
Reflexive, Symmetric and Transitive

\begin{itemize} 
	\item Reflexive
	\item Symmetric
	\item Transitive
\end{itemize}

%--------------------------------------------------%
\subsection{Logarithms}


Here we
assume x and y are positive real numbers.
1. $loga(xy) = loga(x) + loga(y)$
2. $loga(x/y)= loga(x) - loga(y)$
3. $loga (x^r) = r loga(x)$ for any real number r.
%--------------------------------------------------%
Invertible Functions

%--------------------------------------------------%
\subsection{Proof by Induction}

Another sequence is defined by the recurrence relation un = un-1+2n-1 and
u1 = 1.
(i) Calculate u2, u3, u4 and u5.
1,4,9,16,25

(ii) Prove by induction that $u_n = n^2$ for all $n \geq 1$

%--------------------------------------------------%
Exponentials
% http://www.mathsisfun.com/algebra/exponent-laws.html
%----------------------------------------------------------- %
\newpage

\section*{Prepositional Logic}


%-------------------------------------------------------------- %
\begin{itemize}
	\item $p \wedge q$
	\item $p \vee q$
	\item $p \rightarrow q$
\end{itemize}
\newpage
%--------------------------------------------------- %
%--------------------------------------------------- %
\subsection*{1.5 Truth Sets}
\textbf{2009} 

Let $n = \{1, 2,3,4, 5,6,7, 8, 9\}$ and let p, q be the following propositions concerning the integer $n$.
\begin{itemize}
\item p: n is even, 
\item q: $n\geq 5$.
\end{itemize}
By drawing up the appropriate truth table find the truth set for each of the
propositions $p \vee \neg q$ and $ \neg q \rightarrow p$

\begin{tabular}{|c|c|c|c|c|}
\hline n & p & q & $\neg q$ & $p \vee \neg q$ \\ 
\hline 1 & 0 & 0 & 1 & 1\\ 
\hline 2 & 1 & 0 & 1 & 0\\ 
\hline 3 & 0 & 0 & 1 & 1\\ 
\hline 4 & 1 & 0 & 1 & 0\\ 
\hline 5 & 0 & 1 & 0 & 1\\ 
\hline 6 & 1 & 1 & 0 & 1\\ 
\hline 7 & 0 & 1 & 0 & 1\\ 
\hline 8 & 1 & 1 & 0 & 1\\ 
\hline 9 & 0 & 1 & 0 & 1\\ 
\hline 
\end{tabular} 
\[\mbox{Truth Set} = \{1,3,5,6,7,8,9\}\]

\begin{tabular}{|c|c|c|c|c|}
\hline n & p & q & $  q \rightarrow p$ & $  q \rightarrow p$ \\ 
\hline 1 & 0 & 0 & 1 & 0\\ 
\hline 2 & 1 & 0 & 1 & 0\\ 
\hline 3 & 0 & 0 & 1 & 0\\ 
\hline 4 & 1 & 0 & 1 & 0\\ 
\hline 5 & 0 & 1 & 0 & 1\\ 
\hline 6 & 1 & 1 & 1 & 0\\ 
\hline 7 & 0 & 1 & 0 & 1\\ 
\hline 8 & 1 & 1 & 1 & 0\\ 
\hline 9 & 0 & 1 & 0 & 1\\ 
\hline 
\end{tabular} 
\[\mbox{Truth Set} = \{5,7,9\}\]
%---------------------------------------
%--------------------------------------------------- %
%--------------------------------------------------- %
\newpage
\subsection*{1.6 Biconditional}
\emph{See Section 3.2.1}.\\
Use truth tables to prove that $ \neg p \leftrightarrow \neg q $ is equivalent to  $ p \leftrightarrow q $

\begin{tabular}{|c|c|c|}
\hline  p& q & $p \leftrightarrow q$ \\ 
\hline  0& 0 &  1\\ 
\hline  0& 1 &  0\\ 
\hline  1& 0 &  0\\ 
\hline  1& 1 &  1\\ 
\hline 
\end{tabular} 


\begin{tabular}{|c|c|c|c|c|}
\hline  p& q & $\neg p$ & $\neg q$ & $p \leftrightarrow q$ \\ 
\hline  0& 0 & 1& 1 & 1\\ 
\hline  0& 1 &  1& 0& 0\\ 
\hline  1& 0 &  0& 1& 0\\ 
\hline  1& 1 &  0 & 0& 1\\ 
\hline 
\end{tabular} 
\newpage
%--------------------------------------------------- %
%--------------------------------------------------- %
\subsection*{1.7 2008 Q3b Logic Networks }

Construct a logic network that accepts as input p and q, which may independently have the value 0 or 1, and
gives as final input $\neg(p \wedge \not q)$ (i.e. $\equiv p \rightarrow q$).\\
\bigskip

\textbf{Logic Gates}
\begin{itemize}
\item AND
\item OR
\item NOT
\end{itemize}
\bigskip

\emph{\textbf{Examiner's Comments:}Many
diagrams were carefully and clearly drawn and well labelled, gaining full
marks. The logic table was also well done by most, but there were a few marks
lost in the final part by failing to deduce that ‘since the columns of the table are
identical the expressions are equivalent’.}

%--------------------------------------------------- %
\subsection{logical implication}

%- http://whatis.techtarget.com/definition/logical-implication

Logical implication is a type of relationship between two statements or sentences. The relation translates verbally into "logically implies" or "if/then" and is symbolized by a double-lined arrow pointing toward the right ( $\rightarrow$). If A and B represent statements, then $A\rightarrow B$ means "A implies B" or "If A, then B." The word "implies" is used in the strongest possible sense.


As an example of logical implication, suppose the sentences A and B are assigned as follows:

A = The sky is overcast.
B = The sun is not visible.

In this instance, A B is a true statement (assuming we are at the surface of the earth, below the cloud layer.) However, the statement B A is not necessarily true; it might be a clear night. Logical implication does not work both ways. However, the sense of logical implication is reversed if both statements are negated. That is,

$$(A \rightarrow B)  \equiv (-B \rightarrow -A)$$

Using the above sentences as examples, we can say that if the sun is visible, then the sky is not overcast. This is always true. In fact, the two statements A B and -B -A are logically equivalent.

\newpage
%---------------------------------------%
%--------------------------------------------------- %
\subsection*{1.8 2008 Q3b Logic Networks }
Construct a logic network that accepts as input p and q, which may independently have the value 0 or 1, and
gives as final input $(p \wedge  q) \vee \neg q$ (i.e. $\equiv p \rightarrow q$).



\textbf{Important} Label each of the gates appropriately and label the diagram with a symblic expression for the output after each gate.

%----------------------------------------------------------- %
\end{document}
