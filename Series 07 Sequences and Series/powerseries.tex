
\documentclass{beamer}

\usepackage{amsmath}
\usepackage{amssymb}

\begin{document}
%============================================================================================= %
\begin{frame}
\frametitle{Power Series}



In mathematics, a power series (in one variable) is an infinite series of the form
\[f(x) = \sum_{n=0}^\infty a_n \left( x-c \right)^n = a_0 + a_1 (x-c)^1 + a_2 (x-c)^2 + a_3 (x-c)^3 + \cdots\]
where an represents the coefficient of the nth term, c is a constant, and x varies around c (for this reason one sometimes speaks of the series as being centered at c). This series usually arises as the Taylor series of some known function.
\end{frame}
%============================================================================================= %

\begin{frame}
\frametitle{Power Series}
In many situations c is equal to zero, for instance when considering a Maclaurin series. In such cases, the power series takes the simpler form

\[f(x) = \sum_{n=0}^\infty a_n x^n = a_0 + a_1 x + a_2 x^2 + a_3 x^3 + \cdots.\]


\end{frame}
%============================================================================================= %

\end{document}