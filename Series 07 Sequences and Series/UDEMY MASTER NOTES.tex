%\documentclass[11pt, a4paper,dalthesis]{report}    % final 
%\documentclass[11pt,a4paper,dalthesis]{report}
%\documentclass[11pt,a4paper,dalthesis]{book}

\documentclass[11pt,a4paper,titlepage,oneside,openany]{article}

\pagestyle{plain}
%\renewcommand{\baselinestretch}{1.7}

\usepackage{setspace}
%\singlespacing
\onehalfspacing
%\doublespacing
%\setstretch{1.1}

\usepackage{amsmath}
\usepackage{amssymb}
\usepackage{amsthm}

\usepackage[margin=3cm]{geometry}
\usepackage{graphicx,psfrag}%\usepackage{hyperref}


\usepackage{framed}
\include{zcust_preamble}
\include{zcust_commands}

\begin{document}

\section*{Sequences and limits}

Heron's method and the reciprocal algorithm both generated \emph{sequences} of numbers which hopefully converged to a \emph{limit}. We need to study and understand these objects and concepts.

A sequence is an ordered list of numbers given by a definite rule. A sequence of numbers is written in the form
\begin{equation*}
 a_0,a_1,a_2,a_3,a_4,a_5,a_6,\ldots
\end{equation*}
The term $a_n$ denotes the $n^{th}$ \emph{term} in the sequence, \eg $a_{105}$ denotes the $105^{th}$ term. Note that $n$ must be a positive integer ($n \in \mathbb{N}_0$). It cannot be negative or a fraction, \etc. Some sequences start at $n=0$, other can start at $n=1$. A sequence can be \emph{finite}, with the terms ending after a large enough value of $n$ is reach. Sequences can also be \emph{infinite}, with the terms continuing forever without ending.

The \emph{rule} which defines the sequence may take many forms but it must define all terms in the sequence unambiguously.

\subsection*{Simple Rules}

A simple rule for a sequence is a formula which has the same behavior/form for all inputs $n$. For example
\begin{align*}
%  a_n=&n\ , \quad &a_0=0,a_1=1,a_2=2,a_3=3,\ldots \\
  a_n=&2n^2+3n\\%\ , \quad &a_0=0,a_1=5,a_2=14,a_3=27,\ldots\\
  b_n=&\frac{1}{n+1}\\%\ , \quad &a_0=1,a_1=\frac{1}{2},a_2=\frac{1}{3},a_3=\frac{1}{4},\ldots\\
  c_n=&2^n+3^n%\ , \quad &a_0=2,a_1=5,a_2=13,a_3=35,\ldots
\end{align*}
The rule is of the form $a_n=f(n)$ for some simple function $f$. Different letter can be used to distinguish terms in different sequences. Some terms in the sequences above are listed in the tables below.

\begin{center}

\begin{tabular}{|c|c|c|c|c|c|}
\hline
$a_0$ & $a_1$ & $a_2$ & $a_3$ & $a_4$ & $a_5$ \\
\hline
  0 & 5 & 14 & 27 & 44 & 65  \\
\hline
\end{tabular}
\end{center}

\begin{center}
\begin{tabular}{|c|c|c|c|c|c|}
\hline
$b_0$ & $b_1$ & $b_2$ & $b_3$ & $b_4$ & $b_5$ \\
\hline
  1 & $\frac{1}{2}$ & $\frac{1}{3}$  & $\frac{1}{4}$  & $\frac{1}{5}$  & $\frac{1}{6}$   \\
\hline
\end{tabular}
\end{center}

\begin{center}
\begin{tabular}{|c|c|c|c|c|c|}
\hline
$c_0$ & $c_1$ & $c_2$ & $c_3$ & $c_4$ & $c_5$ \\
\hline
  2 & 5 & 13 & 35 & 97 & 275  \\
\hline
\end{tabular}
\end{center}

\subsection*{Piecewise Rules}

A piecewise rule for a sequence is one which changes its formula depending on the input $n$. The input must first be checked and matched to a particular formula before evaluating the term. 

\noindent {\bf Example:}
\begin{align*}
  a_n&= \begin{cases}
    n^2 &,\qquad n < 3\\
    -2n &,\qquad n \ge 3
  \end{cases}
\end{align*}
The sequence $a_n$ is given by the formula $n^2$ when $n<10$ and by the formula $-2n$ when $n \ge 10$. Before evaluation, the input $n$ much be checked to see which of the conditions it satisfies.

For example to evaluate the term $a_2$, the input $n=2$ is seen to satisfy the condition $n<3$. Therefore the formula $n^2$ is used, and so $a_2=2^2=4$. To evaluate the term $a_5$, the input $n=5$ is seen to satisfy the condition $n \ge3$. Therefore the formula $-2n$ is used, and so $a_5=-2(5) = -10$. In this way, the terms of the sequence can be evaluated as long as the inputs $n$ are matched to their proper conditions. Further terms in this sequence are listed in the table below.
\begin{center}
\begin{tabular}{|c|c|c|c|c|c|}
\hline
$a_0$ & $a_1$ & $a_2$ & $a_3$ & $a_4$ & $a_5$ \\
\hline
  0 & 1 & 4 & -6 & -8 & -10  \\
\hline
\end{tabular}
\end{center}

The sequences $b_n$ and $c_n$ below are also defined using piecewise rules. Some terms in the sequences $b_n$ and $c_n$ are listed in the tables below.

The conditions for the sequence $b_n$ check whether the input $n$ is even$(0,2,4,6,\ldots)$ or odd$(1,3,5,7,\ldots)$. The sequence $c_n$ has $3$ formulas, and $3$ corresponding conditions on the input $n$. The input must be checked against all listed conditions until a suitable formula is found.
\begin{align*}
  b_n&= \begin{cases}
    2n^2+3n&,\qquad n \text{ even}\\
    \frac{1}{n} &,\qquad n \text{ odd}
  \end{cases}\\
  c_n&= \begin{cases}
    0 &,\qquad n<2\\
    n^2 &,\qquad n=2\\
    n^3 &,\qquad n>2
  \end{cases}
\end{align*}


\begin{center}
\begin{tabular}{|c|c|c|c|c|c|}
\hline
$b_0$ & $b_1$ & $b_2$ & $b_3$ & $b_4$ & $b_5$ \\
\hline
  0 & $\frac{1}{1}$ & 14 & $\frac{1}{3}$ & 44 & $\frac{1}{5}$ \\
\hline
\end{tabular}
\end{center}

\begin{center}
\begin{tabular}{|c|c|c|c|c|c|}
\hline
$c_0$ & $c_1$ & $c_2$ & $c_3$ & $c_4$ & $c_5$ \\
\hline
  0 & 0 & 4 & 27 & 64 & 125  \\
\hline
\end{tabular}
\end{center}

To represent piecewise rules on a computer, some kind of if/else statement or other control flow statements are needed. For multiple conditions, a switch statement can be used.

Piecewise and simple rules are convenient as they allow any term in a sequence to be immediately evaluated without needed to evaluate any previous terms. For example $b_{100}=2(100)^2+3(100)=20300$, and $a_{1000000} = -2000000$.

Piecewise rules can also be used to define more general function. For example, the modulus or absolute value function $|x|$, or $\text{abs}(x)$, can be defined using the rule.
\begin{equation*}
  |x| =
  \begin{cases}
    x, \qquad x \ge 0\\
    -x, \qquad x<0
  \end{cases}
\end{equation*}
This function strips off the negative sign of any number, but leaves positive numbers unchanged. It gives the ``size'' of a number, or its distance from $0$. The function $|x|$ is plotted in Figure \vref{fig:abs_x}.

\begin{figure}
  \centering
  \includegraphics[width=0.4\textwidth]{images/abs_x.eps}
  \caption{The absolute value function $|x|$, also written $\text{abs}(x)$}
  \label{fig:abs_x}
\end{figure}

\subsection*{Recursive Sequences}

Many sequences are defined by a recursive rule. Here each term in the sequence is defined by previous terms. In order to begin the recursion, initial terms or \emph{base cases} must be specified. For example
\begin{align*}
  &\begin{cases}
    a_0=&1\\
    a_n=&n a_{n-1}
  \end{cases}
\end{align*}
\begin{align*}
  a_0=&1   & &\\
  a_1=&1 \cdot a_0 = 1 &=&1  \\
  a_2=&2 \cdot a_1 = 2 \cdot 1 &=&2  \\
  a_3=&3 \cdot a_2 = 3 \cdot 2 \cdot 1 &=&6  \\
  a_4=&4 \cdot a_3 = 4 \cdot 3 \cdot 2 \cdot 1 &=&24  \\
  a_5=&5 \cdot a_4 = 5 \cdot 4 \cdot 3 \cdot 2 \cdot 1 &=&120  \\
  a_6=&6 \cdot a_5 = 6 \cdot 5 \cdot 4 \cdot 3 \cdot 2 \cdot 1 &=&620
\end{align*}

This sequences defines the \emph{factorial function}. $n!$ is the product of the first $n$ integers.
\[ a_n = n! = n \cdot (n-1) \cdot \left( n-2 \right) \cdot \left( n-3 \right) \cdots 3 \cdot 2 \cdot 1 \]
This sequence grows very rapidly, with $10!=3628800$, and $100! \cong 9.33 \times 10 ^{157}$ is a number greater than the number of atoms in the known universe. The factorial sequence is one of the fastest growing elementary sequences. Note that by convention $0!=1$.

\subsubsection*{Fibonacci sequence/numbers}
The Fibonacci sequence is defined by the following recursive rule.
\begin{align*}
  &\begin{cases}
    f_0=&0; \quad f_1=1\\
    f_n=&f_{n-1}+f_{n-2}
  \end{cases}
\end{align*}
The current term $f_n$ in the sequence is the sum of the previous two terms. The first few elements of this sequence are listed in the table below.
\begin{center}
\begin{tabular}{|c|c|c|c|c|c|c|c|c|c|c|c|}
\hline
$f_0$ & $f_1$ & $f_2$ & $f_3$ & $f_4$ & $f_5$ & $f_6$ & $f_7$ & $f_8$ & $f_9$ & $f_{10}$ & $f_{11}$  \\
\hline
0 & 1 & 1 & 2& 3 & 5 & 8 & 13 & 21 & 34 & 55 & 89 \\
\hline
\end{tabular}
\end{center}
This sequence is named after the Italian mathematician Fibonacci, who introduced the base-ten Hindu-Arabic numeral system into Europe. Fibonacci consider the sequence to arise from a pairs of rabbits, which each produce a new pair of rabbits every month. After two months, each new pair becomes a breeding pair and increases the sequence. He showed that the number of rabbit pairs obeys the recursive rule $f_n=f_{n-1}+f_{n-2}$, though there are other problems in which this sequence emerges -- for example, when computing the golden ratio.

One difficulty with recursive sequences is that in order to find later values in the sequence, all previous terms in the sequence must be computed first. This makes it difficult to compute a term like $f_{1000}$. In the case of the Fibonacci sequence, alternative formulas exist, but this will not be the case for many recursive sequences.

Both Heron's method and the Fast Reciprocal algorithm can be represented as recursive sequences. The base case of the algorithms is the initial guess $G$.

\noindent {\bf Heron's method to find $\sqrt{S}$:}
\begin{align*}
  &\begin{cases}
    h_0=&G\\
    h_{n+1}=&\frac{1}{2} \left( h_n + \frac{S}{h_n} \right)
  \end{cases}
\end{align*}

\noindent {\bf Fast Reciprocal algorithm to find $\dfrac{1}{D}$:}
\begin{align*}
  &\begin{cases}
    r_0=&G\\
    r_{n+1}=&r_n \left( 2-D r_n \right)
  \end{cases}
\end{align*}
The recursive rule can be written in terms of $a_{n}$ or $a_{n+1}$.

\subsection*{Properties of Sequences}

\textbf{Boundedness}\\
A sequence is said to be \emph{bounded} if there exists some finite number $M$ such that $|a_n|<M$ for every $n$. A sequence for which no such $M$ exists is said to be \emph{unbounded}.\\
\textbf{Strictly Positive/Negative}\\
A sequence is \emph{strictly positive} if $a_n > 0$ for all $n$.\\
A sequence is \emph{strictly negative} if $a_n < 0$ for all $n$.\\
\textbf{Equivilence}\\
Two sequences $a_n$ and $b_n$ are \emph{equivilent} if $a_n = b_n$ for all $n$.\\
\textbf{Increasing/Decreasing}\\
A sequence is \emph{increasing} if $a_{n+1} > a_n$ for all $n$\\
A sequence is \emph{decreasing} if $a_{n+1} < a_n$ for all $n$\\

For strictly positive sequences, these conditions mean respectively that $\frac{a_{n+1}}{a_n} > 1$ and $\frac{a_{n+1}}{a_n} < 1$. Therefore, to test whether a sequence is increasing or decreasing, the ratio $\frac{a_{n+1}}{a_n}$ must be evaluated.

\noindent {\bf Example:}\\
To test whether the sequence $a_n=n$ is increasing or decreasing.
\begin{equation*}
  a_n=n,\quad a_{n+1}=n+1,\quad \Rightarrow \frac{a_{n+1}}{a_n}=\frac{n+1}{n} > 1 
\end{equation*}
as $n+1 >n$. Therefore the sequence $a_n=n$ is increasing.

\noindent {\bf Example:}\\
To test whether the sequence $a_n=\frac{n+2}{n+1}$ is increasing or decreasing.
\begin{align*}
  &a_n=\frac{n+2}{n+1},\quad a_{n+1}=\frac{n+3}{n+2},\\
  &\Rightarrow \frac{a_{n+1}}{a_n}=\frac{\frac{n+3}{n+2}}{\frac{n+2}{n+1}}= \frac{n+3}{n+2}\cdot \frac{n+1}{n+2} = \frac{n^2+4n+3}{n^2+4n+4} = \frac{n^2+4n+3}{\left(n^2+4n+3\right)+1}=\frac{x}{x+1}<1 
\end{align*}
where $x=n^2+4n+2$. Therefore $\frac{a_{n+1}}{a_n}<1$ and so the sequence is decreasing.
\pagebreak
\subsection*{Limit of a Sequence}
An important property that a sequence can have is a \emph{limit}. Not all sequences have limits, but those which do can be very useful.

\textbf{Definition}\\
An infinite sequence $a_n$ is said to have a limit $L$ if as $n \to \infty$, $|a_n - L|\to 0$. If this is the case then we write
\begin{equation*}
  \lim_{n\to \infty} a_n = L
\end{equation*}
A sequence with a limit is said to \emph{converge} to that limit as $n$ goes to $\infty$. As $n$ becomes larger and larger, the numbers in the sequence become closer and closer to $L$, until the difference between $a_n$ and $L$ becomes indistinguishable; $|a_n-L|\to 0$.\\
\\
A sequence which has a limit is said to be \emph{convergent}\\
A sequence without a limit is said to be \emph{divergent}\\

\noindent {\bf Example:}\\
Consider the sequence $a_n=\frac{1}{n}$, which is known as the \emph{harmonic sequence}. Some terms in this sequence are given in the table below, and are plotted in figure \vref{fig:harmonic_sequence}
\begin{center}
\begin{tabular}{|c|c|c|c|c|c|c|c|c|c|}
\hline
 $a_1$ & $a_2$ & $a_3$ & $a_4$ & $a_5$ & $\cdots$ & $a_{100}$ & $\cdots$ & $a_{1000000}$ & $\cdots$  \\
\hline
1 & $\frac{1}{2}$ & $\frac{1}{3}$ & $\frac{1}{4}$ & $\frac{1}{5}$  & $\cdots$ & 0.01 & $\cdots $  & 0.000001 &  \\
\hline
\end{tabular}
\end{center}
\begin{figure}
  \centering
  \includegraphics[width=0.4\textwidth]{images/harmonic_sequence.eps}
  \caption{A plot of the terms in the harmonic sequence $\frac{1}{n}$. The terms converge towards the limit $0$. }
  \label{fig:harmonic_sequence}
\end{figure}
As $n$ becomes larger and larger, the terms $\frac{1}{n}$ become smaller and smaller. As $n$ becomes infinitely large, $\frac{1}{n}$ become infinitely small. The sequence converges towards the limit $0$. Hence we write
\begin{equation}
  \label{eq:1}
  \lim_{n \to \infty} \frac{1}{n}  = 0
\end{equation}
As $n$ increases, eventually no computer, machine, or instrument will be able to distinguish the terms in this sequence from $0$, or from one another. This limit of the harmonic sequence will be regarded as a fundamental limit when we calculate the limits of rational formulas.

\noindent {\bf Example:}\\
Consider the sequence $b_n=\frac{n}{n+1}$. Some terms in this sequence are given in the table below, and are plotted in figure \vref{fig:inverse_harmonic_sequence}

\begin{center}
\begin{tabular}{|c|c|c|c|c|c|c|c|c|c|}
\hline
$b_1$ & $b_2$ & $b_3$ & $b_4$ & $b_5$ & $\cdots$ & $b_{99}$ & $\cdots$ & $b_{9999}$ & $\cdots$  \\
\hline
$\frac{1}{2}$ & $\frac{2}{3}$ & $\frac{3}{4}$ & $\frac{4}{5}$  & $\frac{5}{6}$  & $\cdots$ & 0.99 & $\cdots $  & 0.9999 & $\cdots $\\
\hline
\end{tabular}
\end{center}

\begin{figure}
  \centering
  \includegraphics[width=0.4\textwidth]{images/harmonic_sequence.eps}
  \caption{A plot of the terms in sequence $\frac{n}{n+1}$. The terms converge towards $1$. }
  \label{fig:inverse_harmonic_sequence}
\end{figure}
As $n \to \infty$, the numerator and denominator in $\frac{n}{n+1}$ become relatively equal in scale, and so the fraction approaches $1$. Hence we write
\begin{equation*}
  \lim_{n \to \infty} \frac{n}{n+1}  = 1
\end{equation*}
As $n$ increases, eventually no computer, machine, or instrument will be able to distinguish the terms in this sequence from $1$, or from one another.

Not all sequences have limits. A sequence can simply grow infinitely without bound, or can oscillate between values indefinitely.


\noindent {\bf Example:}\\
Let $c_n=n^2$. This sequence grows indefinitely, diverging towards $+ \infty$, and so no limit exists.
\begin{center}
\begin{tabular}{|c|c|c|c|c|c|c|c|}
\hline
$c_0$ & $c_1$ & $c_2$ & $c_3$ & $c_4$ & $c_5$ & $\cdots$ & $c_{1000}$  \\
\hline
$0$ & $1$ & $4$ & $9$ & $16$ & $25$ & $\cdots$ & $1000000$ \\
\hline
\end{tabular}
\end{center}

\noindent {\bf Example:}\\
Let $d_n=\begin{cases} +1, & n \text{ even} \\ -1, & n \text{ odd} \end{cases}$. This sequence oscillate alternately between $+1$ and $-1$. It never converges to any fixed value. Thus the sequence is said to be divergent.
\begin{center}
\begin{tabular}{|c|c|c|c|c|c||c|c|}
\hline
$d_0$ & $d_1$ & $d_2$ & $d_3$ & $d_4$ & $d_5$ & $\cdots$ & $d_{1000}$  \\
\hline
$0$ & $1$ & $4$ & $9$ & $16$ & $25$ & $\cdots$ & $1000000$ \\
\hline
\end{tabular}
\end{center}
\pagebreak
\subsection*{Limits of rational Formulas}

We will often need to calculate the limits of expressions like $\lim_{n\to \infty} \frac{3 n^2 + 5n +1}{2 n^2 +3}$. Numerical investigation reveals that this sequence is tending towards $\frac{3}{2}$, but we must be able to prove this. We cannot set $n$ equal to $\infty$ in the expression, as this would result in $\frac{\infty}{\infty}$ which is undefined/not a number.

Instead we will use the fundamental limit $\lim_{ n \to \infty} \frac{1}{n}=0$ to evaluate the limit of rational formulas. Because we know its limit at infinity is zero, we will adopt the convention that any $\left( \frac{1}{n}\right)$ terms in an expression can be set equal to zero when the limit at infinity is taken.

\noindent {\bf Example:}\\
\[ \lim_{n \to \infty} \left[ 7+\frac{3}{n} \right] =  \lim_{n \to \infty} \left[ 7+ 3 \left( \frac{1}{n} \right) \right] = 7+ 3(0) = 7\]

\noindent {\bf Example:}\\
\[ \lim_{n \to \infty} \left[ 8-\frac{4}{n}+\frac{5}{n^2} \right] =  \lim_{n \to \infty} \left[ 8-4 \left( \frac{1}{n} \right) + 5 \left( \frac{1}{n} \right) \left( \frac{1}{n} \right) \right] = 8-4(0)+5(0)(0) = 8\]

\noindent {\bf Example:}\\
\[
\lim_{n \to \infty } \frac{1}{n^3}= \lim_{n \to \infty } \left( \frac{1}{n} \right)\left( \frac{1}{n} \right)\left( \frac{1}{n} \right) = (0)(0)(0) = 0
\]

\noindent {\bf Example:}\\
\[
\lim_{n \to \infty } \frac{C}{n^p} = 0, \quad \text{if } p \ge 1, \quad \text{where $C$ is any constant}
\]

The limits of more complicated expressions involving $\frac{1}{n}$ are evaluated in the same way
\noindent {\bf Example:}\\
\[
\lim_{n \to \infty } \left[ \frac{5+\frac{2}{n}}{3-\frac{4}{n^2}} \right]= \frac{5+2(0)}{3-4(0)}=\frac{5}{3}
\]

\noindent {\bf Example:}\\
\[
\lim_{n \to \infty } \left[ \frac{21 -\frac{5}{n}+\frac{16}{n^3}}{4+\frac{9}{n^2}-\frac{10}{n^4}} \right]=\frac{21-0+0}{4+0-0}=\frac{21}{4}
\]

For general rational formulas, we need to transform the formula so that the fundamental limit $\left( \frac{1}{n} \right) \to 0$ can be applied. For example, consider the problem of finding the limit of $\frac{n}{n+1}$ as $n \to \infty $. To transform $\frac{n}{n+1}$, note that because
\[
\frac{\ \frac{1}{n}\ }{\ \frac{1}{n}\ }=1
\]
multiplying by this fraction will not change values. Using this we have
\[ \frac{n}{n+1} =   \frac{\ \frac{1}{n}\ }{\ \frac{1}{n}\ }\cdot \frac{n}{n+1} = \frac{\left( \frac{1}{n} \right) n}{\left( \frac{1}{n} \right)(n+1)} = \frac{\frac{n}{n}}{\frac{n}{n}+\frac{1}{n}}= \frac{1}{1+\frac{1}{n}}
\]
As the first and last formulas here are equal, the fundamental limit $\left( \frac{1}{n} \right) \to 0$ can now be applied as $n \to \infty $.

\[
\lim_{n \to \infty } \left[ \frac{n}{n+1} \right]= \lim_{n \to \infty } \left[ \frac{1}{1+\frac{1}{n}} \right]= \frac{1}{1+0}=1
\]
This result confirms the earlier observation that $\lim_{n \to \infty } \left[ \frac{n}{n+1} \right]=1$.

To find the limits of formulas containing higher powers of $n$, we need to divide the formula above and below by the highest power of $n$ in the expression.

\noindent {\bf Example:}\\
\noindent To find $\lim_{n \to \infty } \frac{n^2-7}{2n^2+n+3}$.

The highest power of $n$ in this expression is $n^2$. Therefore, we multiply the fraction above and below by $\frac{1}{n^2}$.
\begin{align*}
\frac{n^2-7}{2n^2+n+3}=\frac{\frac{1}{n^2}}{\frac{1}{n^2}}\cdot \frac{n^2-7}{2n^2+n+3} & = \frac{\left( \frac{1}{n^2} \right) \left( n^2-7 \right)}{ \left( \frac{1}{n^2} \right) \left( 2n^2+n+3 \right)}\\ & = \frac{\frac{n^2}{n^2}-\frac{7}{n^2}}{2 \cdot \frac{n^2}{n^2}+\frac{n}{n^2}+\frac{3}{n^2}} = \frac{1-\frac{7}{n^2}}{2+\frac{1}{n}+\frac{3}{n^2}}
\end{align*}
With these formulas equal, the fundamental limit $\left( \frac{1}{n} \right) \to 0$ can now be applied as $n \to \infty $.

\[
\lim_{n \to \infty } \left[ \frac{n^2-7}{2n^2+n+3} \right]=  \lim_{n \to \infty } \left[ \frac{1-\frac{7}{n^2}}{2+\frac{1}{n}+\frac{3}{n^2}}\right] = \frac{1-0}{2+0+0} = \frac{1}{2}
\]

\noindent {\bf Example:}\\
\noindent To find $\lim_{n \to \infty } \frac{2n-n^2}{5n^3+4n-10}$.

The highest power in the expression is $n^3$. Therefore, we multiply the fraction above and below by $\frac{1}{n^3}$. Before doing this however, note that the only place this term appears is in the denominator. Therefore, as $n \to \infty $, the denominator will grow much faster than other terms. In fact, the fraction will approach $\to \frac{-n^2}{5n^3}  = -\frac{1}{5n} \to 0$. But we will now confirm this formally.

\[
\frac{2n-n^2}{5n^3+4n-10} = \frac{\frac{1}{n^3}}{\frac{1}{n^3}}\cdot \frac{2n-n^2}{5n^3+4n-10} = \frac{2 \frac{n}{n^3}-\frac{n^2}{n^3}}{5 \frac{n^3}{n^3}+4 \frac{n}{n^3}-\frac{10}{n^3}} = \frac{\frac{2}{n^2}-\frac{1}{n}}{5+\frac{4}{n^2}-\frac{10}{n^3}}
\]
Hence
\[
\lim_{n \to \infty } \left[ \frac{2n-n^2}{5n^3+4n-10} \right]= \lim_{n \to \infty } \left[  \frac{\frac{2}{n^2}-\frac{1}{n}}{5+\frac{4}{n^2}-\frac{10}{n^3}} \right]= \frac{0-0}{5+0-0}=0
\]


\subsection*{Limits of Recursive Sequences}
If a recursive sequence has a limit $L$, this can sometimes be found by replacing all terms $a_{n}$ by $L$ in the recursive rule. Consider for example Heron's method to compute $\sqrt{S}$
\begin{align*}
  &\begin{cases}
    h_0=&G\\
    h_{n+1}=&\frac{1}{2} \left( h_n + \frac{S}{h_n} \right)
  \end{cases}
\end{align*}

First we assume that the sequence converges to some number $L$. Then $\lim_{n \to \infty} a_n=L$, and the terms in the sequence are become closer and closer to $L$ as $n$ increases. So for large $n$, $a_{n} \to  L$ and $a_{n+1}\to L$. Inserting these approximations into the recursion formula gives
\begin{align*}
  L&=\frac{1}{2}\left(L+S/L\right)\\
\Rightarrow L&=\frac{L}{2} + \frac{S}{2L}\\
\Rightarrow \frac{L}{2} &= \frac{S}{2L}, \quad \Rightarrow  L^2=S
\end{align*}
And so $L=\sqrt{S}$, as required.

However, it should be noted that this method finds the limit $L$ under the assumption that the limit exists; it will not work for sequences which have no limit. Care should be taken when applying this method to algorithms such as the Fast Reciprocal method, which can easily not have a limit if the initial guess $G$ is too poor.


\pagebreak
\subsection*{Limits in Relation to Computers}
\emph{The following section is a more advanced discussion of limits in relation to computers. This material will not be on the exam.}

The previous definition of the limit of a sequence is not entirely precise. A more precise definition of the limit of a sequence is the following.
\begin{framed}
A sequence $a_n$ is said to have a limit $L$ if:\\
For all $\epsilon > 0$ there is some integer $M>0$ such that for all $n>M$, $|a_n-L|< \epsilon$.
\end{framed}
The value $\epsilon$ can be considered to be the \emph{limit of precision}. This has relevance to computer science and how computers represent numbers.

Consider a computer which stores or represents numbers internally. Since storage space is finite, such a system can only represent numbers to a certain degree of accuracy. Infinite precision is not possible in practice. There is a limit to how precisely the computer(or any real measuring device) can represent numbers and moreover the differences between numbers.

In particular, on any computer, there will be a smallest non zero positive number $\epsilon$. While the computer can represent and store both $0$ and $\epsilon$ it cannot represent any number between them as these numbers are too small and beyond its limit of precision, and they will be equated to zero. Thus if $|a-b|<\epsilon$, the numbers $a$ and $b$ will be indistinguishable from one another on that computer.

Different computers and device have different limits of precision $\epsilon$. For double precision floating point numbers found on most computers $\epsilon \cong 5 \times 10^{-324}$—though numbers will become indistinguishable long before this due to rounding error $\cong  2 \times 10^{-16}$. In principle however, $\epsilon$ can be any non zero number, no matter how small.

But if a sequence has a limit, then for all $\epsilon > 0$ eventually $n$ will become large enough so that $|a_n -L|<\epsilon$, and so eventually no computer, no matter how precise, will be be able to distinguish between the terms $a_n$ and the limit $L$. The terms $a_n$ have come as close to $L$ as is possible on that computer; or put another way, we have estimated $L$ as accurately as we can on that device.

So a sequence with a limit $L$ offers a way to come as close as possible to $L$ on any computer. This is especially important for numbers like $\sqrt{2}$,$\sqrt{10}$ and $\pi$, which can never be represented exactly on any computer. Though these numbers cannot be represented exactly, if we have a sequence whose limit is such a number, then we can obtain as accurate an estimate of these numbers as we wish.

When you read stories of $\pi$ being approximated to the billionth or trillionth decimal place, sequences with $\lim_{n \to \infty} a_n = \pi$ are being used.

\end{document}

