%-------------------------------------------------------------------------%
\newpage

\section{Section 4 Functions}

\subsection{Invertible Functions}
A function is invertible if it fulfils two criteria
\begin{itemize}
\item The function is \textbf{\textit{onto}},
\item The function is \textbf{\textit{one-to-one}}.
\end{itemize}

State the conditions to be satisfied by a function
$f : X \leftarrow Y$ for it to have an inverse function
$f^{-1} : Y \leftarrow X$.
%---------------------------------------------------------%

$\lceil \frac{x^2+1}{4} \rceil$
where $f : A \rightarrow \textbf{Z}$
\begin{itemize}
\item[(i)] Find $f(4)$ and the ancestors of 3.
\item[(ii)] Find the range of $f$.
\item[(iii)] Is f invertible? Justify your answer
\end{itemize}

Given $f : \textbf{R} \rightarrow \textbf{R}$ where f(x) =3x-1,define fully
the inverse of the function f ,i.e.$f^{-1}$. 
State the value of $f^{-1}(2)$
%---------------------------------------------------------%
\subsection{Precision Functions}

\begin{itemize}
\item Absolute Value Function $| x |$
\item Ceiling Function $\lceil x \rceil$
\item Floor Function  $\lfloor x \rfloor $
\end{itemize}
%\[ \lfloorx\rfloor\]

\noindent \textbf{Question1.2}: State the range and domain of the following function
\[ F(x) = \lfloor x-1 \rfloor \]
%---------------------------------------------------------%
\subsection{Powers}

\[  2^ 4 = 2 \times 2 \times 2 \times 2 = 16 \]

\[  5^ 3 = 5 \times 5 \times 5 =125 \]

\subsubsection{Special Cases}

Anything to the power of zero is always 1

\[  X^ 0 = 1 \mbox{ for all values of X} \]

Sometimes the power is a negative number.

\[  X^{-Y} = { 1 \over X^Y}  \]

Example 
\[  2^{-3} = { 1 \over 2^3} = { 1 \over 8}  \]



%---------------------------------------------------------%
\subsection{Exponentials Functions}

\[ e^a \times e^b = e^{a+b}\]

\[ (e^a )^b = e^{ab}\]
%---------------------------------------------------------%
\subsection{Logarithmic Functions}

\subsubsection{Laws for Logarithms}
The following laws are very useful for working with logarithms.
\begin{enumerate}
\item $\mbox{log}_b(X)$ + $\mbox{log}_b(Y)$ = $\mbox{log}_b(X\times Y)$
\item $\mbox{log}_b(X)$ - $\mbox{log}_b(Y)$ = $\mbox{log}_b(X / Y)$
\item $\mbox{log}_b(X^Y)$= $Y \mbox{log}_b(X)$
\end{enumerate}

\noindent \textbf{Question1.3} Compute the Logarithm of the following
\begin{itemize}
\item $\mbox{log}_2(8)$
\item $\mbox{log}_2(\sqrt{128})$
\item $\mbox{log}_2(64)$
\item $\mbox{log}_5(125)$ +   $\mbox{log}_3(729)$
\item $\mbox{log}_2(64/4)$
\end{itemize}