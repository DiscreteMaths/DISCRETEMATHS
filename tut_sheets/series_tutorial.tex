%\documentclass[11pt, a4paper,dalthesis]{report}    % final 
%\documentclass[11pt,a4paper,dalthesis]{report}
%\documentclass[11pt,a4paper,dalthesis]{book}

\documentclass[11pt,a4paper,titlepage,oneside,openany]{article}

\pagestyle{plain}
%\renewcommand{\baselinestretch}{1.7}

\usepackage{setspace}
%\singlespacing
\onehalfspacing
%\doublespacing
%\setstretch{1.1}

\usepackage{amsmath}
\usepackage{amssymb}
\usepackage{amsthm}
\usepackage{multicol}

\usepackage[margin=3cm]{geometry}
\usepackage{graphicx,psfrag}%\usepackage{hyperref}
\usepackage[small]{caption}
\usepackage{subfig}

\usepackage{algorithm}
\usepackage{algorithmic}
\newcommand{\theHalgorithm}{\arabic{algorithm}}

\usepackage{varioref} %NB: FIGURE LABELS MUST ALWAYS COME DIRECTLY AFTER CAPTION!!!
%\newcommand{\vref}{\ref}

\usepackage{index}
\makeindex
\newindex{sym}{adx}{and}{Symbol Index}
%\newcommand{\symindex}{\index[sym]}
%\newcommand{\symindex}[1]{\index[sym]{#1}\hfill}
\newcommand{\symindex}[1]{\index[sym]{#1}}

%\usepackage[breaklinks,dvips]{hyperref}%Always put after varioref, or you'll get nested section headings
%Make sure this is after index package too!
%\hypersetup{colorlinks=false,breaklinks=true}
%\hypersetup{colorlinks=false,breaklinks=true,pdfborder={0 0 0.15}}


%\usepackage{breakurl}

\graphicspath{{./images/}}

\usepackage[subfigure]{tocloft}%For table of contents
\setlength{\cftfignumwidth}{3em}

\input{longdiv}
\usepackage{wrapfig}


%\usepackage{index}
%\makeindex
%\usepackage{makeidx}

%\usepackage{lscape}
\usepackage{pdflscape}
\usepackage{multicol}

\usepackage[utf8]{inputenc}

%\usepackage{fullpage}

%Compulsory packages for the PhD in UL:
%\usepackage{UL Thesis}
\usepackage{natbib}

%\numberwithin{equation}{section}
\numberwithin{equation}{section}
\numberwithin{algorithm}{section}
\numberwithin{figure}{section}
\numberwithin{table}{section}
%\newcommand{\vec}[1]{\ensuremath{\math{#1}}}

%\linespread{1.6} %for double line spacing

\usepackage{afterpage}%fingers crossed

\newtheorem{thm}{Theorem}[section]
\newtheorem{defin}{Definition}[section]
\newtheorem{cor}[thm]{Corollary}
\newtheorem{lem}[thm]{Lemma}

%\newcommand{\dbar}{{\mkern+3mu\mathchar'26\mkern-12mu d}}
\newcommand{\dbar}{{\mkern+3mu\mathchar'26\mkern-12mud}}

\newcommand{\bbSigma}{{\mkern+8mu\mathsf{\Sigma}\mkern-9mu{\Sigma}}}
\newcommand{\thrfor}{{\Rightarrow}}

\newcommand{\mb}{\mathbb}
\newcommand{\bx}{\vec{x}}
\newcommand{\bxi}{\boldsymbol{\xi}}
\newcommand{\bdeta}{\boldsymbol{\eta}}
\newcommand{\bldeta}{\boldsymbol{\eta}}
\newcommand{\bgamma}{\boldsymbol{\gamma}}
\newcommand{\bTheta}{\boldsymbol{\Theta}}
\newcommand{\balpha}{\boldsymbol{\alpha}}
\newcommand{\bmu}{\boldsymbol{\mu}}
\newcommand{\bnu}{\boldsymbol{\nu}}
\newcommand{\bsigma}{\boldsymbol{\sigma}}
\newcommand{\bdiff}{\boldsymbol{\partial}}

\newcommand{\tomega}{\widetilde{\omega}}
\newcommand{\tbdeta}{\widetilde{\bdeta}}
\newcommand{\tbxi}{\widetilde{\bxi}}



\newcommand{\wv}{\vec{w}}

\newcommand{\ie}{i.e. }
\newcommand{\eg}{e.g. }
\newcommand{\etc}{etc}

\newcommand{\viceversa}{vice versa}
\newcommand{\FT}{\mathcal{F}}
\newcommand{\IFT}{\mathcal{F}^{-1}}
%\renewcommand{\vec}[1]{\boldsymbol{#1}}
\renewcommand{\vec}[1]{\mathbf{#1}}
\newcommand{\anged}[1]{\langle #1 \rangle}
\newcommand{\grv}[1]{\grave{#1}}
\newcommand{\asinh}{\sinh^{-1}}

\newcommand{\sgn}{\text{sgn}}
\newcommand{\morm}[1]{|\det #1 |}

\newcommand{\galpha}{\grv{\alpha}}
\newcommand{\gbeta}{\grv{\beta}}
%\newcommand{\rnlessO}{\mb{R}^n \setminus \vec{0}}
\usepackage{listings}
\usepackage{arydshln}

\interfootnotelinepenalty=10000

\newcommand{\sectionline}{%
  \nointerlineskip \vspace{\baselineskip}%
  \hspace{\fill}\rule{0.5\linewidth}{.7pt}\hspace{\fill}%
  \par\nointerlineskip \vspace{\baselineskip}
}

\renewcommand{\labelenumii}{\roman{enumii})}

\begin{document}

\begin{center}
  \textbf{Tutorial Sheet 2}
\end{center}

\begin{enumerate}
\item For each of the following sequences $a_n$, evaluate the corressponding series $s_n=\displaystyle\sum_{k=0}^n a_k$, up to the term $s_5$. In addition, find $s_{100}$.  
  \begin{multicols}{2}
    \begin{enumerate}
    \item $a_n=\frac{9}{10^n}$
    \item $a_n=\frac{1}{2^n}$
    \item $a_n=2^n$
    \item $a_n=\frac{2^n}{3^n}$
    \item $a_n=3+4n$
    \item $a_n=\frac{1}{4}+\frac{5n}{4}$
    \item $a_n=1+2n+2^n$
    \item $a_n=30+40n+8\frac{2^n}{3^n}$
    \end{enumerate}        
  \end{multicols}
\item Use the ratio test to determine whether the following infinite series are convergent or divergent, or say whether the test is inconclusive.
\begin{multicols}{3}
    \begin{enumerate}
    \item $\displaystyle \sum_{n=0}^\infty \frac{1}{3^n}$
    \item $\displaystyle \sum_{n=0}^\infty \frac{n^3}{5^n}$
    \item $\displaystyle \sum_{n=0}^\infty \frac{5^n}{n^3}$
    \item $\displaystyle \sum_{n=1}^\infty \frac{1}{n}$
    \item $\displaystyle \sum_{n=0}^\infty \frac{n}{n+1}$
    \item $\displaystyle \sum_{n=0}^\infty \frac{x^{2n}}{n!}$
    \item $\displaystyle \sum_{n=0}^\infty x^n$
    \item $\displaystyle \sum_{n=1}^\infty \frac{(-1)^n x^n}{n}$
    \item $\displaystyle \sum_{n=0}^\infty \frac{x^{2n}}{(n+1)!}$
    \end{enumerate}        
  \end{multicols}


\item The exponential function $e^x=\text{exp}(x)$ can be defined using an infinite series as
\begin{equation*}
\text{exp}(x)=\sum_{n=0}^\infty \frac{x^n}{n!}
\end{equation*}
Given that this infinite series is convergent for all $x \in \mb{R}$, use the series and the following table to estimate the value of $e^{0.6}$ to $3$ decimal places.
\\
\\

\begin{tabular}{|l|c|c:c:c:c:c:c:c:c:c:c:c|c|c:c:c:c:c:c:c:c:c:c:c|}\hline
$n$ & \ & \multicolumn{11}{c|}{$a_n$} & \ & \multicolumn{11}{c|}{$s_n$} \\\hline
$0$ & \ &
$1$ & $\cdot$ & $0$ & $0$ & $0$ & $0$ & $0$ & $0$ & $0$ & $0$ & $0$ & \ &
$1$ & $\cdot$ & $0$ & $0$ & $0$ & $0$ & $0$ & $0$ & $0$ & $0$ & $0$ \\ \hline
$1$ & \ &
 & $\cdot$ &  &  &  &  &  &  &  &  &  & \ &
 & $\cdot$ &  &  &  &  &  &  &  &  &  \\ \hline
$2$ & \ &
 & $\cdot$ &  &  &  &  &  &  &  &  &  & \ &
 & $\cdot$ &  &  &  &  &  &  &  &  &  \\ \hline
$3$ & \ &
 & $\cdot$ &  &  &  &  &  &  &  &  &  & \ &
 & $\cdot$ &  &  &  &  &  &  &  &  &  \\ \hline
$4$ & \ &
 & $\cdot$ &  &  &  &  &  &  &  &  &  & \ &
 & $\cdot$ &  &  &  &  &  &  &  &  &  \\ \hline
$5$ & \ &
 & $\cdot$ &  &  &  &  &  &  &  &  &  & \ &
 & $\cdot$ &  &  &  &  &  &  &  &  &  \\ \hline
$6$ & \ &
 & $\cdot$ &  &  &  &  &  &  &  &  &  & \ &
 & $\cdot$ &  &  &  &  &  &  &  &  &  \\ \hline
$7$ & \ &
 & $\cdot$ &  &  &  &  &  &  &  &  &  & \ &
 & $\cdot$ &  &  &  &  &  &  &  &  &  \\ \hline
$8$ & \ &
 & $\cdot$ &  &  &  &  &  &  &  &  &  & \ &
 & $\cdot$ &  &  &  &  &  &  &  &  &  \\ \hline
\end{tabular}

\end{enumerate}
\pagebreak

\begin{center}
\textbf{Solutions}
\end{center}
\begin{enumerate}
\item    \begin{enumerate}   \begin{multicols}{2}
 
    \item 
      \begin{tabular}{|l|c|c|}
      $n$ & $a_n$ & $s_n$ \\ \hline
      $0$ & $9.00000$ & $9.00000$ \\
      $1$ & $0.90000$ & $9.90000$ \\
      $2$ & $0.09000$ & $9.99000$ \\
      $3$ & $0.00900$ & $9.99900$ \\
      $4$ & $0.00090$ & $9.99990$ \\
      $5$ & $0.00009$ & $9.99999$ 
      \end{tabular}\\
      $s_n=9\frac{1-\frac{1}{10^{n+1}}}{1-\frac{1}{10}}$\\
      $s_{100}=10\left(1-10^{-101}\right)\cong 10$
      \columnbreak
      \setcounter{enumii}{4}
          \item 
      \begin{tabular}{|l|c|c|}
      $n$ & $a_n$ & $s_n$ \\ \hline
      $0$ & $3$ & $3$ \\
      $1$ & $7$ & $10$ \\
      $2$ & $11$ &$21$ \\
      $3$ & $15$ & $36$ \\
      $4$ & $19$ & $55$ \\
      $5$ & $23$ & $78$ 
      \end{tabular}\\
      $s_n=\frac{n+1}{2}\left( 2(3)+4n \right)$\\
      $s_{100}=20503$
    \end{multicols}
    \setcounter{enumii}{7}

    viii)\  $s_n=\frac{n+1}{2}\left( 2(30)+40n \right)+8\frac{1-\left(\frac{2}{3}\right)^{n+1}}{1-\frac{2}{3}}$\\
    $s_{100}=205030+8\frac{1-\left(\frac{2}{3}\right)^{101}}{\frac{1}{3}}\cong 205030+24=205054$
  \end{enumerate}        

\item
  \begin{multicols}{3}
    \begin{enumerate}
      \item $R=\frac{1}{3}<1$ cgt.
      \item $R=\frac{1}{5}<1$ cgt.
      \item $R=5>1$ dgt.
      \item $R=1$ Inconclusive\footnote{However, this series is the harmonic series, which is in fact divergent.}.
      \item $R=1$ inconclusive.
      \item $R=0<1$ cgt for all $x$.
      \item $R=|x|$ cgt if $|x|<1$.
      \item $R=|x|$ cgt if $|x|<1$.
      \item $R=0<1$ cgt for all $x$.
    \end{enumerate}        
  \end{multicols}

\item
\hfill

\begin{tabular}{|l|c|c:c:c:c:c:c:c:c:c:c:c|c|c:c:c:c:c:c:c:c:c:c:c|}\hline
$n$ & \ & \multicolumn{11}{c|}{$a_n$} & \ & \multicolumn{11}{c|}{$s_n$} \\\hline
$0$ & \ &
$1$ & $\cdot$ & $0$ & $0$ & $0$ & $0$ & $0$ & $0$ & $0$ & $0$ & $0$ & \ &
$1$ & $\cdot$ & $0$ & $0$ & $0$ & $0$ & $0$ & $0$ & $0$ & $0$ & $0$ \\ \hline
$1$ & \ &
$0$ & $\cdot$ & $6$ & $0$ & $0$ & $0$ & $0$ & $0$ & $0$ & $0$ & $0$ & \ &
$1$ & $\cdot$ & $6$ & $0$ & $0$ & $0$ & $0$ & $0$ & $0$ & $0$ & $0$ \\ \hline
$2$ & \ &
$0$ & $\cdot$ & $1$ & $8$ & $0$ & $0$ & $0$ & $0$ & $0$ & $0$ & $0$ & \ &
$1$ & $\cdot$ & $7$ & $8$ & $0$ & $0$ & $0$ & $0$ & $0$ & $0$ & $0$ \\ \hline
$3$ & \ &
$0$ & $\cdot$ & $0$ & $3$ & $6$ & $0$ & $0$ & $0$ & $0$ & $0$ & $0$ & \ &
$1$ & $\cdot$ & $8$ & $1$ & $6$ & $0$ & $0$ & $0$ & $0$ & $0$ & $0$ \\ \hline
$4$ & \ &
$0$ & $\cdot$ & $0$ & $0$ & $5$ & $4$ & $0$ & $0$ & $0$ & $0$ & $0$ & \ &
$1$ & $\cdot$ & $8$ & $2$ & $1$ & $4$ & $0$ & $0$ & $0$ & $0$ & $0$ \\ \hline
$5$ & \ &
$0$ & $\cdot$ & $0$ & $0$ & $0$ & $6$ & $4$ & $8$ & $0$ & $0$ & $0$ & \ &
$1$ & $\cdot$ & $8$ & $2$ & $2$ & $0$ & $4$ & $8$ & $0$ & $0$ & $0$ \\ \hline
$6$ & \ &
$0$ & $\cdot$ & $0$ & $0$ & $0$ & $0$ & $6$ & $4$ & $8$ & $0$ & $0$ & \ &
$1$ & $\cdot$ & $8$ & $2$ & $2$ & $1$ & $1$ & $2$ & $8$ & $0$ & $0$ \\ \hline
$7$ & \ &
$0$ & $\cdot$ & $0$ & $0$ & $0$ & $0$ & $0$ & $5$ & $5$ & $5$ & $4$ & \ &
$1$ & $\cdot$ & $8$ & $2$ & $2$ & $1$ & $1$ & $8$ & $3$ & $5$ & $4$ \\ \hline
$8$ & \ &
 & $\cdot$ &  &  &  &  &  &  &  &  &  & \ &
 & $\cdot$ &  &  &  &  &  &  &  &  &  \\ \hline

\end{tabular}
\\
\\
\\
$\Rightarrow e^{0.6} \cong 1.822$


\end{enumerate}


\end{document}

%%% Local Variables: 
%%% mode: latex
%%% TeX-master: t
%%% End: 
