A. Let S = {0, 1, 2, 3}. 
Define a relation R1 between the elements of S by
“x is related to y if the product xy is even”.

B. Let S = {0, 1, 2, 3}. 
Define a relation R2 between the elements of S by
“x is related to y if x − y \in {0, 3,−3}”.

C. Let S = {{1}, {1, 2}, {1, 2, 3}, {1, 2, 4}}. 
Define a relation R3 between
the elements of S by “X is related to Y if X \subseteq Y ”.
%-------------------------------------------------------------%
(i) Draw the relationship digraph.
(ii) Determine whether the relation is either reflexive, symmetric, transitive
or anti-symmetric. For the cases when one of these properties
does not hold, justify your answer by giving an example to show that
it does not hold.
\end{frame}
%-------------------------------------------------------------%
\begin{frame}
(iii) Determine which of the relations is an equivalence relation. For the
case when it is an equivalence relation, calculate the distinct equivalence
classes and verify that they give a partition of S.
(iv) Determine which of the relations is a partial order or an order.
%------------------------------------------------------------%
\begin{frame}
% Part A
The relation:
is not reflexive e.g. 1 is not related to 1;
is symmetric;
is not transitive e.g. 1 is related to 2, and 2 is related to 3, but 1
is not related to 3;
is not anti-symmetric e.g. 1 is related to 2, and 2 is related to 1,
but 1 6= 2.
\end{frame}
%-------------------------------------------------------------%
\begin{frame}
(iii) R1 is not an equivalence relation because it is neither reflexive
nor transitive.
(iv) R1 is not a partial order or an order.
\end{frame}
%-------------------------------------------------------------%
\begin{frame}
% Part B
The relation:
is reflexive, symmetric and transitive;
is not anti-symmetric e.g. 0 is related to 3, and 3 is related to 0,
but 0 6= 3.
\begin{frame}
(iii) R2 is an equivalence relation. The three equivalence classes are
{0, 3}, {1}, and {2}. These form a partition of {0, 1, 2, 3}.
(iv) R2 is not a partial order or an order as it is not anti-symmetric.
\end{frame}

%-------------------------------------------------------------%
\begin{frame}
% Part C
ii) The relation:
is reflexive, transitive and anti-symmetric;
is not symmetric e.g. {1} is related to {1, 2}, but {1, 2} is not
related to {1}.
\end{frame}
%-------------------------------------------------------------%
\begin{frame}
(iii) R3 is not an equivalence relation as it is not symmetric.
(iv) R3 is a partial order but not an order (since {1, 2, 3} and {1, 2, 4}
are not related to each other).
