\documentclass{beamer}

\usepackage{amsmath}
\usepackage{amssymb}

\begin{document}

% Ex 1
\section{Video 1 :  Set Theory Listing Method}
%--------------------------------------------%
\begin{frame}
\frametitle{Set Theory : Listing Method}
\Large
Describe the following sets using the Listing Method

\begin{itemize}
\item[(i)] $ \{ 10^m : -2 \leq m \leq 4, m \in \mathbb{Z} \} $
\item[(ii)]  $ \{ \frac{1}{n}: 1 < n < 4, n \in \mathbb{Z} \} $
\end{itemize}
\end{frame}

%--------------------------------------------%
\begin{frame}
\frametitle{Set Theory : Listing Method}
\Large
\vspace{-4cm}
\textbf{Part 1:} $ \{ 10^m : -2 \leq m \leq 4, m \in \mathbb{Z} \} $

\end{frame}
%--------------------------------------------%
\begin{frame}
\frametitle{Set Theory : Listing Method}
\Large
\vspace{-4cm}
\textbf{Part 2:} $ \{ \frac{1}{n}: 1 < n < 4, n \in \mathbb{Z} \} $

\end{frame}
% Ex 2
\section{Video 2 : Set Theory}
%--------------------------------------------%
\begin{frame}
\frametitle{Set Theory : Set Operations}
\Large
Given the following sets

\begin{center}
\begin{tabular}{|c|c|} \hline
$\mathcal{U}$ & $\{1,2,3,4,5,6,7,8,9\}$ \\ \hline
$\mathcal{A}$ & $\{1,2,5,6,8\}$ \\ \hline
$\mathcal{B}$ & $\{3,5,7,8\}$ \\ \hline
$\mathcal{C}$ & $\{5,6,7,8,9\}$ \\ \hline
\end{tabular}
\end{center}

List the elements of the following
$A^{\prime} \cap B $\\
$A^{\prime} \cap C $\\
\end{frame}
%--------------------------------------------%
\begin{frame}
\frametitle{Set Theory : Set Operations}
\Large
\begin{itemize}
\item[(i)] $\{5,8\}$
\item[(ii)] $\{1,2,3,4,5,6,7,8,9\}$ 
\end{itemize}
\end{frame}

%Exercise 3
\section{Video 3 : Set Theory}
%--------------------------------------------%
\begin{frame}
\frametitle{Set Theory : Set Operations}
\Large
\begin{itemize}
\item[\textbf{A}] $ \{ 2n : n \in \mathbb{Z^{+}} \} $
\item[\textbf{B}] $ \{ 3,6,9,12,15,18,\ldots \} $
\end{itemize}
\textbf{Questions}
\begin{itemize}
\item[(i)] \textbf{A} is described by the rules of inclusion. Describe \textbf{A} with the listing method.
\item[(ii)] \textbf{B} is described by the listing method. Describe \textbf{B} with the rules of inclusion. 
\end{itemize}
\end{frame}

%--------------------------------------------%
% Exercise 4 
\section{Video 4 : Graph Theory}
\begin{frame}
\frametitle{Graph Theory}
\Large
\vspace{-0.7cm}
Draw the graph \textbf{G}, which has the vertices $v_1$,$v_2$,$v_3$,$\ldots$,$v_7$, and the adjacency list:

\begin{itemize}
\item[$v_1$]: $v_2$,$v_4$
\item[$v_2$]: $v_1$,$v_3$
\item[$v_3$]: $v_2$,$v_4$
\item[$v_4$]: $v_1$,$v_3$,$v_5$
\item[$v_5$]: $v_4$,$v_6$
\item[$v_6$]: $v_5$,$v_7$
\item[$v_7$]: $v_5$,$v_6$
\end{itemize}
%Draw the graph \textbf{G}.
\end{frame}
%------------------------------------------------------------------------- %
\section{Video 6 : Graph Theory - Isomorphic Graphs}
\begin{frame}[fragile]
% http://www3.ul.ie/cemtl/pdf%20files/cm2/IsomorphicGraph.pdf
\frametitle{Isomorphic Graphs}
\begin{itemize}
\item If the graphs are not simple, we need more sophisticated methods to check for when two graphs are isomorphic. 
\item However, it is often straightforward to show that two graphs are not isomorphic. 
\item You can do this by showing any of the following seven conditions are true.
\end{itemize}
\end{frame}
%-------------------------------------------------------------------------- %
\begin{frame}[fragile]
\frametitle{Isomorphic Graphs}

\begin{enumerate}
\item The two graphs have different numbers of vertices.
\item The two graphs have different numbers of edges.
\item One graph has parallel edges and the other does not.
\item One graph has a loop and the other does not.
\item One graph has a vertice of degree k (for example) and the other does not.
\item One graph is connected and the other is not.
\item One graph has a cycle and the other has not.
\end{enumerate}
\end{frame}

% Ex 8 Inequality Operators
\section{Video 8 : Numbers}
%------------------------------------------------------------------------- %
\begin{frame}
\frametitle{Inequality Symbols}
\Large
Given $x = \sqrt{2}$ determine whether the following statements are true or false:

\begin{itemize}
\item[(i)] $x \leq 2$
\item[(ii)] $1.42 > x > 1.41$
\item[(iii)] x is a rational number
\item[(iv)] $\sqrt{2} = 2$
\end{itemize}
\end{frame}
%------------------------------------------------------------------------- %
\section{Video 6}
\begin{frame}
\frametitle{Inequality Symbols}
\Large
Convert the following statements into symbols:

\begin{itemize}
\item $\sqrt{2}$ is less than 1.5 and greater than 1.4
\item $\sqrt{2}$ is greater than or equal to 5
\end{itemize}

\end{frame}

%------------------------------------------------------------------------- %
% Floating Point Notation
\section{Video 7 : Numbers}
\begin{frame}
\frametitle{Floating Point Notation}
\Large

\begin{description}
\item[mantissa]
\item[abscissa]
\item[radix point]
\end{description}
\end{frame}
\end{document}



