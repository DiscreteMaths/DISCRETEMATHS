\documentclass{beamer}

\usepackage{amsmath}
\usepackage{framed}
\usepackage{amssymb}
\usepackage{graphics}

\begin{document}
\begin{frame}
\Huge
\[ \mbox{Mathematics for Computing} \]
\huge
\[ \mbox{Subsets and Elements of Sets} \]

\Large
\[ \mbox{kobriendublin.wordpress.com} \]
\[ \mbox{Twitter: @StatsLabDublin} \]

\end{frame}
%-------------------------------------%
\begin{frame}
\frametitle{Subsets and Elements of Sets}
\Large
\vspace{-0.5cm}
\textbf{Elements of a Set}\\
\begin{itemize}
\item Sets are comprised of members, which are often called \textbf{elements}. 
\item If a particular value ($k$) is an element of set $A$, then we would write this as
\[k \in A \]

\item If a single value $k$ is not an element of set $A$, then we write
\[k \notin A \]
\end{itemize}
\end{frame}
%------------------------------------- %
\begin{frame}
\frametitle{Subsets and Elements of Sets}
\Large
\vspace{-1.5cm}
\textbf{Subsets}\\
Given two sets $A$ and $B$, the set $A$ is a \textbf{subset} of set $B$ if every element of $A$ is also an element of $B$. 


We write this mathematically as
\[A \subseteq B \]


\bigskip
Sets are denoted with curly braces, even if they contain only one element.

\end{frame}
%------------------------------------- %
\begin{frame}
\frametitle{Subsets and Elements of Sets}
\Large
\vspace{-0.4cm}
\textbf{Subsets}\\
Suppose we have the set $A$ comprised of the following elements
\[ A =\{3,5,7,9\}\]
The value $5$ is an element of $A$
\[  5 \in A \]

The single element set $\{5\} $ is a subset of $A$.
\[ \{5\} \subseteq A\]

\end{frame}
%-------------------------------------%
\begin{frame}
\frametitle{Subsets and Elements of Sets}
\Large
\vspace{-2cm}
\textbf{Proper Subsets}\\
Given two sets $A$ and $B$, the set $A$ is a \textbf{proper subset} of set $B$ if every element of $A$ is also an element of $B$, but there are elements of set $B$ that are not in set $A$.


We write this mathematically as
\[A \subset B \]

\end{frame}
%-------------------------------------%
\begin{frame}
\frametitle{Subsets and Elements of Sets}
\Large
\vspace{-2cm}
\textbf{Equivalent Sets}\\
If both of the following two statements are \textbf{true}, 
\[\mbox{1)  } A \subseteq B \]
\[\mbox{2)  } B \subseteq A \]

then $A$ and $B$ are \textbf{equivalent sets}.


\end{frame}


%-------------------------------------%
\begin{frame}
\frametitle{Subsets and Elements of Sets}
\Large
\vspace{-2cm}
\textbf{Non-Comparable Sets}\\
If both of the following two statements are \textbf{false}, 
\[\mbox{1)  } A \subseteq B \]
\[\mbox{2)  } B \subseteq A \]

then $A$ and $B$ are said to be said to be \textbf{noncomparable sets}.


\end{frame}

%-------------------------------------%
\end{document}