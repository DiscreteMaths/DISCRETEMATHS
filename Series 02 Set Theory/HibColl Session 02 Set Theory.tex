\documentclass{article}
\usepackage{amsmath}
\usepackage{amssymb}
\usepackage{graphicx}
\begin{document}
\section{Set Theory}
\begin{enumerate}
\item The Universal Set $\mathcal{U}$
\item Union
\item Intersection
\item Set Difference
\item Relative Difference
\end{enumerate}

(2.2.2) Cardinality
The number of distinct elements in a finite set is called its cardinality.


(2.2.3) Power Set

(2.3) Operations on Sets

(2.3.1) Complement of a set


(2.3.2) Binary Operations on Sets
-Union
-Intersection
-Set Difference
-Symmetric Difference

\newpage
%---------------------------------------%
\subsection*{Dice Rolls}
Consider rolls of a die. What is the universal set?

\[ \mathcal{U} = \{1,2,3,4,5,6\} \]

%--------------------------------------%
\subsection*{Worked Example}

Suppose that the Universal Set $\mathcal{U}$ is the set of integers from 1 to 9.
\[ \mathcal{U} = \{1,2,3,4,5,6,7,8,9\}, \]

and that the set $\mathcal{A}$ contains the prime numbers between 1 to 9 inclusive.

\[ \mathcal{A} = \{1,2,3,5,7\}, \]

and that the set $\mathcal{B}$ contains the even numbers between 1 to 9 inclusive.

\[ \mathcal{B} = \{2,4,6,8\}. \]

%--------------------------------------------------------%
\subsubsection*{Complements}
\begin{itemize}

\item The Complements of A and B are the elements of the universal set not contained in A and B.

\item The complements are denoted $\mathcal{A}^{\prime}$ and $\mathcal{B}^{\prime}$
\[ \mathcal{A}^{\prime} = \{4,6,8,9\}, \]
\[ \mathcal{B}^{\prime} = \{1,3,5,7,9\}, \]

\end{itemize}


%--------------------------------------------------------%

\subsubsection*{Intersection}
\begin{itemize}

\item Intersection of two sets describes the elements that are members of both the specified Sets

\item The intersection is denoted $\mathcal{A\cap B}$ 
\[ \mathcal{A\cap B} = \{2\}\]

\item only one element is a member of both A and B.
\end{itemize}
%--------------------------------------------------------%

\subsubsection*{Set Difference}
\begin{itemize}

\item The Set Difference of A with regard to B are list of elements of A not contained by B.

\item The complements are denoted $\mathcal{A-B}$ and $\mathcal{B-A}$
\[ \mathcal{A-B} = \{1,3,5,7\}, \]

\[ \mathcal{B-A} = \{4,6,8\}, \]
\end{itemize}
\subsection*{symbols}
$\varnothing$,
$\forall$,
$\in$,
$\notin$,
$\cup$
%----------------------------------------------------------- %
\newpage

\section*{Prepositional Logic}


%-------------------------------------------------------------- %
\begin{itemize}
\item $p \wedge q$
\item $p \vee q$
\item $p \rightarrow q$
\end{itemize}
\newpage
\section{Sequence and Series and Proof by Induction}


\[\sum (n^2) \]

%--------------------------------------------------------%
\subsubsection*{Relative Difference}
\begin{itemize}
\item $ A \otimes B$
\end{itemize}
%--------------------------------------------------------%
\subsubsection*{Power Sets}
\begin{itemize}
\item Consider the set A where $ A = \{w,x,y,z\}$
\item There are 4 elements in set A.
\item The power set of A contains 16 element data sets.
\item \[  \mathcal{P}(A) = \{\{ x \}, \{ y \} \}  \]
\item (i.e. 1 null set, 4 single element sets, 6 two -elemnts sets, 4 three lement set and one 4- element set.)
\end{itemize}
%------------------------------------------------%
\newpage
\begin{itemize}
\item $ p \rightarrow q$  p implies q
\item $p \lg q $
\end{itemize}
%---------------------------- %

\section*{Session 05 Graph Theory}
\begin{itemize}
\item Eulerian Path
\item Isomorphism
\item Adjacency matrices
\end{itemize}
Adjacency Matrices
\[ \left( \begin{matrix}
o & 1 & 0 & 1 & 1 \\ 
1 & 0 & 1 & 0 & 1 \\ 
1 & 1 & 0 & 1 & 1 \\ 
0 & 1 & 1 & 1 & 1 \\ 
1 & 1 & 0 & 1 & 0
\end{matrix} \right) \]

\section*{Functions}
\begin{itemize}
\item Domain of a Function
\item Range of a function
\item Inverse of a function
\end{itemize}
\begin{itemize}
\item one-one (surjective)
\item onto (bijective)
\end{itemize}
%------------------------------------------------%
\section*{Probability}
\subsection*{Binomial Coefficients}
\begin{itemize}
\item factorials 
\[ n! = (n)\times (n-1)\times(n-2) \times \ldots \times 1 \]
\begin{itemize}
\item $5! = 5 \times 4 \times 3 \times 2 \times 1 = 120 $
\item $3! = 3 \times 2 \times 1$
\end{itemize}
\item Zero factorial
\[ 0! =  1 \]
\end{itemize}
%---------------------------------------------- %

% \[P(A |B = \frac{P(A \cap B)}{P(B)})\]

The complement rule in Probability

$P(C^{\prime}) = 1- P(C)$

 

If the probability of C is $70 \%$ then the probability of $C^{\prime}$ is $30\%$
\section{Matrices}

What are the dimensions of the following matrix


\[ \left(
\begin{array}{cc}
a_1 & a_2 \\ 
b_1 & b_2
\end{array} \right)\left(
\begin{array}{cc}
c_1 & d_1 \\ 
c_2 & d_2
\end{array} \right) = \left(
\begin{array}{cc}
(a_1 \times c_1) + (a_2 \times c_2) & (a_1 \times d_1) + (a_2 \times d_2) \\ 
(b_1 \times c_1) + (b_2 \times c_2) & (b_1 \times d_1) + (b_2 \times d_2)
\end{array} \right) \]

\bigskip
\large{
\[ \left(
\begin{array}{cc}
1 & 3 \\ 
0 & 2
\end{array} \right)\left(
\begin{array}{cc}
1 & 2 \\ 
4 & 1
\end{array} \right) = \left(
\begin{array}{cc}
(1 \times 1) + (3 \times 4) & (1 \times 2) + (3 \times 1) \\ 
(0 \times 4) + (2 \times 4) & (0 \times 2) + (2 \times 1)
\end{array} \right) = \left(
\begin{array}{cc}
14 & 5 \\ 
8 & 2
\end{array} \right) \]
}

\[ \left(
\left(
\begin{array}{cc}
1 & 2 \\ 
4 & 1
\end{array} \right)
\begin{array}{cc}
1 & 3 \\ 
0 & 2
\end{array} \right) = ? \]


\end{document}

