\documentclass[a4paper,12pt]{article}
%%%%%%%%%%%%%%%%%%%%%%%%%%%%%%%%%%%%%%%%%%%%%%%%%%%%%%%%%%%%%%%%%%%%%%%%%%%%%%%%%%%%%%%%%%%%%%%%%%%%%%%%%%%%%%%%%%%%%%%%%%%%%%%%%%%%%%%%%%%%%%%%%%%%%%%%%%%%%%%%%%%%%%%%%%%%%%%%%%%%%%%%%%%%%%%%%%%%%%%%%%%%%%%%%%%%%%%%%%%%%%%%%%%%%%%%%%%%%%%%%%%%%%%%%%%%
\usepackage{eurosym}
\usepackage{vmargin}
\usepackage{amsmath}
\usepackage{graphics}
\usepackage{epsfig}
\usepackage{subfigure}
\usepackage{fancyhdr}

\setcounter{MaxMatrixCols}{10}
%TCIDATA{OutputFilter=LATEX.DLL}
%TCIDATA{Version=5.00.0.2570}
%TCIDATA{<META NAME="SaveForMode" CONTENT="1">}
%TCIDATA{LastRevised=Wednesday, February 23, 2011 13:24:34}
%TCIDATA{<META NAME="GraphicsSave" CONTENT="32">}
%TCIDATA{Language=American English}

\pagestyle{fancy}
\setmarginsrb{20mm}{0mm}{20mm}{25mm}{12mm}{11mm}{0mm}{11mm}
\lhead{Hibernia College} \rhead{Kevin O'Brien}
\chead{Mathematics for Computing Part 2 Sheet A}
%\input{tcilatex}

\begin{document}

a

%--------------------
%--------------------


%--------------------2011 Question 2a 


Let the two sets A and B be given as

$ A =\{ 3, 1/3, \pi \}$
$ B =\{ x \in Q : x \notin Z   \}$

Let us look at the elements of A

3 (\in N ) \in Z \in Q \in R
1/3 \in  Q in R
\pi in R

B is constructed using the builder method
An real number that is not an integer is an element of B
i.e. 2.45 is an element of B, but 2 and 3 are not.

Questions
$A \cap Q$: 3 and 1/3 are in Q, but $\pi$ is not.

$A \cap R$: All three members of A are also members of R

$A \cap B$: $ 1/3$ and $\pi$ are elements of both A and B.

$A-B$: This describes the elements of A that are not also in B. From the previous question, we can see that this is 3 only.




%------------------------------------------------------------------------------------%

If U = {2, 3, 4, 7, 9, 10, 11, 13} and A = {3, 4, 9, 10}

Then

Compliment of A = A’	 = U - A
= {2, 3, 4, 7, 9, 10, 11, 13} – {3, 4, 9, 10}
= {2, 7, 11, 13}

%------------------------------------------------------------------------------------%
UNIVERSAL SET:

A universal set is a set of all elements under consideration. It is denoted by U. A Universal set is always a non-empty set.

For example: The set of real numbers R is a universal set for the operations related to real numbers.

Example:

Given that U = {5, 6, 7, 8, 9, 10, 11, 12}, list the elements of the following sets.
A = {x : x is a prime number}
B = {x : x is a factor of 60}
Solution:

The elements of sets A and B can only be selected from the given universal set U
A = {5, 7, 11}
B = {5, 6, 10, 12}

%------------------------------------------------------------------------------------%
MUTUALLY EXCLUSIVE SETS OR DISJOINT SETS:

Two sets are called disjoint if they don't have any common element.

For example, A = {2, 3, 4} and B = {5, 6, 7} are disjoint sets.

\section*{Rules of Inclusion, Listing and Cardinality}
For each of the following sets, a set is specified by the rules of inclusion method and listing method respectively. Also stated is the cardinality of that data set.
\subsection*{Worked example 1}
\begin{itemize}
\item $\{ x : x $ is an odd integer $ 5 \leq x \leq 17 \}$
\item $x = \{5,7,9,11,13,15,17\}$
\item The cardinality of set $x$ is 7.
\end{itemize}

\subsection*{Worked example 2}
\begin{itemize}
\item $\{ y : y $ is an even integer $ 6 \leq y < 18 \}$
\item $y = \{6,8,10,12,14,16\}$
\item The cardinality of set $y$ is 6.
\end{itemize}

\subsection*{Worked example 3}
A perfect square is a number that has a integer value as a
square root. 4 and 9 are perfect squares ($\sqrt{4} = 2$,
$\sqrt{9} = 3$).
\begin{itemize}
\item $\{ z : z $ is an perfect square $ 1 < z < 100 \}$
\item $z = \{4,9,16,25,36,49,64,81\}$
\item The cardinality of set $z$ is 8.
\end{itemize}


\subsection*{Exercises}
For each of the following sets, write out the set using the listing method.
Also write down the cardinality of each set.

\begin{itemize}
\item $\{ s : s $ is an negative integer $ -10 \leq s \leq 0 \}$
\item $\{ t : t $ is an even number $ 1 \leq t \leq 20 \}$
\item $\{ u : u $ is a prime number $ 1 \leq u \leq 20 \}$
\item $\{ v : v $ is a multiple of 3 $ 1 \leq v \leq 20 \}$
\end{itemize}
%-------------------------------------------------%

\section*{Power Sets}
\subsection*{Worked Example}
Consider the set $Z$:
\[ Z = \{ a,b,c\}  \]
\begin{itemize}
\item[Q1] How many sets are in the power set of $Z$?
\item[Q2] Write out the power set of $Z$.
\item[Q3] How many elements are in each element set?
\end{itemize}
%----------------------------------------------%
\subsection*{Solutions to Worked Example}

\begin{itemize}


\item[Q1] There are 3 elements in $Z$. So there is $2^3 = 8$ element sets contained in the power set.

\item[Q2] Write out the power set of $Z$.
\[ \mathcal{P}(Z) = \{ \{0\}, \{a\}, \{b\}, \{c\}, \{a,b\}, \{a,c\}, \{b,c\}, \{a,b,c\} \]

\item[Q3]
\begin{itemize}
\item[*] One element set is the null set - i.e. containing no
elements \item[*] Three element sets have only elements \item[*]
Three element sets have two elements \item[*] One element set
contains all three elements \item[*] 1+3+3+1=8
\end{itemize}
\end{itemize}
\subsection*{Exercise}
For the set $Y = \{u,v,w,x\}$ , answer the questions from the
previous exercise


%------------------------------------------------------%

\section*{Complement of a Set}
%(2.3.1)
Consider the universal set $U$ such that
\[U=\{2,4,6,8,10,12,15\} \]
For each of the sets $A$,$B$,$C$ and $D$, specify the complement sets.

\begin{center}
\begin{tabular}{|c|c|}
  \hline
Set & Complement\\
\hline $A=\{4,6,12,15\}$ &
$A^{\prime}=\{2,8,10\}$ \\ \hline $B=\{4,8,10,15\}$ & \\ \hline
$C=\{2,6,12,15\}$ & \\ \hline $D=\{8,10,15\}$ & \\ \hline

\end{tabular}
\end{center}
%-------------------------------------%
 % Binary Operations on Sets (2.3.2)
 % Union , Intersection, Symmetric Difference
 % Set Difference



\section*{Set Operations}
Consider the universal set $U$ such that
\[U=\{1,2,3,4,5,6,7,8,9\} \]
and the sets
\[A=\{2,5,7,9\} \]
\[B=\{2,4,6,8,9\} \]

\begin{itemize}
\item[(a)] $A-B$
\item[(b)] $A \otimes B$
\item[(c)] $A \cap B$
\item[(d)] $A \cup B$
\item[(e)] $A^{\prime} \cap B^{\prime}$
\item[(f)] $A^{\prime} \cup B^{\prime}$
\end{itemize}


\section*{Venn Diagrams}

Draw a Venn Diagram to represent the universal set
$\mathcal{U} = \{0,1,2,3,4,5,6\}$ with subsets
$A = \{2,4,5\}$
$B = \{1,4,5,6\}$

Find each of the following
\begin{itemize}
\item[(a)] $A \cup B $
\item[(b)] $A \cap B $
\item[(c)] $A-B$
\item[(d)] $B-A$
\item[(e)] $A \otimes B$
\end{itemize}

\newpage
\subsection*{Conversion to Binary Form}
\[
\begin{tabular}{|c|c|c|c|}
	&		&	Quotient	&	Remainder	\\
507	&\emph{	253.5	}&	253	&	1	\\
253	&\emph{	126.5	}&	126	&	1	\\
126	&\emph{	63	}&	63	&	0	\\
63	&\emph{	31.5	}&	31	&	1	\\
31	&\emph{	15.5	}&	15	&	1	\\
15	&\emph{	7.5	}&	7	&	1	\\
7	&\emph{	3.5	}&	3	&	1	\\
3	&\emph{	1.5	}&	1	&	1	\\
1	&\emph{	0.5	}&	0	&	1	\\
\end{tabular} \]

Correct Answer : 111111011


2011 Question 1C

List all the set of positive integers with precisely 3 bits in binary notation

\[\begin{array}{|c|c|}
100 & 4 \\
101 & 5 \\
110 & 6 \\
111 & 7 \\
\end{array}\]

Let n be a positive integer. How many positive integers have precisely n bits in binary notation. 

\newpage
Express the binary number (1011.011) as a decimal, showing all of your working
\[\begin{tabular}{|c|c|c|c|}
Digit	&	Power	&	Weight	&		\\
1	&	3	&	2^{3} = 8	&	8	\\
0	&	2	&	2^{2} =4	&	0	\\
1	&	1	&	2^{1}=2	&	2	\\
1	&	0	&	2^{0}=1	&	1	\\
.	&	.	&	.	&	.	\\
0	&	-1	&	2^{-1}=0.5	&	0	\\
1	&	-2	&	2^{-1}=0.25	&	0.25	\\
1	&	-3	&	2^{-3}=0.125	&	0.125	\\
	&		&		&	11.375	\\
\hline 
\end{tabular} \]





\newpage
\subsection*{2011 Question 1b}

Working in Binary, and showing all carries, compute the following:\\
$(11010)_{2} + (111)_{2}$ (i.e. $(26)_10 +7_{10}$)
\\
(Demonstration on whiteboard)
\\
Answer: 11010 + 111 = 100001
\\


\begin{itemize} 
\item Remark 0 is not a positive integer.
\item How many have 1 bit : Answer only one 1.
\item How many have 2 bits :  Answer only two : 10 and 11 (i.e. 2 and 3 in decimal)
\item From above: 4 have 3 bits.
\item What is the first 4 bit numbers? : 1000 (i.e 8 in decimal)
\item What is the last 4 bit number? 1111 (i.e 15 in decimal)
\item How many numbers have 4 bits? Answer :8
\item Answer to question: $2^{n-1}$
\item Return to this after "Proof by Induction".
\end{itemize}
\newpage

\subsection*{Hexadecimal}
\begin{itemize}
\item Hexadecimal basic concepts:	
\item Hexadecimal is base 16.  
\item There are 16 digits in counting (0, 1, 2, 3, 4, 5, 6, 7, 8, 9, A, B, C, D, E, F)
\item When you reach 16, you carry a “1” over to the next column
\item The number after F (decimal 15) is 10 in hex (or 16 in decimal)
\end{itemize}
%---------------------------------------------------------

\begin{tabular}{|c||c|c|c|c|c|c|}
\hline Hex & A & B  & C & D & E & F \\  \hline
\hline Dec & 10 & 11 & 12 & 13  &14  &15  \\  \hline
\hline 
\end{tabular} 

\textbf{Conversion to Decimal}
\[\textbf{5} \times 16^2 + \textbf{A} \times 16^1  + \textbf{9} \times 16^0\]
\[=5 \times 16^2 + 10 \times 16^1  + 9 \times 16^0\]
\[= 1280 + 160 + 9\]


% Use the following steps to perform hexadecimal addition:

\subsection*{Hexadecimal Addition}
\begin{itemize}
\item		Add one column at a time.

\item		Convert to decimal and add the numbers.

\item	If the result of step two is 16 or larger subtract the result from 16 and carry 1 to the next column.

\item	If the result of step two is less than 16, convert the number to hexadecimal.

\end{itemize}

\[\begin{array}{cccc}
	&		&		&		\\	
	&	5	&	A	&	9	\\	
	-&	6	&	9	&	4	\\	\hline
	&		&		&		\\	
\end{array}\]

\subsection*{Hexadecimal Addition}
\begin{itemize}
\item		Add one column at a time.

\item		Convert to decimal and add the numbers.

\item	If the result of step two is 16 or larger subtract the result from 16 and carry 1 to the next column.

\item	If the result of step two is less than 16, convert the number to hexadecimal.

\end{itemize}
\[\begin{array}{cccc}
	&		&		&		\\	
	&	5	&	A	&	9	\\	
	+&	6	&	9	&	4	\\	\hline
	&		&		&		\\	
\end{array} \]

\noindent 1. Add one column at a time\\
2. Convert to decimal and add (9 + 4 = 13)\\
3. Decimal 13 is hexadecimal D\\
4. Next column :Convert to decimal and add (10 + 9 = 19)\\
5. (19 larger than 16) 14 in Hexadecimal \\
6. Leave 4 carry to one\\
7.  Next Colulm add 1 + 5 + 6 = 12)\\
8. Decimal 12 is hexadecimal C\\

\[\begin{array}{cccc}
	&		&		&		\\	
	&	5	&	A	&	9	\\	
	+&	6	&	9	&	4	\\	\hline
	&	C	&	4	&	D	\\	
\end{array} \]

\subsection*{Hexadecimal Subtraction}
\[\begin{array}{cccc}
	&		&		&		\\	
	&	B	&	B	&	B	\\	
	-&	A	&	5	&	D	\\	\hline
	&		&		&		\\	
\end{array} \]


\[\begin{array}{cccc}
	&		&		&		\\	
	&	B	&	\textbf{\emph{A}}	&	\textbf{\emph{1B}}	\\	
	&	A	&	5	&	D	\\	\hline
	&		&		&		\\	
\end{array} \]


\newpage
\begin{verbatim}
If U = {2, 3, 4, 7, 9, 10, 11, 13} and A = {3, 4, 9, 10}

Then

Compliment of A = A’	 = U - A
= {2, 3, 4, 7, 9, 10, 11, 13} – {3, 4, 9, 10}
= {2, 7, 11, 13}

\begin{verbatim}
UNIVERSAL SET: 

A universal set is a set of all elements under consideration. It is denoted by U. 
A Universal set is always a non-empty set.

For example: The set of real numbers R is a universal set
for the operations related to real numbers.

Example: 

Given that U = {5, 6, 7, 8, 9, 10, 11, 12}, list the elements of the following sets. 
A = {x : x is a prime number}
B = {x : x is a factor of 60}
Solution: 

The elements of sets A and B can only be selected 
from the given universal set U 
A = {5, 7, 11}
B = {5, 6, 10, 12}
\end{verbatim}

\newpage
\begin{verbatim}
MUTUALLY EXCLUSIVE SETS OR DISJOINT SETS: 

Two sets are called disjoint if they don't have any common element. 

For example, A = {2, 3, 4} and B = {5, 6, 7} are disjoint sets.

\end{verbatim}

\end{document}








%-------------------------------------% Ellipsis

When using Ellipsis, it should be clear what the pattern is

%-------------------------------------%


%-----------Reference to section 2.2.3 Power Sets




%-------------------------------------%

 %----(Reference to Section 2.2.2 Cardinality)



%-------------------------------------% % Complement of a set


%--------------------------------------%
\subsection*{ Three Sets }

%- Section 2.3.5 %- Associative Law %- Distribution Law





%-------------------------------------% %- Section 3
Propositional Logic A statement is a declarative sentence that
is either true or false.
\begin{itemize}
\item $\tilde q$ not q \item $p \vee q$ \item $p \wedge \tilde
q$
\end{itemize}

%-------------------------------%
------------------------------------------------
Question 1


$\{ s :  s is an odd integer and 2 \leq s \leq 10 \}$
$\{ 2m :  m \element Z and 5 \leq m \leq 10 \}$
$\{ 2^t :  t \element Z and 0 \leq t \leq 5 \}$

------------------------------------------------
Question 2

\{12,13,14,15,16,17\}
{0,5,-5,10,-10,15,-15,.....\}
{6,8,10,12,14,16,18\}
Question 5

-----------------------------------------------------


--------------------------------------------

Let A, B be subsets of the universal set \mathcal{U}.

Use membership tables to prove De Morgan's Laws.

---------------------------------------------
Draw a venn diagram to show three subsets A,B and C of a universal set U intersecting in
the most general way?
How are sets $X$ and $Z$ related?
Can you describe each of the subsets X,Y and Z in terms  of the
sets A,B,C using the operations union intersection and set complement.

Construct Membership tables for each of the sets
(A-B) - C
A-(B- C)

(A-B) -C = A-(B-C)
A



\begin{itemize}
\item[a.] (1 mark) Write out the sample space for the outcomes for both players A and B.
\item[b.] (1 mark) Write out the sample space for the outcomes of C, where C is the difference of the two scores (i.e. B-A)
\item[c.] (1 mark) Are the sample points for the sample space of C equally probable? Provide a brief justification for your answer.
\end{itemize}

%----------------------------------------------------------%
\newpage
\section*{Section B: Set Operations}
\begin{itemize}
\item[B.1] complement of A $A^{\prime}$
\item[B.2] Union $A \cup B$
\item[B.3] Intersection $A \cap B$
\item[B.4] Relative Difference $A \otimes B$
\item[A.5]
\item[A.6]
\item[A.7]
\item[A.8]
\end{itemize}
\newpage


\begin{itemize}
\item Specifying Sets
\item Listing Method
\item Rules of Inclusion method
\end{itemize}


\begin{itemize}
\item Subsets Notation of a subset
\item Cardinality of a set
\item Power of a set
\end{itemize}

\subsection*{Operation on Sets}

\begin{itemize}
\item The complement of Set
\item Binary Operations
\begin{itemize}
\item Union
\item Intersection
\end{itemize}
\item Membership tables
\item Laws for Combining Sets
\end{itemize}
%----------------------------------------------------------%
\subsection*{Membership Tables}
Using membership tables
\begin{tabular}{|ccc|c|c|c|}
  \hline
  % after \\: \hline or \cline{col1-col2} \cline{col3-col4} ...
  A & B & C & x & y & z \\\hline
  0 & 0 & 0 & 1 & 1 & 1 \\
  0 & 0 & 1 & 0 & 0 & 1 \\
  0 & 1 & 0 & 0 & 0 & 1 \\
  0 & 1 & 1 & 0 & 0 & 1 \\
  1 & 0 & 0 & 1 & 0 & 1 \\
  1 & 0 & 1 & 1 & 0 & 1 \\
  1 & 1 & 0 & 0 & 0 & 1 \\
  1 & 1 & 1 & 1 & 0 & 1 \\
  \hline
\end{tabular}

%----------------------------------------------------------%
\section*{Section B: Hexadecimal Numbers}
\begin{itemize}
\item[B.1]
\item[B.2]
\item[B.3]
\item[B.4]
\item[B.5]
\item[B.6]
\item[B.7]
\item[B.8]
\end{itemize}
\newpage


\subsection*{Associative Laws}
\[ (A \cup B) \cup C =  A \cup (B \cup C)  \]
\[ (A \cap B) \cap C =  A \cap (B \cap C)  \]

\subsection*{Distributive Laws}
\[ (A \cup B) \cap C =  (A \cup B) \cap (A \cup C)  \]
\[ (A \cap B) \cup C =  (A \cap B) \cup (A \cap C)  \]


\[ (A \cup B) \cap B^{\prime} \]
%----------------------------------------------------------%
\section*{Section C: Real and Rational Numbers}
\begin{itemize}
\item[C.1]
\item[C.2]
\item[C.3]
\item[C.4]
\item[C.5]
\item[C.6]
\item[C.7]
\item[C.8]
\end{itemize}


\end{document} 