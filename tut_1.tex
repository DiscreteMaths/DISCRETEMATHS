\documentclass[]{article}
\voffset=-1.5cm
\oddsidemargin=0.0cm
\textwidth = 470pt
\usepackage[utf8]{inputenc}
\usepackage[english]{babel}
\usepackage{framed}

\usepackage{multicol}
\usepackage{amsmath}
\usepackage{amssymb}
\usepackage{enumerate}
\usepackage{multicol}




%opening
\title{Binary Arithmetic - Tutorial Sheet}


\begin{document}
\begin{enumerate}
    \item Suppose we have the sets A and B defined as follows:

\[ A = \{ \sqrt{2}, \frac{3}{2}, 2 \}\]

$ \{ B = \{ x \in R :  X \notin Q \}  $

\begin{itemize}
\item $A \cap Q$
\item $A \cap B$
\item $B \cup Q$
\end{itemize}

\item 
\begin{itemize}
	\item[(i)] Express the recurring decimal $0.727272\ldots$ as a rational number in its simplest form.
\end{itemize}
%-------------------------------------------------------- %
\item 
\begin{itemize}
\item[(i)] Given x is the irrational positive number $\sqrt{2}$, express $x^8$ in binary notation\\
\item[(ii)] From part (i), is $x^8$ a rational number?
\end{itemize}
\item Express the recurring decimal $0.727272\ldots$ as a rational number in its simplest form.

\item Express 42900 as a product of its prime factors, using index notation for repeated factors
\begin{framed}
\[
2 \times 2 \times 3 \times 5 \times 5 \times 11 \times 13 \]

Using index notations

\[2^2 \times 3 \times 5^2 \times 11 \times 13 \]
\end{framed}
%-------------------------------------------------------- %
\item 
\begin{itemize}
\item[(i)] Given x is the irrational positive number $\sqrt{2}$, express $x^8$ in binary notation\\
\item[(ii)] From part (i), is $x^8$ a rational number?
\end{itemize}



\item How many symbols are used in each number system. State the symbol set?

%------------------------%
\begin{enumerate}[(i)]
\item Decimal (Base 10)
\item Binary (Base 2) 
\item Hexadecimal (Base 16)
\end{enumerate}
%------------------------%


\end{enumerate}

\end{document}