\documentclass{beamer}

\usepackage{graphics}
\usepackage{amsmath}
\usepackage{amssymb}

\begin{document}

%Slide 1 Title Slide
\begin{frame}
\bigskip
{
\Huge
\[ \mbox{Numbers and Set Theory}\]
}
{
\Large
\[ \mbox{kobriendublin.wordpress.com}\]
\[ \mbox{Youtube: StatsLabDublin}\]
}
\end{frame}



%------------------------------------------------- %

%Slide 2 - How Information is presented - Check
\begin{frame}
\frametitle{Numbers and Set Theory}
\LARGE
Suppose we have the sets \textit{\textbf{A}} and \textbf{\textit{B}} defined as follows:

\[ \boldsymbol{A} = \{\; \sqrt{2},\; \frac{3}{2},\; 2\; \}\]
\begin{center}
$ \boldsymbol{B} = \{ x \in \mathbb{R} :  x \notin  \mathbb{Q} \}  $
\end{center}
\Large
\begin{itemize}
\item[1] $\boldsymbol{A} \cap \mathbb{Q}$
\item[2] $\boldsymbol{A} \cap \boldsymbol{B}$
\item[3] $\boldsymbol{B} \cup \mathbb{Q}$
\end{itemize}
\end{frame}

%Slide 3-------------------- %
\begin{frame}
\frametitle{Numbers and Set Theory}
\LARGE
\begin{itemize}
\item $ \mathbb{R}$ Set of all real numbers.
\item $ \mathbb{Q}$ Set of all quotient numbers.
\end{itemize}

\[ \boldsymbol{B} = \{ x \in \mathbb{R} :  x \notin  \mathbb{Q} \}\]  

Set of all real numbers that are not quotients (i.e. numbers that can not be expressed as a division of one integer by another).


\end{frame}

%Slide 3-------------------- %
\begin{frame}
\frametitle{Numbers and Set Theory}
\LARGE
\[\boldsymbol{A} \cap \mathbb{Q}\]

\end{frame}

\end{document}