

\documentclass[12pt]{article} % use larger type; default would be 10pt


\usepackage{graphicx} 
\usepackage{framed}



\title{Hibernia College}
\author{Kevin O'Brien}

\begin{document}


\begin{center}
\huge{Mathematics for Computing}\\
\LARGE{Logic}
\end{center}

\begin{itemize}
\item (2 Marks) Covert the decimal integer $(407)_{10}$ to binary notation.

\item (2 Marks) Showing your working, express the following number 
\[ 1.024024024024\ldots\]
as a ration number in its simplest form.

\end{itemize}
%---------------------------%

% Propositions
% Logical Operations
% Conditional Connectives
% Logical Proofs
% De Morgan's Law

%-----------------------------------%

% \frametitle{Propositions}

\begin{itemize}
\item Contra-positives
\end{itemize}


% \frametitle{Logical Notation}

\begin{center}
\begin{tabular}{|c|c|}
Logical & Negation \\ \hline
$p$	& $\neg p$ \\ \hline
$T$	& $F$      \\ \hline
$F$	& $T$      \\ \hline
\end{tabular}
\end{center}


%-----------------------------------%

(2.2.2) Cardinality 
The number of distinct elements in a finite set is called its cardinality.


(2.2.3) Power Set

(2.3) Operations on Sets

(2.3.1) Complement of a set
(2.3.2) Binary Operations on Sets
-Union
-Intersection
-Set Difference
-Symmetric Difference 


$ A \otimes B$

{2, 3, 4, 6, 7, 8}

Rules of Inclusion
{1,2,3,4,5,6}
{0,1,2,3,4,5,6,7,8,9}
{-3,0,3,6,9,12}

%================================= %
$ A \in S$
$ A \in S$
$ A \cup B$
$ A \cap B$
%================================= %
\subsection*{Example 1: }
\begin{verbatim}
If $U = {1, 2, 3, 4, 5, 6, 7, 8, 9, 10}$, \\

$A = {2, 4, 6, 8, 10}$, \\
$B = (1, 3, 6, 7, 8}$ \\
$C = {3, 7}$, \\

find $A \cap B$, $A \cup C$, $B \cap A^{c}$, $B \cap C^{c}$
\end{verbatim}
\begin{verbatim}
Solution: 

$U = {1, 2, 3, 4, 5, 6, 7, 8, 9, 10}$\\
$A = {2, 4, 6, 8, 10}$\\
%$B = (1, 3, 6, 7, 8} $\\
$C = {3, 7}$\\

%A n B = {6, 8}
%A \cap C = {2, 3, 4, 6, 7, 8, 10}
%B n A^C = {1, 3, 7}
%B n C^C = {1, 6, 8}
\end{verbatim}
\begin{verbatim}

%> A
%[1]  1  8  9 10
%> B
%[1] 4 7 9
%> C
%[1] 1 2 3 4 9
%> D
%[1] 2 5 6
\end{verbatim}



\section{Logic}
\begin{itemize}
\item The NOT operator $\neg$ (also known as the negation operator)
\end{itemize}
\subsection{Truth Tables}
{
\Large
\begin{center}
\begin{tabular}{|c|c||c|c||c|c|}
\hline $p$ & $q$ & $p \wedge q$ & $p\vee q$ & $\neg p$  & $\neg (p \wedge q)$ \\ 
\hline 0 & 0 & 0 & 0 & 1 & 1 \\ 
\hline 0 & 1 & 0 & 1 & 1 & 1 \\ 
\hline 1 & 0 & 0 & 1 & 0 & 1 \\ 
\hline 1 & 1 & 1 & 1 & 0 & 0 \\ 
\hline 
\end{tabular} 
\end{center}
}
%-------------------------%

\subsection*{Prepositional Logic}







\section{Section 3 Logic}
\subsection{Logical Operations}
\begin{itemize}
\item $\neg p$ the negation of proposition $p$.
\item $p \wedge q$ Both propositions p and q are simultaneously true (Logical State AND)
\item $p \vee q $ One of the propositions is true, or both (Logical State : OR)
\item $p \otimes q$ Only one of the propositions is true (Logical State : exclusive OR (i.e XOR)
\end{itemize}
\begin{center}
\begin{tabular}{|c|c|c|c|c|}
\hline
p & q & $p \vee q$ & $q \wedge p$ & $p \otimes q$ \\
\hline
0 & 0 & 0 & 0 & 0 \\
0 & 1 & 1 & 0 & 1\\
1 & 0 & 1 & 0 & 1 \\
1 & 1 & 1 & 1 & 0\\
\hline
\end{tabular}
\end{center}
%---------------------------------------------------------%
\section{Conditional Connectives}
Construct the truth table for the proposition $p \rightarrow q$.

\begin{center}
\begin{tabular}{|c|c|c|c|}
\hline
p & q & $p \rightarrow q$ & $q \rightarrow p$ \\
\hline
0 & 0 & 1& 1 \\
0 & 1 & 1 & 0 \\
1 & 0 & 0 & 1 \\
1 & 1 & 1 & 1 \\
\hline
\end{tabular}
\end{center}


%-------------------------------------------------------------------------%
\newpage

\section{Section 4 Functions}

\subsection{Invertible Functions}
A function is invertible if it fulfils two criteria
\begin{itemize}
\item The function is \textbf{\textit{onto}},
\item The function is \textbf{\textit{one-to-one}}.
\end{itemize}

State the conditions to be satisfied by a function
$f : X \leftarrow Y$ for it to have an inverse function
$f^{-1} : Y \leftarrow X$.
%---------------------------------------------------------%

$\lceil \frac{x^2+1}{4} \rceil$
where $f : A \rightarrow \textbf{Z}$
\begin{itemize}
\item[(i)] Find $f(4)$ and the ancestors of 3.
\item[(ii)] Find the range of $f$.
\item[(iii)] Is f invertible? Justify your answer
\end{itemize}

Given $f : \textbf{R} \rightarrow \textbf{R}$ where f(x) =3x-1,define fully
the inverse of the function f ,i.e.$f^{-1}$. 
State the value of $f^{-1}(2)$
%---------------------------------------------------------%
\subsection{Precision Functions}

\begin{itemize}
\item Absolute Value Function $| x |$
\item Ceiling Function $\lceil x \rceil$
\item Floor Function  $\lfloor x \rfloor $
\end{itemize}
%\[ \lfloorx\rfloor\]

\noindent \textbf{Question1.2}: State the range and domain of the following function
\[ F(x) = \lfloor x-1 \rfloor \]



%---------------------------------------------------------%

\section{graph theory }
Given the following definitions for simple, connected graphs:
\begin{itemize}
\item $K_n$ is a graph on $n$ vertices where each pair of vertices is connected by an edge;
\item $C_n$ is the graph with vertices $v_1, v_2, v_3, \dots, v_n$ and edges $\{v_1,v_2\}, \{v_2,v_3\}, \dots\{v_n, v_1\}$;
\item $W_n$ is the graph obtained from $C_n$ by adding an extra vertex,$v_{n+1}$, and edges
from this to each of the original vertices in $C_n$.
\end{itemize}
(a) Draw $K_4$, $C_4$, and $W_4$. 
%----------------------------------------------------------------%
\newpage
\section{Digraphs and Relatiosn}
% 2007 Q8
Given a flock of chickens, between any two chickens one of them is
dominant. A relation, R, is defined between chicken x and chicken y as xRy if x is
dominant over y. This gives what is known as a pecking order to the flock. Home
Farm has 5 chickens: Amy, Beth, Carol, Daisy and Eve, with the following relations:

\begin{itemize}
\item Amy is dominant over Beth and Carol
\item Beth is dominant over Eve and Carol
\item Carol is dominant over Eve and Daisy
\item Daisy is dominant over Eve, Amy and Beth
\item Eve is dominant over Amy.
\end{itemize}

\newpage
\section{Counting}
% 2007 Q8
Given S is the set of all 5 digit binary strings, E is the set of a 5 digit
binary strings beginning with a 1 and F is the set of all 5 digit binary strings ending
with two zeroes.
\begin{itemize}
\item[(a)] Find the cardinality of S, E and F.
\item[(b)] Draw a Venn diagram to show the relationship between the sets S, E and F.
Show the relevant number of elements in each region of your diagram.
\end{itemize}
%-------------------------------------------------------------------------%
\newpage
\end{document}