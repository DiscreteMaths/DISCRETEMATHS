
%====================================================================================================================== %
\begin{frame}
	\frametitle{The Master Theorem}
	\end{frame}
%====================================================================================================================== %
\begin{frame}
	\frametitle{The Master Theorem}
	A finite-state machine (FSM) or finite-state automaton (plural: automata), or simply a state machine, is a mathematical model of computation used to design both computer programs and sequential logic circuits. It is conceived as an abstract machine that can be in one of a finite number of states. The machine is in only one state at a time; the state it is in at any given time is called the current state. It can change from one state to another when initiated by a triggering event or condition; this is called a transition. A particular FSM is defined by a list of its states, and the triggering condition for each transition.
\end{frame}
%====================================================================================================================== %
\begin{frame}
	\frametitle{The Master Theorem}
	The behavior of state machines can be observed in many devices in modern society which perform a predetermined sequence of actions depending on a sequence of events with which they are presented. Simple examples are vending machines which dispense products when the proper combination of coins is deposited, elevators which drop riders off at upper floors before going down, traffic lights which change sequence when cars are waiting, and combination locks which require the input of combination numbers in the proper order.
\end{frame}
%====================================================================================================================== %
\begin{frame}
	\frametitle{The Master Theorem}
	Finite-state machines can model a large number of problems, among which are electronic design automation, communication protocol design, language parsing and other engineering applications. In biology and artificial intelligence research, state machines or hierarchies of state machines have been used to describe neurological systems. In linguistics, they are used to describe simple parts of the grammars of natural languages.
\end{frame}
%====================================================================================================================== %
\begin{frame}
	\frametitle{The Master Theorem}
	Considered as an abstract model of computation, the finite state machine is weak; it has less computational power than some other models of computation such as the Turing machine.[1] That is, there are tasks which no FSM can do, but some Turing machines can. This is because the FSM memory is limited by the number of states.
FSMs are studied in the more general field of automata theory.
\end{frame}
%====================================================================================================================== %
\begin{frame}
	\frametitle{The Master Theorem}
	
\end{document}