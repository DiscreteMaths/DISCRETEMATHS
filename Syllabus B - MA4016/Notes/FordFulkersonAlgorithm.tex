\begin{frame}
\frametitle{Ford–Fulkerson algorithm}
The Ford–Fulkerson method or Ford–Fulkerson algorithm (FFA) is an algorithm which computes the maximum flow in a flow network. It is called a "method" instead of an "algorithm" as the approach to finding augmenting paths in a residual graph is not fully specificied[1] or it is specified in several implementations with different running times.[2] It was published in 1956 by L. R. Ford, Jr. and D. R. Fulkerson.[3] 

The name "Ford–Fulkerson" is often also used for the Edmonds–Karp algorithm, which is a specialization of Ford–Fulkerson.
\end{frame}
%========================================================================================== %
\begin{frame}
\frametitle{Ford–Fulkerson algorithm}

The idea behind the algorithm is as follows: 
\begin{itemize}
\item as long as there is a path from the source (\textbf{start node}) to the sink (end node), with available capacity on all edges in the path, we send flow along one of the paths. \item Then we find another path, and so on. 
\item A path with available capacity is called an augmenting path.
\end{itemize}
\end{frame}
\end{document}