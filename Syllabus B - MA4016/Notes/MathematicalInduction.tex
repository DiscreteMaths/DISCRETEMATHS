Mathematical Induction
Mathematical Induction (MI) is a proof method used to establish the truth of a statement
P about the natural number n. Examples of such statements are:
• 1 + 2 + 3 + · · · + n =
n(n+1)
2
• The number of subsets of a set with n elements is 2n
The steps involved in MI are
1. Base Step : Establish P for some initial value n = n0
2. Inductive Step: Show that if P is true for n = k, then P is also true for n = k + 1.
The combination of these two steps ensures that P is true for all n greater than or equal
to n0.(why ?)
Example: Use MI to prove P : 1 + 2 + 3 + · · · + n =
n(n+1)
2
Base Step: When n = 1 the left hand side (LHS) of P is 1. The RHS is 1(1+1)
2 = 1.
Inductive Step: Assume P is true for n = k i.e.
1 + 2 + 3 + · · · + k =
k(k + 1)
2
When n = k + 1, the LHS of P is
1 + 2 + 3 + · · · + k + (k + 1) = [1 + 2 + 3 + · · · + k] + (k + 1)
And using the inductive assumption, this becomes
k(k + 1)
2
+ (k + 1) = (k + 1)(k
2
+ 1) = (k + 1)(k + 2)
2
which is the RHS of P when n = k + 1. This completes the proof, and shows that the
result is true for n ≥ 1.
The version of MI given in the box above is called weak induction. Strong induction
consists of
1. Base Step : Establish P for some initial value n = n0
2. Inductive Step: Show that if P is true for n = 1, 2, 3, . . . , k, then P is also true for
n = k + 1.
Both versions are equivalent.