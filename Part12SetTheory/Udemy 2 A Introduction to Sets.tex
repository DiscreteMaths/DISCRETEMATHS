\documentclass{beamer}

\usepackage{graphicx}
\usepackage{framed}
\begin{document}


%--------------------------------%
\begin{frame}
\frametitle{Set Theory Introduction}
\Large
\begin{itemize}
\item A set is simply a collection of things or objects, of any kind. 
\item These objects
are called elements or members of the set. We refer to the set as an
entity in its own right and often denote it by A, B, C or D, etc.
\item 
If A is a set and x a member of the set, then we say $x \in A$ i.e. x ‘belongs to’
A. The symbol $\notin$ denotes the negation of $in$  i.e. x $\notin$ A means ‘x does not
belong to’ A.
\end{itemize}
\end{frame}
%--------------------------------%
\begin{frame}
\frametitle{Set Theory Introduction}
\Large
\begin{itemize}
\item The elements of a set, and hence the set itself, are characterised by having
one or more properties that distinguish the elements of the set from those
not in the set. 
\item For example, if C is the set of non-negative real numbers, then we
might use the notation
\[C = \{x / x \mbox{ is a real number and} x \neq 0\}\]
\item We would verbalise this as \textit{the set of all x such that x is a real number and non-negative}.
\end{itemize}
\end{frame}

\section{Differences and complements}

%--------------------------------%
\begin{frame}
\frametitle{Differences and complements}
\Large
\begin{itemize}
\item If A and B are sets then the difference set A − B is the set of all elements
of A which do not belong to B.
\item If B is a sub-set of A, then A − B is sometimes called the complement of
B in A. When A is the universal set one may simply refer to the
complement of B to denote all things not in B. \item The complement of a set A
is denoted as $A^c$ or $A^{\prime}$.
\end{itemize}
\end{frame}

%------------------------------------%
\section{De Morgan's Laws}

\end{document}
