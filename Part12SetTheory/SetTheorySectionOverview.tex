\documentclass[12pt]{article}

\usepackage{amsmath}
\usepackage{amssymb}

\begin{document}

\begin{center}
\huge{Mathematics for Computing}\\
\LARGE{Set Thoery}
\end{center}


\section{Set Theory}
\subsection{Important Operations in Set Theory}
%---------------------------------%
\begin{itemize}
\item Union ($\cup$) - also known as the OR operator
\item Intersection ($\cap$) - also known as the AND operator
\end{itemize}
%---------------------------------%
\subsection{Number Sets}
The font that the symbols are written in (i.e. $\mathbb{N}$, $\mathbb{R}$) is known as \textit{\textbf{blackboard font}}.
\begin{itemize}
\item $\mathbb{N}$ Natural Numbers ($0,1,2,3$) 
(Not used in the CIS102 Syllabus)
\item $\mathbb{Z}$ Integers ($-3,-2,-1,0,1,2,3, \ldots$)
\begin{itemize}
\item[$\ast$] $\mathbb{Z}^{+}$ Positive Integers
\item[$\ast$] $\mathbb{Z}^{-}$ Negative Integers
\end{itemize}
\item $\mathbb{Q}$ Rational Numbers
\item $\mathbb{R}$ Real Numbers
\end{itemize}

%---------------------------%
\newpage
\section{Logic}
\begin{itemize}
\item The NOT operator $\neg$ (also known as the negation operator)
\end{itemize}
\subsection{Truth Tables}
{
\Large
\begin{center}
\begin{tabular}{|c|c||c|c||c|c|}
\hline $p$ & $q$ & $p \wedge q$ & $p\vee q$ & $\neg p$  & $\neg (p \wedge q)$ \\ 
\hline 0 & 0 & 0 & 0 & 1 & 1 \\ 
\hline 0 & 1 & 0 & 1 & 1 & 1 \\ 
\hline 1 & 0 & 0 & 1 & 0 & 1 \\ 
\hline 1 & 1 & 1 & 1 & 0 & 0 \\ 
\hline 
\end{tabular} 
\end{center}
}
%-------------------------%


\newpage
\section{Functions}

\subsection{\textit{One-to-One} Functions and \textit{Onto} Functions}


\subsection{Mathematical Operators}
\begin{itemize}
\item The Square Root function
\item The Floor and Ceiling functions
\item The Absolute Value functions
\end{itemize}

\subsection{Exponential and Logarithms}
\subsubsection*{Laws of Logarithms}

\begin{itemize}
\item Law 1 : Multiplcation of Logarithms
\[ Log(a) \times Log(b) = Log(a+b) \]
\item Law 2 : Division of Logarithms
\[ \frac{Log(a)}{Log(b)} = Log(a-b) \]
\item Law 3 : : Powers of Logarithms
\[ Log(a^b) = b \times Log(a) \]
\end{itemize}


\end{document}
